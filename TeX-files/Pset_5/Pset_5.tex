\documentclass[11pt]{article}

    \usepackage[breakable]{tcolorbox}
    \usepackage{parskip} % Stop auto-indenting (to mimic markdown behaviour)
    
    \usepackage{iftex}
    \ifPDFTeX
    	\usepackage[T1]{fontenc}
    	\usepackage{mathpazo}
    \else
    	\usepackage{fontspec}
    \fi

    % Basic figure setup, for now with no caption control since it's done
    % automatically by Pandoc (which extracts ![](path) syntax from Markdown).
    \usepackage{graphicx}
    % Maintain compatibility with old templates. Remove in nbconvert 6.0
    \let\Oldincludegraphics\includegraphics
    % Ensure that by default, figures have no caption (until we provide a
    % proper Figure object with a Caption API and a way to capture that
    % in the conversion process - todo).
    \usepackage{caption}
    \DeclareCaptionFormat{nocaption}{}
    \captionsetup{format=nocaption,aboveskip=0pt,belowskip=0pt}

    \usepackage[Export]{adjustbox} % Used to constrain images to a maximum size
    \adjustboxset{max size={0.9\linewidth}{0.9\paperheight}}
    \usepackage{float}
    \floatplacement{figure}{H} % forces figures to be placed at the correct location
    \usepackage{xcolor} % Allow colors to be defined
    \usepackage{enumerate} % Needed for markdown enumerations to work
    \usepackage{geometry} % Used to adjust the document margins
    \usepackage{amsmath} % Equations
    \usepackage{amssymb} % Equations
    \usepackage{textcomp} % defines textquotesingle
    % Hack from http://tex.stackexchange.com/a/47451/13684:
    \AtBeginDocument{%
        \def\PYZsq{\textquotesingle}% Upright quotes in Pygmentized code
    }
    \usepackage{upquote} % Upright quotes for verbatim code
    \usepackage{eurosym} % defines \euro
    \usepackage[mathletters]{ucs} % Extended unicode (utf-8) support
    \usepackage{fancyvrb} % verbatim replacement that allows latex
    \usepackage{grffile} % extends the file name processing of package graphics 
                         % to support a larger range
    \makeatletter % fix for grffile with XeLaTeX
    \def\Gread@@xetex#1{%
      \IfFileExists{"\Gin@base".bb}%
      {\Gread@eps{\Gin@base.bb}}%
      {\Gread@@xetex@aux#1}%
    }
    \makeatother

    % The hyperref package gives us a pdf with properly built
    % internal navigation ('pdf bookmarks' for the table of contents,
    % internal cross-reference links, web links for URLs, etc.)
    \usepackage{hyperref}
    % The default LaTeX title has an obnoxious amount of whitespace. By default,
    % titling removes some of it. It also provides customization options.
    \usepackage{titling}
    \usepackage{longtable} % longtable support required by pandoc >1.10
    \usepackage{booktabs}  % table support for pandoc > 1.12.2
    \usepackage[inline]{enumitem} % IRkernel/repr support (it uses the enumerate* environment)
    \usepackage[normalem]{ulem} % ulem is needed to support strikethroughs (\sout)
                                % normalem makes italics be italics, not underlines
    \usepackage{mathrsfs}
    

    
    % Colors for the hyperref package
    \definecolor{urlcolor}{rgb}{0,.145,.698}
    \definecolor{linkcolor}{rgb}{.71,0.21,0.01}
    \definecolor{citecolor}{rgb}{.12,.54,.11}

    % ANSI colors
    \definecolor{ansi-black}{HTML}{3E424D}
    \definecolor{ansi-black-intense}{HTML}{282C36}
    \definecolor{ansi-red}{HTML}{E75C58}
    \definecolor{ansi-red-intense}{HTML}{B22B31}
    \definecolor{ansi-green}{HTML}{00A250}
    \definecolor{ansi-green-intense}{HTML}{007427}
    \definecolor{ansi-yellow}{HTML}{DDB62B}
    \definecolor{ansi-yellow-intense}{HTML}{B27D12}
    \definecolor{ansi-blue}{HTML}{208FFB}
    \definecolor{ansi-blue-intense}{HTML}{0065CA}
    \definecolor{ansi-magenta}{HTML}{D160C4}
    \definecolor{ansi-magenta-intense}{HTML}{A03196}
    \definecolor{ansi-cyan}{HTML}{60C6C8}
    \definecolor{ansi-cyan-intense}{HTML}{258F8F}
    \definecolor{ansi-white}{HTML}{C5C1B4}
    \definecolor{ansi-white-intense}{HTML}{A1A6B2}
    \definecolor{ansi-default-inverse-fg}{HTML}{FFFFFF}
    \definecolor{ansi-default-inverse-bg}{HTML}{000000}

    % commands and environments needed by pandoc snippets
    % extracted from the output of `pandoc -s`
    \providecommand{\tightlist}{%
      \setlength{\itemsep}{0pt}\setlength{\parskip}{0pt}}
    \DefineVerbatimEnvironment{Highlighting}{Verbatim}{commandchars=\\\{\}}
    % Add ',fontsize=\small' for more characters per line
    \newenvironment{Shaded}{}{}
    \newcommand{\KeywordTok}[1]{\textcolor[rgb]{0.00,0.44,0.13}{\textbf{{#1}}}}
    \newcommand{\DataTypeTok}[1]{\textcolor[rgb]{0.56,0.13,0.00}{{#1}}}
    \newcommand{\DecValTok}[1]{\textcolor[rgb]{0.25,0.63,0.44}{{#1}}}
    \newcommand{\BaseNTok}[1]{\textcolor[rgb]{0.25,0.63,0.44}{{#1}}}
    \newcommand{\FloatTok}[1]{\textcolor[rgb]{0.25,0.63,0.44}{{#1}}}
    \newcommand{\CharTok}[1]{\textcolor[rgb]{0.25,0.44,0.63}{{#1}}}
    \newcommand{\StringTok}[1]{\textcolor[rgb]{0.25,0.44,0.63}{{#1}}}
    \newcommand{\CommentTok}[1]{\textcolor[rgb]{0.38,0.63,0.69}{\textit{{#1}}}}
    \newcommand{\OtherTok}[1]{\textcolor[rgb]{0.00,0.44,0.13}{{#1}}}
    \newcommand{\AlertTok}[1]{\textcolor[rgb]{1.00,0.00,0.00}{\textbf{{#1}}}}
    \newcommand{\FunctionTok}[1]{\textcolor[rgb]{0.02,0.16,0.49}{{#1}}}
    \newcommand{\RegionMarkerTok}[1]{{#1}}
    \newcommand{\ErrorTok}[1]{\textcolor[rgb]{1.00,0.00,0.00}{\textbf{{#1}}}}
    \newcommand{\NormalTok}[1]{{#1}}
    
    % Additional commands for more recent versions of Pandoc
    \newcommand{\ConstantTok}[1]{\textcolor[rgb]{0.53,0.00,0.00}{{#1}}}
    \newcommand{\SpecialCharTok}[1]{\textcolor[rgb]{0.25,0.44,0.63}{{#1}}}
    \newcommand{\VerbatimStringTok}[1]{\textcolor[rgb]{0.25,0.44,0.63}{{#1}}}
    \newcommand{\SpecialStringTok}[1]{\textcolor[rgb]{0.73,0.40,0.53}{{#1}}}
    \newcommand{\ImportTok}[1]{{#1}}
    \newcommand{\DocumentationTok}[1]{\textcolor[rgb]{0.73,0.13,0.13}{\textit{{#1}}}}
    \newcommand{\AnnotationTok}[1]{\textcolor[rgb]{0.38,0.63,0.69}{\textbf{\textit{{#1}}}}}
    \newcommand{\CommentVarTok}[1]{\textcolor[rgb]{0.38,0.63,0.69}{\textbf{\textit{{#1}}}}}
    \newcommand{\VariableTok}[1]{\textcolor[rgb]{0.10,0.09,0.49}{{#1}}}
    \newcommand{\ControlFlowTok}[1]{\textcolor[rgb]{0.00,0.44,0.13}{\textbf{{#1}}}}
    \newcommand{\OperatorTok}[1]{\textcolor[rgb]{0.40,0.40,0.40}{{#1}}}
    \newcommand{\BuiltInTok}[1]{{#1}}
    \newcommand{\ExtensionTok}[1]{{#1}}
    \newcommand{\PreprocessorTok}[1]{\textcolor[rgb]{0.74,0.48,0.00}{{#1}}}
    \newcommand{\AttributeTok}[1]{\textcolor[rgb]{0.49,0.56,0.16}{{#1}}}
    \newcommand{\InformationTok}[1]{\textcolor[rgb]{0.38,0.63,0.69}{\textbf{\textit{{#1}}}}}
    \newcommand{\WarningTok}[1]{\textcolor[rgb]{0.38,0.63,0.69}{\textbf{\textit{{#1}}}}}
    
    
    % Define a nice break command that doesn't care if a line doesn't already
    % exist.
    \def\br{\hspace*{\fill} \\* }
    % Math Jax compatibility definitions
    \def\gt{>}
    \def\lt{<}
    \let\Oldtex\TeX
    \let\Oldlatex\LaTeX
    \renewcommand{\TeX}{\textrm{\Oldtex}}
    \renewcommand{\LaTeX}{\textrm{\Oldlatex}}
    % Document parameters
    % Document title
    \title{Pset\_5}
    
    
    
    
    
% Pygments definitions
\makeatletter
\def\PY@reset{\let\PY@it=\relax \let\PY@bf=\relax%
    \let\PY@ul=\relax \let\PY@tc=\relax%
    \let\PY@bc=\relax \let\PY@ff=\relax}
\def\PY@tok#1{\csname PY@tok@#1\endcsname}
\def\PY@toks#1+{\ifx\relax#1\empty\else%
    \PY@tok{#1}\expandafter\PY@toks\fi}
\def\PY@do#1{\PY@bc{\PY@tc{\PY@ul{%
    \PY@it{\PY@bf{\PY@ff{#1}}}}}}}
\def\PY#1#2{\PY@reset\PY@toks#1+\relax+\PY@do{#2}}

\expandafter\def\csname PY@tok@w\endcsname{\def\PY@tc##1{\textcolor[rgb]{0.73,0.73,0.73}{##1}}}
\expandafter\def\csname PY@tok@c\endcsname{\let\PY@it=\textit\def\PY@tc##1{\textcolor[rgb]{0.25,0.50,0.50}{##1}}}
\expandafter\def\csname PY@tok@cp\endcsname{\def\PY@tc##1{\textcolor[rgb]{0.74,0.48,0.00}{##1}}}
\expandafter\def\csname PY@tok@k\endcsname{\let\PY@bf=\textbf\def\PY@tc##1{\textcolor[rgb]{0.00,0.50,0.00}{##1}}}
\expandafter\def\csname PY@tok@kp\endcsname{\def\PY@tc##1{\textcolor[rgb]{0.00,0.50,0.00}{##1}}}
\expandafter\def\csname PY@tok@kt\endcsname{\def\PY@tc##1{\textcolor[rgb]{0.69,0.00,0.25}{##1}}}
\expandafter\def\csname PY@tok@o\endcsname{\def\PY@tc##1{\textcolor[rgb]{0.40,0.40,0.40}{##1}}}
\expandafter\def\csname PY@tok@ow\endcsname{\let\PY@bf=\textbf\def\PY@tc##1{\textcolor[rgb]{0.67,0.13,1.00}{##1}}}
\expandafter\def\csname PY@tok@nb\endcsname{\def\PY@tc##1{\textcolor[rgb]{0.00,0.50,0.00}{##1}}}
\expandafter\def\csname PY@tok@nf\endcsname{\def\PY@tc##1{\textcolor[rgb]{0.00,0.00,1.00}{##1}}}
\expandafter\def\csname PY@tok@nc\endcsname{\let\PY@bf=\textbf\def\PY@tc##1{\textcolor[rgb]{0.00,0.00,1.00}{##1}}}
\expandafter\def\csname PY@tok@nn\endcsname{\let\PY@bf=\textbf\def\PY@tc##1{\textcolor[rgb]{0.00,0.00,1.00}{##1}}}
\expandafter\def\csname PY@tok@ne\endcsname{\let\PY@bf=\textbf\def\PY@tc##1{\textcolor[rgb]{0.82,0.25,0.23}{##1}}}
\expandafter\def\csname PY@tok@nv\endcsname{\def\PY@tc##1{\textcolor[rgb]{0.10,0.09,0.49}{##1}}}
\expandafter\def\csname PY@tok@no\endcsname{\def\PY@tc##1{\textcolor[rgb]{0.53,0.00,0.00}{##1}}}
\expandafter\def\csname PY@tok@nl\endcsname{\def\PY@tc##1{\textcolor[rgb]{0.63,0.63,0.00}{##1}}}
\expandafter\def\csname PY@tok@ni\endcsname{\let\PY@bf=\textbf\def\PY@tc##1{\textcolor[rgb]{0.60,0.60,0.60}{##1}}}
\expandafter\def\csname PY@tok@na\endcsname{\def\PY@tc##1{\textcolor[rgb]{0.49,0.56,0.16}{##1}}}
\expandafter\def\csname PY@tok@nt\endcsname{\let\PY@bf=\textbf\def\PY@tc##1{\textcolor[rgb]{0.00,0.50,0.00}{##1}}}
\expandafter\def\csname PY@tok@nd\endcsname{\def\PY@tc##1{\textcolor[rgb]{0.67,0.13,1.00}{##1}}}
\expandafter\def\csname PY@tok@s\endcsname{\def\PY@tc##1{\textcolor[rgb]{0.73,0.13,0.13}{##1}}}
\expandafter\def\csname PY@tok@sd\endcsname{\let\PY@it=\textit\def\PY@tc##1{\textcolor[rgb]{0.73,0.13,0.13}{##1}}}
\expandafter\def\csname PY@tok@si\endcsname{\let\PY@bf=\textbf\def\PY@tc##1{\textcolor[rgb]{0.73,0.40,0.53}{##1}}}
\expandafter\def\csname PY@tok@se\endcsname{\let\PY@bf=\textbf\def\PY@tc##1{\textcolor[rgb]{0.73,0.40,0.13}{##1}}}
\expandafter\def\csname PY@tok@sr\endcsname{\def\PY@tc##1{\textcolor[rgb]{0.73,0.40,0.53}{##1}}}
\expandafter\def\csname PY@tok@ss\endcsname{\def\PY@tc##1{\textcolor[rgb]{0.10,0.09,0.49}{##1}}}
\expandafter\def\csname PY@tok@sx\endcsname{\def\PY@tc##1{\textcolor[rgb]{0.00,0.50,0.00}{##1}}}
\expandafter\def\csname PY@tok@m\endcsname{\def\PY@tc##1{\textcolor[rgb]{0.40,0.40,0.40}{##1}}}
\expandafter\def\csname PY@tok@gh\endcsname{\let\PY@bf=\textbf\def\PY@tc##1{\textcolor[rgb]{0.00,0.00,0.50}{##1}}}
\expandafter\def\csname PY@tok@gu\endcsname{\let\PY@bf=\textbf\def\PY@tc##1{\textcolor[rgb]{0.50,0.00,0.50}{##1}}}
\expandafter\def\csname PY@tok@gd\endcsname{\def\PY@tc##1{\textcolor[rgb]{0.63,0.00,0.00}{##1}}}
\expandafter\def\csname PY@tok@gi\endcsname{\def\PY@tc##1{\textcolor[rgb]{0.00,0.63,0.00}{##1}}}
\expandafter\def\csname PY@tok@gr\endcsname{\def\PY@tc##1{\textcolor[rgb]{1.00,0.00,0.00}{##1}}}
\expandafter\def\csname PY@tok@ge\endcsname{\let\PY@it=\textit}
\expandafter\def\csname PY@tok@gs\endcsname{\let\PY@bf=\textbf}
\expandafter\def\csname PY@tok@gp\endcsname{\let\PY@bf=\textbf\def\PY@tc##1{\textcolor[rgb]{0.00,0.00,0.50}{##1}}}
\expandafter\def\csname PY@tok@go\endcsname{\def\PY@tc##1{\textcolor[rgb]{0.53,0.53,0.53}{##1}}}
\expandafter\def\csname PY@tok@gt\endcsname{\def\PY@tc##1{\textcolor[rgb]{0.00,0.27,0.87}{##1}}}
\expandafter\def\csname PY@tok@err\endcsname{\def\PY@bc##1{\setlength{\fboxsep}{0pt}\fcolorbox[rgb]{1.00,0.00,0.00}{1,1,1}{\strut ##1}}}
\expandafter\def\csname PY@tok@kc\endcsname{\let\PY@bf=\textbf\def\PY@tc##1{\textcolor[rgb]{0.00,0.50,0.00}{##1}}}
\expandafter\def\csname PY@tok@kd\endcsname{\let\PY@bf=\textbf\def\PY@tc##1{\textcolor[rgb]{0.00,0.50,0.00}{##1}}}
\expandafter\def\csname PY@tok@kn\endcsname{\let\PY@bf=\textbf\def\PY@tc##1{\textcolor[rgb]{0.00,0.50,0.00}{##1}}}
\expandafter\def\csname PY@tok@kr\endcsname{\let\PY@bf=\textbf\def\PY@tc##1{\textcolor[rgb]{0.00,0.50,0.00}{##1}}}
\expandafter\def\csname PY@tok@bp\endcsname{\def\PY@tc##1{\textcolor[rgb]{0.00,0.50,0.00}{##1}}}
\expandafter\def\csname PY@tok@fm\endcsname{\def\PY@tc##1{\textcolor[rgb]{0.00,0.00,1.00}{##1}}}
\expandafter\def\csname PY@tok@vc\endcsname{\def\PY@tc##1{\textcolor[rgb]{0.10,0.09,0.49}{##1}}}
\expandafter\def\csname PY@tok@vg\endcsname{\def\PY@tc##1{\textcolor[rgb]{0.10,0.09,0.49}{##1}}}
\expandafter\def\csname PY@tok@vi\endcsname{\def\PY@tc##1{\textcolor[rgb]{0.10,0.09,0.49}{##1}}}
\expandafter\def\csname PY@tok@vm\endcsname{\def\PY@tc##1{\textcolor[rgb]{0.10,0.09,0.49}{##1}}}
\expandafter\def\csname PY@tok@sa\endcsname{\def\PY@tc##1{\textcolor[rgb]{0.73,0.13,0.13}{##1}}}
\expandafter\def\csname PY@tok@sb\endcsname{\def\PY@tc##1{\textcolor[rgb]{0.73,0.13,0.13}{##1}}}
\expandafter\def\csname PY@tok@sc\endcsname{\def\PY@tc##1{\textcolor[rgb]{0.73,0.13,0.13}{##1}}}
\expandafter\def\csname PY@tok@dl\endcsname{\def\PY@tc##1{\textcolor[rgb]{0.73,0.13,0.13}{##1}}}
\expandafter\def\csname PY@tok@s2\endcsname{\def\PY@tc##1{\textcolor[rgb]{0.73,0.13,0.13}{##1}}}
\expandafter\def\csname PY@tok@sh\endcsname{\def\PY@tc##1{\textcolor[rgb]{0.73,0.13,0.13}{##1}}}
\expandafter\def\csname PY@tok@s1\endcsname{\def\PY@tc##1{\textcolor[rgb]{0.73,0.13,0.13}{##1}}}
\expandafter\def\csname PY@tok@mb\endcsname{\def\PY@tc##1{\textcolor[rgb]{0.40,0.40,0.40}{##1}}}
\expandafter\def\csname PY@tok@mf\endcsname{\def\PY@tc##1{\textcolor[rgb]{0.40,0.40,0.40}{##1}}}
\expandafter\def\csname PY@tok@mh\endcsname{\def\PY@tc##1{\textcolor[rgb]{0.40,0.40,0.40}{##1}}}
\expandafter\def\csname PY@tok@mi\endcsname{\def\PY@tc##1{\textcolor[rgb]{0.40,0.40,0.40}{##1}}}
\expandafter\def\csname PY@tok@il\endcsname{\def\PY@tc##1{\textcolor[rgb]{0.40,0.40,0.40}{##1}}}
\expandafter\def\csname PY@tok@mo\endcsname{\def\PY@tc##1{\textcolor[rgb]{0.40,0.40,0.40}{##1}}}
\expandafter\def\csname PY@tok@ch\endcsname{\let\PY@it=\textit\def\PY@tc##1{\textcolor[rgb]{0.25,0.50,0.50}{##1}}}
\expandafter\def\csname PY@tok@cm\endcsname{\let\PY@it=\textit\def\PY@tc##1{\textcolor[rgb]{0.25,0.50,0.50}{##1}}}
\expandafter\def\csname PY@tok@cpf\endcsname{\let\PY@it=\textit\def\PY@tc##1{\textcolor[rgb]{0.25,0.50,0.50}{##1}}}
\expandafter\def\csname PY@tok@c1\endcsname{\let\PY@it=\textit\def\PY@tc##1{\textcolor[rgb]{0.25,0.50,0.50}{##1}}}
\expandafter\def\csname PY@tok@cs\endcsname{\let\PY@it=\textit\def\PY@tc##1{\textcolor[rgb]{0.25,0.50,0.50}{##1}}}

\def\PYZbs{\char`\\}
\def\PYZus{\char`\_}
\def\PYZob{\char`\{}
\def\PYZcb{\char`\}}
\def\PYZca{\char`\^}
\def\PYZam{\char`\&}
\def\PYZlt{\char`\<}
\def\PYZgt{\char`\>}
\def\PYZsh{\char`\#}
\def\PYZpc{\char`\%}
\def\PYZdl{\char`\$}
\def\PYZhy{\char`\-}
\def\PYZsq{\char`\'}
\def\PYZdq{\char`\"}
\def\PYZti{\char`\~}
% for compatibility with earlier versions
\def\PYZat{@}
\def\PYZlb{[}
\def\PYZrb{]}
\makeatother


    % For linebreaks inside Verbatim environment from package fancyvrb. 
    \makeatletter
        \newbox\Wrappedcontinuationbox 
        \newbox\Wrappedvisiblespacebox 
        \newcommand*\Wrappedvisiblespace {\textcolor{red}{\textvisiblespace}} 
        \newcommand*\Wrappedcontinuationsymbol {\textcolor{red}{\llap{\tiny$\m@th\hookrightarrow$}}} 
        \newcommand*\Wrappedcontinuationindent {3ex } 
        \newcommand*\Wrappedafterbreak {\kern\Wrappedcontinuationindent\copy\Wrappedcontinuationbox} 
        % Take advantage of the already applied Pygments mark-up to insert 
        % potential linebreaks for TeX processing. 
        %        {, <, #, %, $, ' and ": go to next line. 
        %        _, }, ^, &, >, - and ~: stay at end of broken line. 
        % Use of \textquotesingle for straight quote. 
        \newcommand*\Wrappedbreaksatspecials {% 
            \def\PYGZus{\discretionary{\char`\_}{\Wrappedafterbreak}{\char`\_}}% 
            \def\PYGZob{\discretionary{}{\Wrappedafterbreak\char`\{}{\char`\{}}% 
            \def\PYGZcb{\discretionary{\char`\}}{\Wrappedafterbreak}{\char`\}}}% 
            \def\PYGZca{\discretionary{\char`\^}{\Wrappedafterbreak}{\char`\^}}% 
            \def\PYGZam{\discretionary{\char`\&}{\Wrappedafterbreak}{\char`\&}}% 
            \def\PYGZlt{\discretionary{}{\Wrappedafterbreak\char`\<}{\char`\<}}% 
            \def\PYGZgt{\discretionary{\char`\>}{\Wrappedafterbreak}{\char`\>}}% 
            \def\PYGZsh{\discretionary{}{\Wrappedafterbreak\char`\#}{\char`\#}}% 
            \def\PYGZpc{\discretionary{}{\Wrappedafterbreak\char`\%}{\char`\%}}% 
            \def\PYGZdl{\discretionary{}{\Wrappedafterbreak\char`\$}{\char`\$}}% 
            \def\PYGZhy{\discretionary{\char`\-}{\Wrappedafterbreak}{\char`\-}}% 
            \def\PYGZsq{\discretionary{}{\Wrappedafterbreak\textquotesingle}{\textquotesingle}}% 
            \def\PYGZdq{\discretionary{}{\Wrappedafterbreak\char`\"}{\char`\"}}% 
            \def\PYGZti{\discretionary{\char`\~}{\Wrappedafterbreak}{\char`\~}}% 
        } 
        % Some characters . , ; ? ! / are not pygmentized. 
        % This macro makes them "active" and they will insert potential linebreaks 
        \newcommand*\Wrappedbreaksatpunct {% 
            \lccode`\~`\.\lowercase{\def~}{\discretionary{\hbox{\char`\.}}{\Wrappedafterbreak}{\hbox{\char`\.}}}% 
            \lccode`\~`\,\lowercase{\def~}{\discretionary{\hbox{\char`\,}}{\Wrappedafterbreak}{\hbox{\char`\,}}}% 
            \lccode`\~`\;\lowercase{\def~}{\discretionary{\hbox{\char`\;}}{\Wrappedafterbreak}{\hbox{\char`\;}}}% 
            \lccode`\~`\:\lowercase{\def~}{\discretionary{\hbox{\char`\:}}{\Wrappedafterbreak}{\hbox{\char`\:}}}% 
            \lccode`\~`\?\lowercase{\def~}{\discretionary{\hbox{\char`\?}}{\Wrappedafterbreak}{\hbox{\char`\?}}}% 
            \lccode`\~`\!\lowercase{\def~}{\discretionary{\hbox{\char`\!}}{\Wrappedafterbreak}{\hbox{\char`\!}}}% 
            \lccode`\~`\/\lowercase{\def~}{\discretionary{\hbox{\char`\/}}{\Wrappedafterbreak}{\hbox{\char`\/}}}% 
            \catcode`\.\active
            \catcode`\,\active 
            \catcode`\;\active
            \catcode`\:\active
            \catcode`\?\active
            \catcode`\!\active
            \catcode`\/\active 
            \lccode`\~`\~ 	
        }
    \makeatother

    \let\OriginalVerbatim=\Verbatim
    \makeatletter
    \renewcommand{\Verbatim}[1][1]{%
        %\parskip\z@skip
        \sbox\Wrappedcontinuationbox {\Wrappedcontinuationsymbol}%
        \sbox\Wrappedvisiblespacebox {\FV@SetupFont\Wrappedvisiblespace}%
        \def\FancyVerbFormatLine ##1{\hsize\linewidth
            \vtop{\raggedright\hyphenpenalty\z@\exhyphenpenalty\z@
                \doublehyphendemerits\z@\finalhyphendemerits\z@
                \strut ##1\strut}%
        }%
        % If the linebreak is at a space, the latter will be displayed as visible
        % space at end of first line, and a continuation symbol starts next line.
        % Stretch/shrink are however usually zero for typewriter font.
        \def\FV@Space {%
            \nobreak\hskip\z@ plus\fontdimen3\font minus\fontdimen4\font
            \discretionary{\copy\Wrappedvisiblespacebox}{\Wrappedafterbreak}
            {\kern\fontdimen2\font}%
        }%
        
        % Allow breaks at special characters using \PYG... macros.
        \Wrappedbreaksatspecials
        % Breaks at punctuation characters . , ; ? ! and / need catcode=\active 	
        \OriginalVerbatim[#1,codes*=\Wrappedbreaksatpunct]%
    }
    \makeatother

    % Exact colors from NB
    \definecolor{incolor}{HTML}{303F9F}
    \definecolor{outcolor}{HTML}{D84315}
    \definecolor{cellborder}{HTML}{CFCFCF}
    \definecolor{cellbackground}{HTML}{F7F7F7}
    
    % prompt
    \makeatletter
    \newcommand{\boxspacing}{\kern\kvtcb@left@rule\kern\kvtcb@boxsep}
    \makeatother
    \newcommand{\prompt}[4]{
        \ttfamily\llap{{\color{#2}[#3]:\hspace{3pt}#4}}\vspace{-\baselineskip}
    }
    

    
    % Prevent overflowing lines due to hard-to-break entities
    \sloppy 
    % Setup hyperref package
    \hypersetup{
      breaklinks=true,  % so long urls are correctly broken across lines
      colorlinks=true,
      urlcolor=urlcolor,
      linkcolor=linkcolor,
      citecolor=citecolor,
      }
    % Slightly bigger margins than the latex defaults
    
    \geometry{verbose,tmargin=1in,bmargin=1in,lmargin=1in,rmargin=1in}
    
\makeatletter
\renewcommand{\@seccntformat}[1]{}
\makeatother  

\begin{document}
    \title{CS 156a - Problem Set 5}
    \author{Samuel Patrone, 2140749}
    \maketitle
    

The following notebook is publicly available at the following
\href{https://github.com/spatrone/CS156A-Caltech.git}{link}.

\tableofcontents

    \hypertarget{problem-1}{%
\section{Problem 1}\label{problem-1}}

\hypertarget{answer-c-100}{%
\subsection{\texorpdfstring{Answer: {[}c{]}
\(100\)}{Answer: {[}c{]} 100}}\label{answer-c-100}}

\hypertarget{derivation}{%
\subsection{Derivation:}\label{derivation}}

The expected value on a data set \(\mathcal{D}\) of \(N\) samples of
in-sample error for a noisy target function with variance \(\sigma^2\)
using linear regression in \(d\) dimension is

\begin{equation}
\mathbb{E}_{\mathcal{D}}[E_{in}]=\sigma^2\left(1-\frac{d+1}{N}\right)\,.
\end{equation}

For \(\sigma=0.1\), \(d=8\) and
\(\mathbb{E}_{\mathcal{D}}[E_{in}]\ge 0.008\), we need at least

\begin{equation}
N\ge\frac{d+1}{\left(1-\frac{\mathbb{E}_{\mathcal{D}}[E_{in}]}{\sigma^2}\right)}=\frac{9}{\left(1-\frac{0.008}{(0.1)^2}\right)}=45\,.
\end{equation}

    \begin{tcolorbox}[breakable, size=fbox, boxrule=1pt, pad at break*=1mm,colback=cellbackground, colframe=cellborder]
\prompt{In}{incolor}{16}{\boxspacing}
\begin{Verbatim}[commandchars=\\\{\}]
\PY{k}{def} \PY{n+nf}{expEmin}\PY{p}{(}\PY{n}{N}\PY{p}{,}\PY{n}{sigma}\PY{o}{=}\PY{l+m+mf}{0.1}\PY{p}{,}\PY{n}{d}\PY{o}{=}\PY{l+m+mi}{8}\PY{p}{)}\PY{p}{:}
    \PY{k}{return} \PY{n}{sigma}\PY{o}{*}\PY{o}{*}\PY{l+m+mi}{2}\PY{o}{*}\PY{p}{(}\PY{l+m+mi}{1}\PY{o}{\PYZhy{}}\PY{p}{(}\PY{n}{d}\PY{o}{+}\PY{l+m+mi}{1}\PY{p}{)}\PY{o}{/}\PY{n}{N}\PY{p}{)}

\PY{n+nb}{print}\PY{p}{(}\PY{n}{f}\PY{l+s+s1}{\PYZsq{}}\PY{l+s+s1}{For N=45, we get an expected value for E\PYZus{}min of }\PY{l+s+s1}{\PYZob{}}\PY{l+s+s1}{expEmin(45):.3f\PYZcb{}}\PY{l+s+s1}{\PYZsq{}}\PY{p}{)}
\end{Verbatim}
\end{tcolorbox}

    \begin{Verbatim}[commandchars=\\\{\}]
For N=45, we get an expected value for E\_min of 0.008
    \end{Verbatim}

    \hypertarget{problem-2}{%
\section{Problem 2}\label{problem-2}}

\hypertarget{answer-d-_10-_20}{%
\subsection{\texorpdfstring{Answer: {[}d{]} $
\tilde{\omega}_1\textless{}0,; \tilde{\omega}_2\textgreater{}0
$}{Answer: {[}d{]} \$ \_1\textless{}0,; \_2\textgreater{}0 \$}}\label{answer-d-_10-_20}}

\hypertarget{derivation}{%
\subsection{Derivation:}\label{derivation}}

A hyperbola is the set of points in a plane whose distances from two
fixed points, called foci, has an absolute difference that is equal to a
positive constant. In formulae:

\begin{equation}
f(x_1,x_2)=x_1^2-x_2^2-r^2=0\,,
\end{equation}

where we assumed that the center is the origin of the coordinates
\((0,0)\), the hyperbola is equilateral, i.e.~the asymptotes have
unitary slopes, and \(r\in \mathbb{R}\). The general case can be
addressed by shifting the origin and/or rescaling the coordinates.

In the \(\mathcal{Z}\) space, the point are classified by the sign of
the following function:

\begin{equation}
\label{Zbounds}
{\rm sgn}\left(\tilde{\omega}_0\Phi(x_0)+\tilde{\omega}_1 \Phi(x_1)+\tilde{\omega}_2 \Phi(x_2)\right)={\rm sgn}\left(\tilde{\omega}_0+\tilde{\omega}_1 x_1^2+\tilde{\omega}_2 x_2^2\right)\,.
\end{equation}

In \(\mathcal{X}\) space, a generic sample \((x_1,x_2)\) is labelled by
computing \({\rm sgn}{(-f(x_1,x_2))}\), as illustrated in the following
plot. In order to agree with the decision boundary in
Eq.\eqref{Zbounds}, i.e.

\begin{equation}
{\rm sgn}\left(\tilde{\omega}_0+\tilde{\omega}_1 x_1^2+\tilde{\omega}_2 x_2^2)\right)={\rm sgn}(r^2-x_1^2+x_2^2)
\end{equation}

we impose the following sets of constraints on the weights
\(\tilde{\omega}\):

\begin{equation}
\tilde{\omega}_0>0,\; \tilde{\omega}_1<0,\; \tilde{\omega}_2>0\,.
\end{equation}

    \begin{tcolorbox}[breakable, size=fbox, boxrule=1pt, pad at break*=1mm,colback=cellbackground, colframe=cellborder]
\prompt{In}{incolor}{26}{\boxspacing}
\begin{Verbatim}[commandchars=\\\{\}]
\PY{k+kn}{import} \PY{n+nn}{numpy} \PY{k}{as} \PY{n+nn}{np}
\PY{k+kn}{import} \PY{n+nn}{matplotlib}\PY{n+nn}{.}\PY{n+nn}{pyplot} \PY{k}{as} \PY{n+nn}{plt}

\PY{k}{def} \PY{n+nf}{gen\PYZus{}uniform\PYZus{}points}\PY{p}{(}\PY{n}{N}\PY{p}{,}\PY{n}{d}\PY{o}{=}\PY{l+m+mi}{2}\PY{p}{,}\PY{n}{vmin}\PY{o}{=}\PY{p}{[}\PY{o}{\PYZhy{}}\PY{l+m+mi}{1}\PY{p}{,}\PY{o}{\PYZhy{}}\PY{l+m+mi}{1}\PY{p}{]}\PY{p}{,}\PY{n}{vmax}\PY{o}{=}\PY{p}{[}\PY{l+m+mi}{1}\PY{p}{,}\PY{l+m+mi}{1}\PY{p}{]}\PY{p}{)}\PY{p}{:}
    \PY{k}{if}\PY{p}{(}\PY{n}{d}\PY{o}{!=}\PY{n+nb}{len}\PY{p}{(}\PY{n}{vmin}\PY{p}{)}\PY{o}{|}\PY{n}{d}\PY{o}{!=}\PY{n+nb}{len}\PY{p}{(}\PY{n}{vmax}\PY{p}{)}\PY{p}{)}\PY{p}{:} 
        \PY{k}{raise} \PY{n+ne}{Exception}\PY{p}{(}\PY{l+s+s1}{\PYZsq{}}\PY{l+s+s1}{WARNING: Boundary values do not match the dimensionality of the problem!}\PY{l+s+s1}{\PYZsq{}}\PY{p}{)}
    \PY{k}{return} \PY{n}{np}\PY{o}{.}\PY{n}{random}\PY{o}{.}\PY{n}{uniform}\PY{p}{(}\PY{n}{low}\PY{o}{=}\PY{n}{vmin}\PY{p}{,}\PY{n}{high}\PY{o}{=}\PY{n}{vmax}\PY{p}{,}\PY{n}{size}\PY{o}{=}\PY{p}{(}\PY{n}{N}\PY{p}{,}\PY{n}{d}\PY{p}{)}\PY{p}{)}

\PY{k}{def} \PY{n+nf}{label\PYZus{}hyperbolic}\PY{p}{(}\PY{n}{pts}\PY{p}{,} \PY{n}{a}\PY{o}{=}\PY{l+m+mi}{1}\PY{p}{,}\PY{n}{b}\PY{o}{=}\PY{l+m+mi}{1}\PY{p}{,}\PY{n}{r}\PY{o}{=}\PY{l+m+mi}{1}\PY{p}{)}\PY{p}{:}
    \PY{k}{return} \PY{p}{[}\PY{o}{\PYZhy{}}\PY{n}{np}\PY{o}{.}\PY{n}{sign}\PY{p}{(}\PY{n}{pts}\PY{p}{[}\PY{n}{i}\PY{p}{]}\PY{p}{[}\PY{l+m+mi}{0}\PY{p}{]}\PY{o}{*}\PY{n}{pts}\PY{p}{[}\PY{n}{i}\PY{p}{]}\PY{p}{[}\PY{l+m+mi}{0}\PY{p}{]}\PY{o}{/}\PY{n}{a}\PY{o}{*}\PY{o}{*}\PY{l+m+mi}{2}\PY{o}{\PYZhy{}}\PY{n}{pts}\PY{p}{[}\PY{n}{i}\PY{p}{]}\PY{p}{[}\PY{l+m+mi}{1}\PY{p}{]}\PY{o}{*}\PY{n}{pts}\PY{p}{[}\PY{n}{i}\PY{p}{]}\PY{p}{[}\PY{l+m+mi}{1}\PY{p}{]}\PY{o}{/}\PY{n}{b}\PY{o}{*}\PY{o}{*}\PY{l+m+mi}{2}\PY{o}{\PYZhy{}}\PY{n}{r}\PY{o}{*}\PY{o}{*}\PY{l+m+mi}{2}\PY{p}{)} \PY{k}{for} \PY{n}{i} \PY{o+ow}{in} \PY{n+nb}{range}\PY{p}{(}\PY{n+nb}{len}\PY{p}{(}\PY{n}{pts}\PY{p}{)}\PY{p}{)}\PY{p}{]}

\PY{k}{def} \PY{n+nf}{color\PYZus{}pts}\PY{p}{(}\PY{n}{label}\PY{p}{)}\PY{p}{:}
    \PY{c+c1}{\PYZsh{}green is +1, red is \PYZhy{}1}
    \PY{n}{col}\PY{o}{=}\PY{p}{[}\PY{p}{]}
    \PY{k}{for} \PY{n}{i} \PY{o+ow}{in} \PY{n+nb}{range}\PY{p}{(}\PY{n+nb}{len}\PY{p}{(}\PY{n}{label}\PY{p}{)}\PY{p}{)}\PY{p}{:}
        \PY{k}{if}\PY{p}{(}\PY{n}{label}\PY{p}{[}\PY{n}{i}\PY{p}{]}\PY{o}{\PYZgt{}}\PY{l+m+mi}{0}\PY{p}{)}\PY{p}{:} \PY{n}{col}\PY{o}{.}\PY{n}{append}\PY{p}{(}\PY{l+s+s1}{\PYZsq{}}\PY{l+s+s1}{green}\PY{l+s+s1}{\PYZsq{}}\PY{p}{)}
        \PY{k}{else}\PY{p}{:} \PY{n}{col}\PY{o}{.}\PY{n}{append}\PY{p}{(}\PY{l+s+s1}{\PYZsq{}}\PY{l+s+s1}{red}\PY{l+s+s1}{\PYZsq{}}\PY{p}{)}
    \PY{k}{return} \PY{n}{col}

\PY{n}{N}\PY{o}{=}\PY{l+m+mi}{1000}
\PY{n}{pts}\PY{o}{=}\PY{n}{gen\PYZus{}uniform\PYZus{}points}\PY{p}{(}\PY{n}{N}\PY{p}{,}\PY{n}{vmin}\PY{o}{=}\PY{p}{[}\PY{o}{\PYZhy{}}\PY{l+m+mi}{10}\PY{p}{,}\PY{o}{\PYZhy{}}\PY{l+m+mi}{10}\PY{p}{]}\PY{p}{,}\PY{n}{vmax}\PY{o}{=}\PY{p}{[}\PY{l+m+mi}{10}\PY{p}{,}\PY{l+m+mi}{10}\PY{p}{]}\PY{p}{)}
\PY{n}{label}\PY{o}{=}\PY{n}{label\PYZus{}hyperbolic}\PY{p}{(}\PY{n}{pts}\PY{p}{)}
\PY{n}{plt}\PY{o}{.}\PY{n}{scatter}\PY{p}{(}\PY{n}{pts}\PY{p}{[}\PY{p}{:}\PY{p}{,}\PY{l+m+mi}{0}\PY{p}{]}\PY{p}{,}\PY{n}{pts}\PY{p}{[}\PY{p}{:}\PY{p}{,}\PY{l+m+mi}{1}\PY{p}{]}\PY{p}{,} \PY{n}{color}\PY{o}{=}\PY{n}{color\PYZus{}pts}\PY{p}{(}\PY{n}{label}\PY{p}{)}\PY{p}{)}
\PY{n}{plt}\PY{o}{.}\PY{n}{show}\PY{p}{(}\PY{p}{)}
\end{Verbatim}
\end{tcolorbox}

    \begin{center}
    \adjustimage{max size={0.5\linewidth}{0.5\paperheight}}{output_4_0.png}
    \end{center}
    { \hspace*{\fill} \\}
    
    \hypertarget{problem-3}{%
\section{Problem 3}\label{problem-3}}

\hypertarget{answer-c-15}{%
\subsection{\texorpdfstring{Answer: {[}c{]}
\(15\)}{Answer: {[}c{]} 15}}\label{answer-c-15}}

\hypertarget{derivation}{%
\subsection{Derivation:}\label{derivation}}

Let's consider the general \(\mathcal{Q}\)th order polynomial transform
\(\Phi_\mathcal{Q}\) for the space \(\mathcal{x}=\mathbb{R}^d\). We can
find the dimensionality \(\tilde{d}\) of the feature space
\(\mathcal{Z}\) by observing that we can form \(C(d,k)\) different
monomials of order \(k\) from the \(d\) initial coordinates, where

\begin{equation}
C(d,k)={d+k-1 \choose k}\,.
\end{equation}

Since \(\Phi_\mathcal{Q}\) will have all possible monomials up to order
\(\mathcal{Q}\) as transformed coordinates, the feature space
\(\mathcal{Z}\) will have a dimensionality

\begin{equation}
\tilde{d}(Q,d)=\sum^Q_{k=1}{d+k-1 \choose k}\,.
\end{equation}

For \(d=2\), we get

\begin{equation}
\tilde{d}(Q,2)=\sum^Q_{k=1}{k+1 \choose k}= \sum^Q_{k=1} k+1 = \frac{Q(Q+3)}{2} \,.
\end{equation}

The VC dimension of the set of the hypothesis in \(\mathcal{Z}\)
\(d_{VC}(\mathcal{H}_{\Phi})\) can be as high as the VC dimension of a
linear model in the transformed space, in formulae:

\begin{equation}
d_{VC}(\mathcal{H}_{\Phi})\le\tilde{d}+1\,.
\end{equation}

For the case examined here, \(Q=4\), hence
\(d_{VC}(\mathcal{H}_{\Phi})\le \tilde{d}(4,2)+1=15\).

    \hypertarget{problem-4}{%
\section{Problem 4}\label{problem-4}}

\hypertarget{answer-e-2uev-2ve-uev2ve-u}{%
\subsection{\texorpdfstring{Answer: {[}e{]}
\(2(ue^v-2ve^{-u})(e^v+2ve^{-u})\)}{Answer: {[}e{]} 2(ue\^{}v-2ve\^{}\{-u\})(e\^{}v+2ve\^{}\{-u\})}}\label{answer-e-2uev-2ve-uev2ve-u}}

\hypertarget{derivation}{%
\subsection{Derivation:}\label{derivation}}

\begin{align}
\begin{split}
\frac{\partial E(u,v)}{\partial u}=&2(ue^v-2ve^{-u})\frac{\partial}{\partial u}(ue^v-2ve^{-u})\\ =&2(ue^v-2ve^{-u})(e^v+2ve^{-u})\,.
\end{split}
\end{align}

    \hypertarget{problems-5-6}{%
\section{Problems 5-6}\label{problems-5-6}}

\hypertarget{answers-d-10-e-0.0450.024}{%
\subsection{\texorpdfstring{Answers: {[}d{]} \(10\) , {[}e{]}
\([0.045,0.024]\)}{Answers: {[}d{]} 10 , {[}e{]} {[}0.045,0.024{]}}}\label{answers-d-10-e-0.0450.024}}

\hypertarget{code}{%
\subsection{Code:}\label{code}}

    \begin{tcolorbox}[breakable, size=fbox, boxrule=1pt, pad at break*=1mm,colback=cellbackground, colframe=cellborder]
\prompt{In}{incolor}{65}{\boxspacing}
\begin{Verbatim}[commandchars=\\\{\}]
\PY{k+kn}{import} \PY{n+nn}{math} \PY{k}{as} \PY{n+nn}{m}

\PY{k}{def} \PY{n+nf}{E}\PY{p}{(}\PY{n}{w}\PY{p}{)}\PY{p}{:}
    \PY{n}{u}\PY{o}{=}\PY{n}{w}\PY{p}{[}\PY{l+m+mi}{0}\PY{p}{]}
    \PY{n}{v}\PY{o}{=}\PY{n}{w}\PY{p}{[}\PY{l+m+mi}{1}\PY{p}{]}
    \PY{k}{return} \PY{n}{np}\PY{o}{.}\PY{n}{double}\PY{p}{(}\PY{p}{(}\PY{n}{u}\PY{o}{*}\PY{n}{m}\PY{o}{.}\PY{n}{e}\PY{o}{*}\PY{o}{*}\PY{n}{v} \PY{o}{\PYZhy{}} \PY{l+m+mi}{2}\PY{o}{*}\PY{n}{v}\PY{o}{*}\PY{n}{m}\PY{o}{.}\PY{n}{e}\PY{o}{*}\PY{o}{*}\PY{p}{(}\PY{o}{\PYZhy{}}\PY{n}{u}\PY{p}{)}\PY{p}{)}\PY{o}{*}\PY{o}{*}\PY{l+m+mi}{2}\PY{p}{)}

\PY{k}{def} \PY{n+nf}{gradE}\PY{p}{(}\PY{n}{w}\PY{p}{)}\PY{p}{:}
    \PY{n}{u}\PY{o}{=}\PY{n}{w}\PY{p}{[}\PY{l+m+mi}{0}\PY{p}{]}
    \PY{n}{v}\PY{o}{=}\PY{n}{w}\PY{p}{[}\PY{l+m+mi}{1}\PY{p}{]}
    \PY{n}{duE}\PY{o}{=}\PY{n}{np}\PY{o}{.}\PY{n}{double}\PY{p}{(}\PY{l+m+mi}{2}\PY{o}{*}\PY{p}{(}\PY{n}{u}\PY{o}{*}\PY{n}{m}\PY{o}{.}\PY{n}{e}\PY{o}{*}\PY{o}{*}\PY{n}{v}\PY{o}{\PYZhy{}}\PY{l+m+mi}{2}\PY{o}{*}\PY{n}{v}\PY{o}{*}\PY{n}{m}\PY{o}{.}\PY{n}{e}\PY{o}{*}\PY{o}{*}\PY{p}{(}\PY{o}{\PYZhy{}}\PY{n}{u}\PY{p}{)}\PY{p}{)}\PY{o}{*}\PY{p}{(}\PY{n}{m}\PY{o}{.}\PY{n}{e}\PY{o}{*}\PY{o}{*}\PY{n}{v}\PY{o}{+}\PY{l+m+mi}{2}\PY{o}{*}\PY{n}{v}\PY{o}{*}\PY{n}{m}\PY{o}{.}\PY{n}{e}\PY{o}{*}\PY{o}{*}\PY{p}{(}\PY{o}{\PYZhy{}}\PY{n}{u}\PY{p}{)}\PY{p}{)}\PY{p}{)}
    \PY{n}{dvE}\PY{o}{=}\PY{n}{np}\PY{o}{.}\PY{n}{double}\PY{p}{(}\PY{l+m+mi}{2}\PY{o}{*}\PY{p}{(}\PY{o}{\PYZhy{}}\PY{l+m+mi}{2}\PY{o}{*}\PY{n}{m}\PY{o}{.}\PY{n}{e}\PY{o}{*}\PY{o}{*}\PY{p}{(}\PY{o}{\PYZhy{}}\PY{n}{u}\PY{p}{)}\PY{o}{+}\PY{n}{m}\PY{o}{.}\PY{n}{e}\PY{o}{*}\PY{o}{*}\PY{n}{v}\PY{o}{*}\PY{n}{u}\PY{p}{)}\PY{o}{*}\PY{p}{(}\PY{n}{m}\PY{o}{.}\PY{n}{e}\PY{o}{*}\PY{o}{*}\PY{n}{v}\PY{o}{*}\PY{n}{u} \PY{o}{\PYZhy{}} \PY{l+m+mi}{2}\PY{o}{*}\PY{n}{m}\PY{o}{.}\PY{n}{e}\PY{o}{*}\PY{o}{*}\PY{p}{(}\PY{o}{\PYZhy{}}\PY{n}{u}\PY{p}{)}\PY{o}{*}\PY{n}{v}\PY{p}{)}\PY{p}{)}
    \PY{k}{return} \PY{n}{np}\PY{o}{.}\PY{n}{array}\PY{p}{(}\PY{p}{[}\PY{n}{duE}\PY{p}{,}\PY{n}{dvE}\PY{p}{]}\PY{p}{)}

\PY{k}{def} \PY{n+nf}{grad\PYZus{}step}\PY{p}{(}\PY{n}{w}\PY{p}{,}\PY{n}{grad}\PY{p}{,}\PY{n}{eta}\PY{p}{)}\PY{p}{:}
    \PY{k}{return} \PY{n}{w}\PY{o}{\PYZhy{}}\PY{n}{eta}\PY{o}{*}\PY{n}{grad}\PY{p}{(}\PY{n}{w}\PY{p}{)}

\PY{k}{def} \PY{n+nf}{grad\PYZus{}desc}\PY{p}{(}\PY{n}{E}\PY{p}{,}\PY{n}{gradE}\PY{p}{,}\PY{n}{Emin}\PY{p}{,}\PY{n}{init}\PY{p}{,}\PY{n}{eta}\PY{o}{=}\PY{l+m+mf}{0.1}\PY{p}{,}\PY{n}{print\PYZus{}ans}\PY{o}{=}\PY{k+kc}{True}\PY{p}{)}\PY{p}{:}
    \PY{n}{w}\PY{o}{=}\PY{n}{init}
    \PY{n}{Niter}\PY{o}{=}\PY{l+m+mi}{0}
    \PY{k}{while}\PY{p}{(}\PY{n}{E}\PY{p}{(}\PY{n}{w}\PY{p}{)}\PY{o}{\PYZgt{}}\PY{n}{Emin}\PY{p}{)}\PY{p}{:}
        \PY{n}{w}\PY{o}{=}\PY{n}{grad\PYZus{}step}\PY{p}{(}\PY{n}{w}\PY{p}{,}\PY{n}{gradE}\PY{p}{,}\PY{n}{eta}\PY{p}{)}
        \PY{n}{Niter}\PY{o}{=}\PY{n}{Niter}\PY{o}{+}\PY{l+m+mi}{1}
    \PY{k}{if}\PY{p}{(}\PY{n}{print\PYZus{}ans}\PY{o}{==}\PY{k+kc}{True}\PY{p}{)}\PY{p}{:}
        \PY{n+nb}{print}\PY{p}{(}\PY{n}{f}\PY{l+s+s1}{\PYZsq{}}\PY{l+s+s1}{After }\PY{l+s+si}{\PYZob{}Niter\PYZcb{}}\PY{l+s+s1}{ iterations, we found a minimum of the function at [}\PY{l+s+si}{\PYZob{}w[0]:.3f\PYZcb{}}\PY{l+s+s1}{,}\PY{l+s+si}{\PYZob{}w[1]:.3f\PYZcb{}}\PY{l+s+s1}{] with error value of }\PY{l+s+s1}{\PYZob{}}\PY{l+s+s1}{E(w):.2e\PYZcb{}}\PY{l+s+s1}{\PYZsq{}}\PY{p}{)}
    \PY{k}{return} \PY{n}{w}\PY{p}{,}\PY{n}{Niter}

\PY{n}{eps}\PY{o}{=}\PY{n}{np}\PY{o}{.}\PY{n}{double}\PY{p}{(}\PY{l+m+mi}{10}\PY{o}{*}\PY{o}{*}\PY{p}{(}\PY{o}{\PYZhy{}}\PY{l+m+mi}{14}\PY{p}{)}\PY{p}{)}
\PY{n}{startpt}\PY{o}{=}\PY{n}{np}\PY{o}{.}\PY{n}{array}\PY{p}{(}\PY{p}{[}\PY{l+m+mi}{1}\PY{p}{,}\PY{l+m+mi}{1}\PY{p}{]}\PY{p}{)}
\PY{n}{w}\PY{p}{,}\PY{n}{Niter}\PY{o}{=}\PY{n}{grad\PYZus{}desc}\PY{p}{(}\PY{n}{E}\PY{p}{,}\PY{n}{gradE}\PY{p}{,}\PY{n}{eps}\PY{p}{,}\PY{n}{startpt}\PY{p}{)}
\end{Verbatim}
\end{tcolorbox}

    \begin{Verbatim}[commandchars=\\\{\}]
After 10 iterations, we found a minimum of the function at [0.045,0.024] with
error value of 1.21e-15
    \end{Verbatim}

    \hypertarget{problem-7}{%
\section{Problem 7}\label{problem-7}}

\hypertarget{answer-a-10-1}{%
\subsection{\texorpdfstring{Answer: {[}a{]}
\(10^{-1}\)}{Answer: {[}a{]} 10\^{}\{-1\}}}\label{answer-a-10-1}}

\hypertarget{code}{%
\subsection{Code:}\label{code}}

    \begin{tcolorbox}[breakable, size=fbox, boxrule=1pt, pad at break*=1mm,colback=cellbackground, colframe=cellborder]
\prompt{In}{incolor}{74}{\boxspacing}
\begin{Verbatim}[commandchars=\\\{\}]
\PY{k}{def} \PY{n+nf}{grad\PYZus{}step\PYZus{}coord}\PY{p}{(}\PY{n}{w}\PY{p}{,}\PY{n}{gradE}\PY{p}{,}\PY{n}{eta}\PY{p}{)}\PY{p}{:}
    \PY{n}{ustep}\PY{o}{=}\PY{n}{w}\PY{p}{[}\PY{l+m+mi}{0}\PY{p}{]}\PY{o}{\PYZhy{}}\PY{n}{eta}\PY{o}{*}\PY{n}{gradE}\PY{p}{(}\PY{n}{w}\PY{p}{)}\PY{p}{[}\PY{l+m+mi}{0}\PY{p}{]}
    \PY{n}{w}\PY{o}{=}\PY{n}{np}\PY{o}{.}\PY{n}{array}\PY{p}{(}\PY{p}{[}\PY{n}{ustep}\PY{p}{,}\PY{n}{w}\PY{p}{[}\PY{l+m+mi}{1}\PY{p}{]}\PY{p}{]}\PY{p}{,}\PY{n}{dtype}\PY{o}{=}\PY{n}{np}\PY{o}{.}\PY{n}{double}\PY{p}{)}
    \PY{n}{vstep}\PY{o}{=}\PY{n}{w}\PY{p}{[}\PY{l+m+mi}{1}\PY{p}{]}\PY{o}{\PYZhy{}}\PY{n}{eta}\PY{o}{*}\PY{n}{gradE}\PY{p}{(}\PY{n}{w}\PY{p}{)}\PY{p}{[}\PY{l+m+mi}{1}\PY{p}{]}
    \PY{k}{return} \PY{n}{np}\PY{o}{.}\PY{n}{array}\PY{p}{(}\PY{p}{[}\PY{n}{ustep}\PY{p}{,}\PY{n}{vstep}\PY{p}{]}\PY{p}{,}\PY{n}{dtype}\PY{o}{=}\PY{n}{np}\PY{o}{.}\PY{n}{double}\PY{p}{)}

\PY{k}{def} \PY{n+nf}{grad\PYZus{}desc\PYZus{}coord}\PY{p}{(}\PY{n}{E}\PY{p}{,}\PY{n}{gradE}\PY{p}{,}\PY{n}{max\PYZus{}ite}\PY{p}{,}\PY{n}{init}\PY{p}{,}\PY{n}{eta}\PY{o}{=}\PY{l+m+mf}{0.1}\PY{p}{,}\PY{n}{print\PYZus{}ans}\PY{o}{=}\PY{k+kc}{True}\PY{p}{)}\PY{p}{:}
    \PY{n}{w}\PY{o}{=}\PY{n}{init}
    \PY{n}{Niter}\PY{o}{=}\PY{l+m+mi}{0}
    \PY{k}{while}\PY{p}{(}\PY{n}{Niter}\PY{o}{\PYZlt{}}\PY{n}{max\PYZus{}ite}\PY{p}{)}\PY{p}{:}
        \PY{n}{w}\PY{o}{=}\PY{n}{grad\PYZus{}step\PYZus{}coord}\PY{p}{(}\PY{n}{w}\PY{p}{,}\PY{n}{gradE}\PY{p}{,}\PY{n}{eta}\PY{p}{)}
        \PY{n}{Niter}\PY{o}{=}\PY{n}{Niter}\PY{o}{+}\PY{l+m+mi}{1}
    \PY{k}{if}\PY{p}{(}\PY{n}{print\PYZus{}ans}\PY{o}{==}\PY{k+kc}{True}\PY{p}{)}\PY{p}{:}
        \PY{n+nb}{print}\PY{p}{(}\PY{n}{f}\PY{l+s+s1}{\PYZsq{}}\PY{l+s+s1}{After }\PY{l+s+si}{\PYZob{}Niter\PYZcb{}}\PY{l+s+s1}{ iterations, we found a minimum of the function at [}\PY{l+s+si}{\PYZob{}w[0]:.3f\PYZcb{}}\PY{l+s+s1}{,}\PY{l+s+si}{\PYZob{}w[1]:.3f\PYZcb{}}\PY{l+s+s1}{] with error value of }\PY{l+s+s1}{\PYZob{}}\PY{l+s+s1}{E(w):.2e\PYZcb{}}\PY{l+s+s1}{\PYZsq{}}\PY{p}{)}
    \PY{k}{return} \PY{n}{w}\PY{p}{,}\PY{n}{Niter}

\PY{n}{w}\PY{p}{,}\PY{n}{Niter}\PY{o}{=}\PY{n}{grad\PYZus{}desc\PYZus{}coord}\PY{p}{(}\PY{n}{E}\PY{p}{,}\PY{n}{gradE}\PY{p}{,}\PY{l+m+mi}{15}\PY{p}{,}\PY{n}{startpt}\PY{p}{)}
\end{Verbatim}
\end{tcolorbox}

    \begin{Verbatim}[commandchars=\\\{\}]
After 15 iterations, we found a minimum of the function at [6.297,-2.852] with
error value of 1.40e-01
    \end{Verbatim}

    \hypertarget{problems-8-9}{%
\section{Problems 8-9}\label{problems-8-9}}

\hypertarget{answers-d-0.100-a-350}{%
\subsection{\texorpdfstring{Answers: {[}d{]} \(0.100\), {[}a{]}
\(350\)}{Answers: {[}d{]} 0.100, {[}a{]} 350}}\label{answers-d-0.100-a-350}}

\hypertarget{code}{%
\subsection{Code:}\label{code}}

    \begin{tcolorbox}[breakable, size=fbox, boxrule=1pt, pad at break*=1mm,colback=cellbackground, colframe=cellborder]
\prompt{In}{incolor}{59}{\boxspacing}
\begin{Verbatim}[commandchars=\\\{\}]
\PY{k+kn}{import} \PY{n+nn}{numpy} \PY{k}{as} \PY{n+nn}{np}
\PY{k+kn}{import} \PY{n+nn}{matplotlib}\PY{n+nn}{.}\PY{n+nn}{pyplot} \PY{k}{as} \PY{n+nn}{plt}
\PY{k+kn}{import} \PY{n+nn}{math} \PY{k}{as} \PY{n+nn}{m}

\PY{k}{def} \PY{n+nf}{gen\PYZus{}uniform\PYZus{}points}\PY{p}{(}\PY{n}{N}\PY{p}{,}\PY{n}{d}\PY{o}{=}\PY{l+m+mi}{2}\PY{p}{,}\PY{n}{vmin}\PY{o}{=}\PY{p}{[}\PY{o}{\PYZhy{}}\PY{l+m+mi}{1}\PY{p}{,}\PY{o}{\PYZhy{}}\PY{l+m+mi}{1}\PY{p}{]}\PY{p}{,}\PY{n}{vmax}\PY{o}{=}\PY{p}{[}\PY{l+m+mi}{1}\PY{p}{,}\PY{l+m+mi}{1}\PY{p}{]}\PY{p}{)}\PY{p}{:}
    \PY{k}{if}\PY{p}{(}\PY{n}{d}\PY{o}{!=}\PY{n+nb}{len}\PY{p}{(}\PY{n}{vmin}\PY{p}{)}\PY{o}{|}\PY{n}{d}\PY{o}{!=}\PY{n+nb}{len}\PY{p}{(}\PY{n}{vmax}\PY{p}{)}\PY{p}{)}\PY{p}{:} 
        \PY{k}{raise} \PY{n+ne}{Exception}\PY{p}{(}\PY{l+s+s1}{\PYZsq{}}\PY{l+s+s1}{WARNING: Boundary values do not match the dimensionality of the problem!}\PY{l+s+s1}{\PYZsq{}}\PY{p}{)}
    \PY{n}{pts}\PY{o}{=}\PY{n}{np}\PY{o}{.}\PY{n}{random}\PY{o}{.}\PY{n}{uniform}\PY{p}{(}\PY{n}{low}\PY{o}{=}\PY{n}{vmin}\PY{p}{,}\PY{n}{high}\PY{o}{=}\PY{n}{vmax}\PY{p}{,}\PY{n}{size}\PY{o}{=}\PY{p}{(}\PY{n}{N}\PY{p}{,}\PY{n}{d}\PY{p}{)}\PY{p}{)}
    \PY{k}{return}  \PY{n}{np}\PY{o}{.}\PY{n}{concatenate}\PY{p}{(}\PY{p}{(}\PY{n}{np}\PY{o}{.}\PY{n}{ones}\PY{p}{(}\PY{n}{N}\PY{p}{)}\PY{p}{[}\PY{p}{:}\PY{p}{,} \PY{n}{np}\PY{o}{.}\PY{n}{newaxis}\PY{p}{]}\PY{p}{,} \PY{n}{pts}\PY{p}{)}\PY{p}{,} \PY{n}{axis}\PY{o}{=}\PY{l+m+mi}{1}\PY{p}{)}

\PY{k}{def} \PY{n+nf}{gen\PYZus{}line}\PY{p}{(}\PY{p}{)}\PY{p}{:} 
    \PY{n}{pts}\PY{o}{=}\PY{n}{gen\PYZus{}uniform\PYZus{}points}\PY{p}{(}\PY{l+m+mi}{2}\PY{p}{)}
    \PY{n}{x1}\PY{p}{,}\PY{n}{x2}\PY{o}{=}\PY{n}{pts}\PY{p}{[}\PY{l+m+mi}{0}\PY{p}{]}\PY{p}{[}\PY{l+m+mi}{1}\PY{p}{]}\PY{p}{,}\PY{n}{pts}\PY{p}{[}\PY{l+m+mi}{1}\PY{p}{]}\PY{p}{[}\PY{l+m+mi}{1}\PY{p}{]}
    \PY{n}{y1}\PY{p}{,}\PY{n}{y2}\PY{o}{=}\PY{n}{pts}\PY{p}{[}\PY{l+m+mi}{0}\PY{p}{]}\PY{p}{[}\PY{l+m+mi}{2}\PY{p}{]}\PY{p}{,}\PY{n}{pts}\PY{p}{[}\PY{l+m+mi}{1}\PY{p}{]}\PY{p}{[}\PY{l+m+mi}{2}\PY{p}{]}
    \PY{n}{m}\PY{o}{=}\PY{p}{(}\PY{n}{y2}\PY{o}{\PYZhy{}}\PY{n}{y1}\PY{p}{)}\PY{o}{/}\PY{p}{(}\PY{n}{x2}\PY{o}{\PYZhy{}}\PY{n}{x1}\PY{p}{)}
    \PY{n}{b}\PY{o}{=}\PY{p}{(}\PY{n}{y1}\PY{o}{*}\PY{n}{x2}\PY{o}{\PYZhy{}}\PY{n}{y2}\PY{o}{*}\PY{n}{x1}\PY{p}{)}\PY{o}{/}\PY{p}{(}\PY{n}{x2}\PY{o}{\PYZhy{}}\PY{n}{x1}\PY{p}{)}
    \PY{k}{return} \PY{n}{m}\PY{p}{,}\PY{n}{b}

\PY{k}{def} \PY{n+nf}{line}\PY{p}{(}\PY{n}{x}\PY{p}{,}\PY{n}{m}\PY{p}{,}\PY{n}{b}\PY{p}{)}\PY{p}{:}
    \PY{k}{return} \PY{n}{x}\PY{o}{*}\PY{n}{m}\PY{o}{+}\PY{n}{b}

\PY{k}{def} \PY{n+nf}{label\PYZus{}linear}\PY{p}{(}\PY{n}{pts}\PY{p}{,}\PY{n}{m}\PY{p}{,}\PY{n}{b}\PY{p}{)}\PY{p}{:}
    \PY{k}{return} \PY{n}{np}\PY{o}{.}\PY{n}{array}\PY{p}{(}\PY{p}{[}\PY{n}{np}\PY{o}{.}\PY{n}{sign}\PY{p}{(}\PY{n}{pts}\PY{p}{[}\PY{n}{i}\PY{p}{]}\PY{p}{[}\PY{l+m+mi}{2}\PY{p}{]}\PY{o}{\PYZhy{}}\PY{p}{(}\PY{n}{m}\PY{o}{*}\PY{n}{pts}\PY{p}{[}\PY{n}{i}\PY{p}{]}\PY{p}{[}\PY{l+m+mi}{1}\PY{p}{]}\PY{o}{+}\PY{n}{b}\PY{p}{)}\PY{p}{)} \PY{k}{for} \PY{n}{i} \PY{o+ow}{in} \PY{n+nb}{range}\PY{p}{(}\PY{n+nb}{len}\PY{p}{(}\PY{n}{pts}\PY{p}{)}\PY{p}{)}\PY{p}{]}\PY{p}{)}

\PY{k}{def} \PY{n+nf}{color\PYZus{}pts}\PY{p}{(}\PY{n}{label}\PY{p}{)}\PY{p}{:}
    \PY{c+c1}{\PYZsh{}green is +1, red is \PYZhy{}1}
    \PY{n}{col}\PY{o}{=}\PY{p}{[}\PY{p}{]}
    \PY{k}{for} \PY{n}{i} \PY{o+ow}{in} \PY{n+nb}{range}\PY{p}{(}\PY{n+nb}{len}\PY{p}{(}\PY{n}{label}\PY{p}{)}\PY{p}{)}\PY{p}{:}
        \PY{k}{if}\PY{p}{(}\PY{n}{label}\PY{p}{[}\PY{n}{i}\PY{p}{]}\PY{o}{\PYZgt{}}\PY{l+m+mi}{0}\PY{p}{)}\PY{p}{:} \PY{n}{col}\PY{o}{.}\PY{n}{append}\PY{p}{(}\PY{l+s+s1}{\PYZsq{}}\PY{l+s+s1}{green}\PY{l+s+s1}{\PYZsq{}}\PY{p}{)}
        \PY{k}{else}\PY{p}{:} \PY{n}{col}\PY{o}{.}\PY{n}{append}\PY{p}{(}\PY{l+s+s1}{\PYZsq{}}\PY{l+s+s1}{red}\PY{l+s+s1}{\PYZsq{}}\PY{p}{)}
    \PY{k}{return} \PY{n}{col}

\PY{k}{def} \PY{n+nf}{SGD\PYZus{}step}\PY{p}{(}\PY{n}{pt}\PY{p}{,}\PY{n}{label}\PY{p}{,}\PY{n}{w}\PY{p}{,}\PY{n}{eta}\PY{p}{)}\PY{p}{:}
    \PY{k}{return} \PY{n}{w}\PY{o}{\PYZhy{}}\PY{n}{eta}\PY{o}{*}\PY{p}{(}\PY{o}{\PYZhy{}}\PY{n}{label}\PY{o}{*}\PY{n}{pt}\PY{o}{/}\PY{p}{(}\PY{l+m+mi}{1}\PY{o}{+}\PY{n}{m}\PY{o}{.}\PY{n}{e}\PY{o}{*}\PY{o}{*}\PY{p}{(}\PY{n}{label}\PY{o}{*}\PY{n}{np}\PY{o}{.}\PY{n}{dot}\PY{p}{(}\PY{n}{w}\PY{p}{,}\PY{n}{pt}\PY{p}{)}\PY{p}{)}\PY{p}{)}\PY{p}{)}

\PY{k}{def} \PY{n+nf}{SGD\PYZus{}epoch}\PY{p}{(}\PY{n}{pts}\PY{p}{,}\PY{n}{label}\PY{p}{,}\PY{n}{w}\PY{p}{,}\PY{n}{eta}\PY{p}{)}\PY{p}{:}
    \PY{c+c1}{\PYZsh{}set epoch random indices}
    \PY{n}{randindex}\PY{o}{=}\PY{n}{np}\PY{o}{.}\PY{n}{random}\PY{o}{.}\PY{n}{choice}\PY{p}{(}\PY{n+nb}{range}\PY{p}{(}\PY{n+nb}{len}\PY{p}{(}\PY{n}{label}\PY{p}{)}\PY{p}{)}\PY{p}{,}\PY{n+nb}{len}\PY{p}{(}\PY{n}{label}\PY{p}{)}\PY{p}{,} \PY{n}{replace}\PY{o}{=}\PY{k+kc}{False}\PY{p}{)}
    \PY{k}{for} \PY{n}{i} \PY{o+ow}{in} \PY{n+nb}{range}\PY{p}{(}\PY{n+nb}{len}\PY{p}{(}\PY{n}{label}\PY{p}{)}\PY{p}{)}\PY{p}{:}
        \PY{n}{w}\PY{o}{=}\PY{n}{SGD\PYZus{}step}\PY{p}{(}\PY{n}{pts}\PY{p}{[}\PY{n}{i}\PY{p}{]}\PY{p}{,}\PY{n}{label}\PY{p}{[}\PY{n}{i}\PY{p}{]}\PY{p}{,}\PY{n}{w}\PY{p}{,}\PY{n}{eta}\PY{p}{)}
    \PY{k}{return} \PY{n}{w}

\PY{k}{def} \PY{n+nf}{cross\PYZus{}entropy}\PY{p}{(}\PY{n}{pts}\PY{p}{,}\PY{n}{label}\PY{p}{,}\PY{n}{w}\PY{p}{)}\PY{p}{:}
    \PY{n}{Npts}\PY{o}{=}\PY{n+nb}{len}\PY{p}{(}\PY{n}{label}\PY{p}{)}
    \PY{n}{Ein}\PY{o}{=}\PY{l+m+mi}{0}
    \PY{k}{for} \PY{n}{i} \PY{o+ow}{in} \PY{n+nb}{range}\PY{p}{(}\PY{n}{Npts}\PY{p}{)}\PY{p}{:}
        \PY{n}{Ein}\PY{o}{+}\PY{o}{=}\PY{n}{m}\PY{o}{.}\PY{n}{log}\PY{p}{(}\PY{l+m+mi}{1}\PY{o}{+}\PY{n}{m}\PY{o}{.}\PY{n}{e}\PY{o}{*}\PY{o}{*}\PY{p}{(}\PY{o}{\PYZhy{}}\PY{n}{label}\PY{p}{[}\PY{n}{i}\PY{p}{]}\PY{o}{*}\PY{n}{np}\PY{o}{.}\PY{n}{dot}\PY{p}{(}\PY{n}{w}\PY{p}{,}\PY{n}{pts}\PY{p}{[}\PY{n}{i}\PY{p}{]}\PY{p}{)}\PY{p}{)}\PY{p}{)}
    \PY{k}{return} \PY{n}{Ein}\PY{o}{/}\PY{n}{Npts}

\PY{k}{def} \PY{n+nf}{Eout\PYZus{}estimate}\PY{p}{(}\PY{n}{m}\PY{p}{,}\PY{n}{b}\PY{p}{,}\PY{n}{w}\PY{p}{,}\PY{n}{Nval}\PY{o}{=}\PY{l+m+mi}{1000}\PY{p}{)}\PY{p}{:}
    \PY{n}{pts}\PY{o}{=}\PY{n}{gen\PYZus{}uniform\PYZus{}points}\PY{p}{(}\PY{n}{Nval}\PY{p}{)}
    \PY{n}{true}\PY{o}{=}\PY{n}{label\PYZus{}linear}\PY{p}{(}\PY{n}{pts}\PY{p}{,}\PY{n}{m}\PY{p}{,}\PY{n}{b}\PY{p}{)}
    \PY{k}{return} \PY{n}{cross\PYZus{}entropy}\PY{p}{(}\PY{n}{pts}\PY{p}{,}\PY{n}{true}\PY{p}{,}\PY{n}{w}\PY{p}{)}

\PY{k}{def} \PY{n+nf}{LogRegr\PYZus{}SGD}\PY{p}{(}\PY{n}{Npts}\PY{p}{,}\PY{n}{eta}\PY{p}{,}\PY{n}{stop}\PY{p}{,}\PY{n}{Nval}\PY{o}{=}\PY{l+m+mi}{1000}\PY{p}{,}\PY{n}{plot}\PY{o}{=}\PY{k+kc}{True}\PY{p}{)}\PY{p}{:}
    
    \PY{c+c1}{\PYZsh{}generate data}
    \PY{n}{pts}\PY{o}{=}\PY{n}{gen\PYZus{}uniform\PYZus{}points}\PY{p}{(}\PY{n}{Npts}\PY{p}{)}
    \PY{n}{m}\PY{p}{,}\PY{n}{b}\PY{o}{=}\PY{n}{gen\PYZus{}line}\PY{p}{(}\PY{p}{)}
    \PY{n}{label}\PY{o}{=}\PY{n}{label\PYZus{}linear}\PY{p}{(}\PY{n}{pts}\PY{p}{,}\PY{n}{m}\PY{p}{,}\PY{n}{b}\PY{p}{)}
    
    \PY{c+c1}{\PYZsh{}initialization of SGD}
    \PY{n}{epoch}\PY{o}{=}\PY{l+m+mi}{0}
    \PY{n}{w}\PY{o}{=}\PY{n}{np}\PY{o}{.}\PY{n}{zeros}\PY{p}{(}\PY{l+m+mi}{3}\PY{p}{)}
    
    \PY{c+c1}{\PYZsh{}SGD epochs}
    \PY{k}{while} \PY{k+kc}{True}\PY{p}{:}
        \PY{n}{wtemp}\PY{o}{=}\PY{n}{w}
        \PY{n}{w}\PY{o}{=}\PY{n}{SGD\PYZus{}epoch}\PY{p}{(}\PY{n}{pts}\PY{p}{,}\PY{n}{label}\PY{p}{,}\PY{n}{w}\PY{p}{,}\PY{n}{eta}\PY{p}{)}
        \PY{n}{dw}\PY{o}{=}\PY{n}{wtemp}\PY{o}{\PYZhy{}}\PY{n}{w}
        \PY{n}{epoch}\PY{o}{+}\PY{o}{=}\PY{l+m+mi}{1}
        \PY{k}{if}\PY{p}{(}\PY{n}{np}\PY{o}{.}\PY{n}{sqrt}\PY{p}{(}\PY{n}{np}\PY{o}{.}\PY{n}{dot}\PY{p}{(}\PY{n}{dw}\PY{p}{,}\PY{n}{dw}\PY{p}{)}\PY{p}{)}\PY{o}{\PYZlt{}}\PY{n}{stop}\PY{p}{)}\PY{p}{:} \PY{k}{break}
    
    \PY{c+c1}{\PYZsh{}Eout estimate}
    \PY{n}{Eout}\PY{o}{=}\PY{n}{Eout\PYZus{}estimate}\PY{p}{(}\PY{n}{m}\PY{p}{,}\PY{n}{b}\PY{p}{,}\PY{n}{w}\PY{p}{,}\PY{n}{Nval}\PY{o}{=}\PY{n}{Nval}\PY{p}{)}
    
    \PY{c+c1}{\PYZsh{}plots}
    \PY{k}{if}\PY{p}{(}\PY{n}{plot}\PY{o}{==}\PY{k+kc}{True}\PY{p}{)}\PY{p}{:}
        \PY{n}{xaxis}\PY{o}{=}\PY{n}{np}\PY{o}{.}\PY{n}{linspace}\PY{p}{(}\PY{o}{\PYZhy{}}\PY{l+m+mi}{1}\PY{p}{,}\PY{l+m+mi}{1}\PY{p}{,}\PY{l+m+mi}{100}\PY{p}{)}
        \PY{n}{col}\PY{o}{=}\PY{n}{color\PYZus{}pts}\PY{p}{(}\PY{n}{label}\PY{p}{)}
        \PY{n}{plt}\PY{o}{.}\PY{n}{scatter}\PY{p}{(}\PY{n}{pts}\PY{p}{[}\PY{p}{:}\PY{p}{,}\PY{l+m+mi}{1}\PY{p}{]}\PY{p}{,}\PY{n}{pts}\PY{p}{[}\PY{p}{:}\PY{p}{,}\PY{l+m+mi}{2}\PY{p}{]}\PY{p}{,}\PY{n}{color}\PY{o}{=}\PY{n}{col}\PY{p}{)}
        \PY{n}{plt}\PY{o}{.}\PY{n}{plot}\PY{p}{(}\PY{n}{xaxis}\PY{p}{,}\PY{n}{line}\PY{p}{(}\PY{n}{xaxis}\PY{p}{,}\PY{n}{m}\PY{p}{,}\PY{n}{b}\PY{p}{)}\PY{p}{,}\PY{n}{label}\PY{o}{=}\PY{l+s+s1}{\PYZsq{}}\PY{l+s+s1}{True}\PY{l+s+s1}{\PYZsq{}}\PY{p}{)}
        \PY{n}{plt}\PY{o}{.}\PY{n}{plot}\PY{p}{(}\PY{n}{xaxis}\PY{p}{,}\PY{n}{line}\PY{p}{(}\PY{n}{xaxis}\PY{p}{,}\PY{o}{\PYZhy{}}\PY{n}{w}\PY{p}{[}\PY{l+m+mi}{1}\PY{p}{]}\PY{o}{/}\PY{n}{w}\PY{p}{[}\PY{l+m+mi}{2}\PY{p}{]}\PY{p}{,}\PY{o}{\PYZhy{}}\PY{n}{w}\PY{p}{[}\PY{l+m+mi}{0}\PY{p}{]}\PY{o}{/}\PY{n}{w}\PY{p}{[}\PY{l+m+mi}{2}\PY{p}{]}\PY{p}{)}\PY{p}{,}\PY{n}{color}\PY{o}{=}\PY{l+s+s1}{\PYZsq{}}\PY{l+s+s1}{blue}\PY{l+s+s1}{\PYZsq{}}\PY{p}{,}\PY{n}{linestyle}\PY{o}{=}\PY{l+s+s1}{\PYZsq{}}\PY{l+s+s1}{dashed}\PY{l+s+s1}{\PYZsq{}}\PY{p}{,}\PY{n}{label}\PY{o}{=}\PY{l+s+s1}{\PYZsq{}}\PY{l+s+s1}{grad\PYZus{}desc}\PY{l+s+s1}{\PYZsq{}}\PY{p}{)}
        \PY{n}{plt}\PY{o}{.}\PY{n}{xlim}\PY{p}{(}\PY{p}{[}\PY{o}{\PYZhy{}}\PY{l+m+mi}{1}\PY{p}{,} \PY{l+m+mi}{1}\PY{p}{]}\PY{p}{)}
        \PY{n}{plt}\PY{o}{.}\PY{n}{ylim}\PY{p}{(}\PY{p}{[}\PY{o}{\PYZhy{}}\PY{l+m+mi}{1}\PY{p}{,} \PY{l+m+mi}{1}\PY{p}{]}\PY{p}{)}
        \PY{n}{plt}\PY{o}{.}\PY{n}{legend}\PY{p}{(}\PY{p}{)}
        \PY{n}{plt}\PY{o}{.}\PY{n}{show}\PY{p}{(}\PY{p}{)}
    
    \PY{k}{return} \PY{n}{w}\PY{p}{,} \PY{n}{epoch}\PY{p}{,} \PY{n}{Eout}
\end{Verbatim}
\end{tcolorbox}

    \begin{tcolorbox}[breakable, size=fbox, boxrule=1pt, pad at break*=1mm,colback=cellbackground, colframe=cellborder]
\prompt{In}{incolor}{69}{\boxspacing}
\begin{Verbatim}[commandchars=\\\{\}]
\PY{n}{Nruns}\PY{o}{=}\PY{l+m+mi}{100}
\PY{n}{Npts}\PY{o}{=}\PY{l+m+mi}{100}
\PY{n}{eta}\PY{o}{=}\PY{l+m+mf}{0.01}
\PY{n}{stop}\PY{o}{=}\PY{l+m+mf}{0.01}
\PY{n}{Eavg}\PY{o}{=}\PY{l+m+mi}{0}
\PY{n}{epochs}\PY{o}{=}\PY{l+m+mi}{0}

\PY{k}{for} \PY{n}{i} \PY{o+ow}{in} \PY{n+nb}{range}\PY{p}{(}\PY{n}{Nruns}\PY{p}{)}\PY{p}{:}
    \PY{n}{w}\PY{p}{,}\PY{n}{ep}\PY{p}{,}\PY{n}{Eout} \PY{o}{=} \PY{n}{LogRegr\PYZus{}SGD}\PY{p}{(}\PY{n}{Npts}\PY{p}{,}\PY{n}{eta}\PY{p}{,}\PY{n}{stop}\PY{p}{,}\PY{n}{plot}\PY{o}{=}\PY{k+kc}{False}\PY{p}{)}
    \PY{n}{Eavg}\PY{o}{+}\PY{o}{=}\PY{n}{Eout}
    \PY{n}{epochs}\PY{o}{+}\PY{o}{=}\PY{n}{ep}
    
\PY{n+nb}{print}\PY{p}{(}\PY{n}{f}\PY{l+s+s1}{\PYZsq{}}\PY{l+s+s1}{For }\PY{l+s+si}{\PYZob{}Nruns\PYZcb{}}\PY{l+s+s1}{ runs of Logistic Regression with SGD, with }\PY{l+s+si}{\PYZob{}Npts\PYZcb{}}\PY{l+s+s1}{ points each run and a learning rate of }\PY{l+s+si}{\PYZob{}eta\PYZcb{}}\PY{l+s+s1}{, we get the following average results: }\PY{l+s+se}{\PYZbs{}n}\PY{l+s+s1}{ Average Eout (cross\PYZhy{}entropy error)=}\PY{l+s+s1}{\PYZob{}}\PY{l+s+s1}{Eavg/Nruns:.3f\PYZcb{} }\PY{l+s+se}{\PYZbs{}n}\PY{l+s+s1}{ Average epoch of convergence (with dwstop=}\PY{l+s+si}{\PYZob{}stop\PYZcb{}}\PY{l+s+s1}{)=}\PY{l+s+s1}{\PYZob{}}\PY{l+s+s1}{epochs/Nruns:.0f\PYZcb{}}\PY{l+s+s1}{\PYZsq{}}\PY{p}{)}
\end{Verbatim}
\end{tcolorbox}

    \begin{Verbatim}[commandchars=\\\{\}]
For 100 runs of Logistic Regression with SGD, with 100 points each run and a
learning rate of 0.01, we get the following average results:
 Average Eout (cross-entropy error)=0.104
 Average epoch of convergence (with dwstop=0.01)=349
    \end{Verbatim}

    \begin{tcolorbox}[breakable, size=fbox, boxrule=1pt, pad at break*=1mm,colback=cellbackground, colframe=cellborder]
\prompt{In}{incolor}{75}{\boxspacing}
\begin{Verbatim}[commandchars=\\\{\}]
\PY{c+c1}{\PYZsh{}plotted example}
\PY{n}{ex}\PY{o}{=}\PY{n}{LogRegr\PYZus{}SGD}\PY{p}{(}\PY{n}{Npts}\PY{p}{,}\PY{n}{eta}\PY{p}{,}\PY{n}{stop}\PY{p}{)}
\end{Verbatim}
\end{tcolorbox}

    \begin{center}
    \adjustimage{max size={0.5\linewidth}{0.5\paperheight}}{output_14_0.png}
    \end{center}
    { \hspace*{\fill} \\}
    
    \hypertarget{problem-10}{%
\section{Problem 10}\label{problem-10}}

\hypertarget{answers-e-e_nmathbfw-rm-min0y_nmathbfwtmathbfx_n}{%
\subsection{\texorpdfstring{Answers: {[}e{]}
\(e_n(\mathbf{w})=-{\rm min}(0,y_n\mathbf{w}^T\mathbf{x}_n)\)}{Answers: {[}e{]} e\_n(\textbackslash{}mathbf\{w\})=-\{\textbackslash{}rm min\}(0,y\_n\textbackslash{}mathbf\{w\}\^{}T\textbackslash{}mathbf\{x\}\_n)}}\label{answers-e-e_nmathbfw-rm-min0y_nmathbfwtmathbfx_n}}

\hypertarget{derivation}{%
\subsection{Derivation:}\label{derivation}}

In Stochastic Gradient Descent (SGD) method, the weights are updated by
picking one random point of the sample and computing the gradient of the
error function \(e_n(\mathbf{w})\). In formulae:

\begin{equation}
\mathbf{w}\rightarrow \mathbf{w}-\eta \nabla e_n(\mathbf{w})\,,
\end{equation}

where \(\eta\) is the learning rate.

For the error function proposed {[}e{]}, the gradient is:

\begin{equation}
 \nabla e_n(\mathbf{w})= -\nabla \left({\rm min}(0,y_n\mathbf{w}^T\mathbf{x}_n)\right)=
\begin{cases}
-y_n\mathbf{x}_n & \text{for } y_n\mathbf{w}^T\mathbf{x}_n < 0\\
0  & \text{for } y_n\mathbf{w}^T\mathbf{x}_n \ge 0
\end{cases}
\end{equation}

We observe that \(y_n\mathbf{w}^T\mathbf{x}_n<0\) if and only if the
point \(\mathbf{x}_n\) is misclassified by the current weights.

In other words, if the point \(\mathbf{x}_n\) is misclassified, the SGD
algorithm with the error function defined in {[}e{]} and a learning rate
of \(\eta=1\) will update the weights as the following:

\begin{equation}
\mathbf{w}\rightarrow \mathbf{w}+y_n\mathbf{x}_n\,.
\end{equation}

This reproduces exactly the Perceptron Linear Algorithm (PLA) step.

    \begin{tcolorbox}[breakable, size=fbox, boxrule=1pt, pad at break*=1mm,colback=cellbackground, colframe=cellborder]
\prompt{In}{incolor}{ }{\boxspacing}
\begin{Verbatim}[commandchars=\\\{\}]

\end{Verbatim}
\end{tcolorbox}


    % Add a bibliography block to the postdoc
    
    
    
\end{document}
