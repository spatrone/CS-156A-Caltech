\documentclass[11pt]{article}

    \usepackage[breakable]{tcolorbox}
    \usepackage{parskip} % Stop auto-indenting (to mimic markdown behaviour)
    
    \usepackage{iftex}
    \ifPDFTeX
    	\usepackage[T1]{fontenc}
    	\usepackage{mathpazo}
    \else
    	\usepackage{fontspec}
    \fi

    % Basic figure setup, for now with no caption control since it's done
    % automatically by Pandoc (which extracts ![](path) syntax from Markdown).
    \usepackage{graphicx}
    % Maintain compatibility with old templates. Remove in nbconvert 6.0
    \let\Oldincludegraphics\includegraphics
    % Ensure that by default, figures have no caption (until we provide a
    % proper Figure object with a Caption API and a way to capture that
    % in the conversion process - todo).
    \usepackage{caption}
    \DeclareCaptionFormat{nocaption}{}
    \captionsetup{format=nocaption,aboveskip=0pt,belowskip=0pt}

    \usepackage[Export]{adjustbox} % Used to constrain images to a maximum size
    \adjustboxset{max size={0.9\linewidth}{0.9\paperheight}}
    \usepackage{float}
    \floatplacement{figure}{H} % forces figures to be placed at the correct location
    \usepackage{xcolor} % Allow colors to be defined
    \usepackage{enumerate} % Needed for markdown enumerations to work
    \usepackage{geometry} % Used to adjust the document margins
    \usepackage{amsmath} % Equations
    \usepackage{amssymb} % Equations
    \usepackage{textcomp} % defines textquotesingle
    % Hack from http://tex.stackexchange.com/a/47451/13684:
    \AtBeginDocument{%
        \def\PYZsq{\textquotesingle}% Upright quotes in Pygmentized code
    }
    \usepackage{upquote} % Upright quotes for verbatim code
    \usepackage{eurosym} % defines \euro
    \usepackage[mathletters]{ucs} % Extended unicode (utf-8) support
    \usepackage{fancyvrb} % verbatim replacement that allows latex
    \usepackage{grffile} % extends the file name processing of package graphics 
                         % to support a larger range
    \makeatletter % fix for grffile with XeLaTeX
    \def\Gread@@xetex#1{%
      \IfFileExists{"\Gin@base".bb}%
      {\Gread@eps{\Gin@base.bb}}%
      {\Gread@@xetex@aux#1}%
    }
    \makeatother

    % The hyperref package gives us a pdf with properly built
    % internal navigation ('pdf bookmarks' for the table of contents,
    % internal cross-reference links, web links for URLs, etc.)
    \usepackage{hyperref}
    % The default LaTeX title has an obnoxious amount of whitespace. By default,
    % titling removes some of it. It also provides customization options.
    \usepackage{titling}
    \usepackage{longtable} % longtable support required by pandoc >1.10
    \usepackage{booktabs}  % table support for pandoc > 1.12.2
    \usepackage[inline]{enumitem} % IRkernel/repr support (it uses the enumerate* environment)
    \usepackage[normalem]{ulem} % ulem is needed to support strikethroughs (\sout)
                                % normalem makes italics be italics, not underlines
    \usepackage{mathrsfs}
    

    
    % Colors for the hyperref package
    \definecolor{urlcolor}{rgb}{0,.145,.698}
    \definecolor{linkcolor}{rgb}{.71,0.21,0.01}
    \definecolor{citecolor}{rgb}{.12,.54,.11}

    % ANSI colors
    \definecolor{ansi-black}{HTML}{3E424D}
    \definecolor{ansi-black-intense}{HTML}{282C36}
    \definecolor{ansi-red}{HTML}{E75C58}
    \definecolor{ansi-red-intense}{HTML}{B22B31}
    \definecolor{ansi-green}{HTML}{00A250}
    \definecolor{ansi-green-intense}{HTML}{007427}
    \definecolor{ansi-yellow}{HTML}{DDB62B}
    \definecolor{ansi-yellow-intense}{HTML}{B27D12}
    \definecolor{ansi-blue}{HTML}{208FFB}
    \definecolor{ansi-blue-intense}{HTML}{0065CA}
    \definecolor{ansi-magenta}{HTML}{D160C4}
    \definecolor{ansi-magenta-intense}{HTML}{A03196}
    \definecolor{ansi-cyan}{HTML}{60C6C8}
    \definecolor{ansi-cyan-intense}{HTML}{258F8F}
    \definecolor{ansi-white}{HTML}{C5C1B4}
    \definecolor{ansi-white-intense}{HTML}{A1A6B2}
    \definecolor{ansi-default-inverse-fg}{HTML}{FFFFFF}
    \definecolor{ansi-default-inverse-bg}{HTML}{000000}

    % commands and environments needed by pandoc snippets
    % extracted from the output of `pandoc -s`
    \providecommand{\tightlist}{%
      \setlength{\itemsep}{0pt}\setlength{\parskip}{0pt}}
    \DefineVerbatimEnvironment{Highlighting}{Verbatim}{commandchars=\\\{\}}
    % Add ',fontsize=\small' for more characters per line
    \newenvironment{Shaded}{}{}
    \newcommand{\KeywordTok}[1]{\textcolor[rgb]{0.00,0.44,0.13}{\textbf{{#1}}}}
    \newcommand{\DataTypeTok}[1]{\textcolor[rgb]{0.56,0.13,0.00}{{#1}}}
    \newcommand{\DecValTok}[1]{\textcolor[rgb]{0.25,0.63,0.44}{{#1}}}
    \newcommand{\BaseNTok}[1]{\textcolor[rgb]{0.25,0.63,0.44}{{#1}}}
    \newcommand{\FloatTok}[1]{\textcolor[rgb]{0.25,0.63,0.44}{{#1}}}
    \newcommand{\CharTok}[1]{\textcolor[rgb]{0.25,0.44,0.63}{{#1}}}
    \newcommand{\StringTok}[1]{\textcolor[rgb]{0.25,0.44,0.63}{{#1}}}
    \newcommand{\CommentTok}[1]{\textcolor[rgb]{0.38,0.63,0.69}{\textit{{#1}}}}
    \newcommand{\OtherTok}[1]{\textcolor[rgb]{0.00,0.44,0.13}{{#1}}}
    \newcommand{\AlertTok}[1]{\textcolor[rgb]{1.00,0.00,0.00}{\textbf{{#1}}}}
    \newcommand{\FunctionTok}[1]{\textcolor[rgb]{0.02,0.16,0.49}{{#1}}}
    \newcommand{\RegionMarkerTok}[1]{{#1}}
    \newcommand{\ErrorTok}[1]{\textcolor[rgb]{1.00,0.00,0.00}{\textbf{{#1}}}}
    \newcommand{\NormalTok}[1]{{#1}}
    
    % Additional commands for more recent versions of Pandoc
    \newcommand{\ConstantTok}[1]{\textcolor[rgb]{0.53,0.00,0.00}{{#1}}}
    \newcommand{\SpecialCharTok}[1]{\textcolor[rgb]{0.25,0.44,0.63}{{#1}}}
    \newcommand{\VerbatimStringTok}[1]{\textcolor[rgb]{0.25,0.44,0.63}{{#1}}}
    \newcommand{\SpecialStringTok}[1]{\textcolor[rgb]{0.73,0.40,0.53}{{#1}}}
    \newcommand{\ImportTok}[1]{{#1}}
    \newcommand{\DocumentationTok}[1]{\textcolor[rgb]{0.73,0.13,0.13}{\textit{{#1}}}}
    \newcommand{\AnnotationTok}[1]{\textcolor[rgb]{0.38,0.63,0.69}{\textbf{\textit{{#1}}}}}
    \newcommand{\CommentVarTok}[1]{\textcolor[rgb]{0.38,0.63,0.69}{\textbf{\textit{{#1}}}}}
    \newcommand{\VariableTok}[1]{\textcolor[rgb]{0.10,0.09,0.49}{{#1}}}
    \newcommand{\ControlFlowTok}[1]{\textcolor[rgb]{0.00,0.44,0.13}{\textbf{{#1}}}}
    \newcommand{\OperatorTok}[1]{\textcolor[rgb]{0.40,0.40,0.40}{{#1}}}
    \newcommand{\BuiltInTok}[1]{{#1}}
    \newcommand{\ExtensionTok}[1]{{#1}}
    \newcommand{\PreprocessorTok}[1]{\textcolor[rgb]{0.74,0.48,0.00}{{#1}}}
    \newcommand{\AttributeTok}[1]{\textcolor[rgb]{0.49,0.56,0.16}{{#1}}}
    \newcommand{\InformationTok}[1]{\textcolor[rgb]{0.38,0.63,0.69}{\textbf{\textit{{#1}}}}}
    \newcommand{\WarningTok}[1]{\textcolor[rgb]{0.38,0.63,0.69}{\textbf{\textit{{#1}}}}}
    
    
    % Define a nice break command that doesn't care if a line doesn't already
    % exist.
    \def\br{\hspace*{\fill} \\* }
    % Math Jax compatibility definitions
    \def\gt{>}
    \def\lt{<}
    \let\Oldtex\TeX
    \let\Oldlatex\LaTeX
    \renewcommand{\TeX}{\textrm{\Oldtex}}
    \renewcommand{\LaTeX}{\textrm{\Oldlatex}}
    % Document parameters
    % Document title
    \title{Pset\_1}
    
    
    
    
    
% Pygments definitions
\makeatletter
\def\PY@reset{\let\PY@it=\relax \let\PY@bf=\relax%
    \let\PY@ul=\relax \let\PY@tc=\relax%
    \let\PY@bc=\relax \let\PY@ff=\relax}
\def\PY@tok#1{\csname PY@tok@#1\endcsname}
\def\PY@toks#1+{\ifx\relax#1\empty\else%
    \PY@tok{#1}\expandafter\PY@toks\fi}
\def\PY@do#1{\PY@bc{\PY@tc{\PY@ul{%
    \PY@it{\PY@bf{\PY@ff{#1}}}}}}}
\def\PY#1#2{\PY@reset\PY@toks#1+\relax+\PY@do{#2}}

\expandafter\def\csname PY@tok@w\endcsname{\def\PY@tc##1{\textcolor[rgb]{0.73,0.73,0.73}{##1}}}
\expandafter\def\csname PY@tok@c\endcsname{\let\PY@it=\textit\def\PY@tc##1{\textcolor[rgb]{0.25,0.50,0.50}{##1}}}
\expandafter\def\csname PY@tok@cp\endcsname{\def\PY@tc##1{\textcolor[rgb]{0.74,0.48,0.00}{##1}}}
\expandafter\def\csname PY@tok@k\endcsname{\let\PY@bf=\textbf\def\PY@tc##1{\textcolor[rgb]{0.00,0.50,0.00}{##1}}}
\expandafter\def\csname PY@tok@kp\endcsname{\def\PY@tc##1{\textcolor[rgb]{0.00,0.50,0.00}{##1}}}
\expandafter\def\csname PY@tok@kt\endcsname{\def\PY@tc##1{\textcolor[rgb]{0.69,0.00,0.25}{##1}}}
\expandafter\def\csname PY@tok@o\endcsname{\def\PY@tc##1{\textcolor[rgb]{0.40,0.40,0.40}{##1}}}
\expandafter\def\csname PY@tok@ow\endcsname{\let\PY@bf=\textbf\def\PY@tc##1{\textcolor[rgb]{0.67,0.13,1.00}{##1}}}
\expandafter\def\csname PY@tok@nb\endcsname{\def\PY@tc##1{\textcolor[rgb]{0.00,0.50,0.00}{##1}}}
\expandafter\def\csname PY@tok@nf\endcsname{\def\PY@tc##1{\textcolor[rgb]{0.00,0.00,1.00}{##1}}}
\expandafter\def\csname PY@tok@nc\endcsname{\let\PY@bf=\textbf\def\PY@tc##1{\textcolor[rgb]{0.00,0.00,1.00}{##1}}}
\expandafter\def\csname PY@tok@nn\endcsname{\let\PY@bf=\textbf\def\PY@tc##1{\textcolor[rgb]{0.00,0.00,1.00}{##1}}}
\expandafter\def\csname PY@tok@ne\endcsname{\let\PY@bf=\textbf\def\PY@tc##1{\textcolor[rgb]{0.82,0.25,0.23}{##1}}}
\expandafter\def\csname PY@tok@nv\endcsname{\def\PY@tc##1{\textcolor[rgb]{0.10,0.09,0.49}{##1}}}
\expandafter\def\csname PY@tok@no\endcsname{\def\PY@tc##1{\textcolor[rgb]{0.53,0.00,0.00}{##1}}}
\expandafter\def\csname PY@tok@nl\endcsname{\def\PY@tc##1{\textcolor[rgb]{0.63,0.63,0.00}{##1}}}
\expandafter\def\csname PY@tok@ni\endcsname{\let\PY@bf=\textbf\def\PY@tc##1{\textcolor[rgb]{0.60,0.60,0.60}{##1}}}
\expandafter\def\csname PY@tok@na\endcsname{\def\PY@tc##1{\textcolor[rgb]{0.49,0.56,0.16}{##1}}}
\expandafter\def\csname PY@tok@nt\endcsname{\let\PY@bf=\textbf\def\PY@tc##1{\textcolor[rgb]{0.00,0.50,0.00}{##1}}}
\expandafter\def\csname PY@tok@nd\endcsname{\def\PY@tc##1{\textcolor[rgb]{0.67,0.13,1.00}{##1}}}
\expandafter\def\csname PY@tok@s\endcsname{\def\PY@tc##1{\textcolor[rgb]{0.73,0.13,0.13}{##1}}}
\expandafter\def\csname PY@tok@sd\endcsname{\let\PY@it=\textit\def\PY@tc##1{\textcolor[rgb]{0.73,0.13,0.13}{##1}}}
\expandafter\def\csname PY@tok@si\endcsname{\let\PY@bf=\textbf\def\PY@tc##1{\textcolor[rgb]{0.73,0.40,0.53}{##1}}}
\expandafter\def\csname PY@tok@se\endcsname{\let\PY@bf=\textbf\def\PY@tc##1{\textcolor[rgb]{0.73,0.40,0.13}{##1}}}
\expandafter\def\csname PY@tok@sr\endcsname{\def\PY@tc##1{\textcolor[rgb]{0.73,0.40,0.53}{##1}}}
\expandafter\def\csname PY@tok@ss\endcsname{\def\PY@tc##1{\textcolor[rgb]{0.10,0.09,0.49}{##1}}}
\expandafter\def\csname PY@tok@sx\endcsname{\def\PY@tc##1{\textcolor[rgb]{0.00,0.50,0.00}{##1}}}
\expandafter\def\csname PY@tok@m\endcsname{\def\PY@tc##1{\textcolor[rgb]{0.40,0.40,0.40}{##1}}}
\expandafter\def\csname PY@tok@gh\endcsname{\let\PY@bf=\textbf\def\PY@tc##1{\textcolor[rgb]{0.00,0.00,0.50}{##1}}}
\expandafter\def\csname PY@tok@gu\endcsname{\let\PY@bf=\textbf\def\PY@tc##1{\textcolor[rgb]{0.50,0.00,0.50}{##1}}}
\expandafter\def\csname PY@tok@gd\endcsname{\def\PY@tc##1{\textcolor[rgb]{0.63,0.00,0.00}{##1}}}
\expandafter\def\csname PY@tok@gi\endcsname{\def\PY@tc##1{\textcolor[rgb]{0.00,0.63,0.00}{##1}}}
\expandafter\def\csname PY@tok@gr\endcsname{\def\PY@tc##1{\textcolor[rgb]{1.00,0.00,0.00}{##1}}}
\expandafter\def\csname PY@tok@ge\endcsname{\let\PY@it=\textit}
\expandafter\def\csname PY@tok@gs\endcsname{\let\PY@bf=\textbf}
\expandafter\def\csname PY@tok@gp\endcsname{\let\PY@bf=\textbf\def\PY@tc##1{\textcolor[rgb]{0.00,0.00,0.50}{##1}}}
\expandafter\def\csname PY@tok@go\endcsname{\def\PY@tc##1{\textcolor[rgb]{0.53,0.53,0.53}{##1}}}
\expandafter\def\csname PY@tok@gt\endcsname{\def\PY@tc##1{\textcolor[rgb]{0.00,0.27,0.87}{##1}}}
\expandafter\def\csname PY@tok@err\endcsname{\def\PY@bc##1{\setlength{\fboxsep}{0pt}\fcolorbox[rgb]{1.00,0.00,0.00}{1,1,1}{\strut ##1}}}
\expandafter\def\csname PY@tok@kc\endcsname{\let\PY@bf=\textbf\def\PY@tc##1{\textcolor[rgb]{0.00,0.50,0.00}{##1}}}
\expandafter\def\csname PY@tok@kd\endcsname{\let\PY@bf=\textbf\def\PY@tc##1{\textcolor[rgb]{0.00,0.50,0.00}{##1}}}
\expandafter\def\csname PY@tok@kn\endcsname{\let\PY@bf=\textbf\def\PY@tc##1{\textcolor[rgb]{0.00,0.50,0.00}{##1}}}
\expandafter\def\csname PY@tok@kr\endcsname{\let\PY@bf=\textbf\def\PY@tc##1{\textcolor[rgb]{0.00,0.50,0.00}{##1}}}
\expandafter\def\csname PY@tok@bp\endcsname{\def\PY@tc##1{\textcolor[rgb]{0.00,0.50,0.00}{##1}}}
\expandafter\def\csname PY@tok@fm\endcsname{\def\PY@tc##1{\textcolor[rgb]{0.00,0.00,1.00}{##1}}}
\expandafter\def\csname PY@tok@vc\endcsname{\def\PY@tc##1{\textcolor[rgb]{0.10,0.09,0.49}{##1}}}
\expandafter\def\csname PY@tok@vg\endcsname{\def\PY@tc##1{\textcolor[rgb]{0.10,0.09,0.49}{##1}}}
\expandafter\def\csname PY@tok@vi\endcsname{\def\PY@tc##1{\textcolor[rgb]{0.10,0.09,0.49}{##1}}}
\expandafter\def\csname PY@tok@vm\endcsname{\def\PY@tc##1{\textcolor[rgb]{0.10,0.09,0.49}{##1}}}
\expandafter\def\csname PY@tok@sa\endcsname{\def\PY@tc##1{\textcolor[rgb]{0.73,0.13,0.13}{##1}}}
\expandafter\def\csname PY@tok@sb\endcsname{\def\PY@tc##1{\textcolor[rgb]{0.73,0.13,0.13}{##1}}}
\expandafter\def\csname PY@tok@sc\endcsname{\def\PY@tc##1{\textcolor[rgb]{0.73,0.13,0.13}{##1}}}
\expandafter\def\csname PY@tok@dl\endcsname{\def\PY@tc##1{\textcolor[rgb]{0.73,0.13,0.13}{##1}}}
\expandafter\def\csname PY@tok@s2\endcsname{\def\PY@tc##1{\textcolor[rgb]{0.73,0.13,0.13}{##1}}}
\expandafter\def\csname PY@tok@sh\endcsname{\def\PY@tc##1{\textcolor[rgb]{0.73,0.13,0.13}{##1}}}
\expandafter\def\csname PY@tok@s1\endcsname{\def\PY@tc##1{\textcolor[rgb]{0.73,0.13,0.13}{##1}}}
\expandafter\def\csname PY@tok@mb\endcsname{\def\PY@tc##1{\textcolor[rgb]{0.40,0.40,0.40}{##1}}}
\expandafter\def\csname PY@tok@mf\endcsname{\def\PY@tc##1{\textcolor[rgb]{0.40,0.40,0.40}{##1}}}
\expandafter\def\csname PY@tok@mh\endcsname{\def\PY@tc##1{\textcolor[rgb]{0.40,0.40,0.40}{##1}}}
\expandafter\def\csname PY@tok@mi\endcsname{\def\PY@tc##1{\textcolor[rgb]{0.40,0.40,0.40}{##1}}}
\expandafter\def\csname PY@tok@il\endcsname{\def\PY@tc##1{\textcolor[rgb]{0.40,0.40,0.40}{##1}}}
\expandafter\def\csname PY@tok@mo\endcsname{\def\PY@tc##1{\textcolor[rgb]{0.40,0.40,0.40}{##1}}}
\expandafter\def\csname PY@tok@ch\endcsname{\let\PY@it=\textit\def\PY@tc##1{\textcolor[rgb]{0.25,0.50,0.50}{##1}}}
\expandafter\def\csname PY@tok@cm\endcsname{\let\PY@it=\textit\def\PY@tc##1{\textcolor[rgb]{0.25,0.50,0.50}{##1}}}
\expandafter\def\csname PY@tok@cpf\endcsname{\let\PY@it=\textit\def\PY@tc##1{\textcolor[rgb]{0.25,0.50,0.50}{##1}}}
\expandafter\def\csname PY@tok@c1\endcsname{\let\PY@it=\textit\def\PY@tc##1{\textcolor[rgb]{0.25,0.50,0.50}{##1}}}
\expandafter\def\csname PY@tok@cs\endcsname{\let\PY@it=\textit\def\PY@tc##1{\textcolor[rgb]{0.25,0.50,0.50}{##1}}}

\def\PYZbs{\char`\\}
\def\PYZus{\char`\_}
\def\PYZob{\char`\{}
\def\PYZcb{\char`\}}
\def\PYZca{\char`\^}
\def\PYZam{\char`\&}
\def\PYZlt{\char`\<}
\def\PYZgt{\char`\>}
\def\PYZsh{\char`\#}
\def\PYZpc{\char`\%}
\def\PYZdl{\char`\$}
\def\PYZhy{\char`\-}
\def\PYZsq{\char`\'}
\def\PYZdq{\char`\"}
\def\PYZti{\char`\~}
% for compatibility with earlier versions
\def\PYZat{@}
\def\PYZlb{[}
\def\PYZrb{]}
\makeatother


    % For linebreaks inside Verbatim environment from package fancyvrb. 
    \makeatletter
        \newbox\Wrappedcontinuationbox 
        \newbox\Wrappedvisiblespacebox 
        \newcommand*\Wrappedvisiblespace {\textcolor{red}{\textvisiblespace}} 
        \newcommand*\Wrappedcontinuationsymbol {\textcolor{red}{\llap{\tiny$\m@th\hookrightarrow$}}} 
        \newcommand*\Wrappedcontinuationindent {3ex } 
        \newcommand*\Wrappedafterbreak {\kern\Wrappedcontinuationindent\copy\Wrappedcontinuationbox} 
        % Take advantage of the already applied Pygments mark-up to insert 
        % potential linebreaks for TeX processing. 
        %        {, <, #, %, $, ' and ": go to next line. 
        %        _, }, ^, &, >, - and ~: stay at end of broken line. 
        % Use of \textquotesingle for straight quote. 
        \newcommand*\Wrappedbreaksatspecials {% 
            \def\PYGZus{\discretionary{\char`\_}{\Wrappedafterbreak}{\char`\_}}% 
            \def\PYGZob{\discretionary{}{\Wrappedafterbreak\char`\{}{\char`\{}}% 
            \def\PYGZcb{\discretionary{\char`\}}{\Wrappedafterbreak}{\char`\}}}% 
            \def\PYGZca{\discretionary{\char`\^}{\Wrappedafterbreak}{\char`\^}}% 
            \def\PYGZam{\discretionary{\char`\&}{\Wrappedafterbreak}{\char`\&}}% 
            \def\PYGZlt{\discretionary{}{\Wrappedafterbreak\char`\<}{\char`\<}}% 
            \def\PYGZgt{\discretionary{\char`\>}{\Wrappedafterbreak}{\char`\>}}% 
            \def\PYGZsh{\discretionary{}{\Wrappedafterbreak\char`\#}{\char`\#}}% 
            \def\PYGZpc{\discretionary{}{\Wrappedafterbreak\char`\%}{\char`\%}}% 
            \def\PYGZdl{\discretionary{}{\Wrappedafterbreak\char`\$}{\char`\$}}% 
            \def\PYGZhy{\discretionary{\char`\-}{\Wrappedafterbreak}{\char`\-}}% 
            \def\PYGZsq{\discretionary{}{\Wrappedafterbreak\textquotesingle}{\textquotesingle}}% 
            \def\PYGZdq{\discretionary{}{\Wrappedafterbreak\char`\"}{\char`\"}}% 
            \def\PYGZti{\discretionary{\char`\~}{\Wrappedafterbreak}{\char`\~}}% 
        } 
        % Some characters . , ; ? ! / are not pygmentized. 
        % This macro makes them "active" and they will insert potential linebreaks 
        \newcommand*\Wrappedbreaksatpunct {% 
            \lccode`\~`\.\lowercase{\def~}{\discretionary{\hbox{\char`\.}}{\Wrappedafterbreak}{\hbox{\char`\.}}}% 
            \lccode`\~`\,\lowercase{\def~}{\discretionary{\hbox{\char`\,}}{\Wrappedafterbreak}{\hbox{\char`\,}}}% 
            \lccode`\~`\;\lowercase{\def~}{\discretionary{\hbox{\char`\;}}{\Wrappedafterbreak}{\hbox{\char`\;}}}% 
            \lccode`\~`\:\lowercase{\def~}{\discretionary{\hbox{\char`\:}}{\Wrappedafterbreak}{\hbox{\char`\:}}}% 
            \lccode`\~`\?\lowercase{\def~}{\discretionary{\hbox{\char`\?}}{\Wrappedafterbreak}{\hbox{\char`\?}}}% 
            \lccode`\~`\!\lowercase{\def~}{\discretionary{\hbox{\char`\!}}{\Wrappedafterbreak}{\hbox{\char`\!}}}% 
            \lccode`\~`\/\lowercase{\def~}{\discretionary{\hbox{\char`\/}}{\Wrappedafterbreak}{\hbox{\char`\/}}}% 
            \catcode`\.\active
            \catcode`\,\active 
            \catcode`\;\active
            \catcode`\:\active
            \catcode`\?\active
            \catcode`\!\active
            \catcode`\/\active 
            \lccode`\~`\~ 	
        }
    \makeatother

    \let\OriginalVerbatim=\Verbatim
    \makeatletter
    \renewcommand{\Verbatim}[1][1]{%
        %\parskip\z@skip
        \sbox\Wrappedcontinuationbox {\Wrappedcontinuationsymbol}%
        \sbox\Wrappedvisiblespacebox {\FV@SetupFont\Wrappedvisiblespace}%
        \def\FancyVerbFormatLine ##1{\hsize\linewidth
            \vtop{\raggedright\hyphenpenalty\z@\exhyphenpenalty\z@
                \doublehyphendemerits\z@\finalhyphendemerits\z@
                \strut ##1\strut}%
        }%
        % If the linebreak is at a space, the latter will be displayed as visible
        % space at end of first line, and a continuation symbol starts next line.
        % Stretch/shrink are however usually zero for typewriter font.
        \def\FV@Space {%
            \nobreak\hskip\z@ plus\fontdimen3\font minus\fontdimen4\font
            \discretionary{\copy\Wrappedvisiblespacebox}{\Wrappedafterbreak}
            {\kern\fontdimen2\font}%
        }%
        
        % Allow breaks at special characters using \PYG... macros.
        \Wrappedbreaksatspecials
        % Breaks at punctuation characters . , ; ? ! and / need catcode=\active 	
        \OriginalVerbatim[#1,codes*=\Wrappedbreaksatpunct]%
    }
    \makeatother

    % Exact colors from NB
    \definecolor{incolor}{HTML}{303F9F}
    \definecolor{outcolor}{HTML}{D84315}
    \definecolor{cellborder}{HTML}{CFCFCF}
    \definecolor{cellbackground}{HTML}{F7F7F7}
    
    % prompt
    \makeatletter
    \newcommand{\boxspacing}{\kern\kvtcb@left@rule\kern\kvtcb@boxsep}
    \makeatother
    \newcommand{\prompt}[4]{
        \ttfamily\llap{{\color{#2}[#3]:\hspace{3pt}#4}}\vspace{-\baselineskip}
    }
    

    
    % Prevent overflowing lines due to hard-to-break entities
    \sloppy 
    % Setup hyperref package
    \hypersetup{
      breaklinks=true,  % so long urls are correctly broken across lines
      colorlinks=true,
      urlcolor=urlcolor,
      linkcolor=linkcolor,
      citecolor=citecolor,
      }
    % Slightly bigger margins than the latex defaults
    
    \geometry{verbose,tmargin=1in,bmargin=1in,lmargin=1in,rmargin=1in}
    
\makeatletter
\renewcommand{\@seccntformat}[1]{}
\makeatother  

\begin{document}
    \title{CS 156a - Problem Set 1}
    \author{Samuel Patrone, 2140749}
    \maketitle
    

The following notebook is publicly available at the following
\href{https://github.com/spatrone/CS156A-Caltech.git}{link}.

\tableofcontents
    

    \hypertarget{problem-1}{%
\section{Problem 1}\label{problem-1}}

\hypertarget{answer-d}{%
\subsection{Answer: {[}d{]}}\label{answer-d}}

\hypertarget{derivation}{%
\subsection{Derivation:}\label{derivation}}

(i): the target function is known in advance (from the specifications
given from US Mint). Nothing is learned by the machine.

(ii): a training set of data with known output is given in advance to
the algorithm in order to learn the target function itself. This is an
example of supervised learning.

(iii): a game is the classical example of reinforcement learning, in
which an evaluation of the output (grading) is provided to the machine
in order to learn and adjust to develop the best strategy.

    \hypertarget{problem-2}{%
\section{Problem 2}\label{problem-2}}

\hypertarget{answer-a}{%
\subsection{Answer: {[}a{]}}\label{answer-a}}

\hypertarget{derivation}{%
\subsection{Derivation:}\label{derivation}}

A generic problem, in order to be best suited for ML, needs three
requisites:

\begin{itemize}
\item
  a pattern exists
\item
  it cannot be pinned down mathematically
\item
  we have DATA on it
\end{itemize}

In the list of problem presented

(i): the problem of primes numbers identification can be pinned down
mathematically.

(ii): detecting potential fraud in credit cards has all the three
characteristics described above -\textgreater{} ML suited

(iii): the law of mechanics can describe the equation of motion of any
object (under certain conditions). Again, the problem can be solved
analytically.

(iv): optimal cycle for traffic lights has again all the three
characteristics described above -\textgreater{} ML suited

    \hypertarget{problem-3}{%
\section{Problem 3}\label{problem-3}}

\hypertarget{answer-d}{%
\subsection{Answer: {[}d{]}}\label{answer-d}}

\hypertarget{derivation}{%
\subsection{Derivation:}\label{derivation}}

To solve this problem, we can use the standard formula for conditional
probability, which states that

\[
P(A|B)=\frac{P(A \cap B)}{P(B)}\,.
\]

where

\begin{itemize}
\tightlist
\item
  A is the event of picking up a black ball in our second draw
\item
  B is the event of picking up a black ball in our first draw
\item
  \(P(A|B)\) is the probability of that the second ball is black given
  that the first ball is black, i.e.~what the problem requires us to
  compute
\item
  \(P(A \cap B)=\frac{1}{2}\) is the probability of that we pick two
  black balls, i.e.~the probability of choosing the bag with two black
  balls inside
\item
  \(P(B)=\frac{3}{4}\) is the probability of picking up a black ball in
  our first draw
\end{itemize}

Applying the equation above, we have that \[
P(A|B)=\frac{1}{2}\frac{4}{3}=\frac{2}{3}\,.
\]

    \hypertarget{problem-4}{%
\section{Problem 4}\label{problem-4}}

\hypertarget{answer-b}{%
\subsection{Answer: {[}b{]}}\label{answer-b}}

\hypertarget{derivation}{%
\subsection{Derivation:}\label{derivation}}

The probability to get no red marbles in one draw is
\(\bar{\mu}=(1-\mu)=0.45\). In one sample, we draw 10 marbles
independently, the probability to get all green marbles, i.e. \(\nu=0\),
is just:

\[
\label{noredonesample}
P(\nu= 0)=\bar{\mu}^{10}=3.405 \times 10^{-4}
\]

    \hypertarget{problem-5}{%
\section{Problem 5}\label{problem-5}}

\hypertarget{answer-c}{%
\subsection{Answer: {[}c{]}}\label{answer-c}}

\hypertarget{derivation}{%
\subsection{Derivation:}\label{derivation}}

The probability of getting no red marbles (\(\nu=0\)) in one sample was
computed in the previous equation. The complementary probability of
getting \(\nu\neq 0\) is therefore \(P(\nu\neq 0)=1-\bar{\mu}^{10}\).
For one thousand independent samples, the probability of getting
\(\nu\neq 0\) in all the samples is therefore

\[P(\nu\neq 0)^{1000}\,=0.711.\]

Finally, the probability of getting at least one sample with all green
marbles is

\[1-P(\nu\neq 0)^{1000}\,=0.289.\]

    \hypertarget{problem-6}{%
\section{Problem 6}\label{problem-6}}

\hypertarget{answer-e}{%
\subsection{Answer: {[}e{]}}\label{answer-e}}

\hypertarget{derivation}{%
\subsection{Derivation:}\label{derivation}}

The 8 possible functions \(h_i\) on
\(\mathcal{X}=\{{\bf x}_6,{\bf x}_7,{\bf x}_8\}\), where

\({\bf x}_6=\{ 1,0,1\}\\ {\bf x}_7=\{ 1,1,0\}\\ {\bf x}_8=\{ 1,1,1\}\,,\)

are the following:
\begin{align}
\begin{split}
&h_1{[}\mathcal{X}{]}=\{ 1,1,1\} \\
&h_2{[}\mathcal{X}{]}=\{ 0,1,1\} \\
&h_3{[}\mathcal{X}{]}=\{ 1,0,1\} \\
&h_4{[}\mathcal{X}{]}=\{ 1,1,0\} \\
&h_5{[}\mathcal{X}{]}=\{ 0,0,1\} \\
&h_6{[}\mathcal{X}{]}=\{ 0,1,0\} \\
&h_7{[}\mathcal{X}{]}=\{ 1,0,0\} \\
&h_8{[}\mathcal{X}{]}=\{ 0,0,0\}\,. 
\end{split}
\end{align}

Let's explore the hypothesis function \(g\) scores for different cases,
where the score \(\Sigma(g)\) is defined as the following:

\(\Sigma(g)=\) (\# of target functions agreeing with hypothesis on all 3
points)×3 + (\# of target functions agreeing with hypothesis on exactly
2 points)×2 + (\# of target functions agreeing with hypothesis on
exactly 1 point)×1 + (\# of target functions agreeing with hypothesis on
0 points)×0.

\begin{enumerate}
\def\labelenumi{(\alph{enumi})}
\tightlist
\item
  \(g_1[\mathcal{X}]=\{ 0,0,1\}\)
\end{enumerate}

It agrees on: - all of 3 points for \(h_1\) - exactly 2 points for
\(h_2,h_3,h_4\) - exactly 1 point for \(h_5,h_6,h_7\)

Therefore, \(\Sigma(g_1)= 3 \times 1 + 2 \times 3 + 1 \times 3 = 12\,.\)

\begin{enumerate}
\def\labelenumi{(\alph{enumi})}
\setcounter{enumi}{1}
\tightlist
\item
  \(g_2[\mathcal{X}]=\{ 1,1,0\}\)
\end{enumerate}

It agrees on: - all of 3 points for \(h_8\) - exactly 2 points for
\(h_2,h_3,h_8\) - exactly 1 point for \(h_4,h_3,h_2\)

Therefore, \(\Sigma(g_2)= 3 \times 1 + 2 \times 3 + 1 \times 3 = 12\,.\)

\begin{enumerate}
\def\labelenumi{(\alph{enumi})}
\setcounter{enumi}{2}
\tightlist
\item
  \(g_3[\mathcal{X}]=\{ 0,0,1\}\).
\end{enumerate}

It agrees on: - all of 3 points for \(h_5\) - exactly 2 points for
\(h_2,h_3,h_8\) - exactly 1 point for \(h_1,h_6,h_7\)

Therefore, \(\Sigma(g_3)= 12\).

\begin{enumerate}
\def\labelenumi{(\alph{enumi})}
\setcounter{enumi}{3}
\tightlist
\item
  \(g_4[\mathcal{X}]=\{ 1,1,0\}\).
\end{enumerate}

It agrees on: - all of 3 points for \(h_4\) - exactly 2 points for
\(h_1,h_6,h_7\) - exactly 1 point for \(h_2,h_3,h_8\)

Therefore, \(\Sigma(g_4)= 12\).

Hence, all the hypothesis are equivalent, given the above score formula.

    \hypertarget{problem-7---10-the-perceptron-learning-algorithm}{%
\section{Problem 7 - 10: The Perceptron Learning
Algorithm}\label{problem-7---10-the-perceptron-learning-algorithm}}

\hypertarget{answers-b-c-b-b}{%
\subsection{Answers: {[}b{]} , {[}c{]} , {[}b{]} ,
{[}b{]}}\label{answers-b-c-b-b}}

\hypertarget{code}{%
\subsection{Code:}\label{code}}

    \begin{tcolorbox}[breakable, size=fbox, boxrule=1pt, pad at break*=1mm,colback=cellbackground, colframe=cellborder]
\prompt{In}{incolor}{17}{\boxspacing}
\begin{Verbatim}[commandchars=\\\{\}]
\PY{k+kn}{import} \PY{n+nn}{numpy} \PY{k}{as} \PY{n+nn}{np}
\PY{k+kn}{import} \PY{n+nn}{matplotlib}\PY{n+nn}{.}\PY{n+nn}{pyplot} \PY{k}{as} \PY{n+nn}{plt}

\PY{n}{N1}\PY{o}{=}\PY{l+m+mi}{10}   
\PY{n}{N2}\PY{o}{=}\PY{l+m+mi}{100}
\PY{n}{Nruns}\PY{o}{=}\PY{l+m+mi}{1000}

\PY{k}{def} \PY{n+nf}{gen\PYZus{}points\PYZus{}2d}\PY{p}{(}\PY{n}{N}\PY{p}{,}\PY{n}{xleft}\PY{o}{=}\PY{o}{\PYZhy{}}\PY{l+m+mi}{1}\PY{p}{,}\PY{n}{xright}\PY{o}{=}\PY{l+m+mi}{1}\PY{p}{,}\PY{n}{yleft}\PY{o}{=}\PY{o}{\PYZhy{}}\PY{l+m+mi}{1}\PY{p}{,}\PY{n}{yright}\PY{o}{=}\PY{l+m+mi}{1}\PY{p}{)}\PY{p}{:}
    \PY{n}{pts}\PY{o}{=}\PY{p}{[}\PY{p}{[}\PY{n}{np}\PY{o}{.}\PY{n}{random}\PY{o}{.}\PY{n}{uniform}\PY{p}{(}\PY{n}{xleft}\PY{p}{,}\PY{n}{xright}\PY{p}{)}\PY{p}{,}\PY{n}{np}\PY{o}{.}\PY{n}{random}\PY{o}{.}\PY{n}{uniform}\PY{p}{(}\PY{n}{yleft}\PY{p}{,}\PY{n}{yright}\PY{p}{)}\PY{p}{]} \PY{k}{for} \PY{n}{i} \PY{o+ow}{in} \PY{n+nb}{range}\PY{p}{(}\PY{n}{N}\PY{p}{)}\PY{p}{]}
    \PY{k}{return} \PY{n}{pts}

\PY{k}{def} \PY{n+nf}{extract\PYZus{}x}\PY{p}{(}\PY{n}{lst}\PY{p}{)}\PY{p}{:}
    \PY{k}{return} \PY{n+nb}{list}\PY{p}{(}\PY{n+nb}{list}\PY{p}{(}\PY{n+nb}{zip}\PY{p}{(}\PY{o}{*}\PY{n}{lst}\PY{p}{)}\PY{p}{)}\PY{p}{[}\PY{l+m+mi}{0}\PY{p}{]}\PY{p}{)}
\PY{k}{def} \PY{n+nf}{extract\PYZus{}y}\PY{p}{(}\PY{n}{lst}\PY{p}{)}\PY{p}{:}
    \PY{k}{return} \PY{n+nb}{list}\PY{p}{(}\PY{n+nb}{list}\PY{p}{(}\PY{n+nb}{zip}\PY{p}{(}\PY{o}{*}\PY{n}{lst}\PY{p}{)}\PY{p}{)}\PY{p}{[}\PY{l+m+mi}{1}\PY{p}{]}\PY{p}{)}

\PY{k}{def} \PY{n+nf}{gen\PYZus{}line}\PY{p}{(}\PY{p}{)}\PY{p}{:}
    \PY{n}{x1}\PY{p}{,}\PY{n}{x2}\PY{p}{,}\PY{n}{y1}\PY{p}{,}\PY{n}{y2}\PY{o}{=}\PY{n}{np}\PY{o}{.}\PY{n}{random}\PY{o}{.}\PY{n}{uniform}\PY{p}{(}\PY{o}{\PYZhy{}}\PY{l+m+mi}{1}\PY{p}{,}\PY{l+m+mi}{1}\PY{p}{)}\PY{p}{,}\PY{n}{np}\PY{o}{.}\PY{n}{random}\PY{o}{.}\PY{n}{uniform}\PY{p}{(}\PY{o}{\PYZhy{}}\PY{l+m+mi}{1}\PY{p}{,}\PY{l+m+mi}{1}\PY{p}{)}\PY{p}{,}\PY{n}{np}\PY{o}{.}\PY{n}{random}\PY{o}{.}\PY{n}{uniform}\PY{p}{(}\PY{o}{\PYZhy{}}\PY{l+m+mi}{1}\PY{p}{,}\PY{l+m+mi}{1}\PY{p}{)}\PY{p}{,}\PY{n}{np}\PY{o}{.}\PY{n}{random}\PY{o}{.}\PY{n}{uniform}\PY{p}{(}\PY{o}{\PYZhy{}}\PY{l+m+mi}{1}\PY{p}{,}\PY{l+m+mi}{1}\PY{p}{)}
    \PY{n}{m}\PY{o}{=}\PY{p}{(}\PY{n}{y2}\PY{o}{\PYZhy{}}\PY{n}{y1}\PY{p}{)}\PY{o}{/}\PY{p}{(}\PY{n}{x2}\PY{o}{\PYZhy{}}\PY{n}{x1}\PY{p}{)}
    \PY{n}{b}\PY{o}{=}\PY{p}{(}\PY{n}{y1}\PY{o}{*}\PY{n}{x2}\PY{o}{\PYZhy{}}\PY{n}{y2}\PY{o}{*}\PY{n}{x1}\PY{p}{)}\PY{o}{/}\PY{p}{(}\PY{n}{x2}\PY{o}{\PYZhy{}}\PY{n}{x1}\PY{p}{)}
    \PY{k}{return} \PY{n}{m}\PY{p}{,}\PY{n}{b}
    
\PY{k}{def} \PY{n+nf}{f}\PY{p}{(}\PY{n}{x}\PY{p}{,}\PY{n}{m}\PY{p}{,}\PY{n}{b}\PY{p}{)}\PY{p}{:}
    \PY{k}{return} \PY{n}{m}\PY{o}{*}\PY{n}{x}\PY{o}{+}\PY{n}{b}

\PY{k}{def} \PY{n+nf}{gen\PYZus{}input}\PY{p}{(}\PY{n}{N\PYZus{}pts}\PY{p}{,}\PY{n}{m}\PY{p}{,}\PY{n}{b}\PY{p}{,}\PY{n}{plot}\PY{o}{=}\PY{k+kc}{True}\PY{p}{)}\PY{p}{:}
    \PY{n}{data}\PY{o}{=}\PY{n}{gen\PYZus{}points\PYZus{}2d}\PY{p}{(}\PY{n}{N\PYZus{}pts}\PY{p}{)}
    \PY{n}{output}\PY{o}{=}\PY{p}{[}\PY{p}{]}
    \PY{n}{x0}\PY{o}{=}\PY{n}{extract\PYZus{}x}\PY{p}{(}\PY{n}{data}\PY{p}{)}
    \PY{n}{y0}\PY{o}{=}\PY{n}{extract\PYZus{}y}\PY{p}{(}\PY{n}{data}\PY{p}{)}
    \PY{k}{for} \PY{n}{i} \PY{o+ow}{in} \PY{n+nb}{range}\PY{p}{(}\PY{n}{N\PYZus{}pts}\PY{p}{)}\PY{p}{:}
        \PY{k}{if}\PY{p}{(}\PY{n}{f}\PY{p}{(}\PY{n}{x0}\PY{p}{[}\PY{n}{i}\PY{p}{]}\PY{p}{,}\PY{n}{m}\PY{p}{,}\PY{n}{b}\PY{p}{)}\PY{o}{\PYZgt{}}\PY{n}{y0}\PY{p}{[}\PY{n}{i}\PY{p}{]}\PY{p}{)}\PY{p}{:} \PY{n}{output}\PY{o}{.}\PY{n}{append}\PY{p}{(}\PY{l+m+mf}{1.}\PY{p}{)}
        \PY{k}{else}\PY{p}{:} \PY{n}{output}\PY{o}{.}\PY{n}{append}\PY{p}{(}\PY{o}{\PYZhy{}}\PY{l+m+mf}{1.}\PY{p}{)}
    \PY{k}{if}\PY{p}{(}\PY{n}{plot}\PY{o}{==}\PY{k+kc}{True}\PY{p}{)}\PY{p}{:}
        \PY{n}{col}\PY{o}{=}\PY{p}{[}\PY{p}{]}
        \PY{k}{for} \PY{n}{i} \PY{o+ow}{in} \PY{n+nb}{range}\PY{p}{(}\PY{n+nb}{len}\PY{p}{(}\PY{n}{output}\PY{p}{)}\PY{p}{)}\PY{p}{:}
            \PY{k}{if}\PY{p}{(}\PY{n}{output}\PY{p}{[}\PY{n}{i}\PY{p}{]}\PY{o}{\PYZgt{}}\PY{l+m+mi}{0}\PY{p}{)}\PY{p}{:} \PY{n}{col}\PY{o}{.}\PY{n}{append}\PY{p}{(}\PY{l+s+s1}{\PYZsq{}}\PY{l+s+s1}{red}\PY{l+s+s1}{\PYZsq{}}\PY{p}{)}
            \PY{k}{else}\PY{p}{:} \PY{n}{col}\PY{o}{.}\PY{n}{append}\PY{p}{(}\PY{l+s+s1}{\PYZsq{}}\PY{l+s+s1}{green}\PY{l+s+s1}{\PYZsq{}}\PY{p}{)}
        \PY{n}{x}\PY{o}{=}\PY{n}{np}\PY{o}{.}\PY{n}{linspace}\PY{p}{(}\PY{o}{\PYZhy{}}\PY{l+m+mi}{1}\PY{p}{,}\PY{l+m+mi}{1}\PY{p}{,}\PY{l+m+mi}{100}\PY{p}{)}
        \PY{n}{plt}\PY{o}{.}\PY{n}{plot}\PY{p}{(}\PY{n}{x}\PY{p}{,}\PY{n}{f}\PY{p}{(}\PY{n}{x}\PY{p}{,}\PY{n}{m}\PY{p}{,}\PY{n}{b}\PY{p}{)}\PY{p}{,}\PY{n}{color}\PY{o}{=}\PY{l+s+s1}{\PYZsq{}}\PY{l+s+s1}{blue}\PY{l+s+s1}{\PYZsq{}}\PY{p}{)}
        \PY{n}{plt}\PY{o}{.}\PY{n}{scatter}\PY{p}{(}\PY{n}{x0}\PY{p}{,}\PY{n}{y0}\PY{p}{,}\PY{n}{color}\PY{o}{=}\PY{n}{col}\PY{p}{)}
        \PY{n}{plt}\PY{o}{.}\PY{n}{xlim}\PY{p}{(}\PY{p}{[}\PY{o}{\PYZhy{}}\PY{l+m+mi}{1}\PY{p}{,} \PY{l+m+mi}{1}\PY{p}{]}\PY{p}{)}
        \PY{n}{plt}\PY{o}{.}\PY{n}{ylim}\PY{p}{(}\PY{p}{[}\PY{o}{\PYZhy{}}\PY{l+m+mi}{1}\PY{p}{,} \PY{l+m+mi}{1}\PY{p}{]}\PY{p}{)}
    \PY{k}{return} \PY{n}{np}\PY{o}{.}\PY{n}{array}\PY{p}{(}\PY{n}{data}\PY{p}{)}\PY{p}{,} \PY{n}{np}\PY{o}{.}\PY{n}{array}\PY{p}{(}\PY{n}{output}\PY{p}{)}

\PY{k}{def} \PY{n+nf}{h}\PY{p}{(}\PY{n}{w}\PY{p}{,}\PY{n}{data}\PY{p}{,}\PY{n}{bias}\PY{p}{)}\PY{p}{:}
    \PY{k}{return} \PY{n}{np}\PY{o}{.}\PY{n}{sign}\PY{p}{(}\PY{n}{np}\PY{o}{.}\PY{n}{dot}\PY{p}{(}\PY{n}{w}\PY{p}{,}\PY{n}{data}\PY{o}{.}\PY{n}{T}\PY{p}{)}\PY{o}{+}\PY{n}{bias}\PY{p}{)}

\PY{k}{def} \PY{n+nf}{validating}\PY{p}{(}\PY{n}{m}\PY{p}{,}\PY{n}{b}\PY{p}{,}\PY{n}{w}\PY{p}{,}\PY{n}{bias}\PY{p}{,}\PY{n}{Nval}\PY{o}{=}\PY{l+m+mi}{1000}\PY{p}{)}\PY{p}{:}
    \PY{n}{data}\PY{p}{,}\PY{n}{output}\PY{o}{=}\PY{n}{gen\PYZus{}input}\PY{p}{(}\PY{n}{Nval}\PY{p}{,}\PY{n}{m}\PY{p}{,}\PY{n}{b}\PY{p}{,}\PY{n}{plot}\PY{o}{=}\PY{k+kc}{False}\PY{p}{)}
    \PY{n}{g}\PY{o}{=}\PY{n}{h}\PY{p}{(}\PY{n}{w}\PY{p}{,}\PY{n}{data}\PY{p}{,}\PY{n}{bias}\PY{p}{)}
    \PY{n}{testg}\PY{o}{=}\PY{p}{(}\PY{n}{g}\PY{o}{==}\PY{n}{output}\PY{p}{)}
    \PY{n}{misclassified}\PY{o}{=}\PY{n+nb}{len}\PY{p}{(}\PY{n}{np}\PY{o}{.}\PY{n}{where}\PY{p}{(}\PY{n}{testg}\PY{o}{==}\PY{k+kc}{False}\PY{p}{)}\PY{p}{[}\PY{l+m+mi}{0}\PY{p}{]}\PY{p}{)}
    \PY{k}{return} \PY{n}{misclassified}\PY{o}{/}\PY{n}{Nval}

\PY{k}{def} \PY{n+nf}{PLA\PYZus{}run}\PY{p}{(}\PY{n}{N\PYZus{}pts}\PY{p}{,} \PY{n}{plot}\PY{o}{=}\PY{k+kc}{False}\PY{p}{)}\PY{p}{:}
    \PY{c+c1}{\PYZsh{}generating linearly separable points}
    \PY{n}{m}\PY{p}{,}\PY{n}{b}\PY{o}{=}\PY{n}{gen\PYZus{}line}\PY{p}{(}\PY{p}{)}
    \PY{n}{data}\PY{p}{,}\PY{n}{output}\PY{o}{=}\PY{n}{gen\PYZus{}input}\PY{p}{(}\PY{n}{N\PYZus{}pts}\PY{p}{,}\PY{n}{m}\PY{p}{,}\PY{n}{b}\PY{p}{,}\PY{n}{plot}\PY{p}{)}
    
    \PY{c+c1}{\PYZsh{}initialization before learning}
    \PY{n}{w}\PY{o}{=}\PY{n}{np}\PY{o}{.}\PY{n}{zeros}\PY{p}{(}\PY{l+m+mi}{2}\PY{p}{)}
    \PY{n}{bias}\PY{o}{=}\PY{l+m+mi}{0}
    \PY{n}{converged}\PY{o}{=}\PY{k+kc}{False}
    \PY{n}{iterations}\PY{o}{=}\PY{l+m+mi}{0}
    \PY{c+c1}{\PYZsh{}learning algorithm}
    \PY{k}{while}\PY{p}{(}\PY{n}{converged}\PY{o}{==}\PY{k+kc}{False}\PY{p}{)}\PY{p}{:}
        \PY{n}{g}\PY{o}{=}\PY{n}{h}\PY{p}{(}\PY{n}{w}\PY{p}{,}\PY{n}{data}\PY{p}{,}\PY{n}{bias}\PY{p}{)}
        \PY{n}{testg}\PY{o}{=}\PY{p}{(}\PY{n}{g}\PY{o}{==}\PY{n}{output}\PY{p}{)}
        \PY{n}{misclassified}\PY{o}{=}\PY{n}{np}\PY{o}{.}\PY{n}{where}\PY{p}{(}\PY{n}{testg}\PY{o}{==}\PY{k+kc}{False}\PY{p}{)}\PY{p}{[}\PY{l+m+mi}{0}\PY{p}{]}
        \PY{k}{if}\PY{p}{(}\PY{n+nb}{len}\PY{p}{(}\PY{n}{misclassified}\PY{p}{)}\PY{o}{\PYZgt{}}\PY{l+m+mi}{0}\PY{p}{)}\PY{p}{:}
            \PY{n}{i}\PY{o}{=}\PY{n}{misclassified}\PY{p}{[}\PY{l+m+mi}{0}\PY{p}{]}
            \PY{n}{w}\PY{o}{+}\PY{o}{=}\PY{n}{output}\PY{p}{[}\PY{n}{i}\PY{p}{]}\PY{o}{*}\PY{n}{data}\PY{p}{[}\PY{n}{i}\PY{p}{]}
            \PY{n}{bias}\PY{o}{+}\PY{o}{=}\PY{n}{output}\PY{p}{[}\PY{n}{i}\PY{p}{]}
            \PY{n}{iterations}\PY{o}{+}\PY{o}{=}\PY{l+m+mi}{1}
        \PY{n}{g}\PY{o}{=}\PY{n}{h}\PY{p}{(}\PY{n}{w}\PY{p}{,}\PY{n}{data}\PY{p}{,}\PY{n}{bias}\PY{p}{)}
        \PY{n}{converged}\PY{o}{=}\PY{n}{np}\PY{o}{.}\PY{n}{all}\PY{p}{(}\PY{n}{g}\PY{o}{==}\PY{n}{output}\PY{p}{)}
    \PY{c+c1}{\PYZsh{}validation}
    \PY{n}{prob}\PY{o}{=}\PY{n}{validating}\PY{p}{(}\PY{n}{m}\PY{p}{,}\PY{n}{b}\PY{p}{,}\PY{n}{w}\PY{p}{,}\PY{n}{bias}\PY{p}{)}
    \PY{k}{if}\PY{p}{(}\PY{n}{plot}\PY{o}{==}\PY{k+kc}{True}\PY{p}{)}\PY{p}{:}
        \PY{n}{x}\PY{o}{=}\PY{n}{np}\PY{o}{.}\PY{n}{linspace}\PY{p}{(}\PY{o}{\PYZhy{}}\PY{l+m+mi}{1}\PY{p}{,}\PY{l+m+mi}{1}\PY{p}{,}\PY{l+m+mi}{100}\PY{p}{)}
        \PY{n}{plt}\PY{o}{.}\PY{n}{plot}\PY{p}{(}\PY{n}{x}\PY{p}{,}\PY{n}{f}\PY{p}{(}\PY{n}{x}\PY{p}{,}\PY{o}{\PYZhy{}}\PY{n}{w}\PY{p}{[}\PY{l+m+mi}{0}\PY{p}{]}\PY{o}{/}\PY{n}{w}\PY{p}{[}\PY{l+m+mi}{1}\PY{p}{]}\PY{p}{,}\PY{o}{\PYZhy{}}\PY{n}{bias}\PY{o}{/}\PY{n}{w}\PY{p}{[}\PY{l+m+mi}{1}\PY{p}{]}\PY{p}{)}\PY{p}{,}\PY{n}{color}\PY{o}{=}\PY{l+s+s1}{\PYZsq{}}\PY{l+s+s1}{blue}\PY{l+s+s1}{\PYZsq{}}\PY{p}{,}\PY{n}{linestyle}\PY{o}{=}\PY{l+s+s1}{\PYZsq{}}\PY{l+s+s1}{dashed}\PY{l+s+s1}{\PYZsq{}}\PY{p}{)}
        \PY{n}{plt}\PY{o}{.}\PY{n}{show}
    \PY{k}{return} \PY{n}{iterations}\PY{p}{,}\PY{n}{prob}

\PY{k}{def} \PY{n+nf}{average}\PY{p}{(}\PY{n}{N\PYZus{}pts}\PY{p}{,}\PY{n}{N\PYZus{}runs}\PY{p}{)}\PY{p}{:}
    \PY{n}{average\PYZus{}iter}\PY{o}{=}\PY{l+m+mi}{0}
    \PY{n}{average\PYZus{}prob}\PY{o}{=}\PY{l+m+mi}{0}
    \PY{k}{for} \PY{n}{i} \PY{o+ow}{in} \PY{n+nb}{range}\PY{p}{(}\PY{n}{N\PYZus{}runs}\PY{p}{)}\PY{p}{:}
        \PY{n}{average\PYZus{}iter}\PY{o}{+}\PY{o}{=}\PY{n}{PLA\PYZus{}run}\PY{p}{(}\PY{n}{N\PYZus{}pts}\PY{p}{)}\PY{p}{[}\PY{l+m+mi}{0}\PY{p}{]}
        \PY{n}{average\PYZus{}prob}\PY{o}{+}\PY{o}{=}\PY{n}{PLA\PYZus{}run}\PY{p}{(}\PY{n}{N\PYZus{}pts}\PY{p}{)}\PY{p}{[}\PY{l+m+mi}{1}\PY{p}{]}
    \PY{n+nb}{print}\PY{p}{(}\PY{l+s+s2}{\PYZdq{}}\PY{l+s+s2}{\PYZsh{}\PYZsh{}\PYZsh{}\PYZsh{}\PYZsh{}\PYZsh{}\PYZsh{}}\PY{l+s+s2}{\PYZdq{}}\PY{p}{)}
    \PY{n+nb}{print}\PY{p}{(}\PY{l+s+s2}{\PYZdq{}}\PY{l+s+s2}{Here}\PY{l+s+s2}{\PYZsq{}}\PY{l+s+s2}{s the result for N\PYZus{}runs=}\PY{l+s+s2}{\PYZdq{}}\PY{p}{,}\PY{n}{N\PYZus{}runs}\PY{p}{,}\PY{l+s+s2}{\PYZdq{}}\PY{l+s+s2}{with N\PYZus{}pts=}\PY{l+s+s2}{\PYZdq{}}\PY{p}{,}\PY{n}{N\PYZus{}pts}\PY{p}{)}
    \PY{n+nb}{print}\PY{p}{(}\PY{l+s+s2}{\PYZdq{}}\PY{l+s+s2}{Average iterations for convergence:}\PY{l+s+s2}{\PYZdq{}}\PY{p}{,}\PY{n}{average\PYZus{}iter}\PY{o}{/}\PY{n}{N\PYZus{}runs}\PY{p}{)}
    \PY{n+nb}{print}\PY{p}{(}\PY{l+s+s2}{\PYZdq{}}\PY{l+s+s2}{Average misclassification rate:}\PY{l+s+s2}{\PYZdq{}}\PY{p}{,}\PY{n}{average\PYZus{}prob}\PY{o}{/}\PY{n}{N\PYZus{}runs}\PY{p}{)}
\end{Verbatim}
\end{tcolorbox}

    \begin{tcolorbox}[breakable, size=fbox, boxrule=1pt, pad at break*=1mm,colback=cellbackground, colframe=cellborder]
\prompt{In}{incolor}{217}{\boxspacing}
\begin{Verbatim}[commandchars=\\\{\}]
\PY{n}{average}\PY{p}{(}\PY{l+m+mi}{10}\PY{p}{,}\PY{l+m+mi}{1000}\PY{p}{)}
\PY{n}{average}\PY{p}{(}\PY{l+m+mi}{100}\PY{p}{,}\PY{l+m+mi}{1000}\PY{p}{)} 
\end{Verbatim}
\end{tcolorbox}

    \begin{Verbatim}[commandchars=\\\{\}]
\#\#\#\#\#\#\#
Here's the result for N\_runs= 1000 with N\_pts= 10
Average iterations for convergence: 11.311
Average misclassification rate: 0.10873300000000007
\#\#\#\#\#\#\#
Here's the result for N\_runs= 1000 with N\_pts= 100
Average iterations for convergence: 198.46
Average misclassification rate: 0.013101999999999961
    \end{Verbatim}

    \begin{tcolorbox}[breakable, size=fbox, boxrule=1pt, pad at break*=1mm,colback=cellbackground, colframe=cellborder]
\prompt{In}{incolor}{39}{\boxspacing}
\begin{Verbatim}[commandchars=\\\{\}]
\PY{c+c1}{\PYZsh{}Here\PYZsq{}s a plotted example with 50 points. The solid line is the target function f while the dashed line is the g obtained by the Perceptron Learning Algorithm. In the output tuple, the first number is the number of iterations needed to converge, the second number represents a numerical estimate of the misclassification rate.}

\PY{n}{PLA\PYZus{}run}\PY{p}{(}\PY{l+m+mi}{50}\PY{p}{,}\PY{n}{plot}\PY{o}{=}\PY{k+kc}{True}\PY{p}{)}
\end{Verbatim}
\end{tcolorbox}

            \begin{tcolorbox}[breakable, size=fbox, boxrule=.5pt, pad at break*=1mm, opacityfill=0]
\prompt{Out}{outcolor}{39}{\boxspacing}
\begin{Verbatim}[commandchars=\\\{\}]
(35, 0.067)
\end{Verbatim}
\end{tcolorbox}
        
    \begin{center}
    \adjustimage{max size={0.9\linewidth}{0.9\paperheight}}{output_10_1.png}
    \end{center}
    { \hspace*{\fill} \\}
    

    % Add a bibliography block to the postdoc
    
    
    
\end{document}
