\documentclass[11pt]{article}

    \usepackage[breakable]{tcolorbox}
    \usepackage{parskip} % Stop auto-indenting (to mimic markdown behaviour)
    
    \usepackage{iftex}
    \ifPDFTeX
    	\usepackage[T1]{fontenc}
    	\usepackage{mathpazo}
    \else
    	\usepackage{fontspec}
    \fi

    % Basic figure setup, for now with no caption control since it's done
    % automatically by Pandoc (which extracts ![](path) syntax from Markdown).
    \usepackage{graphicx}
    % Maintain compatibility with old templates. Remove in nbconvert 6.0
    \let\Oldincludegraphics\includegraphics
    % Ensure that by default, figures have no caption (until we provide a
    % proper Figure object with a Caption API and a way to capture that
    % in the conversion process - todo).
    \usepackage{caption}
    \DeclareCaptionFormat{nocaption}{}
    \captionsetup{format=nocaption,aboveskip=0pt,belowskip=0pt}

    \usepackage[Export]{adjustbox} % Used to constrain images to a maximum size
    \adjustboxset{max size={0.9\linewidth}{0.9\paperheight}}
    \usepackage{float}
    \floatplacement{figure}{H} % forces figures to be placed at the correct location
    \usepackage{xcolor} % Allow colors to be defined
    \usepackage{enumerate} % Needed for markdown enumerations to work
    \usepackage{geometry} % Used to adjust the document margins
    \usepackage{amsmath} % Equations
    \usepackage{amssymb} % Equations
    \usepackage{textcomp} % defines textquotesingle
    % Hack from http://tex.stackexchange.com/a/47451/13684:
    \AtBeginDocument{%
        \def\PYZsq{\textquotesingle}% Upright quotes in Pygmentized code
    }
    \usepackage{upquote} % Upright quotes for verbatim code
    \usepackage{eurosym} % defines \euro
    \usepackage[mathletters]{ucs} % Extended unicode (utf-8) support
    \usepackage{fancyvrb} % verbatim replacement that allows latex
    \usepackage{grffile} % extends the file name processing of package graphics 
                         % to support a larger range
    \makeatletter % fix for grffile with XeLaTeX
    \def\Gread@@xetex#1{%
      \IfFileExists{"\Gin@base".bb}%
      {\Gread@eps{\Gin@base.bb}}%
      {\Gread@@xetex@aux#1}%
    }
    \makeatother

    % The hyperref package gives us a pdf with properly built
    % internal navigation ('pdf bookmarks' for the table of contents,
    % internal cross-reference links, web links for URLs, etc.)
    \usepackage{hyperref}
    % The default LaTeX title has an obnoxious amount of whitespace. By default,
    % titling removes some of it. It also provides customization options.
    \usepackage{titling}
    \usepackage{longtable} % longtable support required by pandoc >1.10
    \usepackage{booktabs}  % table support for pandoc > 1.12.2
    \usepackage[inline]{enumitem} % IRkernel/repr support (it uses the enumerate* environment)
    \usepackage[normalem]{ulem} % ulem is needed to support strikethroughs (\sout)
                                % normalem makes italics be italics, not underlines
    \usepackage{mathrsfs}
    

    
    % Colors for the hyperref package
    \definecolor{urlcolor}{rgb}{0,.145,.698}
    \definecolor{linkcolor}{rgb}{.71,0.21,0.01}
    \definecolor{citecolor}{rgb}{.12,.54,.11}

    % ANSI colors
    \definecolor{ansi-black}{HTML}{3E424D}
    \definecolor{ansi-black-intense}{HTML}{282C36}
    \definecolor{ansi-red}{HTML}{E75C58}
    \definecolor{ansi-red-intense}{HTML}{B22B31}
    \definecolor{ansi-green}{HTML}{00A250}
    \definecolor{ansi-green-intense}{HTML}{007427}
    \definecolor{ansi-yellow}{HTML}{DDB62B}
    \definecolor{ansi-yellow-intense}{HTML}{B27D12}
    \definecolor{ansi-blue}{HTML}{208FFB}
    \definecolor{ansi-blue-intense}{HTML}{0065CA}
    \definecolor{ansi-magenta}{HTML}{D160C4}
    \definecolor{ansi-magenta-intense}{HTML}{A03196}
    \definecolor{ansi-cyan}{HTML}{60C6C8}
    \definecolor{ansi-cyan-intense}{HTML}{258F8F}
    \definecolor{ansi-white}{HTML}{C5C1B4}
    \definecolor{ansi-white-intense}{HTML}{A1A6B2}
    \definecolor{ansi-default-inverse-fg}{HTML}{FFFFFF}
    \definecolor{ansi-default-inverse-bg}{HTML}{000000}

    % commands and environments needed by pandoc snippets
    % extracted from the output of `pandoc -s`
    \providecommand{\tightlist}{%
      \setlength{\itemsep}{0pt}\setlength{\parskip}{0pt}}
    \DefineVerbatimEnvironment{Highlighting}{Verbatim}{commandchars=\\\{\}}
    % Add ',fontsize=\small' for more characters per line
    \newenvironment{Shaded}{}{}
    \newcommand{\KeywordTok}[1]{\textcolor[rgb]{0.00,0.44,0.13}{\textbf{{#1}}}}
    \newcommand{\DataTypeTok}[1]{\textcolor[rgb]{0.56,0.13,0.00}{{#1}}}
    \newcommand{\DecValTok}[1]{\textcolor[rgb]{0.25,0.63,0.44}{{#1}}}
    \newcommand{\BaseNTok}[1]{\textcolor[rgb]{0.25,0.63,0.44}{{#1}}}
    \newcommand{\FloatTok}[1]{\textcolor[rgb]{0.25,0.63,0.44}{{#1}}}
    \newcommand{\CharTok}[1]{\textcolor[rgb]{0.25,0.44,0.63}{{#1}}}
    \newcommand{\StringTok}[1]{\textcolor[rgb]{0.25,0.44,0.63}{{#1}}}
    \newcommand{\CommentTok}[1]{\textcolor[rgb]{0.38,0.63,0.69}{\textit{{#1}}}}
    \newcommand{\OtherTok}[1]{\textcolor[rgb]{0.00,0.44,0.13}{{#1}}}
    \newcommand{\AlertTok}[1]{\textcolor[rgb]{1.00,0.00,0.00}{\textbf{{#1}}}}
    \newcommand{\FunctionTok}[1]{\textcolor[rgb]{0.02,0.16,0.49}{{#1}}}
    \newcommand{\RegionMarkerTok}[1]{{#1}}
    \newcommand{\ErrorTok}[1]{\textcolor[rgb]{1.00,0.00,0.00}{\textbf{{#1}}}}
    \newcommand{\NormalTok}[1]{{#1}}
    
    % Additional commands for more recent versions of Pandoc
    \newcommand{\ConstantTok}[1]{\textcolor[rgb]{0.53,0.00,0.00}{{#1}}}
    \newcommand{\SpecialCharTok}[1]{\textcolor[rgb]{0.25,0.44,0.63}{{#1}}}
    \newcommand{\VerbatimStringTok}[1]{\textcolor[rgb]{0.25,0.44,0.63}{{#1}}}
    \newcommand{\SpecialStringTok}[1]{\textcolor[rgb]{0.73,0.40,0.53}{{#1}}}
    \newcommand{\ImportTok}[1]{{#1}}
    \newcommand{\DocumentationTok}[1]{\textcolor[rgb]{0.73,0.13,0.13}{\textit{{#1}}}}
    \newcommand{\AnnotationTok}[1]{\textcolor[rgb]{0.38,0.63,0.69}{\textbf{\textit{{#1}}}}}
    \newcommand{\CommentVarTok}[1]{\textcolor[rgb]{0.38,0.63,0.69}{\textbf{\textit{{#1}}}}}
    \newcommand{\VariableTok}[1]{\textcolor[rgb]{0.10,0.09,0.49}{{#1}}}
    \newcommand{\ControlFlowTok}[1]{\textcolor[rgb]{0.00,0.44,0.13}{\textbf{{#1}}}}
    \newcommand{\OperatorTok}[1]{\textcolor[rgb]{0.40,0.40,0.40}{{#1}}}
    \newcommand{\BuiltInTok}[1]{{#1}}
    \newcommand{\ExtensionTok}[1]{{#1}}
    \newcommand{\PreprocessorTok}[1]{\textcolor[rgb]{0.74,0.48,0.00}{{#1}}}
    \newcommand{\AttributeTok}[1]{\textcolor[rgb]{0.49,0.56,0.16}{{#1}}}
    \newcommand{\InformationTok}[1]{\textcolor[rgb]{0.38,0.63,0.69}{\textbf{\textit{{#1}}}}}
    \newcommand{\WarningTok}[1]{\textcolor[rgb]{0.38,0.63,0.69}{\textbf{\textit{{#1}}}}}
    
    
    % Define a nice break command that doesn't care if a line doesn't already
    % exist.
    \def\br{\hspace*{\fill} \\* }
    % Math Jax compatibility definitions
    \def\gt{>}
    \def\lt{<}
    \let\Oldtex\TeX
    \let\Oldlatex\LaTeX
    \renewcommand{\TeX}{\textrm{\Oldtex}}
    \renewcommand{\LaTeX}{\textrm{\Oldlatex}}
    % Document parameters
    % Document title
    \title{Pset\_6}
    
    
    
    
    
% Pygments definitions
\makeatletter
\def\PY@reset{\let\PY@it=\relax \let\PY@bf=\relax%
    \let\PY@ul=\relax \let\PY@tc=\relax%
    \let\PY@bc=\relax \let\PY@ff=\relax}
\def\PY@tok#1{\csname PY@tok@#1\endcsname}
\def\PY@toks#1+{\ifx\relax#1\empty\else%
    \PY@tok{#1}\expandafter\PY@toks\fi}
\def\PY@do#1{\PY@bc{\PY@tc{\PY@ul{%
    \PY@it{\PY@bf{\PY@ff{#1}}}}}}}
\def\PY#1#2{\PY@reset\PY@toks#1+\relax+\PY@do{#2}}

\expandafter\def\csname PY@tok@w\endcsname{\def\PY@tc##1{\textcolor[rgb]{0.73,0.73,0.73}{##1}}}
\expandafter\def\csname PY@tok@c\endcsname{\let\PY@it=\textit\def\PY@tc##1{\textcolor[rgb]{0.25,0.50,0.50}{##1}}}
\expandafter\def\csname PY@tok@cp\endcsname{\def\PY@tc##1{\textcolor[rgb]{0.74,0.48,0.00}{##1}}}
\expandafter\def\csname PY@tok@k\endcsname{\let\PY@bf=\textbf\def\PY@tc##1{\textcolor[rgb]{0.00,0.50,0.00}{##1}}}
\expandafter\def\csname PY@tok@kp\endcsname{\def\PY@tc##1{\textcolor[rgb]{0.00,0.50,0.00}{##1}}}
\expandafter\def\csname PY@tok@kt\endcsname{\def\PY@tc##1{\textcolor[rgb]{0.69,0.00,0.25}{##1}}}
\expandafter\def\csname PY@tok@o\endcsname{\def\PY@tc##1{\textcolor[rgb]{0.40,0.40,0.40}{##1}}}
\expandafter\def\csname PY@tok@ow\endcsname{\let\PY@bf=\textbf\def\PY@tc##1{\textcolor[rgb]{0.67,0.13,1.00}{##1}}}
\expandafter\def\csname PY@tok@nb\endcsname{\def\PY@tc##1{\textcolor[rgb]{0.00,0.50,0.00}{##1}}}
\expandafter\def\csname PY@tok@nf\endcsname{\def\PY@tc##1{\textcolor[rgb]{0.00,0.00,1.00}{##1}}}
\expandafter\def\csname PY@tok@nc\endcsname{\let\PY@bf=\textbf\def\PY@tc##1{\textcolor[rgb]{0.00,0.00,1.00}{##1}}}
\expandafter\def\csname PY@tok@nn\endcsname{\let\PY@bf=\textbf\def\PY@tc##1{\textcolor[rgb]{0.00,0.00,1.00}{##1}}}
\expandafter\def\csname PY@tok@ne\endcsname{\let\PY@bf=\textbf\def\PY@tc##1{\textcolor[rgb]{0.82,0.25,0.23}{##1}}}
\expandafter\def\csname PY@tok@nv\endcsname{\def\PY@tc##1{\textcolor[rgb]{0.10,0.09,0.49}{##1}}}
\expandafter\def\csname PY@tok@no\endcsname{\def\PY@tc##1{\textcolor[rgb]{0.53,0.00,0.00}{##1}}}
\expandafter\def\csname PY@tok@nl\endcsname{\def\PY@tc##1{\textcolor[rgb]{0.63,0.63,0.00}{##1}}}
\expandafter\def\csname PY@tok@ni\endcsname{\let\PY@bf=\textbf\def\PY@tc##1{\textcolor[rgb]{0.60,0.60,0.60}{##1}}}
\expandafter\def\csname PY@tok@na\endcsname{\def\PY@tc##1{\textcolor[rgb]{0.49,0.56,0.16}{##1}}}
\expandafter\def\csname PY@tok@nt\endcsname{\let\PY@bf=\textbf\def\PY@tc##1{\textcolor[rgb]{0.00,0.50,0.00}{##1}}}
\expandafter\def\csname PY@tok@nd\endcsname{\def\PY@tc##1{\textcolor[rgb]{0.67,0.13,1.00}{##1}}}
\expandafter\def\csname PY@tok@s\endcsname{\def\PY@tc##1{\textcolor[rgb]{0.73,0.13,0.13}{##1}}}
\expandafter\def\csname PY@tok@sd\endcsname{\let\PY@it=\textit\def\PY@tc##1{\textcolor[rgb]{0.73,0.13,0.13}{##1}}}
\expandafter\def\csname PY@tok@si\endcsname{\let\PY@bf=\textbf\def\PY@tc##1{\textcolor[rgb]{0.73,0.40,0.53}{##1}}}
\expandafter\def\csname PY@tok@se\endcsname{\let\PY@bf=\textbf\def\PY@tc##1{\textcolor[rgb]{0.73,0.40,0.13}{##1}}}
\expandafter\def\csname PY@tok@sr\endcsname{\def\PY@tc##1{\textcolor[rgb]{0.73,0.40,0.53}{##1}}}
\expandafter\def\csname PY@tok@ss\endcsname{\def\PY@tc##1{\textcolor[rgb]{0.10,0.09,0.49}{##1}}}
\expandafter\def\csname PY@tok@sx\endcsname{\def\PY@tc##1{\textcolor[rgb]{0.00,0.50,0.00}{##1}}}
\expandafter\def\csname PY@tok@m\endcsname{\def\PY@tc##1{\textcolor[rgb]{0.40,0.40,0.40}{##1}}}
\expandafter\def\csname PY@tok@gh\endcsname{\let\PY@bf=\textbf\def\PY@tc##1{\textcolor[rgb]{0.00,0.00,0.50}{##1}}}
\expandafter\def\csname PY@tok@gu\endcsname{\let\PY@bf=\textbf\def\PY@tc##1{\textcolor[rgb]{0.50,0.00,0.50}{##1}}}
\expandafter\def\csname PY@tok@gd\endcsname{\def\PY@tc##1{\textcolor[rgb]{0.63,0.00,0.00}{##1}}}
\expandafter\def\csname PY@tok@gi\endcsname{\def\PY@tc##1{\textcolor[rgb]{0.00,0.63,0.00}{##1}}}
\expandafter\def\csname PY@tok@gr\endcsname{\def\PY@tc##1{\textcolor[rgb]{1.00,0.00,0.00}{##1}}}
\expandafter\def\csname PY@tok@ge\endcsname{\let\PY@it=\textit}
\expandafter\def\csname PY@tok@gs\endcsname{\let\PY@bf=\textbf}
\expandafter\def\csname PY@tok@gp\endcsname{\let\PY@bf=\textbf\def\PY@tc##1{\textcolor[rgb]{0.00,0.00,0.50}{##1}}}
\expandafter\def\csname PY@tok@go\endcsname{\def\PY@tc##1{\textcolor[rgb]{0.53,0.53,0.53}{##1}}}
\expandafter\def\csname PY@tok@gt\endcsname{\def\PY@tc##1{\textcolor[rgb]{0.00,0.27,0.87}{##1}}}
\expandafter\def\csname PY@tok@err\endcsname{\def\PY@bc##1{\setlength{\fboxsep}{0pt}\fcolorbox[rgb]{1.00,0.00,0.00}{1,1,1}{\strut ##1}}}
\expandafter\def\csname PY@tok@kc\endcsname{\let\PY@bf=\textbf\def\PY@tc##1{\textcolor[rgb]{0.00,0.50,0.00}{##1}}}
\expandafter\def\csname PY@tok@kd\endcsname{\let\PY@bf=\textbf\def\PY@tc##1{\textcolor[rgb]{0.00,0.50,0.00}{##1}}}
\expandafter\def\csname PY@tok@kn\endcsname{\let\PY@bf=\textbf\def\PY@tc##1{\textcolor[rgb]{0.00,0.50,0.00}{##1}}}
\expandafter\def\csname PY@tok@kr\endcsname{\let\PY@bf=\textbf\def\PY@tc##1{\textcolor[rgb]{0.00,0.50,0.00}{##1}}}
\expandafter\def\csname PY@tok@bp\endcsname{\def\PY@tc##1{\textcolor[rgb]{0.00,0.50,0.00}{##1}}}
\expandafter\def\csname PY@tok@fm\endcsname{\def\PY@tc##1{\textcolor[rgb]{0.00,0.00,1.00}{##1}}}
\expandafter\def\csname PY@tok@vc\endcsname{\def\PY@tc##1{\textcolor[rgb]{0.10,0.09,0.49}{##1}}}
\expandafter\def\csname PY@tok@vg\endcsname{\def\PY@tc##1{\textcolor[rgb]{0.10,0.09,0.49}{##1}}}
\expandafter\def\csname PY@tok@vi\endcsname{\def\PY@tc##1{\textcolor[rgb]{0.10,0.09,0.49}{##1}}}
\expandafter\def\csname PY@tok@vm\endcsname{\def\PY@tc##1{\textcolor[rgb]{0.10,0.09,0.49}{##1}}}
\expandafter\def\csname PY@tok@sa\endcsname{\def\PY@tc##1{\textcolor[rgb]{0.73,0.13,0.13}{##1}}}
\expandafter\def\csname PY@tok@sb\endcsname{\def\PY@tc##1{\textcolor[rgb]{0.73,0.13,0.13}{##1}}}
\expandafter\def\csname PY@tok@sc\endcsname{\def\PY@tc##1{\textcolor[rgb]{0.73,0.13,0.13}{##1}}}
\expandafter\def\csname PY@tok@dl\endcsname{\def\PY@tc##1{\textcolor[rgb]{0.73,0.13,0.13}{##1}}}
\expandafter\def\csname PY@tok@s2\endcsname{\def\PY@tc##1{\textcolor[rgb]{0.73,0.13,0.13}{##1}}}
\expandafter\def\csname PY@tok@sh\endcsname{\def\PY@tc##1{\textcolor[rgb]{0.73,0.13,0.13}{##1}}}
\expandafter\def\csname PY@tok@s1\endcsname{\def\PY@tc##1{\textcolor[rgb]{0.73,0.13,0.13}{##1}}}
\expandafter\def\csname PY@tok@mb\endcsname{\def\PY@tc##1{\textcolor[rgb]{0.40,0.40,0.40}{##1}}}
\expandafter\def\csname PY@tok@mf\endcsname{\def\PY@tc##1{\textcolor[rgb]{0.40,0.40,0.40}{##1}}}
\expandafter\def\csname PY@tok@mh\endcsname{\def\PY@tc##1{\textcolor[rgb]{0.40,0.40,0.40}{##1}}}
\expandafter\def\csname PY@tok@mi\endcsname{\def\PY@tc##1{\textcolor[rgb]{0.40,0.40,0.40}{##1}}}
\expandafter\def\csname PY@tok@il\endcsname{\def\PY@tc##1{\textcolor[rgb]{0.40,0.40,0.40}{##1}}}
\expandafter\def\csname PY@tok@mo\endcsname{\def\PY@tc##1{\textcolor[rgb]{0.40,0.40,0.40}{##1}}}
\expandafter\def\csname PY@tok@ch\endcsname{\let\PY@it=\textit\def\PY@tc##1{\textcolor[rgb]{0.25,0.50,0.50}{##1}}}
\expandafter\def\csname PY@tok@cm\endcsname{\let\PY@it=\textit\def\PY@tc##1{\textcolor[rgb]{0.25,0.50,0.50}{##1}}}
\expandafter\def\csname PY@tok@cpf\endcsname{\let\PY@it=\textit\def\PY@tc##1{\textcolor[rgb]{0.25,0.50,0.50}{##1}}}
\expandafter\def\csname PY@tok@c1\endcsname{\let\PY@it=\textit\def\PY@tc##1{\textcolor[rgb]{0.25,0.50,0.50}{##1}}}
\expandafter\def\csname PY@tok@cs\endcsname{\let\PY@it=\textit\def\PY@tc##1{\textcolor[rgb]{0.25,0.50,0.50}{##1}}}

\def\PYZbs{\char`\\}
\def\PYZus{\char`\_}
\def\PYZob{\char`\{}
\def\PYZcb{\char`\}}
\def\PYZca{\char`\^}
\def\PYZam{\char`\&}
\def\PYZlt{\char`\<}
\def\PYZgt{\char`\>}
\def\PYZsh{\char`\#}
\def\PYZpc{\char`\%}
\def\PYZdl{\char`\$}
\def\PYZhy{\char`\-}
\def\PYZsq{\char`\'}
\def\PYZdq{\char`\"}
\def\PYZti{\char`\~}
% for compatibility with earlier versions
\def\PYZat{@}
\def\PYZlb{[}
\def\PYZrb{]}
\makeatother


    % For linebreaks inside Verbatim environment from package fancyvrb. 
    \makeatletter
        \newbox\Wrappedcontinuationbox 
        \newbox\Wrappedvisiblespacebox 
        \newcommand*\Wrappedvisiblespace {\textcolor{red}{\textvisiblespace}} 
        \newcommand*\Wrappedcontinuationsymbol {\textcolor{red}{\llap{\tiny$\m@th\hookrightarrow$}}} 
        \newcommand*\Wrappedcontinuationindent {3ex } 
        \newcommand*\Wrappedafterbreak {\kern\Wrappedcontinuationindent\copy\Wrappedcontinuationbox} 
        % Take advantage of the already applied Pygments mark-up to insert 
        % potential linebreaks for TeX processing. 
        %        {, <, #, %, $, ' and ": go to next line. 
        %        _, }, ^, &, >, - and ~: stay at end of broken line. 
        % Use of \textquotesingle for straight quote. 
        \newcommand*\Wrappedbreaksatspecials {% 
            \def\PYGZus{\discretionary{\char`\_}{\Wrappedafterbreak}{\char`\_}}% 
            \def\PYGZob{\discretionary{}{\Wrappedafterbreak\char`\{}{\char`\{}}% 
            \def\PYGZcb{\discretionary{\char`\}}{\Wrappedafterbreak}{\char`\}}}% 
            \def\PYGZca{\discretionary{\char`\^}{\Wrappedafterbreak}{\char`\^}}% 
            \def\PYGZam{\discretionary{\char`\&}{\Wrappedafterbreak}{\char`\&}}% 
            \def\PYGZlt{\discretionary{}{\Wrappedafterbreak\char`\<}{\char`\<}}% 
            \def\PYGZgt{\discretionary{\char`\>}{\Wrappedafterbreak}{\char`\>}}% 
            \def\PYGZsh{\discretionary{}{\Wrappedafterbreak\char`\#}{\char`\#}}% 
            \def\PYGZpc{\discretionary{}{\Wrappedafterbreak\char`\%}{\char`\%}}% 
            \def\PYGZdl{\discretionary{}{\Wrappedafterbreak\char`\$}{\char`\$}}% 
            \def\PYGZhy{\discretionary{\char`\-}{\Wrappedafterbreak}{\char`\-}}% 
            \def\PYGZsq{\discretionary{}{\Wrappedafterbreak\textquotesingle}{\textquotesingle}}% 
            \def\PYGZdq{\discretionary{}{\Wrappedafterbreak\char`\"}{\char`\"}}% 
            \def\PYGZti{\discretionary{\char`\~}{\Wrappedafterbreak}{\char`\~}}% 
        } 
        % Some characters . , ; ? ! / are not pygmentized. 
        % This macro makes them "active" and they will insert potential linebreaks 
        \newcommand*\Wrappedbreaksatpunct {% 
            \lccode`\~`\.\lowercase{\def~}{\discretionary{\hbox{\char`\.}}{\Wrappedafterbreak}{\hbox{\char`\.}}}% 
            \lccode`\~`\,\lowercase{\def~}{\discretionary{\hbox{\char`\,}}{\Wrappedafterbreak}{\hbox{\char`\,}}}% 
            \lccode`\~`\;\lowercase{\def~}{\discretionary{\hbox{\char`\;}}{\Wrappedafterbreak}{\hbox{\char`\;}}}% 
            \lccode`\~`\:\lowercase{\def~}{\discretionary{\hbox{\char`\:}}{\Wrappedafterbreak}{\hbox{\char`\:}}}% 
            \lccode`\~`\?\lowercase{\def~}{\discretionary{\hbox{\char`\?}}{\Wrappedafterbreak}{\hbox{\char`\?}}}% 
            \lccode`\~`\!\lowercase{\def~}{\discretionary{\hbox{\char`\!}}{\Wrappedafterbreak}{\hbox{\char`\!}}}% 
            \lccode`\~`\/\lowercase{\def~}{\discretionary{\hbox{\char`\/}}{\Wrappedafterbreak}{\hbox{\char`\/}}}% 
            \catcode`\.\active
            \catcode`\,\active 
            \catcode`\;\active
            \catcode`\:\active
            \catcode`\?\active
            \catcode`\!\active
            \catcode`\/\active 
            \lccode`\~`\~ 	
        }
    \makeatother

    \let\OriginalVerbatim=\Verbatim
    \makeatletter
    \renewcommand{\Verbatim}[1][1]{%
        %\parskip\z@skip
        \sbox\Wrappedcontinuationbox {\Wrappedcontinuationsymbol}%
        \sbox\Wrappedvisiblespacebox {\FV@SetupFont\Wrappedvisiblespace}%
        \def\FancyVerbFormatLine ##1{\hsize\linewidth
            \vtop{\raggedright\hyphenpenalty\z@\exhyphenpenalty\z@
                \doublehyphendemerits\z@\finalhyphendemerits\z@
                \strut ##1\strut}%
        }%
        % If the linebreak is at a space, the latter will be displayed as visible
        % space at end of first line, and a continuation symbol starts next line.
        % Stretch/shrink are however usually zero for typewriter font.
        \def\FV@Space {%
            \nobreak\hskip\z@ plus\fontdimen3\font minus\fontdimen4\font
            \discretionary{\copy\Wrappedvisiblespacebox}{\Wrappedafterbreak}
            {\kern\fontdimen2\font}%
        }%
        
        % Allow breaks at special characters using \PYG... macros.
        \Wrappedbreaksatspecials
        % Breaks at punctuation characters . , ; ? ! and / need catcode=\active 	
        \OriginalVerbatim[#1,codes*=\Wrappedbreaksatpunct]%
    }
    \makeatother

    % Exact colors from NB
    \definecolor{incolor}{HTML}{303F9F}
    \definecolor{outcolor}{HTML}{D84315}
    \definecolor{cellborder}{HTML}{CFCFCF}
    \definecolor{cellbackground}{HTML}{F7F7F7}
    
    % prompt
    \makeatletter
    \newcommand{\boxspacing}{\kern\kvtcb@left@rule\kern\kvtcb@boxsep}
    \makeatother
    \newcommand{\prompt}[4]{
        \ttfamily\llap{{\color{#2}[#3]:\hspace{3pt}#4}}\vspace{-\baselineskip}
    }
    

    
    % Prevent overflowing lines due to hard-to-break entities
    \sloppy 
    % Setup hyperref package
    \hypersetup{
      breaklinks=true,  % so long urls are correctly broken across lines
      colorlinks=true,
      urlcolor=urlcolor,
      linkcolor=linkcolor,
      citecolor=citecolor,
      }
    % Slightly bigger margins than the latex defaults
    
    \geometry{verbose,tmargin=1in,bmargin=1in,lmargin=1in,rmargin=1in}
    
\makeatletter
\renewcommand{\@seccntformat}[1]{}
\makeatother  

\begin{document}
    \title{CS 156a - Problem Set 6}
    \author{Samuel Patrone, 2140749}
    \maketitle
    

The following notebook is publicly available at the following
\href{https://github.com/spatrone/CS156A-Caltech.git}{link}.

\tableofcontents

    \hypertarget{problem-1}{%
\section{Problem 1}\label{problem-1}}

\hypertarget{answer-b-in-general-deterministic-noise-will-increase.}{%
\subsection{Answer: {[}b{]} In general, deterministic noise will
increase.}\label{answer-b-in-general-deterministic-noise-will-increase.}}

\hypertarget{derivation}{%
\subsection{Derivation:}\label{derivation}}

Given that we are using a less complex hypothesis in
\(\mathcal{H}^\prime\subset\mathcal{H}\), we expect the best fit
\(g^\prime\in\mathcal{H}^\prime\) in this less complex hypothesis space
to have an higher deterministic noise since there will be more of the
target function that cannot be captured by it. In more mathematical
terms, the deterministic noise is represented by the bias, which
naturally increases for less complex hypothesis. The bias can be seen as
the asymptotic value at which both \(E_{in}\) and \(E_{out}\) converge
for \(N\to\infty\) (in the case of zero stochastic noise) which is lower
for more complex hypothesis.

    \hypertarget{problem-2}{%
\section{Problem 2}\label{problem-2}}

\hypertarget{answer-a-0.030.08}{%
\subsection{\texorpdfstring{Answer: {[}a{]}
\(0.03,\,0.08\)}{Answer: {[}a{]} 0.03,\textbackslash{},0.08}}\label{answer-a-0.030.08}}

\hypertarget{code}{%
\subsection{Code:}\label{code}}

    \begin{tcolorbox}[breakable, size=fbox, boxrule=1pt, pad at break*=1mm,colback=cellbackground, colframe=cellborder]
\prompt{In}{incolor}{72}{\boxspacing}
\begin{Verbatim}[commandchars=\\\{\}]
\PY{k+kn}{import} \PY{n+nn}{numpy} \PY{k}{as} \PY{n+nn}{np}
\PY{k+kn}{import} \PY{n+nn}{matplotlib}\PY{n+nn}{.}\PY{n+nn}{pyplot} \PY{k}{as} \PY{n+nn}{plt}
\PY{k+kn}{import} \PY{n+nn}{pandas} \PY{k}{as} \PY{n+nn}{pd}

\PY{c+c1}{\PYZsh{}import data}

\PY{n}{training\PYZus{}set}\PY{o}{=}\PY{n}{pd}\PY{o}{.}\PY{n}{read\PYZus{}csv}\PY{p}{(}\PY{l+s+s1}{\PYZsq{}}\PY{l+s+s1}{in.dta}\PY{l+s+s1}{\PYZsq{}}\PY{p}{,}\PY{n}{header}\PY{o}{=}\PY{k+kc}{None}\PY{p}{,}\PY{n}{delim\PYZus{}whitespace}\PY{o}{=}\PY{k+kc}{True}\PY{p}{)}
\PY{n}{testing\PYZus{}set}\PY{o}{=}\PY{n}{pd}\PY{o}{.}\PY{n}{read\PYZus{}csv}\PY{p}{(}\PY{l+s+s1}{\PYZsq{}}\PY{l+s+s1}{out.dta}\PY{l+s+s1}{\PYZsq{}}\PY{p}{,}\PY{n}{header}\PY{o}{=}\PY{k+kc}{None}\PY{p}{,}\PY{n}{delim\PYZus{}whitespace}\PY{o}{=}\PY{k+kc}{True}\PY{p}{)}

\PY{n}{train\PYZus{}pts}\PY{o}{=}\PY{n}{training\PYZus{}set}\PY{p}{[}\PY{p}{[}\PY{l+m+mi}{0}\PY{p}{,} \PY{l+m+mi}{1}\PY{p}{]}\PY{p}{]}\PY{o}{.}\PY{n}{to\PYZus{}numpy}\PY{p}{(}\PY{p}{)}
\PY{n}{train\PYZus{}y}\PY{o}{=}\PY{n}{training\PYZus{}set}\PY{p}{[}\PY{l+m+mi}{2}\PY{p}{]}\PY{o}{.}\PY{n}{to\PYZus{}numpy}\PY{p}{(}\PY{p}{)}

\PY{n}{test\PYZus{}pts}\PY{o}{=}\PY{n}{testing\PYZus{}set}\PY{p}{[}\PY{p}{[}\PY{l+m+mi}{0}\PY{p}{,} \PY{l+m+mi}{1}\PY{p}{]}\PY{p}{]}\PY{o}{.}\PY{n}{to\PYZus{}numpy}\PY{p}{(}\PY{p}{)}
\PY{n}{test\PYZus{}y}\PY{o}{=}\PY{n}{testing\PYZus{}set}\PY{p}{[}\PY{l+m+mi}{2}\PY{p}{]}\PY{o}{.}\PY{n}{to\PYZus{}numpy}\PY{p}{(}\PY{p}{)}


\PY{k}{def} \PY{n+nf}{color\PYZus{}pts}\PY{p}{(}\PY{n}{y}\PY{p}{)}\PY{p}{:}
    \PY{c+c1}{\PYZsh{}green is +1, red is \PYZhy{}1}
    \PY{n}{col}\PY{o}{=}\PY{p}{[}\PY{p}{]}
    \PY{k}{for} \PY{n}{i} \PY{o+ow}{in} \PY{n+nb}{range}\PY{p}{(}\PY{n+nb}{len}\PY{p}{(}\PY{n}{y}\PY{p}{)}\PY{p}{)}\PY{p}{:}
        \PY{k}{if}\PY{p}{(}\PY{n}{y}\PY{p}{[}\PY{n}{i}\PY{p}{]}\PY{o}{\PYZgt{}}\PY{l+m+mi}{0}\PY{p}{)}\PY{p}{:} \PY{n}{col}\PY{o}{.}\PY{n}{append}\PY{p}{(}\PY{l+s+s1}{\PYZsq{}}\PY{l+s+s1}{green}\PY{l+s+s1}{\PYZsq{}}\PY{p}{)}
        \PY{k}{else}\PY{p}{:} \PY{n}{col}\PY{o}{.}\PY{n}{append}\PY{p}{(}\PY{l+s+s1}{\PYZsq{}}\PY{l+s+s1}{red}\PY{l+s+s1}{\PYZsq{}}\PY{p}{)}
    \PY{k}{return} \PY{n}{col}

\PY{k}{def} \PY{n+nf}{plot\PYZus{}pts}\PY{p}{(}\PY{n}{pts}\PY{p}{,}\PY{n}{y}\PY{p}{)}\PY{p}{:}
    \PY{n}{col}\PY{o}{=}\PY{n}{color\PYZus{}pts}\PY{p}{(}\PY{n}{y}\PY{p}{)}
    \PY{n}{plt}\PY{o}{.}\PY{n}{scatter}\PY{p}{(}\PY{n}{pts}\PY{p}{[}\PY{p}{:}\PY{p}{,}\PY{l+m+mi}{0}\PY{p}{]}\PY{p}{,}\PY{n}{pts}\PY{p}{[}\PY{p}{:}\PY{p}{,}\PY{l+m+mi}{1}\PY{p}{]}\PY{p}{,}\PY{n}{color}\PY{o}{=}\PY{n}{col}\PY{p}{)}
    \PY{c+c1}{\PYZsh{}plt.xlim([\PYZhy{}1, 1])}
    \PY{c+c1}{\PYZsh{}plt.ylim([\PYZhy{}1, 1])}
    \PY{n}{plt}\PY{o}{.}\PY{n}{legend}\PY{p}{(}\PY{p}{)}
    \PY{n}{plt}\PY{o}{.}\PY{n}{show}\PY{p}{(}\PY{p}{)}
    
\PY{c+c1}{\PYZsh{}non\PYZhy{}linear transformation}

\PY{k}{def} \PY{n+nf}{transform}\PY{p}{(}\PY{n}{pts}\PY{p}{)}\PY{p}{:}
    \PY{n}{res}\PY{o}{=}\PY{p}{[}\PY{p}{]}
    \PY{k}{for} \PY{n}{i} \PY{o+ow}{in} \PY{n+nb}{range}\PY{p}{(}\PY{n+nb}{len}\PY{p}{(}\PY{n}{pts}\PY{p}{)}\PY{p}{)}\PY{p}{:}
        \PY{n}{x1}\PY{o}{=}\PY{n}{pts}\PY{p}{[}\PY{n}{i}\PY{p}{]}\PY{p}{[}\PY{l+m+mi}{0}\PY{p}{]}
        \PY{n}{x2}\PY{o}{=}\PY{n}{pts}\PY{p}{[}\PY{n}{i}\PY{p}{]}\PY{p}{[}\PY{l+m+mi}{1}\PY{p}{]}
        \PY{n}{res}\PY{o}{.}\PY{n}{append}\PY{p}{(}\PY{p}{[}\PY{l+m+mi}{1}\PY{p}{,}\PY{n}{x1}\PY{p}{,}\PY{n}{x2}\PY{p}{,}\PY{n}{x1}\PY{o}{*}\PY{o}{*}\PY{l+m+mi}{2}\PY{p}{,}\PY{n}{x2}\PY{o}{*}\PY{o}{*}\PY{l+m+mi}{2}\PY{p}{,}\PY{n}{x1}\PY{o}{*}\PY{n}{x2}\PY{p}{,}\PY{n}{np}\PY{o}{.}\PY{n}{abs}\PY{p}{(}\PY{n}{x1}\PY{o}{\PYZhy{}}\PY{n}{x2}\PY{p}{)}\PY{p}{,}\PY{n}{np}\PY{o}{.}\PY{n}{abs}\PY{p}{(}\PY{n}{x1}\PY{o}{+}\PY{n}{x2}\PY{p}{)}\PY{p}{]}\PY{p}{)}
    \PY{k}{return} \PY{n}{np}\PY{o}{.}\PY{n}{array}\PY{p}{(}\PY{n}{res}\PY{p}{)}
        
\PY{k}{def} \PY{n+nf}{lin\PYZus{}reg\PYZus{}w}\PY{p}{(}\PY{n}{X}\PY{p}{,}\PY{n}{y}\PY{p}{)}\PY{p}{:}
    \PY{k}{return} \PY{n}{np}\PY{o}{.}\PY{n}{dot}\PY{p}{(}\PY{n}{np}\PY{o}{.}\PY{n}{linalg}\PY{o}{.}\PY{n}{pinv}\PY{p}{(}\PY{n}{X}\PY{p}{)}\PY{p}{,}\PY{n}{y}\PY{p}{)}

\PY{k}{def} \PY{n+nf}{h}\PY{p}{(}\PY{n}{pts}\PY{p}{,}\PY{n}{w}\PY{p}{)}\PY{p}{:}
    \PY{k}{return} \PY{n}{np}\PY{o}{.}\PY{n}{sign}\PY{p}{(}\PY{n}{np}\PY{o}{.}\PY{n}{dot}\PY{p}{(}\PY{n}{w}\PY{p}{,}\PY{n}{pts}\PY{o}{.}\PY{n}{T}\PY{p}{)}\PY{p}{)}

\PY{k}{def} \PY{n+nf}{lin\PYZus{}reg}\PY{p}{(}\PY{n}{train\PYZus{}pts}\PY{p}{,}\PY{n}{train\PYZus{}y}\PY{p}{,}\PY{n}{test\PYZus{}pts}\PY{p}{,}\PY{n}{test\PYZus{}y}\PY{p}{,}\PY{n}{res}\PY{o}{=}\PY{k+kc}{True}\PY{p}{)}\PY{p}{:}
    \PY{n}{N\PYZus{}train}\PY{o}{=}\PY{n+nb}{len}\PY{p}{(}\PY{n}{train\PYZus{}pts}\PY{p}{)}
    \PY{n}{N\PYZus{}test}\PY{o}{=}\PY{n+nb}{len}\PY{p}{(}\PY{n}{test\PYZus{}pts}\PY{p}{)}
    
    \PY{n}{w}\PY{o}{=}\PY{n}{lin\PYZus{}reg\PYZus{}w}\PY{p}{(}\PY{n}{train\PYZus{}pts}\PY{p}{,}\PY{n}{train\PYZus{}y}\PY{p}{)}
    
    \PY{c+c1}{\PYZsh{}Ein computation}
    \PY{n}{gin}\PY{o}{=}\PY{n}{h}\PY{p}{(}\PY{n}{train\PYZus{}pts}\PY{p}{,}\PY{n}{w}\PY{p}{)}
    \PY{n}{testgin}\PY{o}{=}\PY{p}{(}\PY{n}{gin}\PY{o}{==}\PY{n}{train\PYZus{}y}\PY{p}{)}
    \PY{n}{Ein}\PY{o}{=}\PY{n+nb}{len}\PY{p}{(}\PY{n}{np}\PY{o}{.}\PY{n}{where}\PY{p}{(}\PY{n}{testgin}\PY{o}{==}\PY{k+kc}{False}\PY{p}{)}\PY{p}{[}\PY{l+m+mi}{0}\PY{p}{]}\PY{p}{)}\PY{o}{/}\PY{n}{N\PYZus{}train}
    
    \PY{c+c1}{\PYZsh{}Eout computation}
    \PY{n}{gout}\PY{o}{=}\PY{n}{h}\PY{p}{(}\PY{n}{test\PYZus{}pts}\PY{p}{,}\PY{n}{w}\PY{p}{)}
    \PY{n}{testgout}\PY{o}{=}\PY{p}{(}\PY{n}{gout}\PY{o}{==}\PY{n}{test\PYZus{}y}\PY{p}{)}
    \PY{n}{Eout}\PY{o}{=}\PY{n+nb}{len}\PY{p}{(}\PY{n}{np}\PY{o}{.}\PY{n}{where}\PY{p}{(}\PY{n}{testgout}\PY{o}{==}\PY{k+kc}{False}\PY{p}{)}\PY{p}{[}\PY{l+m+mi}{0}\PY{p}{]}\PY{p}{)}\PY{o}{/}\PY{n}{N\PYZus{}test}
    
    \PY{c+c1}{\PYZsh{}print results}
    \PY{k}{if}\PY{p}{(}\PY{n}{res}\PY{o}{==}\PY{k+kc}{True}\PY{p}{)}\PY{p}{:}
        \PY{n+nb}{print}\PY{p}{(}\PY{n}{f}\PY{l+s+s1}{\PYZsq{}}\PY{l+s+s1}{Linear Regression results:}\PY{l+s+se}{\PYZbs{}n}\PY{l+s+s1}{Ein=}\PY{l+s+si}{\PYZob{}Ein:.2f\PYZcb{}}\PY{l+s+se}{\PYZbs{}n}\PY{l+s+s1}{Eout=}\PY{l+s+si}{\PYZob{}Eout:.2f\PYZcb{}}\PY{l+s+s1}{\PYZsq{}}\PY{p}{)}
    
    \PY{k}{return} \PY{n}{w}\PY{p}{,}\PY{n}{Ein}\PY{p}{,}\PY{n}{Eout}
\end{Verbatim}
\end{tcolorbox}

    \begin{tcolorbox}[breakable, size=fbox, boxrule=1pt, pad at break*=1mm,colback=cellbackground, colframe=cellborder]
\prompt{In}{incolor}{151}{\boxspacing}
\begin{Verbatim}[commandchars=\\\{\}]
\PY{n}{train\PYZus{}pts\PYZus{}transf}\PY{o}{=}\PY{n}{transform}\PY{p}{(}\PY{n}{train\PYZus{}pts}\PY{p}{)}
\PY{n}{test\PYZus{}pts\PYZus{}transf}\PY{o}{=}\PY{n}{transform}\PY{p}{(}\PY{n}{test\PYZus{}pts}\PY{p}{)}

\PY{n}{ex2}\PY{o}{=}\PY{n}{lin\PYZus{}reg}\PY{p}{(}\PY{n}{train\PYZus{}pts\PYZus{}transf}\PY{p}{,}\PY{n}{train\PYZus{}y}\PY{p}{,}\PY{n}{test\PYZus{}pts\PYZus{}transf}\PY{p}{,}\PY{n}{test\PYZus{}y}\PY{p}{)}
\end{Verbatim}
\end{tcolorbox}

    \begin{Verbatim}[commandchars=\\\{\}]
Linear Regression results:
Ein=0.03
Eout=0.08
    \end{Verbatim}

    \hypertarget{problems-3-6}{%
\section{Problems 3-6}\label{problems-3-6}}

\hypertarget{answers-d-0.030.08-e-0.40.4-d--1-b-0.06}{%
\subsection{\texorpdfstring{Answers: {[}d{]} \([0.03,\,0.08]\),
{[}e{]} \([0.4,\,0.4]\), {[}d{]} \(-1\), {[}b{]}
\(0.06\)}{Answers: {[}d{]} {[}0.03,\textbackslash{},0.08{]}, {[}e{]} {[}0.4,\textbackslash{},0.4{]}, {[}d{]} -1, {[}b{]} 0.06}}\label{answers-d-0.030.08-e-0.40.4-d--1-b-0.06}}

\hypertarget{code}{%
\subsection{Code:}\label{code}}

    \begin{tcolorbox}[breakable, size=fbox, boxrule=1pt, pad at break*=1mm,colback=cellbackground, colframe=cellborder]
\prompt{In}{incolor}{121}{\boxspacing}
\begin{Verbatim}[commandchars=\\\{\}]
\PY{k}{def} \PY{n+nf}{lin\PYZus{}reg\PYZus{}w\PYZus{}lam}\PY{p}{(}\PY{n}{X}\PY{p}{,}\PY{n}{y}\PY{p}{,}\PY{n}{lam}\PY{p}{)}\PY{p}{:}
    \PY{n}{pinv\PYZus{}decay}\PY{o}{=}\PY{n}{np}\PY{o}{.}\PY{n}{dot}\PY{p}{(}\PY{n}{np}\PY{o}{.}\PY{n}{linalg}\PY{o}{.}\PY{n}{inv}\PY{p}{(}\PY{n}{np}\PY{o}{.}\PY{n}{dot}\PY{p}{(}\PY{n}{X}\PY{o}{.}\PY{n}{T}\PY{p}{,}\PY{n}{X}\PY{p}{)}\PY{o}{+}\PY{n}{lam}\PY{o}{*}\PY{n}{np}\PY{o}{.}\PY{n}{identity}\PY{p}{(}\PY{n+nb}{len}\PY{p}{(}\PY{n}{X}\PY{o}{.}\PY{n}{T}\PY{p}{)}\PY{p}{)}\PY{p}{)}\PY{p}{,}\PY{n}{X}\PY{o}{.}\PY{n}{T}\PY{p}{)}
    \PY{k}{return} \PY{n}{np}\PY{o}{.}\PY{n}{dot}\PY{p}{(}\PY{n}{pinv\PYZus{}decay}\PY{p}{,}\PY{n}{y}\PY{p}{)}

\PY{k}{def} \PY{n+nf}{lin\PYZus{}reg\PYZus{}wdecay}\PY{p}{(}\PY{n}{train\PYZus{}pts}\PY{p}{,}\PY{n}{train\PYZus{}y}\PY{p}{,}\PY{n}{test\PYZus{}pts}\PY{p}{,}\PY{n}{test\PYZus{}y}\PY{p}{,}\PY{n}{lam}\PY{p}{,}\PY{n}{res}\PY{o}{=}\PY{k+kc}{True}\PY{p}{)}\PY{p}{:}
    \PY{n}{N\PYZus{}train}\PY{o}{=}\PY{n+nb}{len}\PY{p}{(}\PY{n}{train\PYZus{}pts}\PY{p}{)}
    \PY{n}{N\PYZus{}test}\PY{o}{=}\PY{n+nb}{len}\PY{p}{(}\PY{n}{test\PYZus{}pts}\PY{p}{)}
    
    \PY{n}{w}\PY{o}{=}\PY{n}{lin\PYZus{}reg\PYZus{}w\PYZus{}lam}\PY{p}{(}\PY{n}{train\PYZus{}pts}\PY{p}{,}\PY{n}{train\PYZus{}y}\PY{p}{,}\PY{n}{lam}\PY{p}{)}
    
    \PY{c+c1}{\PYZsh{}Ein computation}
    \PY{n}{gin}\PY{o}{=}\PY{n}{h}\PY{p}{(}\PY{n}{train\PYZus{}pts}\PY{p}{,}\PY{n}{w}\PY{p}{)}
    \PY{n}{testgin}\PY{o}{=}\PY{p}{(}\PY{n}{gin}\PY{o}{==}\PY{n}{train\PYZus{}y}\PY{p}{)}
    \PY{n}{Ein}\PY{o}{=}\PY{n+nb}{len}\PY{p}{(}\PY{n}{np}\PY{o}{.}\PY{n}{where}\PY{p}{(}\PY{n}{testgin}\PY{o}{==}\PY{k+kc}{False}\PY{p}{)}\PY{p}{[}\PY{l+m+mi}{0}\PY{p}{]}\PY{p}{)}\PY{o}{/}\PY{n}{N\PYZus{}train}
    
    \PY{c+c1}{\PYZsh{}Eout computation}
    \PY{n}{gout}\PY{o}{=}\PY{n}{h}\PY{p}{(}\PY{n}{test\PYZus{}pts}\PY{p}{,}\PY{n}{w}\PY{p}{)}
    \PY{n}{testgout}\PY{o}{=}\PY{p}{(}\PY{n}{gout}\PY{o}{==}\PY{n}{test\PYZus{}y}\PY{p}{)}
    \PY{n}{Eout}\PY{o}{=}\PY{n+nb}{len}\PY{p}{(}\PY{n}{np}\PY{o}{.}\PY{n}{where}\PY{p}{(}\PY{n}{testgout}\PY{o}{==}\PY{k+kc}{False}\PY{p}{)}\PY{p}{[}\PY{l+m+mi}{0}\PY{p}{]}\PY{p}{)}\PY{o}{/}\PY{n}{N\PYZus{}test}
    
    \PY{c+c1}{\PYZsh{}print results}
    \PY{k}{if}\PY{p}{(}\PY{n}{res}\PY{o}{==}\PY{k+kc}{True}\PY{p}{)}\PY{p}{:}
        \PY{n+nb}{print}\PY{p}{(}\PY{n}{f}\PY{l+s+s1}{\PYZsq{}}\PY{l+s+s1}{Linear Regression results with k=}\PY{l+s+s1}{\PYZob{}}\PY{l+s+s1}{np.log10(lam):.0f\PYZcb{}:}\PY{l+s+se}{\PYZbs{}n}\PY{l+s+s1}{Ein=}\PY{l+s+si}{\PYZob{}Ein:.2f\PYZcb{}}\PY{l+s+se}{\PYZbs{}n}\PY{l+s+s1}{Eout=}\PY{l+s+si}{\PYZob{}Eout:.2f\PYZcb{}}\PY{l+s+s1}{\PYZsq{}}\PY{p}{)}
    
    \PY{k}{return} \PY{n}{w}\PY{p}{,}\PY{n}{Ein}\PY{p}{,}\PY{n}{Eout}
\end{Verbatim}
\end{tcolorbox}

    \begin{tcolorbox}[breakable, size=fbox, boxrule=1pt, pad at break*=1mm,colback=cellbackground, colframe=cellborder]
\prompt{In}{incolor}{165}{\boxspacing}
\begin{Verbatim}[commandchars=\\\{\}]
\PY{n}{k}\PY{o}{=}\PY{p}{[}\PY{l+m+mi}{4}\PY{p}{,}\PY{l+m+mi}{3}\PY{p}{,}\PY{l+m+mi}{2}\PY{p}{,}\PY{l+m+mi}{1}\PY{p}{,}\PY{l+m+mi}{0}\PY{p}{,}\PY{o}{\PYZhy{}}\PY{l+m+mi}{1}\PY{p}{,}\PY{o}{\PYZhy{}}\PY{l+m+mi}{2}\PY{p}{,}\PY{o}{\PYZhy{}}\PY{l+m+mi}{3}\PY{p}{,}\PY{o}{\PYZhy{}}\PY{l+m+mi}{4}\PY{p}{]}
\PY{n}{Ein}\PY{o}{=}\PY{p}{[}\PY{p}{]}
\PY{n}{Eout}\PY{o}{=}\PY{p}{[}\PY{p}{]}
\PY{n}{wnorm}\PY{o}{=}\PY{p}{[}\PY{p}{]}

\PY{k}{for} \PY{n}{i} \PY{o+ow}{in} \PY{n}{k}\PY{p}{:}
    \PY{n}{ex5}\PY{p}{,}\PY{n}{Ein5}\PY{p}{,}\PY{n}{Eout5}\PY{o}{=}\PY{n}{lin\PYZus{}reg\PYZus{}wdecay}\PY{p}{(}\PY{n}{train\PYZus{}pts\PYZus{}transf}\PY{p}{,}\PY{n}{train\PYZus{}y}\PY{p}{,}\PY{n}{test\PYZus{}pts\PYZus{}transf}\PY{p}{,}\PY{n}{test\PYZus{}y}\PY{p}{,}\PY{l+m+mi}{10}\PY{o}{*}\PY{o}{*}\PY{p}{(}\PY{n}{i}\PY{p}{)}\PY{p}{,}\PY{n}{res}\PY{o}{=}\PY{k+kc}{False}\PY{p}{)}
    \PY{n}{Ein}\PY{o}{.}\PY{n}{append}\PY{p}{(}\PY{n}{Ein5}\PY{p}{)}
    \PY{n}{Eout}\PY{o}{.}\PY{n}{append}\PY{p}{(}\PY{n}{Eout5}\PY{p}{)}
    \PY{n}{wnorm}\PY{o}{.}\PY{n}{append}\PY{p}{(}\PY{n}{np}\PY{o}{.}\PY{n}{dot}\PY{p}{(}\PY{n}{ex5}\PY{p}{,}\PY{n}{ex5}\PY{p}{)}\PY{p}{)}
    
    
\PY{n}{pd}\PY{o}{.}\PY{n}{options}\PY{o}{.}\PY{n}{display}\PY{o}{.}\PY{n}{float\PYZus{}format} \PY{o}{=} \PY{l+s+s1}{\PYZsq{}}\PY{l+s+si}{\PYZob{}:,.2f\PYZcb{}}\PY{l+s+s1}{\PYZsq{}}\PY{o}{.}\PY{n}{format}   
\PY{n}{pd}\PY{o}{.}\PY{n}{DataFrame}\PY{p}{(}\PY{n+nb}{list}\PY{p}{(}\PY{n+nb}{zip}\PY{p}{(}\PY{n}{k}\PY{p}{,} \PY{n}{Ein}\PY{p}{,}\PY{n}{Eout}\PY{p}{,}\PY{n}{wnorm}\PY{p}{)}\PY{p}{)}\PY{p}{,} \PY{n}{columns} \PY{o}{=}\PY{p}{[}\PY{l+s+s1}{\PYZsq{}}\PY{l+s+s1}{k}\PY{l+s+s1}{\PYZsq{}}\PY{p}{,} \PY{l+s+s1}{\PYZsq{}}\PY{l+s+s1}{Ein}\PY{l+s+s1}{\PYZsq{}}\PY{p}{,}\PY{l+s+s1}{\PYZsq{}}\PY{l+s+s1}{Eout}\PY{l+s+s1}{\PYZsq{}}\PY{p}{,}\PY{l+s+s1}{\PYZsq{}}\PY{l+s+s1}{w.w}\PY{l+s+s1}{\PYZsq{}}\PY{p}{]}\PY{p}{)}
\end{Verbatim}
\end{tcolorbox}

            \begin{tcolorbox}[breakable, size=fbox, boxrule=.5pt, pad at break*=1mm, opacityfill=0]
\prompt{Out}{outcolor}{165}{\boxspacing}
\begin{Verbatim}[commandchars=\\\{\}]
   k  Ein  Eout   w.w
0  4 0.43  0.45  0.00
1  3 0.37  0.44  0.00
2  2 0.20  0.23  0.03
3  1 0.06  0.12  0.50
4  0 0.00  0.09  2.52
5 -1 0.03  0.06 15.37
6 -2 0.03  0.08 30.39
7 -3 0.03  0.08 33.40
8 -4 0.03  0.08 33.74
\end{Verbatim}
\end{tcolorbox}
        
    \hypertarget{problem-7}{%
\section{Problem 7}\label{problem-7}}

\hypertarget{answer-c-mathcalh1003cap-mathcalh1004mathcalh_2}{%
\subsection{\texorpdfstring{Answer: {[}c{]}
\(\mathcal{H}(10,0,3)\cap \mathcal{H}(10,0,4)=\mathcal{H}_2\)}{Answer: {[}c{]} \textbackslash{}mathcal\{H\}(10,0,3)\textbackslash{}cap \textbackslash{}mathcal\{H\}(10,0,4)=\textbackslash{}mathcal\{H\}\_2}}\label{answer-c-mathcalh1003cap-mathcalh1004mathcalh_2}}

\hypertarget{derivation}{%
\subsection{Derivation:}\label{derivation}}

We first observe that the following identity holds:

\begin{equation}
\mathcal{H}(Q,C=0,Q_0)=\mathcal{H}_{Q_0-1}\,.
\end{equation}

where \(Q \ge Q_0\).

Furthermore, by definition we have that:

\begin{equation}
\mathcal{H}_{N-1}\subset\mathcal{H}_{N}\,.
\end{equation}

Therefore, \begin{equation}
\mathcal{H}(10,0,3)\cap \mathcal{H}(10,0,4)=\mathcal{H}_2\cap \mathcal{H}_3=\mathcal{H}_2\,.
\end{equation}

    \hypertarget{problem-8}{%
\section{Problem 8}\label{problem-8}}

\hypertarget{answer-d-45}{%
\subsection{\texorpdfstring{Answer: {[}d{]}
\(45\)}{Answer: {[}d{]} 45}}\label{answer-d-45}}

\hypertarget{derivation}{%
\subsection{Derivation:}\label{derivation}}

One single iteration of the backpropagation algorithm using one data
point starts with the forward propagation, i.e.~the computation of the
neural network hypothesis which is obtained by computing the output
vector at each layer

\begin{equation}
\mathbf{x}^{(l)}=
\begin{pmatrix}
1\\
\theta(\mathbf{s}^{(l)})
\end{pmatrix}
\end{equation}

where \(\mathbf{s}^{(l)}\) is a vector whose component are

\begin{equation}
s_i^{(l)}=\sum_{j=0}^{d^{(l-1)}}w_{ij}^{(l)}x_j^{(l-1)}\,.
\end{equation}

For each layer \(l\), there are \(d^{(l)}(d^{(l-1)}+1)\) multiplications
to do, which is exactly the number of weights between the layer \(l-1\)
and the layer \(l\). Hence, for a complete forward propagation, we have
\(N_w\) total multiplications, where

\begin{equation}
N_w=\sum_{l=1}^L d^{(l)}(d^{(l-1)}+1)\,.
\end{equation}

In the stochastic gradient descent algorithm, the weights are updated by
taking a step in the negative gradient direction. To compute the
gradient, we have to take the derivatives of the error function with
respect to each single weight \(w_{ij}^{(l)}\), which is given by

\begin{equation}
\frac{\partial e}{\partial w_{ij}^{(l)}} = x^{(l-1)}_i \delta^{(l)}_j\,.
\end{equation}

There are as many derivatives as weights, so we have other \(N_w\)
operations. Finally, the sensitivity \(\delta^{(l)}_j\) can be computed
using the backpropagation algorithm by recursion with the following
formula

\begin{equation}
\delta^{(l)}_j=\theta^\prime(s^{(l)}_j)\sum_{k=1}^{d^{(l+1)}} w_{jk}^{(l+1)}\delta^{(l+1)}_k\,.
\end{equation} where
\(\delta^{(L)}=2(x^{(L)}-y)\theta^\prime(s^{(L)})\).

There are \(L-1\) sensitivities to compute, for a total number of
additional operations equal to

\begin{equation}
N_b=\sum_{l=1}^{L-1} d^{(l)}d^{(l+1)}\,.
\end{equation}

Hence, the total number of multiplications needed to carry out a single
iteration of backpropagation is \(N_{tot}=2N_w+N_b\), being \(N_w\) the
number of weights of the network and \(N_b\) the number of operations
needed to compute the sensitivities.

In the example given, \(d^{(0)}=5\), \(d^{(1)}=3\), \(d^{(2)}=1\).
Hence, \(N_w=22\), \(N_b=3\) and \(N_{tot}=47\).

    \hypertarget{problem-9}{%
\section{Problem 9}\label{problem-9}}

\hypertarget{answer-a-46}{%
\subsection{\texorpdfstring{Answer: {[}a{]}
\(46\)}{Answer: {[}a{]} 46}}\label{answer-a-46}}

\hypertarget{derivation}{%
\subsection{Derivation:}\label{derivation}}

To minimize the number of weights, we want to reduce the number of
connections. In order to achieve that, we consider the case in which the
36 units are equally distributed in 18 hidden layers of 2 units each.

The following code allows as to compute the number of weights in this
case, which turns out to be 46.

    \begin{tcolorbox}[breakable, size=fbox, boxrule=1pt, pad at break*=1mm,colback=cellbackground, colframe=cellborder]
\prompt{In}{incolor}{195}{\boxspacing}
\begin{Verbatim}[commandchars=\\\{\}]
\PY{k}{def} \PY{n+nf}{nweights}\PY{p}{(}\PY{n}{d}\PY{p}{)}\PY{p}{:}
    \PY{n}{w}\PY{o}{=}\PY{l+m+mi}{0}
    \PY{k}{for} \PY{n}{i} \PY{o+ow}{in} \PY{n+nb}{range}\PY{p}{(}\PY{n+nb}{len}\PY{p}{(}\PY{n}{d}\PY{p}{)}\PY{o}{\PYZhy{}}\PY{l+m+mi}{1}\PY{p}{)}\PY{p}{:}
        \PY{n}{w}\PY{o}{+}\PY{o}{=}\PY{p}{(}\PY{n}{d}\PY{p}{[}\PY{n}{i}\PY{p}{]}\PY{o}{+}\PY{l+m+mi}{1}\PY{p}{)}\PY{o}{*}\PY{n}{d}\PY{p}{[}\PY{n}{i}\PY{o}{+}\PY{l+m+mi}{1}\PY{p}{]}
    \PY{k}{return} \PY{n}{w}

\PY{n}{d}\PY{o}{=}\PY{p}{[}\PY{l+m+mi}{9}\PY{p}{]}
\PY{k}{for} \PY{n}{i} \PY{o+ow}{in} \PY{n+nb}{range}\PY{p}{(}\PY{l+m+mi}{19}\PY{p}{)}\PY{p}{:}
    \PY{n}{d}\PY{o}{.}\PY{n}{append}\PY{p}{(}\PY{l+m+mi}{1}\PY{p}{)}

\PY{n+nb}{print}\PY{p}{(}\PY{n}{nweights}\PY{p}{(}\PY{n}{d}\PY{p}{)}\PY{p}{)}
\end{Verbatim}
\end{tcolorbox}

    \begin{Verbatim}[commandchars=\\\{\}]
46
    \end{Verbatim}

    \hypertarget{problem-10}{%
\section{Problem 10}\label{problem-10}}

\hypertarget{answer-e-510}{%
\subsection{\texorpdfstring{Answer: {[}e{]}
\(510\)}{Answer: {[}e{]} 510}}\label{answer-e-510}}

\hypertarget{derivation}{%
\subsection{Derivation:}\label{derivation}}

To solve this problem, we will analytically compute the number of
weights for all the possible configurations of one, two or three hidden
layers.

For one inner layer of 36 units,
\(N^{(1)}_{max}=10\times35+36\times1=386\).

For two inner layers, let \(d^{(1)}=x\) be the dimension of the first
layer (i.e.~the number of units minus one). The number of weights is
then an analytic function of \(x\), specifically:

\begin{equation}
N^{(2)}(x)=10x+(x+1)(36-x-2)+(36-x-1)=69 + 42 x - x^2\,.
\end{equation}

This function is maximized for \(x_{max}=21\), where
\(N^{(2)}(x_{max})=N^{(2)}_{max}=510\), which correspond to a first
layer of 22 units followed by a second layer of 14 units.

For three layers, let \(d^{(1)}=x\) and \(d^{(2)}=y\) be the dimension
of the first and second layer respectively (i.e.~the number of units
minus one). The number of weights now is the following two-dimensional
surface:

\begin{equation}
N^{(3)}(x)=10x+(x+1)y+(y+1)(36-x-y-3)+(36-x-y-2)=67 + 8 x + 32 y - y^2\,.
\end{equation}

Using the script below, we obtain that the function is maximized for
\(x_{max}=20\) and \(y_{max}=12\), giving \(N^{(3)}_{max}=467\).

Hence, the maximum number of weights are \(N^{(2)}_{max}=510\).

    \begin{tcolorbox}[breakable, size=fbox, boxrule=1pt, pad at break*=1mm,colback=cellbackground, colframe=cellborder]
\prompt{In}{incolor}{194}{\boxspacing}
\begin{Verbatim}[commandchars=\\\{\}]
\PY{n}{Nmax}\PY{o}{=}\PY{l+m+mi}{0}
\PY{n}{dmax}\PY{o}{=}\PY{p}{[}\PY{p}{]}

\PY{k}{for} \PY{n}{i} \PY{o+ow}{in} \PY{n+nb}{range}\PY{p}{(}\PY{l+m+mi}{31}\PY{p}{)}\PY{p}{:}
    \PY{k}{for} \PY{n}{j} \PY{o+ow}{in} \PY{n+nb}{range}\PY{p}{(}\PY{l+m+mi}{33}\PY{o}{\PYZhy{}}\PY{n}{i}\PY{p}{)}\PY{p}{:}
        \PY{n}{k}\PY{o}{=}\PY{l+m+mi}{36}\PY{o}{\PYZhy{}}\PY{p}{(}\PY{n}{i}\PY{o}{+}\PY{n}{j}\PY{o}{+}\PY{l+m+mi}{2}\PY{p}{)}\PY{o}{\PYZhy{}}\PY{l+m+mi}{1}
        \PY{n}{d}\PY{o}{=}\PY{p}{[}\PY{l+m+mi}{9}\PY{p}{,}\PY{n}{i}\PY{p}{,}\PY{n}{j}\PY{p}{,}\PY{n}{k}\PY{p}{,}\PY{l+m+mi}{1}\PY{p}{]}
        \PY{k}{if}\PY{p}{(}\PY{n}{nweights}\PY{p}{(}\PY{n}{d}\PY{p}{)}\PY{o}{\PYZgt{}}\PY{n}{Nmax}\PY{p}{)}\PY{p}{:}
            \PY{n}{Nmax}\PY{o}{=}\PY{n}{nweights}\PY{p}{(}\PY{n}{d}\PY{p}{)}
            \PY{n}{dmax}\PY{o}{=}\PY{n}{d}

\PY{n+nb}{print}\PY{p}{(}\PY{n}{f}\PY{l+s+s1}{\PYZsq{}}\PY{l+s+s1}{N3max=}\PY{l+s+si}{\PYZob{}Nmax\PYZcb{}}\PY{l+s+s1}{ found at d=}\PY{l+s+si}{\PYZob{}dmax\PYZcb{}}\PY{l+s+s1}{!}\PY{l+s+s1}{\PYZsq{}}\PY{p}{)}
\end{Verbatim}
\end{tcolorbox}

    \begin{Verbatim}[commandchars=\\\{\}]
N3max=467 found at d=[9, 20, 12, 1, 1]!
    \end{Verbatim}

    \begin{tcolorbox}[breakable, size=fbox, boxrule=1pt, pad at break*=1mm,colback=cellbackground, colframe=cellborder]
\prompt{In}{incolor}{ }{\boxspacing}
\begin{Verbatim}[commandchars=\\\{\}]

\end{Verbatim}
\end{tcolorbox}


    % Add a bibliography block to the postdoc
    
    
    
\end{document}
