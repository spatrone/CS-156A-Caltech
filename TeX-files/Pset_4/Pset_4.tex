\documentclass[11pt]{article}

    \usepackage[breakable]{tcolorbox}
    \usepackage{parskip} % Stop auto-indenting (to mimic markdown behaviour)
    
    \usepackage{iftex}
    \ifPDFTeX
    	\usepackage[T1]{fontenc}
    	\usepackage{mathpazo}
    \else
    	\usepackage{fontspec}
    \fi

    % Basic figure setup, for now with no caption control since it's done
    % automatically by Pandoc (which extracts ![](path) syntax from Markdown).
    \usepackage{graphicx}
    % Maintain compatibility with old templates. Remove in nbconvert 6.0
    \let\Oldincludegraphics\includegraphics
    % Ensure that by default, figures have no caption (until we provide a
    % proper Figure object with a Caption API and a way to capture that
    % in the conversion process - todo).
    \usepackage{caption}
    \DeclareCaptionFormat{nocaption}{}
    \captionsetup{format=nocaption,aboveskip=0pt,belowskip=0pt}

    \usepackage[Export]{adjustbox} % Used to constrain images to a maximum size
    \adjustboxset{max size={0.9\linewidth}{0.9\paperheight}}
    \usepackage{float}
    \floatplacement{figure}{H} % forces figures to be placed at the correct location
    \usepackage{xcolor} % Allow colors to be defined
    \usepackage{enumerate} % Needed for markdown enumerations to work
    \usepackage{geometry} % Used to adjust the document margins
    \usepackage{amsmath} % Equations
    \usepackage{amssymb} % Equations
    \usepackage{textcomp} % defines textquotesingle
    % Hack from http://tex.stackexchange.com/a/47451/13684:
    \AtBeginDocument{%
        \def\PYZsq{\textquotesingle}% Upright quotes in Pygmentized code
    }
    \usepackage{upquote} % Upright quotes for verbatim code
    \usepackage{eurosym} % defines \euro
    \usepackage[mathletters]{ucs} % Extended unicode (utf-8) support
    \usepackage{fancyvrb} % verbatim replacement that allows latex
    \usepackage{grffile} % extends the file name processing of package graphics 
                         % to support a larger range
    \makeatletter % fix for grffile with XeLaTeX
    \def\Gread@@xetex#1{%
      \IfFileExists{"\Gin@base".bb}%
      {\Gread@eps{\Gin@base.bb}}%
      {\Gread@@xetex@aux#1}%
    }
    \makeatother

    % The hyperref package gives us a pdf with properly built
    % internal navigation ('pdf bookmarks' for the table of contents,
    % internal cross-reference links, web links for URLs, etc.)
    \usepackage{hyperref}
    % The default LaTeX title has an obnoxious amount of whitespace. By default,
    % titling removes some of it. It also provides customization options.
    \usepackage{titling}
    \usepackage{longtable} % longtable support required by pandoc >1.10
    \usepackage{booktabs}  % table support for pandoc > 1.12.2
    \usepackage[inline]{enumitem} % IRkernel/repr support (it uses the enumerate* environment)
    \usepackage[normalem]{ulem} % ulem is needed to support strikethroughs (\sout)
                                % normalem makes italics be italics, not underlines
    \usepackage{mathrsfs}
    

    
    % Colors for the hyperref package
    \definecolor{urlcolor}{rgb}{0,.145,.698}
    \definecolor{linkcolor}{rgb}{.71,0.21,0.01}
    \definecolor{citecolor}{rgb}{.12,.54,.11}

    % ANSI colors
    \definecolor{ansi-black}{HTML}{3E424D}
    \definecolor{ansi-black-intense}{HTML}{282C36}
    \definecolor{ansi-red}{HTML}{E75C58}
    \definecolor{ansi-red-intense}{HTML}{B22B31}
    \definecolor{ansi-green}{HTML}{00A250}
    \definecolor{ansi-green-intense}{HTML}{007427}
    \definecolor{ansi-yellow}{HTML}{DDB62B}
    \definecolor{ansi-yellow-intense}{HTML}{B27D12}
    \definecolor{ansi-blue}{HTML}{208FFB}
    \definecolor{ansi-blue-intense}{HTML}{0065CA}
    \definecolor{ansi-magenta}{HTML}{D160C4}
    \definecolor{ansi-magenta-intense}{HTML}{A03196}
    \definecolor{ansi-cyan}{HTML}{60C6C8}
    \definecolor{ansi-cyan-intense}{HTML}{258F8F}
    \definecolor{ansi-white}{HTML}{C5C1B4}
    \definecolor{ansi-white-intense}{HTML}{A1A6B2}
    \definecolor{ansi-default-inverse-fg}{HTML}{FFFFFF}
    \definecolor{ansi-default-inverse-bg}{HTML}{000000}

    % commands and environments needed by pandoc snippets
    % extracted from the output of `pandoc -s`
    \providecommand{\tightlist}{%
      \setlength{\itemsep}{0pt}\setlength{\parskip}{0pt}}
    \DefineVerbatimEnvironment{Highlighting}{Verbatim}{commandchars=\\\{\}}
    % Add ',fontsize=\small' for more characters per line
    \newenvironment{Shaded}{}{}
    \newcommand{\KeywordTok}[1]{\textcolor[rgb]{0.00,0.44,0.13}{\textbf{{#1}}}}
    \newcommand{\DataTypeTok}[1]{\textcolor[rgb]{0.56,0.13,0.00}{{#1}}}
    \newcommand{\DecValTok}[1]{\textcolor[rgb]{0.25,0.63,0.44}{{#1}}}
    \newcommand{\BaseNTok}[1]{\textcolor[rgb]{0.25,0.63,0.44}{{#1}}}
    \newcommand{\FloatTok}[1]{\textcolor[rgb]{0.25,0.63,0.44}{{#1}}}
    \newcommand{\CharTok}[1]{\textcolor[rgb]{0.25,0.44,0.63}{{#1}}}
    \newcommand{\StringTok}[1]{\textcolor[rgb]{0.25,0.44,0.63}{{#1}}}
    \newcommand{\CommentTok}[1]{\textcolor[rgb]{0.38,0.63,0.69}{\textit{{#1}}}}
    \newcommand{\OtherTok}[1]{\textcolor[rgb]{0.00,0.44,0.13}{{#1}}}
    \newcommand{\AlertTok}[1]{\textcolor[rgb]{1.00,0.00,0.00}{\textbf{{#1}}}}
    \newcommand{\FunctionTok}[1]{\textcolor[rgb]{0.02,0.16,0.49}{{#1}}}
    \newcommand{\RegionMarkerTok}[1]{{#1}}
    \newcommand{\ErrorTok}[1]{\textcolor[rgb]{1.00,0.00,0.00}{\textbf{{#1}}}}
    \newcommand{\NormalTok}[1]{{#1}}
    
    % Additional commands for more recent versions of Pandoc
    \newcommand{\ConstantTok}[1]{\textcolor[rgb]{0.53,0.00,0.00}{{#1}}}
    \newcommand{\SpecialCharTok}[1]{\textcolor[rgb]{0.25,0.44,0.63}{{#1}}}
    \newcommand{\VerbatimStringTok}[1]{\textcolor[rgb]{0.25,0.44,0.63}{{#1}}}
    \newcommand{\SpecialStringTok}[1]{\textcolor[rgb]{0.73,0.40,0.53}{{#1}}}
    \newcommand{\ImportTok}[1]{{#1}}
    \newcommand{\DocumentationTok}[1]{\textcolor[rgb]{0.73,0.13,0.13}{\textit{{#1}}}}
    \newcommand{\AnnotationTok}[1]{\textcolor[rgb]{0.38,0.63,0.69}{\textbf{\textit{{#1}}}}}
    \newcommand{\CommentVarTok}[1]{\textcolor[rgb]{0.38,0.63,0.69}{\textbf{\textit{{#1}}}}}
    \newcommand{\VariableTok}[1]{\textcolor[rgb]{0.10,0.09,0.49}{{#1}}}
    \newcommand{\ControlFlowTok}[1]{\textcolor[rgb]{0.00,0.44,0.13}{\textbf{{#1}}}}
    \newcommand{\OperatorTok}[1]{\textcolor[rgb]{0.40,0.40,0.40}{{#1}}}
    \newcommand{\BuiltInTok}[1]{{#1}}
    \newcommand{\ExtensionTok}[1]{{#1}}
    \newcommand{\PreprocessorTok}[1]{\textcolor[rgb]{0.74,0.48,0.00}{{#1}}}
    \newcommand{\AttributeTok}[1]{\textcolor[rgb]{0.49,0.56,0.16}{{#1}}}
    \newcommand{\InformationTok}[1]{\textcolor[rgb]{0.38,0.63,0.69}{\textbf{\textit{{#1}}}}}
    \newcommand{\WarningTok}[1]{\textcolor[rgb]{0.38,0.63,0.69}{\textbf{\textit{{#1}}}}}
    
    
    % Define a nice break command that doesn't care if a line doesn't already
    % exist.
    \def\br{\hspace*{\fill} \\* }
    % Math Jax compatibility definitions
    \def\gt{>}
    \def\lt{<}
    \let\Oldtex\TeX
    \let\Oldlatex\LaTeX
    \renewcommand{\TeX}{\textrm{\Oldtex}}
    \renewcommand{\LaTeX}{\textrm{\Oldlatex}}
    % Document parameters
    % Document title
    \title{Pset\_4}
    
    
    
    
    
% Pygments definitions
\makeatletter
\def\PY@reset{\let\PY@it=\relax \let\PY@bf=\relax%
    \let\PY@ul=\relax \let\PY@tc=\relax%
    \let\PY@bc=\relax \let\PY@ff=\relax}
\def\PY@tok#1{\csname PY@tok@#1\endcsname}
\def\PY@toks#1+{\ifx\relax#1\empty\else%
    \PY@tok{#1}\expandafter\PY@toks\fi}
\def\PY@do#1{\PY@bc{\PY@tc{\PY@ul{%
    \PY@it{\PY@bf{\PY@ff{#1}}}}}}}
\def\PY#1#2{\PY@reset\PY@toks#1+\relax+\PY@do{#2}}

\expandafter\def\csname PY@tok@w\endcsname{\def\PY@tc##1{\textcolor[rgb]{0.73,0.73,0.73}{##1}}}
\expandafter\def\csname PY@tok@c\endcsname{\let\PY@it=\textit\def\PY@tc##1{\textcolor[rgb]{0.25,0.50,0.50}{##1}}}
\expandafter\def\csname PY@tok@cp\endcsname{\def\PY@tc##1{\textcolor[rgb]{0.74,0.48,0.00}{##1}}}
\expandafter\def\csname PY@tok@k\endcsname{\let\PY@bf=\textbf\def\PY@tc##1{\textcolor[rgb]{0.00,0.50,0.00}{##1}}}
\expandafter\def\csname PY@tok@kp\endcsname{\def\PY@tc##1{\textcolor[rgb]{0.00,0.50,0.00}{##1}}}
\expandafter\def\csname PY@tok@kt\endcsname{\def\PY@tc##1{\textcolor[rgb]{0.69,0.00,0.25}{##1}}}
\expandafter\def\csname PY@tok@o\endcsname{\def\PY@tc##1{\textcolor[rgb]{0.40,0.40,0.40}{##1}}}
\expandafter\def\csname PY@tok@ow\endcsname{\let\PY@bf=\textbf\def\PY@tc##1{\textcolor[rgb]{0.67,0.13,1.00}{##1}}}
\expandafter\def\csname PY@tok@nb\endcsname{\def\PY@tc##1{\textcolor[rgb]{0.00,0.50,0.00}{##1}}}
\expandafter\def\csname PY@tok@nf\endcsname{\def\PY@tc##1{\textcolor[rgb]{0.00,0.00,1.00}{##1}}}
\expandafter\def\csname PY@tok@nc\endcsname{\let\PY@bf=\textbf\def\PY@tc##1{\textcolor[rgb]{0.00,0.00,1.00}{##1}}}
\expandafter\def\csname PY@tok@nn\endcsname{\let\PY@bf=\textbf\def\PY@tc##1{\textcolor[rgb]{0.00,0.00,1.00}{##1}}}
\expandafter\def\csname PY@tok@ne\endcsname{\let\PY@bf=\textbf\def\PY@tc##1{\textcolor[rgb]{0.82,0.25,0.23}{##1}}}
\expandafter\def\csname PY@tok@nv\endcsname{\def\PY@tc##1{\textcolor[rgb]{0.10,0.09,0.49}{##1}}}
\expandafter\def\csname PY@tok@no\endcsname{\def\PY@tc##1{\textcolor[rgb]{0.53,0.00,0.00}{##1}}}
\expandafter\def\csname PY@tok@nl\endcsname{\def\PY@tc##1{\textcolor[rgb]{0.63,0.63,0.00}{##1}}}
\expandafter\def\csname PY@tok@ni\endcsname{\let\PY@bf=\textbf\def\PY@tc##1{\textcolor[rgb]{0.60,0.60,0.60}{##1}}}
\expandafter\def\csname PY@tok@na\endcsname{\def\PY@tc##1{\textcolor[rgb]{0.49,0.56,0.16}{##1}}}
\expandafter\def\csname PY@tok@nt\endcsname{\let\PY@bf=\textbf\def\PY@tc##1{\textcolor[rgb]{0.00,0.50,0.00}{##1}}}
\expandafter\def\csname PY@tok@nd\endcsname{\def\PY@tc##1{\textcolor[rgb]{0.67,0.13,1.00}{##1}}}
\expandafter\def\csname PY@tok@s\endcsname{\def\PY@tc##1{\textcolor[rgb]{0.73,0.13,0.13}{##1}}}
\expandafter\def\csname PY@tok@sd\endcsname{\let\PY@it=\textit\def\PY@tc##1{\textcolor[rgb]{0.73,0.13,0.13}{##1}}}
\expandafter\def\csname PY@tok@si\endcsname{\let\PY@bf=\textbf\def\PY@tc##1{\textcolor[rgb]{0.73,0.40,0.53}{##1}}}
\expandafter\def\csname PY@tok@se\endcsname{\let\PY@bf=\textbf\def\PY@tc##1{\textcolor[rgb]{0.73,0.40,0.13}{##1}}}
\expandafter\def\csname PY@tok@sr\endcsname{\def\PY@tc##1{\textcolor[rgb]{0.73,0.40,0.53}{##1}}}
\expandafter\def\csname PY@tok@ss\endcsname{\def\PY@tc##1{\textcolor[rgb]{0.10,0.09,0.49}{##1}}}
\expandafter\def\csname PY@tok@sx\endcsname{\def\PY@tc##1{\textcolor[rgb]{0.00,0.50,0.00}{##1}}}
\expandafter\def\csname PY@tok@m\endcsname{\def\PY@tc##1{\textcolor[rgb]{0.40,0.40,0.40}{##1}}}
\expandafter\def\csname PY@tok@gh\endcsname{\let\PY@bf=\textbf\def\PY@tc##1{\textcolor[rgb]{0.00,0.00,0.50}{##1}}}
\expandafter\def\csname PY@tok@gu\endcsname{\let\PY@bf=\textbf\def\PY@tc##1{\textcolor[rgb]{0.50,0.00,0.50}{##1}}}
\expandafter\def\csname PY@tok@gd\endcsname{\def\PY@tc##1{\textcolor[rgb]{0.63,0.00,0.00}{##1}}}
\expandafter\def\csname PY@tok@gi\endcsname{\def\PY@tc##1{\textcolor[rgb]{0.00,0.63,0.00}{##1}}}
\expandafter\def\csname PY@tok@gr\endcsname{\def\PY@tc##1{\textcolor[rgb]{1.00,0.00,0.00}{##1}}}
\expandafter\def\csname PY@tok@ge\endcsname{\let\PY@it=\textit}
\expandafter\def\csname PY@tok@gs\endcsname{\let\PY@bf=\textbf}
\expandafter\def\csname PY@tok@gp\endcsname{\let\PY@bf=\textbf\def\PY@tc##1{\textcolor[rgb]{0.00,0.00,0.50}{##1}}}
\expandafter\def\csname PY@tok@go\endcsname{\def\PY@tc##1{\textcolor[rgb]{0.53,0.53,0.53}{##1}}}
\expandafter\def\csname PY@tok@gt\endcsname{\def\PY@tc##1{\textcolor[rgb]{0.00,0.27,0.87}{##1}}}
\expandafter\def\csname PY@tok@err\endcsname{\def\PY@bc##1{\setlength{\fboxsep}{0pt}\fcolorbox[rgb]{1.00,0.00,0.00}{1,1,1}{\strut ##1}}}
\expandafter\def\csname PY@tok@kc\endcsname{\let\PY@bf=\textbf\def\PY@tc##1{\textcolor[rgb]{0.00,0.50,0.00}{##1}}}
\expandafter\def\csname PY@tok@kd\endcsname{\let\PY@bf=\textbf\def\PY@tc##1{\textcolor[rgb]{0.00,0.50,0.00}{##1}}}
\expandafter\def\csname PY@tok@kn\endcsname{\let\PY@bf=\textbf\def\PY@tc##1{\textcolor[rgb]{0.00,0.50,0.00}{##1}}}
\expandafter\def\csname PY@tok@kr\endcsname{\let\PY@bf=\textbf\def\PY@tc##1{\textcolor[rgb]{0.00,0.50,0.00}{##1}}}
\expandafter\def\csname PY@tok@bp\endcsname{\def\PY@tc##1{\textcolor[rgb]{0.00,0.50,0.00}{##1}}}
\expandafter\def\csname PY@tok@fm\endcsname{\def\PY@tc##1{\textcolor[rgb]{0.00,0.00,1.00}{##1}}}
\expandafter\def\csname PY@tok@vc\endcsname{\def\PY@tc##1{\textcolor[rgb]{0.10,0.09,0.49}{##1}}}
\expandafter\def\csname PY@tok@vg\endcsname{\def\PY@tc##1{\textcolor[rgb]{0.10,0.09,0.49}{##1}}}
\expandafter\def\csname PY@tok@vi\endcsname{\def\PY@tc##1{\textcolor[rgb]{0.10,0.09,0.49}{##1}}}
\expandafter\def\csname PY@tok@vm\endcsname{\def\PY@tc##1{\textcolor[rgb]{0.10,0.09,0.49}{##1}}}
\expandafter\def\csname PY@tok@sa\endcsname{\def\PY@tc##1{\textcolor[rgb]{0.73,0.13,0.13}{##1}}}
\expandafter\def\csname PY@tok@sb\endcsname{\def\PY@tc##1{\textcolor[rgb]{0.73,0.13,0.13}{##1}}}
\expandafter\def\csname PY@tok@sc\endcsname{\def\PY@tc##1{\textcolor[rgb]{0.73,0.13,0.13}{##1}}}
\expandafter\def\csname PY@tok@dl\endcsname{\def\PY@tc##1{\textcolor[rgb]{0.73,0.13,0.13}{##1}}}
\expandafter\def\csname PY@tok@s2\endcsname{\def\PY@tc##1{\textcolor[rgb]{0.73,0.13,0.13}{##1}}}
\expandafter\def\csname PY@tok@sh\endcsname{\def\PY@tc##1{\textcolor[rgb]{0.73,0.13,0.13}{##1}}}
\expandafter\def\csname PY@tok@s1\endcsname{\def\PY@tc##1{\textcolor[rgb]{0.73,0.13,0.13}{##1}}}
\expandafter\def\csname PY@tok@mb\endcsname{\def\PY@tc##1{\textcolor[rgb]{0.40,0.40,0.40}{##1}}}
\expandafter\def\csname PY@tok@mf\endcsname{\def\PY@tc##1{\textcolor[rgb]{0.40,0.40,0.40}{##1}}}
\expandafter\def\csname PY@tok@mh\endcsname{\def\PY@tc##1{\textcolor[rgb]{0.40,0.40,0.40}{##1}}}
\expandafter\def\csname PY@tok@mi\endcsname{\def\PY@tc##1{\textcolor[rgb]{0.40,0.40,0.40}{##1}}}
\expandafter\def\csname PY@tok@il\endcsname{\def\PY@tc##1{\textcolor[rgb]{0.40,0.40,0.40}{##1}}}
\expandafter\def\csname PY@tok@mo\endcsname{\def\PY@tc##1{\textcolor[rgb]{0.40,0.40,0.40}{##1}}}
\expandafter\def\csname PY@tok@ch\endcsname{\let\PY@it=\textit\def\PY@tc##1{\textcolor[rgb]{0.25,0.50,0.50}{##1}}}
\expandafter\def\csname PY@tok@cm\endcsname{\let\PY@it=\textit\def\PY@tc##1{\textcolor[rgb]{0.25,0.50,0.50}{##1}}}
\expandafter\def\csname PY@tok@cpf\endcsname{\let\PY@it=\textit\def\PY@tc##1{\textcolor[rgb]{0.25,0.50,0.50}{##1}}}
\expandafter\def\csname PY@tok@c1\endcsname{\let\PY@it=\textit\def\PY@tc##1{\textcolor[rgb]{0.25,0.50,0.50}{##1}}}
\expandafter\def\csname PY@tok@cs\endcsname{\let\PY@it=\textit\def\PY@tc##1{\textcolor[rgb]{0.25,0.50,0.50}{##1}}}

\def\PYZbs{\char`\\}
\def\PYZus{\char`\_}
\def\PYZob{\char`\{}
\def\PYZcb{\char`\}}
\def\PYZca{\char`\^}
\def\PYZam{\char`\&}
\def\PYZlt{\char`\<}
\def\PYZgt{\char`\>}
\def\PYZsh{\char`\#}
\def\PYZpc{\char`\%}
\def\PYZdl{\char`\$}
\def\PYZhy{\char`\-}
\def\PYZsq{\char`\'}
\def\PYZdq{\char`\"}
\def\PYZti{\char`\~}
% for compatibility with earlier versions
\def\PYZat{@}
\def\PYZlb{[}
\def\PYZrb{]}
\makeatother


    % For linebreaks inside Verbatim environment from package fancyvrb. 
    \makeatletter
        \newbox\Wrappedcontinuationbox 
        \newbox\Wrappedvisiblespacebox 
        \newcommand*\Wrappedvisiblespace {\textcolor{red}{\textvisiblespace}} 
        \newcommand*\Wrappedcontinuationsymbol {\textcolor{red}{\llap{\tiny$\m@th\hookrightarrow$}}} 
        \newcommand*\Wrappedcontinuationindent {3ex } 
        \newcommand*\Wrappedafterbreak {\kern\Wrappedcontinuationindent\copy\Wrappedcontinuationbox} 
        % Take advantage of the already applied Pygments mark-up to insert 
        % potential linebreaks for TeX processing. 
        %        {, <, #, %, $, ' and ": go to next line. 
        %        _, }, ^, &, >, - and ~: stay at end of broken line. 
        % Use of \textquotesingle for straight quote. 
        \newcommand*\Wrappedbreaksatspecials {% 
            \def\PYGZus{\discretionary{\char`\_}{\Wrappedafterbreak}{\char`\_}}% 
            \def\PYGZob{\discretionary{}{\Wrappedafterbreak\char`\{}{\char`\{}}% 
            \def\PYGZcb{\discretionary{\char`\}}{\Wrappedafterbreak}{\char`\}}}% 
            \def\PYGZca{\discretionary{\char`\^}{\Wrappedafterbreak}{\char`\^}}% 
            \def\PYGZam{\discretionary{\char`\&}{\Wrappedafterbreak}{\char`\&}}% 
            \def\PYGZlt{\discretionary{}{\Wrappedafterbreak\char`\<}{\char`\<}}% 
            \def\PYGZgt{\discretionary{\char`\>}{\Wrappedafterbreak}{\char`\>}}% 
            \def\PYGZsh{\discretionary{}{\Wrappedafterbreak\char`\#}{\char`\#}}% 
            \def\PYGZpc{\discretionary{}{\Wrappedafterbreak\char`\%}{\char`\%}}% 
            \def\PYGZdl{\discretionary{}{\Wrappedafterbreak\char`\$}{\char`\$}}% 
            \def\PYGZhy{\discretionary{\char`\-}{\Wrappedafterbreak}{\char`\-}}% 
            \def\PYGZsq{\discretionary{}{\Wrappedafterbreak\textquotesingle}{\textquotesingle}}% 
            \def\PYGZdq{\discretionary{}{\Wrappedafterbreak\char`\"}{\char`\"}}% 
            \def\PYGZti{\discretionary{\char`\~}{\Wrappedafterbreak}{\char`\~}}% 
        } 
        % Some characters . , ; ? ! / are not pygmentized. 
        % This macro makes them "active" and they will insert potential linebreaks 
        \newcommand*\Wrappedbreaksatpunct {% 
            \lccode`\~`\.\lowercase{\def~}{\discretionary{\hbox{\char`\.}}{\Wrappedafterbreak}{\hbox{\char`\.}}}% 
            \lccode`\~`\,\lowercase{\def~}{\discretionary{\hbox{\char`\,}}{\Wrappedafterbreak}{\hbox{\char`\,}}}% 
            \lccode`\~`\;\lowercase{\def~}{\discretionary{\hbox{\char`\;}}{\Wrappedafterbreak}{\hbox{\char`\;}}}% 
            \lccode`\~`\:\lowercase{\def~}{\discretionary{\hbox{\char`\:}}{\Wrappedafterbreak}{\hbox{\char`\:}}}% 
            \lccode`\~`\?\lowercase{\def~}{\discretionary{\hbox{\char`\?}}{\Wrappedafterbreak}{\hbox{\char`\?}}}% 
            \lccode`\~`\!\lowercase{\def~}{\discretionary{\hbox{\char`\!}}{\Wrappedafterbreak}{\hbox{\char`\!}}}% 
            \lccode`\~`\/\lowercase{\def~}{\discretionary{\hbox{\char`\/}}{\Wrappedafterbreak}{\hbox{\char`\/}}}% 
            \catcode`\.\active
            \catcode`\,\active 
            \catcode`\;\active
            \catcode`\:\active
            \catcode`\?\active
            \catcode`\!\active
            \catcode`\/\active 
            \lccode`\~`\~ 	
        }
    \makeatother

    \let\OriginalVerbatim=\Verbatim
    \makeatletter
    \renewcommand{\Verbatim}[1][1]{%
        %\parskip\z@skip
        \sbox\Wrappedcontinuationbox {\Wrappedcontinuationsymbol}%
        \sbox\Wrappedvisiblespacebox {\FV@SetupFont\Wrappedvisiblespace}%
        \def\FancyVerbFormatLine ##1{\hsize\linewidth
            \vtop{\raggedright\hyphenpenalty\z@\exhyphenpenalty\z@
                \doublehyphendemerits\z@\finalhyphendemerits\z@
                \strut ##1\strut}%
        }%
        % If the linebreak is at a space, the latter will be displayed as visible
        % space at end of first line, and a continuation symbol starts next line.
        % Stretch/shrink are however usually zero for typewriter font.
        \def\FV@Space {%
            \nobreak\hskip\z@ plus\fontdimen3\font minus\fontdimen4\font
            \discretionary{\copy\Wrappedvisiblespacebox}{\Wrappedafterbreak}
            {\kern\fontdimen2\font}%
        }%
        
        % Allow breaks at special characters using \PYG... macros.
        \Wrappedbreaksatspecials
        % Breaks at punctuation characters . , ; ? ! and / need catcode=\active 	
        \OriginalVerbatim[#1,codes*=\Wrappedbreaksatpunct]%
    }
    \makeatother

    % Exact colors from NB
    \definecolor{incolor}{HTML}{303F9F}
    \definecolor{outcolor}{HTML}{D84315}
    \definecolor{cellborder}{HTML}{CFCFCF}
    \definecolor{cellbackground}{HTML}{F7F7F7}
    
    % prompt
    \makeatletter
    \newcommand{\boxspacing}{\kern\kvtcb@left@rule\kern\kvtcb@boxsep}
    \makeatother
    \newcommand{\prompt}[4]{
        \ttfamily\llap{{\color{#2}[#3]:\hspace{3pt}#4}}\vspace{-\baselineskip}
    }
    

    
    % Prevent overflowing lines due to hard-to-break entities
    \sloppy 
    % Setup hyperref package
    \hypersetup{
      breaklinks=true,  % so long urls are correctly broken across lines
      colorlinks=true,
      urlcolor=urlcolor,
      linkcolor=linkcolor,
      citecolor=citecolor,
      }
    % Slightly bigger margins than the latex defaults
    
    \geometry{verbose,tmargin=1in,bmargin=1in,lmargin=1in,rmargin=1in}
    
\makeatletter
\renewcommand{\@seccntformat}[1]{}
\makeatother  

\begin{document}
    \title{CS 156a - Problem Set 4}
    \author{Samuel Patrone, 2140749}
    \maketitle
    

The following notebook is publicly available at the following
\href{https://github.com/spatrone/CS156A-Caltech.git}{link}.

\tableofcontents

    \hypertarget{problem-1}{%
\section{Problem 1}\label{problem-1}}

\hypertarget{answer-d-460000}{%
\subsection{\texorpdfstring{Answer: {[}d{]}
\(460,000\)}{Answer: {[}d{]} 460,000}}\label{answer-d-460000}}

\hypertarget{derivation}{%
\subsection{Derivation:}\label{derivation}}

The VC generalization bound states that, with probability at least
\(1-\delta\):

\[
E_{out}(g)\le E_{in}(g) + \Omega(N,\mathcal{H},\delta)\,,
\]

where \(\Omega(N,\mathcal{H},\delta)\) is the generalization error that
can be written as a function of the number of samples \(N\), the growth
function \(m_{\mathcal{H}}(N)\) for the hypothesis set \(\mathcal{H}\)
and the confidence parameter \(\delta\) as the following

\[
\Omega(N,\mathcal{H},\delta)=\sqrt{\frac{1}{8N}\log{\left(\frac{4 m_{\mathcal{H}}(2N)}{\delta}\right)}}\le \sqrt{\frac{1}{8N}\log{\left(\frac{4 (2N)^{d_{VC}}}{\delta}\right)}}\,.
\]

For the growth function \(m_{\mathcal{H}}(N)\) we used the polynomial
bound

\[
m_{\mathcal{H}}(N)\le N^{d_{VC}}
\]

where \(d_{VC}\) is the VC dimension of the hypothesis set
\(\mathcal{H}\).

In the following, a simple code to estimate the required number of
sample \(N\) if we want \(\Omega\) to be a specific value
\texttt{Omega\_value} within an error given by \texttt{eps}.

    \begin{tcolorbox}[breakable, size=fbox, boxrule=1pt, pad at break*=1mm,colback=cellbackground, colframe=cellborder]
\prompt{In}{incolor}{73}{\boxspacing}
\begin{Verbatim}[commandchars=\\\{\}]
\PY{k+kn}{import} \PY{n+nn}{math} \PY{k}{as} \PY{n+nn}{m}
\PY{k+kn}{import} \PY{n+nn}{numpy} \PY{k}{as} \PY{n+nn}{np}

\PY{k}{def} \PY{n+nf}{VC\PYZus{}bound}\PY{p}{(}\PY{n}{N}\PY{p}{,}\PY{n}{d\PYZus{}vc}\PY{p}{,}\PY{n}{delta}\PY{p}{)}\PY{p}{:}
    \PY{k}{return} \PY{n}{m}\PY{o}{.}\PY{n}{sqrt}\PY{p}{(}\PY{l+m+mi}{8}\PY{o}{/}\PY{n}{N}\PY{o}{*}\PY{p}{(}\PY{n}{m}\PY{o}{.}\PY{n}{log}\PY{p}{(}\PY{l+m+mi}{4}\PY{p}{)}\PY{o}{+}\PY{n}{d\PYZus{}vc}\PY{o}{*}\PY{n}{m}\PY{o}{.}\PY{n}{log}\PY{p}{(}\PY{l+m+mi}{2}\PY{o}{*}\PY{n}{N}\PY{p}{)} \PY{o}{\PYZhy{}} \PY{n}{m}\PY{o}{.}\PY{n}{log}\PY{p}{(}\PY{n}{delta}\PY{p}{)}\PY{p}{)}\PY{p}{)}

\PY{k}{def} \PY{n+nf}{find\PYZus{}N}\PY{p}{(}\PY{n}{d\PYZus{}vc}\PY{p}{,}\PY{n}{delta}\PY{p}{,}\PY{n}{Omega\PYZus{}value}\PY{p}{,}\PY{n}{eps}\PY{p}{)}\PY{p}{:}
    \PY{n}{error}\PY{o}{=}\PY{l+m+mi}{1}
    \PY{n}{N}\PY{o}{=}\PY{l+m+mi}{1}
    \PY{k}{while}\PY{p}{(}\PY{n}{error}\PY{o}{\PYZgt{}}\PY{n}{eps}\PY{p}{)}\PY{p}{:}
        \PY{n}{error}\PY{o}{=}\PY{n}{np}\PY{o}{.}\PY{n}{abs}\PY{p}{(}\PY{p}{(}\PY{n}{VC\PYZus{}bound}\PY{p}{(}\PY{n}{N}\PY{p}{,}\PY{n}{d\PYZus{}vc}\PY{p}{,}\PY{n}{delta}\PY{p}{)}\PY{o}{\PYZhy{}}\PY{n}{Omega\PYZus{}value}\PY{p}{)}\PY{o}{/}\PY{n}{Omega\PYZus{}value}\PY{p}{)}
        \PY{k}{if}\PY{p}{(}\PY{n}{error}\PY{o}{\PYZgt{}}\PY{n}{eps}\PY{p}{)}\PY{p}{:} \PY{n}{N}\PY{o}{+}\PY{o}{=}\PY{l+m+mi}{1}
    \PY{k}{return} \PY{n}{N}  

\PY{n}{find\PYZus{}N}\PY{p}{(}\PY{l+m+mi}{10}\PY{p}{,}\PY{l+m+mf}{0.05}\PY{p}{,}\PY{l+m+mf}{0.05}\PY{p}{,}\PY{l+m+mf}{0.00001}\PY{p}{)}
\end{Verbatim}
\end{tcolorbox}

            \begin{tcolorbox}[breakable, size=fbox, boxrule=.5pt, pad at break*=1mm, opacityfill=0]
\prompt{Out}{outcolor}{73}{\boxspacing}
\begin{Verbatim}[commandchars=\\\{\}]
452948
\end{Verbatim}
\end{tcolorbox}
        
    \hypertarget{problem-2}{%
\section{Problem 2}\label{problem-2}}

\hypertarget{answer-d-rm-devroye-epsilon-le-sqrtfrac12nbig4epsilon1epsilonlogfrac4-m_mathcalhn2deltabig}{%
\subsection{\texorpdfstring{Answer: {[}d{]}
\({\rm Devroye}\; \epsilon \le \sqrt{\frac{1}{2N}\big(4\epsilon(1+\epsilon)+\log{\frac{4 m_{\mathcal{H}}(N^2)}{\delta}}\big)}\)}{Answer: {[}d{]} \{\textbackslash{}rm Devroye\}\textbackslash{}; \textbackslash{}epsilon \textbackslash{}le \textbackslash{}sqrt\{\textbackslash{}frac\{1\}\{2N\}\textbackslash{}big(4\textbackslash{}epsilon(1+\textbackslash{}epsilon)+\textbackslash{}log\{\textbackslash{}frac\{4 m\_\{\textbackslash{}mathcal\{H\}\}(N\^{}2)\}\{\textbackslash{}delta\}\}\textbackslash{}big)\}}}\label{answer-d-rm-devroye-epsilon-le-sqrtfrac12nbig4epsilon1epsilonlogfrac4-m_mathcalhn2deltabig}}

\hypertarget{code}{%
\subsection{Code:}\label{code}}

    \begin{tcolorbox}[breakable, size=fbox, boxrule=1pt, pad at break*=1mm,colback=cellbackground, colframe=cellborder]
\prompt{In}{incolor}{171}{\boxspacing}
\begin{Verbatim}[commandchars=\\\{\}]
\PY{k+kn}{import} \PY{n+nn}{matplotlib}\PY{n+nn}{.}\PY{n+nn}{pyplot} \PY{k}{as} \PY{n+nn}{plt}

\PY{k}{def} \PY{n+nf}{Rademacher}\PY{p}{(}\PY{n}{N}\PY{p}{,}\PY{n}{d\PYZus{}vc}\PY{p}{,}\PY{n}{delta}\PY{p}{)}\PY{p}{:}
    \PY{k}{return} \PY{n}{m}\PY{o}{.}\PY{n}{sqrt}\PY{p}{(}\PY{l+m+mi}{2}\PY{o}{/}\PY{n}{N}\PY{o}{*}\PY{p}{(}\PY{n}{m}\PY{o}{.}\PY{n}{log}\PY{p}{(}\PY{l+m+mi}{2}\PY{o}{*}\PY{n}{N}\PY{p}{)}\PY{o}{+}\PY{n}{m}\PY{o}{.}\PY{n}{log}\PY{p}{(}\PY{l+m+mi}{2}\PY{o}{*}\PY{n}{N}\PY{p}{)}\PY{o}{*}\PY{n}{d\PYZus{}vc}\PY{p}{)}\PY{p}{)}\PY{o}{+}\PY{n}{m}\PY{o}{.}\PY{n}{sqrt}\PY{p}{(}\PY{l+m+mi}{2}\PY{o}{/}\PY{n}{N}\PY{o}{*}\PY{n}{m}\PY{o}{.}\PY{n}{log}\PY{p}{(}\PY{l+m+mi}{1}\PY{o}{/}\PY{n}{delta}\PY{p}{)}\PY{p}{)}\PY{o}{+}\PY{l+m+mi}{1}\PY{o}{/}\PY{n}{N}

\PY{k}{def} \PY{n+nf}{Parrondo\PYZus{}Broek\PYZus{}impl}\PY{p}{(}\PY{n}{x}\PY{p}{,}\PY{n}{N}\PY{p}{,}\PY{n}{d\PYZus{}vc}\PY{p}{,}\PY{n}{delta}\PY{p}{)}\PY{p}{:}
    \PY{k}{return} \PY{p}{[}\PY{n}{m}\PY{o}{.}\PY{n}{sqrt}\PY{p}{(}\PY{l+m+mi}{1}\PY{o}{/}\PY{n}{N}\PY{o}{*}\PY{p}{(}\PY{l+m+mi}{2}\PY{o}{*}\PY{n}{x}\PY{p}{[}\PY{n}{i}\PY{p}{]}\PY{o}{+}\PY{p}{(}\PY{n}{m}\PY{o}{.}\PY{n}{log}\PY{p}{(}\PY{l+m+mi}{6}\PY{p}{)}\PY{o}{+}\PY{n}{d\PYZus{}vc}\PY{o}{*}\PY{n}{m}\PY{o}{.}\PY{n}{log}\PY{p}{(}\PY{l+m+mi}{2}\PY{o}{*}\PY{n}{N}\PY{p}{)}\PY{o}{\PYZhy{}}\PY{n}{m}\PY{o}{.}\PY{n}{log}\PY{p}{(}\PY{n}{delta}\PY{p}{)}\PY{p}{)}\PY{p}{)}\PY{p}{)} \PY{k}{for} \PY{n}{i} \PY{o+ow}{in} \PY{n+nb}{range}\PY{p}{(}\PY{n+nb}{len}\PY{p}{(}\PY{n}{x}\PY{p}{)}\PY{p}{)}\PY{p}{]}

\PY{k}{def} \PY{n+nf}{Devroye\PYZus{}impl}\PY{p}{(}\PY{n}{x}\PY{p}{,}\PY{n}{N}\PY{p}{,}\PY{n}{d\PYZus{}vc}\PY{p}{,}\PY{n}{delta}\PY{p}{)}\PY{p}{:}
    \PY{k}{return} \PY{p}{[}\PY{n}{m}\PY{o}{.}\PY{n}{sqrt}\PY{p}{(}\PY{l+m+mi}{1}\PY{o}{/}\PY{p}{(}\PY{l+m+mi}{2}\PY{o}{*}\PY{n}{N}\PY{p}{)}\PY{o}{*}\PY{p}{(}\PY{l+m+mi}{4}\PY{o}{*}\PY{n}{x}\PY{p}{[}\PY{n}{i}\PY{p}{]}\PY{o}{*}\PY{p}{(}\PY{l+m+mi}{1}\PY{o}{+}\PY{n}{x}\PY{p}{[}\PY{n}{i}\PY{p}{]}\PY{p}{)}\PY{o}{+}\PY{p}{(}\PY{n}{m}\PY{o}{.}\PY{n}{log}\PY{p}{(}\PY{l+m+mi}{4}\PY{p}{)}\PY{o}{+}\PY{n}{m}\PY{o}{.}\PY{n}{log}\PY{p}{(}\PY{n}{N}\PY{p}{)}\PY{o}{*}\PY{l+m+mi}{2}\PY{o}{*}\PY{n}{d\PYZus{}vc}\PY{o}{\PYZhy{}}\PY{n}{m}\PY{o}{.}\PY{n}{log}\PY{p}{(}\PY{n}{delta}\PY{p}{)}\PY{p}{)}\PY{p}{)}\PY{p}{)} \PY{k}{for} \PY{n}{i} \PY{o+ow}{in} \PY{n+nb}{range}\PY{p}{(}\PY{n+nb}{len}\PY{p}{(}\PY{n}{x}\PY{p}{)}\PY{p}{)}\PY{p}{]}

\PY{k}{def} \PY{n+nf}{implicit\PYZus{}solve}\PY{p}{(}\PY{n}{func}\PY{p}{,}\PY{n}{N}\PY{p}{,}\PY{n}{d\PYZus{}vc}\PY{p}{,}\PY{n}{delta}\PY{p}{,}\PY{n}{eps}\PY{o}{=}\PY{l+m+mf}{0.001}\PY{p}{,}\PY{n}{xleft}\PY{o}{=}\PY{l+m+mi}{0}\PY{p}{,}\PY{n}{xright}\PY{o}{=}\PY{l+m+mi}{1}\PY{p}{)}\PY{p}{:}
    \PY{n}{x} \PY{o}{=} \PY{n}{np}\PY{o}{.}\PY{n}{linspace}\PY{p}{(}\PY{n}{xleft}\PY{p}{,} \PY{n}{xright}\PY{p}{,} \PY{n}{np}\PY{o}{.}\PY{n}{int}\PY{p}{(}\PY{l+m+mi}{1}\PY{o}{/}\PY{n}{eps}\PY{p}{)}\PY{o}{*}\PY{n}{xright}\PY{p}{)}
    \PY{n}{y}\PY{o}{=}\PY{n}{func}\PY{p}{(}\PY{n}{x}\PY{p}{,}\PY{n}{N}\PY{p}{,}\PY{n}{d\PYZus{}vc}\PY{p}{,}\PY{n}{delta}\PY{p}{)}\PY{o}{\PYZhy{}}\PY{n}{x}
    \PY{n}{root} \PY{o}{=} \PY{k+kc}{None}  
    \PY{k}{for} \PY{n}{i} \PY{o+ow}{in} \PY{n+nb}{range}\PY{p}{(}\PY{n+nb}{len}\PY{p}{(}\PY{n}{x}\PY{p}{)}\PY{o}{\PYZhy{}}\PY{l+m+mi}{1}\PY{p}{)}\PY{p}{:}
        \PY{k}{if} \PY{n}{y}\PY{p}{[}\PY{n}{i}\PY{p}{]}\PY{o}{*}\PY{n}{y}\PY{p}{[}\PY{n}{i}\PY{o}{+}\PY{l+m+mi}{1}\PY{p}{]} \PY{o}{\PYZlt{}} \PY{l+m+mi}{0}\PY{p}{:}
            \PY{n}{root} \PY{o}{=} \PY{n}{x}\PY{p}{[}\PY{n}{i}\PY{p}{]} \PY{o}{\PYZhy{}} \PY{p}{(}\PY{n}{x}\PY{p}{[}\PY{n}{i}\PY{o}{+}\PY{l+m+mi}{1}\PY{p}{]} \PY{o}{\PYZhy{}} \PY{n}{x}\PY{p}{[}\PY{n}{i}\PY{p}{]}\PY{p}{)}\PY{o}{/}\PY{p}{(}\PY{n}{y}\PY{p}{[}\PY{n}{i}\PY{o}{+}\PY{l+m+mi}{1}\PY{p}{]} \PY{o}{\PYZhy{}} \PY{n}{y}\PY{p}{[}\PY{n}{i}\PY{p}{]}\PY{p}{)}\PY{o}{*}\PY{n}{y}\PY{p}{[}\PY{n}{i}\PY{p}{]}
            \PY{k}{break} 
    \PY{k}{if} \PY{n}{root} \PY{o+ow}{is} \PY{k+kc}{None}\PY{p}{:}
        \PY{n+nb}{print}\PY{p}{(}\PY{n}{f}\PY{l+s+s1}{\PYZsq{}}\PY{l+s+s1}{Error: Could not find any root in the interval[}\PY{l+s+si}{\PYZob{}xleft\PYZcb{}}\PY{l+s+s1}{,}\PY{l+s+si}{\PYZob{}xright\PYZcb{}}\PY{l+s+s1}{] for }\PY{l+s+si}{\PYZob{}func\PYZcb{}}\PY{l+s+s1}{\PYZsq{}}\PY{p}{)}
    \PY{k}{else}\PY{p}{:} \PY{k}{return} \PY{n}{root}

\PY{k}{def} \PY{n+nf}{print\PYZus{}bounds}\PY{p}{(}\PY{n}{N}\PY{p}{,}\PY{n}{xleft}\PY{o}{=}\PY{l+m+mi}{0}\PY{p}{,}\PY{n}{xright}\PY{o}{=}\PY{l+m+mi}{20}\PY{p}{)}\PY{p}{:}
    \PY{n+nb}{print}\PY{p}{(}\PY{n}{f}\PY{l+s+s1}{\PYZsq{}}\PY{l+s+s1}{For N=}\PY{l+s+si}{\PYZob{}N\PYZcb{}}\PY{l+s+se}{\PYZbs{}n}\PY{l+s+s1}{ Original VC Bound=}\PY{l+s+s1}{\PYZob{}}\PY{l+s+s1}{VC\PYZus{}bound(N,d\PYZus{}vc,delta):.3f\PYZcb{} }\PY{l+s+se}{\PYZbs{}n}\PY{l+s+s1}{ Rademacher Bound=}\PY{l+s+s1}{\PYZob{}}\PY{l+s+s1}{Rademacher(N,d\PYZus{}vc,delta):.3f\PYZcb{} }\PY{l+s+se}{\PYZbs{}n}\PY{l+s+s1}{ Parrondo Broek Bound =}\PY{l+s+s1}{\PYZob{}}\PY{l+s+s1}{implicit\PYZus{}solve(Parrondo\PYZus{}Broek\PYZus{}impl,N,d\PYZus{}vc,delta,xleft=xleft,xright=xright):.3f\PYZcb{} }\PY{l+s+se}{\PYZbs{}n}\PY{l+s+s1}{ Devroye Bound=}\PY{l+s+s1}{\PYZob{}}\PY{l+s+s1}{implicit\PYZus{}solve(Devroye\PYZus{}impl,N,d\PYZus{}vc,delta,xleft=xleft,xright=xright):.3f\PYZcb{}}\PY{l+s+s1}{\PYZsq{}}\PY{p}{)}
\end{Verbatim}
\end{tcolorbox}

    \begin{tcolorbox}[breakable, size=fbox, boxrule=1pt, pad at break*=1mm,colback=cellbackground, colframe=cellborder]
\prompt{In}{incolor}{173}{\boxspacing}
\begin{Verbatim}[commandchars=\\\{\}]
\PY{n}{N}\PY{o}{=}\PY{n}{np}\PY{o}{.}\PY{n}{arange}\PY{p}{(}\PY{l+m+mi}{5000}\PY{p}{,}\PY{l+m+mi}{10001}\PY{p}{)}
\PY{n}{d\PYZus{}vc}\PY{o}{=}\PY{l+m+mi}{50}
\PY{n}{delta}\PY{o}{=}\PY{l+m+mf}{0.05}
\PY{n}{xleft}\PY{o}{=}\PY{l+m+mi}{0}
\PY{n}{xright}\PY{o}{=}\PY{l+m+mi}{1}

\PY{n}{VC\PYZus{}bound\PYZus{}vals}\PY{o}{=}\PY{p}{[}\PY{n}{VC\PYZus{}bound}\PY{p}{(}\PY{n}{N}\PY{p}{[}\PY{n}{i}\PY{p}{]}\PY{p}{,}\PY{n}{d\PYZus{}vc}\PY{p}{,}\PY{n}{delta}\PY{p}{)} \PY{k}{for} \PY{n}{i} \PY{o+ow}{in} \PY{n+nb}{range}\PY{p}{(}\PY{n+nb}{len}\PY{p}{(}\PY{n}{N}\PY{p}{)}\PY{p}{)}\PY{p}{]}
\PY{n}{Rademacher\PYZus{}vals}\PY{o}{=}\PY{p}{[}\PY{n}{Rademacher}\PY{p}{(}\PY{n}{N}\PY{p}{[}\PY{n}{i}\PY{p}{]}\PY{p}{,}\PY{n}{d\PYZus{}vc}\PY{p}{,}\PY{n}{delta}\PY{p}{)} \PY{k}{for} \PY{n}{i} \PY{o+ow}{in} \PY{n+nb}{range}\PY{p}{(}\PY{n+nb}{len}\PY{p}{(}\PY{n}{N}\PY{p}{)}\PY{p}{)}\PY{p}{]}
\PY{n}{Parrondo\PYZus{}Broek\PYZus{}vals}\PY{o}{=}\PY{p}{[}\PY{n}{implicit\PYZus{}solve}\PY{p}{(}\PY{n}{Parrondo\PYZus{}Broek\PYZus{}impl}\PY{p}{,}\PY{n}{N}\PY{p}{[}\PY{n}{i}\PY{p}{]}\PY{p}{,}\PY{n}{d\PYZus{}vc}\PY{p}{,}\PY{n}{delta}\PY{p}{,}\PY{n}{xleft}\PY{o}{=}\PY{n}{xleft}\PY{p}{,}\PY{n}{xright}\PY{o}{=}\PY{n}{xright}\PY{p}{)} \PY{k}{for} \PY{n}{i} \PY{o+ow}{in} \PY{n+nb}{range}\PY{p}{(}\PY{n+nb}{len}\PY{p}{(}\PY{n}{N}\PY{p}{)}\PY{p}{)}\PY{p}{]}
\PY{n}{Devroye\PYZus{}vals}\PY{o}{=}\PY{p}{[}\PY{n}{implicit\PYZus{}solve}\PY{p}{(}\PY{n}{Devroye\PYZus{}impl}\PY{p}{,}\PY{n}{N}\PY{p}{[}\PY{n}{i}\PY{p}{]}\PY{p}{,}\PY{n}{d\PYZus{}vc}\PY{p}{,}\PY{n}{delta}\PY{p}{,}\PY{n}{xleft}\PY{o}{=}\PY{n}{xleft}\PY{p}{,}\PY{n}{xright}\PY{o}{=}\PY{n}{xright}\PY{p}{)} \PY{k}{for} \PY{n}{i} \PY{o+ow}{in} \PY{n+nb}{range}\PY{p}{(}\PY{n+nb}{len}\PY{p}{(}\PY{n}{N}\PY{p}{)}\PY{p}{)}\PY{p}{]}

\PY{n}{plt}\PY{o}{.}\PY{n}{plot}\PY{p}{(}\PY{n}{N}\PY{p}{,}\PY{n}{VC\PYZus{}bound\PYZus{}vals}\PY{p}{,}\PY{n}{label}\PY{o}{=}\PY{l+s+s2}{\PYZdq{}}\PY{l+s+s2}{VC bound}\PY{l+s+s2}{\PYZdq{}}\PY{p}{)}
\PY{n}{plt}\PY{o}{.}\PY{n}{plot}\PY{p}{(}\PY{n}{N}\PY{p}{,}\PY{n}{Rademacher\PYZus{}vals}\PY{p}{,}\PY{n}{label}\PY{o}{=}\PY{l+s+s2}{\PYZdq{}}\PY{l+s+s2}{Rademacher bound}\PY{l+s+s2}{\PYZdq{}}\PY{p}{)}
\PY{n}{plt}\PY{o}{.}\PY{n}{plot}\PY{p}{(}\PY{n}{N}\PY{p}{,}\PY{n}{Parrondo\PYZus{}Broek\PYZus{}vals}\PY{p}{,}\PY{n}{label}\PY{o}{=}\PY{l+s+s2}{\PYZdq{}}\PY{l+s+s2}{Parrondo/Broek Bound}\PY{l+s+s2}{\PYZdq{}}\PY{p}{)}
\PY{n}{plt}\PY{o}{.}\PY{n}{plot}\PY{p}{(}\PY{n}{N}\PY{p}{,}\PY{n}{Devroye\PYZus{}vals}\PY{p}{,}\PY{n}{label}\PY{o}{=}\PY{l+s+s2}{\PYZdq{}}\PY{l+s+s2}{Devroye Bound}\PY{l+s+s2}{\PYZdq{}}\PY{p}{)}
\PY{n}{plt}\PY{o}{.}\PY{n}{legend}\PY{p}{(}\PY{n}{loc}\PY{o}{=}\PY{p}{[}\PY{l+m+mf}{1.05}\PY{p}{,}\PY{l+m+mf}{0.7}\PY{p}{]}\PY{p}{)}
\PY{n}{plt}\PY{o}{.}\PY{n}{show}\PY{p}{(}\PY{p}{)}
\end{Verbatim}
\end{tcolorbox}

    \begin{center}
    \adjustimage{max size={0.9\linewidth}{0.9\paperheight}}{output_5_0.png}
    \end{center}
    { \hspace*{\fill} \\}
    
    \begin{tcolorbox}[breakable, size=fbox, boxrule=1pt, pad at break*=1mm,colback=cellbackground, colframe=cellborder]
\prompt{In}{incolor}{178}{\boxspacing}
\begin{Verbatim}[commandchars=\\\{\}]
\PY{n}{print\PYZus{}bounds}\PY{p}{(}\PY{l+m+mi}{10000}\PY{p}{)}
\end{Verbatim}
\end{tcolorbox}

    \begin{Verbatim}[commandchars=\\\{\}]
For N=10000
 Original VC Bound=0.632
 Rademacher Bound=0.342
 Parrondo Broek Bound =0.224
 Devroye Bound=0.215
    \end{Verbatim}

    \hypertarget{problem-3}{%
\section{Problem 3}\label{problem-3}}

\hypertarget{answer-c-rm-parrondoand-van-den-broek-epsilon-le-sqrtfrac1nbig2epsilonlogfrac6-m_mathcalh2ndeltabig}{%
\subsection{\texorpdfstring{Answer: {[}c{]}
\({\rm Parrondo\,and \, Van\, den \,Broek:}\; \epsilon \le \sqrt{\frac{1}{N}\big(2\epsilon+\log{\frac{6 m_{\mathcal{H}}(2N)}{\delta}}\big)}\)}{Answer: {[}c{]} \{\textbackslash{}rm Parrondo\textbackslash{},and \textbackslash{}, Van\textbackslash{}, den \textbackslash{},Broek:\}\textbackslash{}; \textbackslash{}epsilon \textbackslash{}le \textbackslash{}sqrt\{\textbackslash{}frac\{1\}\{N\}\textbackslash{}big(2\textbackslash{}epsilon+\textbackslash{}log\{\textbackslash{}frac\{6 m\_\{\textbackslash{}mathcal\{H\}\}(2N)\}\{\textbackslash{}delta\}\}\textbackslash{}big)\}}}\label{answer-c-rm-parrondoand-van-den-broek-epsilon-le-sqrtfrac1nbig2epsilonlogfrac6-m_mathcalh2ndeltabig}}

\hypertarget{code}{%
\subsection{Code:}\label{code}}

    \begin{tcolorbox}[breakable, size=fbox, boxrule=1pt, pad at break*=1mm,colback=cellbackground, colframe=cellborder]
\prompt{In}{incolor}{177}{\boxspacing}
\begin{Verbatim}[commandchars=\\\{\}]
\PY{n}{print\PYZus{}bounds}\PY{p}{(}\PY{l+m+mi}{5}\PY{p}{)}
\end{Verbatim}
\end{tcolorbox}

    \begin{Verbatim}[commandchars=\\\{\}]
For N=5
 Original VC Bound=13.828
 Rademacher Bound=8.148
 Parrondo Broek Bound =5.101
 Devroye Bound=5.593
    \end{Verbatim}

    \hypertarget{problem-4}{%
\section{Problem 4}\label{problem-4}}

\hypertarget{answer-e-none-of-the-above}{%
\subsection{Answer: {[}e{]} None of the
above}\label{answer-e-none-of-the-above}}

\hypertarget{derivation}{%
\subsection{Derivation:}\label{derivation}}

We want to minimize the mean squared error on the two examples given, in
formulae:

\[
E_{in}=\frac{1}{N}\sum^N_{i=1}(a x_i-y_i)^2\,.
\]

We can find the \(\bar{a}\) that minimizes the error by setting
\(\frac{d}{da}E_{in}(\bar{a})=0\). For two points, we get:

\[
\bar{a}=\frac{x_1 y_1 + x_2 y_2}{x_1^2 + x_2^2}\,.
\]

Notice that the same result is obtained by applying the Linear
Regression Algorithm (which by definition minimizes the mean squared
errors). Let \(X\) be the matrix of the data points, i.e.

\[
X=
\begin{pmatrix}
x_1\\
x_2
\end{pmatrix}
\,.
\]

The single weight \(\bar{a}\) is then found by computing
\(\bar{a}=X^\dagger y\), where

\[
X^\dagger=(X^T X)^{-1}X^T=
\begin{pmatrix}
\frac{x_1}{x_1^2 + x_2^2}\\
\frac{x_2}{x_1^2 + x_2^2}
\end{pmatrix}
\,
\]

is the pseudo-inverse matrix of \(X\).

Finally, we want to find the average hypothesis \(\bar{g}(x)\). Given N
two-point samples, it can be written as:

\[\bar{g}(x)=\frac{1}{N}\sum_{x\in\chi}h(x)=\frac{\sum^N_{i=0}\bar{a}}{N} x=\bar{a}_{avg}x\,.\]

    \begin{tcolorbox}[breakable, size=fbox, boxrule=1pt, pad at break*=1mm,colback=cellbackground, colframe=cellborder]
\prompt{In}{incolor}{422}{\boxspacing}
\begin{Verbatim}[commandchars=\\\{\}]
\PY{k}{def} \PY{n+nf}{a\PYZus{}min}\PY{p}{(}\PY{n}{x}\PY{p}{,}\PY{n}{y}\PY{p}{)}\PY{p}{:}
    \PY{k}{return} \PY{n}{np}\PY{o}{.}\PY{n}{dot}\PY{p}{(}\PY{n}{x}\PY{p}{,}\PY{n}{y}\PY{p}{)}\PY{o}{/}\PY{n}{np}\PY{o}{.}\PY{n}{dot}\PY{p}{(}\PY{n}{x}\PY{p}{,}\PY{n}{x}\PY{p}{)}

\PY{k}{def} \PY{n+nf}{f}\PY{p}{(}\PY{n}{x}\PY{p}{)}\PY{p}{:}
    \PY{k}{return} \PY{n}{np}\PY{o}{.}\PY{n}{sin}\PY{p}{(}\PY{n}{m}\PY{o}{.}\PY{n}{pi}\PY{o}{*}\PY{n}{x}\PY{p}{)}

\PY{k}{def} \PY{n+nf}{gen\PYZus{}data}\PY{p}{(}\PY{n}{func}\PY{p}{,}\PY{n}{N}\PY{p}{)}\PY{p}{:}
    \PY{n}{x}\PY{o}{=}\PY{n}{np}\PY{o}{.}\PY{n}{random}\PY{o}{.}\PY{n}{uniform}\PY{p}{(}\PY{o}{\PYZhy{}}\PY{l+m+mi}{1}\PY{p}{,}\PY{l+m+mi}{1}\PY{p}{,}\PY{n}{N}\PY{p}{)}
    \PY{k}{return} \PY{n}{x}\PY{p}{,}\PY{n}{func}\PY{p}{(}\PY{n}{x}\PY{p}{)}

\PY{n}{a}\PY{o}{=}\PY{p}{[}\PY{p}{]}
\PY{n}{Nsamples}\PY{o}{=}\PY{l+m+mi}{10000}

\PY{k}{for} \PY{n}{i} \PY{o+ow}{in} \PY{n+nb}{range}\PY{p}{(}\PY{n}{Nsamples}\PY{p}{)}\PY{p}{:}
    \PY{n}{x}\PY{p}{,}\PY{n}{y}\PY{o}{=}\PY{n}{gen\PYZus{}data}\PY{p}{(}\PY{n}{f}\PY{p}{,}\PY{l+m+mi}{2}\PY{p}{)}
    \PY{n}{a}\PY{o}{.}\PY{n}{append}\PY{p}{(}\PY{n}{a\PYZus{}min}\PY{p}{(}\PY{n}{x}\PY{p}{,}\PY{n}{y}\PY{p}{)}\PY{p}{)}

\PY{n}{a\PYZus{}avg}\PY{o}{=}\PY{n}{np}\PY{o}{.}\PY{n}{mean}\PY{p}{(}\PY{n}{a}\PY{p}{)}

\PY{n+nb}{print}\PY{p}{(}\PY{n}{f}\PY{l+s+s1}{\PYZsq{}}\PY{l+s+s1}{For }\PY{l+s+si}{\PYZob{}Nsamples\PYZcb{}}\PY{l+s+s1}{ samples of two points, a\PYZus{}avg=}\PY{l+s+si}{\PYZob{}a\PYZus{}avg:.2f\PYZcb{}}\PY{l+s+s1}{\PYZsq{}}\PY{p}{)}   
\end{Verbatim}
\end{tcolorbox}

    \begin{Verbatim}[commandchars=\\\{\}]
For 10000 samples of two points, a\_avg=1.43
    \end{Verbatim}

    \hypertarget{problem-5}{%
\section{Problem 5}\label{problem-5}}

\hypertarget{answer-b-0.3}{%
\subsection{\texorpdfstring{Answer: {[}b{]}
\(0.3\)}{Answer: {[}b{]} 0.3}}\label{answer-b-0.3}}

\hypertarget{derivation}{%
\subsection{Derivation:}\label{derivation}}

We want to compute the bias for N sample-points:

\[
{\rm bias}=\frac{1}{N}\sum_{x\in\chi}{\rm bias}(x)=\frac{1}{N}\sum^N_{i=1}(\bar{g}(x_i)-f(x_i))^2=\frac{1}{N}\sum^N_{i=1}(\bar{a}_{avg}x_i-f(x_i))^2\,.
\]

    \begin{tcolorbox}[breakable, size=fbox, boxrule=1pt, pad at break*=1mm,colback=cellbackground, colframe=cellborder]
\prompt{In}{incolor}{423}{\boxspacing}
\begin{Verbatim}[commandchars=\\\{\}]
\PY{n}{bias}\PY{o}{=}\PY{l+m+mi}{0}

\PY{k}{for} \PY{n}{i} \PY{o+ow}{in} \PY{n+nb}{range}\PY{p}{(}\PY{n}{Nsamples}\PY{p}{)}\PY{p}{:}
    \PY{n}{x}\PY{p}{,}\PY{n}{y}\PY{o}{=}\PY{n}{gen\PYZus{}data}\PY{p}{(}\PY{n}{f}\PY{p}{,}\PY{l+m+mi}{1}\PY{p}{)}
    \PY{n}{bias}\PY{o}{+}\PY{o}{=}\PY{p}{(}\PY{n}{a\PYZus{}avg}\PY{o}{*}\PY{n}{x}\PY{p}{[}\PY{l+m+mi}{0}\PY{p}{]}\PY{o}{\PYZhy{}}\PY{n}{y}\PY{p}{[}\PY{l+m+mi}{0}\PY{p}{]}\PY{p}{)}\PY{o}{*}\PY{o}{*}\PY{l+m+mi}{2}

\PY{n}{bias}\PY{o}{=}\PY{n}{bias}\PY{o}{/}\PY{n}{Nsamples}

\PY{n+nb}{print}\PY{p}{(}\PY{n}{f}\PY{l+s+s1}{\PYZsq{}}\PY{l+s+s1}{For }\PY{l+s+si}{\PYZob{}Nsamples\PYZcb{}}\PY{l+s+s1}{ sample\PYZhy{}points, bias=}\PY{l+s+si}{\PYZob{}bias:.2f\PYZcb{}}\PY{l+s+s1}{\PYZsq{}}\PY{p}{)}   
\end{Verbatim}
\end{tcolorbox}

    \begin{Verbatim}[commandchars=\\\{\}]
For 10000 sample-points, bias=0.27
    \end{Verbatim}

    \hypertarget{problem-6}{%
\section{Problem 6}\label{problem-6}}

\hypertarget{answer-a-0.2}{%
\subsection{\texorpdfstring{Answer: {[}a{]}
\(0.2\)}{Answer: {[}a{]} 0.2}}\label{answer-a-0.2}}

\hypertarget{derivation}{%
\subsection{Derivation:}\label{derivation}}

We want to compute the variance for N sample-points:

\[
{\rm var}=\frac{1}{N}\sum_{x\in\chi}{\rm var}(x)=\frac{1}{N}\sum^N_{i=1}(h(x)-\bar{g}(x))^2=\frac{1}{N}\sum^N_{i=1}((\bar{a}_i-\bar{a}_{avg})x_i)^2\,.
\]

    \begin{tcolorbox}[breakable, size=fbox, boxrule=1pt, pad at break*=1mm,colback=cellbackground, colframe=cellborder]
\prompt{In}{incolor}{426}{\boxspacing}
\begin{Verbatim}[commandchars=\\\{\}]
\PY{n}{var}\PY{o}{=}\PY{l+m+mi}{0}

\PY{k}{for} \PY{n}{i} \PY{o+ow}{in} \PY{n+nb}{range}\PY{p}{(}\PY{n}{Nsamples}\PY{p}{)}\PY{p}{:}
    \PY{n}{x}\PY{p}{,}\PY{n}{y}\PY{o}{=}\PY{n}{gen\PYZus{}data}\PY{p}{(}\PY{n}{f}\PY{p}{,}\PY{l+m+mi}{1}\PY{p}{)}
    \PY{n}{var}\PY{o}{+}\PY{o}{=}\PY{p}{(}\PY{n}{a}\PY{p}{[}\PY{n}{i}\PY{p}{]}\PY{o}{*}\PY{n}{x}\PY{p}{[}\PY{l+m+mi}{0}\PY{p}{]}\PY{o}{\PYZhy{}}\PY{n}{a\PYZus{}avg}\PY{o}{*}\PY{n}{x}\PY{p}{[}\PY{l+m+mi}{0}\PY{p}{]}\PY{p}{)}\PY{o}{*}\PY{o}{*}\PY{l+m+mi}{2}

\PY{n}{var}\PY{o}{=}\PY{n}{var}\PY{o}{/}\PY{n}{Nsamples}

\PY{n+nb}{print}\PY{p}{(}\PY{n}{f}\PY{l+s+s1}{\PYZsq{}}\PY{l+s+s1}{For }\PY{l+s+si}{\PYZob{}Nsamples\PYZcb{}}\PY{l+s+s1}{ sample\PYZhy{}points, bias=}\PY{l+s+si}{\PYZob{}var:.2f\PYZcb{}}\PY{l+s+s1}{\PYZsq{}}\PY{p}{)}   
\end{Verbatim}
\end{tcolorbox}

    \begin{Verbatim}[commandchars=\\\{\}]
For 10000 sample-points, bias=0.24
    \end{Verbatim}

    \hypertarget{problem-7}{%
\section{Problem 7}\label{problem-7}}

\hypertarget{answer-b-hxax}{%
\subsection{\texorpdfstring{Answer: {[}b{]}
\(h(x)=ax\)}{Answer: {[}b{]} h(x)=ax}}\label{answer-b-hxax}}

\hypertarget{derivation}{%
\subsection{Derivation:}\label{derivation}}

The expected value of out-of-sample error \(E_{out}\) in the
bias-variance analysis can be written as

\[
E_{out}\sim{\rm bias}+{\rm var}
\]

For the proposed models, taking the other data from the lecture notes,
we have:

\begin{itemize}
\tightlist
\item
  {[}a{]} \(h(x)=b\) has
  \(({\rm bias},{\rm var})=(0.5,0.25)\rightarrow E_{out}\sim 0.75\)
\item
  {[}b{]} \(h(x)=ax\) has
  \(({\rm bias},{\rm var})=(0.3,0.2)\rightarrow E_{out}\sim 0.5\)
\item
  {[}c{]} \(h(x)=ax+b\) has
  \(({\rm bias},{\rm var})=(0.21,1.69)\rightarrow E_{out}\sim 1.9\)
\end{itemize}

where we excluded the quadratic hypothesis since they are too complex
for the amount of points provided.

    \hypertarget{problem-8}{%
\section{Problem 8}\label{problem-8}}

\hypertarget{answer-c-q}{%
\subsection{\texorpdfstring{Answer: {[}c{]}
\(q\)}{Answer: {[}c{]} q}}\label{answer-c-q}}

\hypertarget{derivation}{%
\subsection{Derivation:}\label{derivation}}

When \(N<q\), the recursive relation can be solved exactly with initial
condition \(m_{\mathcal{H}}(1)=2\), giving

\[
m_{\mathcal{H}}(N\le q)=2^N\,.
\]

For \(N=q+1\):

\[
m_{\mathcal{H}}(q+1)=2*2^{q}-{q\choose q}=2^{q+1}-1\,.
\]

Hence, \(k=q+1\) is a break point for the hypothesis set
\(\mathcal{H}\). By definition, the VC dimension of the set is then:

\[
d_{VC}=k-1=q\,.
\]

    \hypertarget{problem-9}{%
\section{Problem 9}\label{problem-9}}

\hypertarget{answer-b-0le-d_vccap_k-mathcalh_kle-rm-mind_vcmathcalh_k}{%
\subsection{\texorpdfstring{Answer: {[}b{]}
\(0\le d_{VC}(\cap_k \mathcal{H}_k)\le {\rm min}\{d_{VC}(\mathcal{H}_k)\}\)}{Answer: {[}b{]} 0\textbackslash{}le d\_\{VC\}(\textbackslash{}cap\_k \textbackslash{}mathcal\{H\}\_k)\textbackslash{}le \{\textbackslash{}rm min\}\textbackslash{}\{d\_\{VC\}(\textbackslash{}mathcal\{H\}\_k)\textbackslash{}\}}}\label{answer-b-0le-d_vccap_k-mathcalh_kle-rm-mind_vcmathcalh_k}}

\hypertarget{derivation}{%
\subsection{Derivation:}\label{derivation}}

Since the intersection can be the empty set (consider disjoint sets),
and by definition the singleton set has VC dimension 0, the lower bound
of the intersection has to be 0.

The upper bound should be the minimum VC dimension among all the sets,
which we call \(d_{min}\). Suppose instead the VC dimension of the
intersection set is \(d_{min}+1\). This implies that the intersection
set hypothesis can shatter \(d_{min}+1\) points. Hence, every hypothesis
in the intersection should be able to shatter \(d_{min}+1\) which
contradicts the assumption that \(d_{min}\) is the VC dimension of one
of the sets.

    \hypertarget{problem-10}{%
\section{Problem 10}\label{problem-10}}

\hypertarget{answer-e-rm-maxd_vcmathcalh_kle-d_vccup_k-mathcalh_kle-k-1sum_k-d_vcmathcalh_k}{%
\subsection{\texorpdfstring{Answer: {[}e{]}
\({\rm max}\{d_{VC}(\mathcal{H}_k)\}\le d_{VC}(\cup_k \mathcal{H}_k)\le K-1+\sum_k d_{VC}(\mathcal{H}_k)\)}{Answer: {[}e{]} \{\textbackslash{}rm max\}\textbackslash{}\{d\_\{VC\}(\textbackslash{}mathcal\{H\}\_k)\textbackslash{}\}\textbackslash{}le d\_\{VC\}(\textbackslash{}cup\_k \textbackslash{}mathcal\{H\}\_k)\textbackslash{}le K-1+\textbackslash{}sum\_k d\_\{VC\}(\textbackslash{}mathcal\{H\}\_k)}}\label{answer-e-rm-maxd_vcmathcalh_kle-d_vccup_k-mathcalh_kle-k-1sum_k-d_vcmathcalh_k}}

\hypertarget{derivation}{%
\subsection{Derivation:}\label{derivation}}

The lower bound of the union set has to be the maximum VC dimension
among all the sets \(d_{max}\), since we can always shatter \(d_{max}\)
points with the union set by taking the subset that has \(d_{max}\) as
its VC dimension.

For the upper bound, we first prove that the union set of two hypothesis
sets \(\mathcal{H}_1\) and \(\mathcal{H}_2\) with VC dimensions \(d_1\)
and \(d_2\) respectively can shatter at most \(N-1=d_1+d_2+1\) points.
First of all, we observe that the growth function of the union set can
be at most the sum of the two growth functions of the two subsets for
any \(N\). In particular, for \(N=d_1+d_2+2\) we find:

\begin{equation}
\begin{split}
m_{\mathcal{H}_1\cup\mathcal{H}_2}(N)&\le m_{\mathcal{H}_1}(N)+m_{\mathcal{H}_2}(N) \\&=\sum_{k=0}^{d_{1}}{N\choose k}+\sum_{k=0}^{d_{2}}{N\choose k}=\sum_{k=0}^{d_{1}}{N\choose k}+\sum_{k=0}^{d_{2}}{N\choose N-k}=\sum_{k=0}^{d_{1}}{N\choose k}+\sum_{k=0}^{d_{2}}{N\choose N-k}\\&=\sum_{k=0}^{d_{1}}{N\choose k}+\sum_{k=d_1+2}^{d_1+d_{2}+2}{N\choose k}=\sum_{k=0}^{d_1+d_2+2}{N\choose k}-{N\choose d_1+1}=2^N-{N\choose d_1+1}<2^N\,.
\end{split}
\end{equation}

Therefore, if the first relation is an equality, \(N=d_1+d_2+2\) is
necessarily a break point for the union set and its VC dimension is
\(d_{VC}(\mathcal{H}_1\cup \mathcal{H}_2)=d_1+d_2+1\). Therefore, we
proved that

\[
d_{VC}(\mathcal{H}_1\cup \mathcal{H}_2)\le d_{VC}(\mathcal{H}_1)+d_{VC}(\mathcal{H}_2)+1\,.
\]

Now, we use induction to prove the upper bound for a generalized union
of \(K\) hypothesis sets. Let's assume the relation is valid for
\(K-1\), in formulae:

\[
d_{VC}(\cup_{k=0}^{K-1} \mathcal{H}_k)\le K-2+\sum_{k=0}^{K-1} d_{VC}(\mathcal{H}_k)\,.
\]

For \(d_{VC}(\cup_{k=0}^{K} \mathcal{H}_k)\), we can write:

\begin{equation}
\begin{split}
d_{VC}(\cup_{k=0}^{K} \mathcal{H}_k)= d_{VC}(\cup_{k=0}^{K-1} \mathcal{H}_k \cup \mathcal{H}_K)\le d_{VC}(\cup_{k=0}^{K-1} \mathcal{H}_k)+d_{VC}(\mathcal{H}_K)+1\\ \le K-2+\sum_{k=0}^{K-1} d_{VC}(\mathcal{H}_k)+d_{VC}(\mathcal{H}_K)+1=K-1+\sum_{k=0}^{K} d_{VC}(\mathcal{H}_k)\,.
\end{split}
\end{equation}

    \begin{tcolorbox}[breakable, size=fbox, boxrule=1pt, pad at break*=1mm,colback=cellbackground, colframe=cellborder]
\prompt{In}{incolor}{ }{\boxspacing}
\begin{Verbatim}[commandchars=\\\{\}]

\end{Verbatim}
\end{tcolorbox}


    % Add a bibliography block to the postdoc
    
    
    
\end{document}
