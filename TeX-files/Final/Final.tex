\documentclass[11pt]{article}

    \usepackage[breakable]{tcolorbox}
    \usepackage{parskip} % Stop auto-indenting (to mimic markdown behaviour)
    
    \usepackage{iftex}
    \ifPDFTeX
    	\usepackage[T1]{fontenc}
    	\usepackage{mathpazo}
    \else
    	\usepackage{fontspec}
    \fi

    % Basic figure setup, for now with no caption control since it's done
    % automatically by Pandoc (which extracts ![](path) syntax from Markdown).
    \usepackage{graphicx}
    % Maintain compatibility with old templates. Remove in nbconvert 6.0
    \let\Oldincludegraphics\includegraphics
    % Ensure that by default, figures have no caption (until we provide a
    % proper Figure object with a Caption API and a way to capture that
    % in the conversion process - todo).
    \usepackage{caption}
    \DeclareCaptionFormat{nocaption}{}
    \captionsetup{format=nocaption,aboveskip=0pt,belowskip=0pt}

    \usepackage[Export]{adjustbox} % Used to constrain images to a maximum size
    \adjustboxset{max size={0.9\linewidth}{0.9\paperheight}}
    \usepackage{float}
    \floatplacement{figure}{H} % forces figures to be placed at the correct location
    \usepackage{xcolor} % Allow colors to be defined
    \usepackage{enumerate} % Needed for markdown enumerations to work
    \usepackage{geometry} % Used to adjust the document margins
    \usepackage{amsmath} % Equations
    \usepackage{amssymb} % Equations
    \usepackage{textcomp} % defines textquotesingle
    % Hack from http://tex.stackexchange.com/a/47451/13684:
    \AtBeginDocument{%
        \def\PYZsq{\textquotesingle}% Upright quotes in Pygmentized code
    }
    \usepackage{upquote} % Upright quotes for verbatim code
    \usepackage{eurosym} % defines \euro
    \usepackage[mathletters]{ucs} % Extended unicode (utf-8) support
    \usepackage{fancyvrb} % verbatim replacement that allows latex
    \usepackage{grffile} % extends the file name processing of package graphics 
                         % to support a larger range
    \makeatletter % fix for grffile with XeLaTeX
    \def\Gread@@xetex#1{%
      \IfFileExists{"\Gin@base".bb}%
      {\Gread@eps{\Gin@base.bb}}%
      {\Gread@@xetex@aux#1}%
    }
    \makeatother

    % The hyperref package gives us a pdf with properly built
    % internal navigation ('pdf bookmarks' for the table of contents,
    % internal cross-reference links, web links for URLs, etc.)
    \usepackage{hyperref}
    % The default LaTeX title has an obnoxious amount of whitespace. By default,
    % titling removes some of it. It also provides customization options.
    \usepackage{titling}
    \usepackage{longtable} % longtable support required by pandoc >1.10
    \usepackage{booktabs}  % table support for pandoc > 1.12.2
    \usepackage[inline]{enumitem} % IRkernel/repr support (it uses the enumerate* environment)
    \usepackage[normalem]{ulem} % ulem is needed to support strikethroughs (\sout)
                                % normalem makes italics be italics, not underlines
    \usepackage{mathrsfs}
    

    
    % Colors for the hyperref package
    \definecolor{urlcolor}{rgb}{0,.145,.698}
    \definecolor{linkcolor}{rgb}{.71,0.21,0.01}
    \definecolor{citecolor}{rgb}{.12,.54,.11}

    % ANSI colors
    \definecolor{ansi-black}{HTML}{3E424D}
    \definecolor{ansi-black-intense}{HTML}{282C36}
    \definecolor{ansi-red}{HTML}{E75C58}
    \definecolor{ansi-red-intense}{HTML}{B22B31}
    \definecolor{ansi-green}{HTML}{00A250}
    \definecolor{ansi-green-intense}{HTML}{007427}
    \definecolor{ansi-yellow}{HTML}{DDB62B}
    \definecolor{ansi-yellow-intense}{HTML}{B27D12}
    \definecolor{ansi-blue}{HTML}{208FFB}
    \definecolor{ansi-blue-intense}{HTML}{0065CA}
    \definecolor{ansi-magenta}{HTML}{D160C4}
    \definecolor{ansi-magenta-intense}{HTML}{A03196}
    \definecolor{ansi-cyan}{HTML}{60C6C8}
    \definecolor{ansi-cyan-intense}{HTML}{258F8F}
    \definecolor{ansi-white}{HTML}{C5C1B4}
    \definecolor{ansi-white-intense}{HTML}{A1A6B2}
    \definecolor{ansi-default-inverse-fg}{HTML}{FFFFFF}
    \definecolor{ansi-default-inverse-bg}{HTML}{000000}

    % commands and environments needed by pandoc snippets
    % extracted from the output of `pandoc -s`
    \providecommand{\tightlist}{%
      \setlength{\itemsep}{0pt}\setlength{\parskip}{0pt}}
    \DefineVerbatimEnvironment{Highlighting}{Verbatim}{commandchars=\\\{\}}
    % Add ',fontsize=\small' for more characters per line
    \newenvironment{Shaded}{}{}
    \newcommand{\KeywordTok}[1]{\textcolor[rgb]{0.00,0.44,0.13}{\textbf{{#1}}}}
    \newcommand{\DataTypeTok}[1]{\textcolor[rgb]{0.56,0.13,0.00}{{#1}}}
    \newcommand{\DecValTok}[1]{\textcolor[rgb]{0.25,0.63,0.44}{{#1}}}
    \newcommand{\BaseNTok}[1]{\textcolor[rgb]{0.25,0.63,0.44}{{#1}}}
    \newcommand{\FloatTok}[1]{\textcolor[rgb]{0.25,0.63,0.44}{{#1}}}
    \newcommand{\CharTok}[1]{\textcolor[rgb]{0.25,0.44,0.63}{{#1}}}
    \newcommand{\StringTok}[1]{\textcolor[rgb]{0.25,0.44,0.63}{{#1}}}
    \newcommand{\CommentTok}[1]{\textcolor[rgb]{0.38,0.63,0.69}{\textit{{#1}}}}
    \newcommand{\OtherTok}[1]{\textcolor[rgb]{0.00,0.44,0.13}{{#1}}}
    \newcommand{\AlertTok}[1]{\textcolor[rgb]{1.00,0.00,0.00}{\textbf{{#1}}}}
    \newcommand{\FunctionTok}[1]{\textcolor[rgb]{0.02,0.16,0.49}{{#1}}}
    \newcommand{\RegionMarkerTok}[1]{{#1}}
    \newcommand{\ErrorTok}[1]{\textcolor[rgb]{1.00,0.00,0.00}{\textbf{{#1}}}}
    \newcommand{\NormalTok}[1]{{#1}}
    
    % Additional commands for more recent versions of Pandoc
    \newcommand{\ConstantTok}[1]{\textcolor[rgb]{0.53,0.00,0.00}{{#1}}}
    \newcommand{\SpecialCharTok}[1]{\textcolor[rgb]{0.25,0.44,0.63}{{#1}}}
    \newcommand{\VerbatimStringTok}[1]{\textcolor[rgb]{0.25,0.44,0.63}{{#1}}}
    \newcommand{\SpecialStringTok}[1]{\textcolor[rgb]{0.73,0.40,0.53}{{#1}}}
    \newcommand{\ImportTok}[1]{{#1}}
    \newcommand{\DocumentationTok}[1]{\textcolor[rgb]{0.73,0.13,0.13}{\textit{{#1}}}}
    \newcommand{\AnnotationTok}[1]{\textcolor[rgb]{0.38,0.63,0.69}{\textbf{\textit{{#1}}}}}
    \newcommand{\CommentVarTok}[1]{\textcolor[rgb]{0.38,0.63,0.69}{\textbf{\textit{{#1}}}}}
    \newcommand{\VariableTok}[1]{\textcolor[rgb]{0.10,0.09,0.49}{{#1}}}
    \newcommand{\ControlFlowTok}[1]{\textcolor[rgb]{0.00,0.44,0.13}{\textbf{{#1}}}}
    \newcommand{\OperatorTok}[1]{\textcolor[rgb]{0.40,0.40,0.40}{{#1}}}
    \newcommand{\BuiltInTok}[1]{{#1}}
    \newcommand{\ExtensionTok}[1]{{#1}}
    \newcommand{\PreprocessorTok}[1]{\textcolor[rgb]{0.74,0.48,0.00}{{#1}}}
    \newcommand{\AttributeTok}[1]{\textcolor[rgb]{0.49,0.56,0.16}{{#1}}}
    \newcommand{\InformationTok}[1]{\textcolor[rgb]{0.38,0.63,0.69}{\textbf{\textit{{#1}}}}}
    \newcommand{\WarningTok}[1]{\textcolor[rgb]{0.38,0.63,0.69}{\textbf{\textit{{#1}}}}}
    
    
    % Define a nice break command that doesn't care if a line doesn't already
    % exist.
    \def\br{\hspace*{\fill} \\* }
    % Math Jax compatibility definitions
    \def\gt{>}
    \def\lt{<}
    \let\Oldtex\TeX
    \let\Oldlatex\LaTeX
    \renewcommand{\TeX}{\textrm{\Oldtex}}
    \renewcommand{\LaTeX}{\textrm{\Oldlatex}}
    % Document parameters
    % Document title
    \title{Final}
    
    
    
    
    
% Pygments definitions
\makeatletter
\def\PY@reset{\let\PY@it=\relax \let\PY@bf=\relax%
    \let\PY@ul=\relax \let\PY@tc=\relax%
    \let\PY@bc=\relax \let\PY@ff=\relax}
\def\PY@tok#1{\csname PY@tok@#1\endcsname}
\def\PY@toks#1+{\ifx\relax#1\empty\else%
    \PY@tok{#1}\expandafter\PY@toks\fi}
\def\PY@do#1{\PY@bc{\PY@tc{\PY@ul{%
    \PY@it{\PY@bf{\PY@ff{#1}}}}}}}
\def\PY#1#2{\PY@reset\PY@toks#1+\relax+\PY@do{#2}}

\expandafter\def\csname PY@tok@w\endcsname{\def\PY@tc##1{\textcolor[rgb]{0.73,0.73,0.73}{##1}}}
\expandafter\def\csname PY@tok@c\endcsname{\let\PY@it=\textit\def\PY@tc##1{\textcolor[rgb]{0.25,0.50,0.50}{##1}}}
\expandafter\def\csname PY@tok@cp\endcsname{\def\PY@tc##1{\textcolor[rgb]{0.74,0.48,0.00}{##1}}}
\expandafter\def\csname PY@tok@k\endcsname{\let\PY@bf=\textbf\def\PY@tc##1{\textcolor[rgb]{0.00,0.50,0.00}{##1}}}
\expandafter\def\csname PY@tok@kp\endcsname{\def\PY@tc##1{\textcolor[rgb]{0.00,0.50,0.00}{##1}}}
\expandafter\def\csname PY@tok@kt\endcsname{\def\PY@tc##1{\textcolor[rgb]{0.69,0.00,0.25}{##1}}}
\expandafter\def\csname PY@tok@o\endcsname{\def\PY@tc##1{\textcolor[rgb]{0.40,0.40,0.40}{##1}}}
\expandafter\def\csname PY@tok@ow\endcsname{\let\PY@bf=\textbf\def\PY@tc##1{\textcolor[rgb]{0.67,0.13,1.00}{##1}}}
\expandafter\def\csname PY@tok@nb\endcsname{\def\PY@tc##1{\textcolor[rgb]{0.00,0.50,0.00}{##1}}}
\expandafter\def\csname PY@tok@nf\endcsname{\def\PY@tc##1{\textcolor[rgb]{0.00,0.00,1.00}{##1}}}
\expandafter\def\csname PY@tok@nc\endcsname{\let\PY@bf=\textbf\def\PY@tc##1{\textcolor[rgb]{0.00,0.00,1.00}{##1}}}
\expandafter\def\csname PY@tok@nn\endcsname{\let\PY@bf=\textbf\def\PY@tc##1{\textcolor[rgb]{0.00,0.00,1.00}{##1}}}
\expandafter\def\csname PY@tok@ne\endcsname{\let\PY@bf=\textbf\def\PY@tc##1{\textcolor[rgb]{0.82,0.25,0.23}{##1}}}
\expandafter\def\csname PY@tok@nv\endcsname{\def\PY@tc##1{\textcolor[rgb]{0.10,0.09,0.49}{##1}}}
\expandafter\def\csname PY@tok@no\endcsname{\def\PY@tc##1{\textcolor[rgb]{0.53,0.00,0.00}{##1}}}
\expandafter\def\csname PY@tok@nl\endcsname{\def\PY@tc##1{\textcolor[rgb]{0.63,0.63,0.00}{##1}}}
\expandafter\def\csname PY@tok@ni\endcsname{\let\PY@bf=\textbf\def\PY@tc##1{\textcolor[rgb]{0.60,0.60,0.60}{##1}}}
\expandafter\def\csname PY@tok@na\endcsname{\def\PY@tc##1{\textcolor[rgb]{0.49,0.56,0.16}{##1}}}
\expandafter\def\csname PY@tok@nt\endcsname{\let\PY@bf=\textbf\def\PY@tc##1{\textcolor[rgb]{0.00,0.50,0.00}{##1}}}
\expandafter\def\csname PY@tok@nd\endcsname{\def\PY@tc##1{\textcolor[rgb]{0.67,0.13,1.00}{##1}}}
\expandafter\def\csname PY@tok@s\endcsname{\def\PY@tc##1{\textcolor[rgb]{0.73,0.13,0.13}{##1}}}
\expandafter\def\csname PY@tok@sd\endcsname{\let\PY@it=\textit\def\PY@tc##1{\textcolor[rgb]{0.73,0.13,0.13}{##1}}}
\expandafter\def\csname PY@tok@si\endcsname{\let\PY@bf=\textbf\def\PY@tc##1{\textcolor[rgb]{0.73,0.40,0.53}{##1}}}
\expandafter\def\csname PY@tok@se\endcsname{\let\PY@bf=\textbf\def\PY@tc##1{\textcolor[rgb]{0.73,0.40,0.13}{##1}}}
\expandafter\def\csname PY@tok@sr\endcsname{\def\PY@tc##1{\textcolor[rgb]{0.73,0.40,0.53}{##1}}}
\expandafter\def\csname PY@tok@ss\endcsname{\def\PY@tc##1{\textcolor[rgb]{0.10,0.09,0.49}{##1}}}
\expandafter\def\csname PY@tok@sx\endcsname{\def\PY@tc##1{\textcolor[rgb]{0.00,0.50,0.00}{##1}}}
\expandafter\def\csname PY@tok@m\endcsname{\def\PY@tc##1{\textcolor[rgb]{0.40,0.40,0.40}{##1}}}
\expandafter\def\csname PY@tok@gh\endcsname{\let\PY@bf=\textbf\def\PY@tc##1{\textcolor[rgb]{0.00,0.00,0.50}{##1}}}
\expandafter\def\csname PY@tok@gu\endcsname{\let\PY@bf=\textbf\def\PY@tc##1{\textcolor[rgb]{0.50,0.00,0.50}{##1}}}
\expandafter\def\csname PY@tok@gd\endcsname{\def\PY@tc##1{\textcolor[rgb]{0.63,0.00,0.00}{##1}}}
\expandafter\def\csname PY@tok@gi\endcsname{\def\PY@tc##1{\textcolor[rgb]{0.00,0.63,0.00}{##1}}}
\expandafter\def\csname PY@tok@gr\endcsname{\def\PY@tc##1{\textcolor[rgb]{1.00,0.00,0.00}{##1}}}
\expandafter\def\csname PY@tok@ge\endcsname{\let\PY@it=\textit}
\expandafter\def\csname PY@tok@gs\endcsname{\let\PY@bf=\textbf}
\expandafter\def\csname PY@tok@gp\endcsname{\let\PY@bf=\textbf\def\PY@tc##1{\textcolor[rgb]{0.00,0.00,0.50}{##1}}}
\expandafter\def\csname PY@tok@go\endcsname{\def\PY@tc##1{\textcolor[rgb]{0.53,0.53,0.53}{##1}}}
\expandafter\def\csname PY@tok@gt\endcsname{\def\PY@tc##1{\textcolor[rgb]{0.00,0.27,0.87}{##1}}}
\expandafter\def\csname PY@tok@err\endcsname{\def\PY@bc##1{\setlength{\fboxsep}{0pt}\fcolorbox[rgb]{1.00,0.00,0.00}{1,1,1}{\strut ##1}}}
\expandafter\def\csname PY@tok@kc\endcsname{\let\PY@bf=\textbf\def\PY@tc##1{\textcolor[rgb]{0.00,0.50,0.00}{##1}}}
\expandafter\def\csname PY@tok@kd\endcsname{\let\PY@bf=\textbf\def\PY@tc##1{\textcolor[rgb]{0.00,0.50,0.00}{##1}}}
\expandafter\def\csname PY@tok@kn\endcsname{\let\PY@bf=\textbf\def\PY@tc##1{\textcolor[rgb]{0.00,0.50,0.00}{##1}}}
\expandafter\def\csname PY@tok@kr\endcsname{\let\PY@bf=\textbf\def\PY@tc##1{\textcolor[rgb]{0.00,0.50,0.00}{##1}}}
\expandafter\def\csname PY@tok@bp\endcsname{\def\PY@tc##1{\textcolor[rgb]{0.00,0.50,0.00}{##1}}}
\expandafter\def\csname PY@tok@fm\endcsname{\def\PY@tc##1{\textcolor[rgb]{0.00,0.00,1.00}{##1}}}
\expandafter\def\csname PY@tok@vc\endcsname{\def\PY@tc##1{\textcolor[rgb]{0.10,0.09,0.49}{##1}}}
\expandafter\def\csname PY@tok@vg\endcsname{\def\PY@tc##1{\textcolor[rgb]{0.10,0.09,0.49}{##1}}}
\expandafter\def\csname PY@tok@vi\endcsname{\def\PY@tc##1{\textcolor[rgb]{0.10,0.09,0.49}{##1}}}
\expandafter\def\csname PY@tok@vm\endcsname{\def\PY@tc##1{\textcolor[rgb]{0.10,0.09,0.49}{##1}}}
\expandafter\def\csname PY@tok@sa\endcsname{\def\PY@tc##1{\textcolor[rgb]{0.73,0.13,0.13}{##1}}}
\expandafter\def\csname PY@tok@sb\endcsname{\def\PY@tc##1{\textcolor[rgb]{0.73,0.13,0.13}{##1}}}
\expandafter\def\csname PY@tok@sc\endcsname{\def\PY@tc##1{\textcolor[rgb]{0.73,0.13,0.13}{##1}}}
\expandafter\def\csname PY@tok@dl\endcsname{\def\PY@tc##1{\textcolor[rgb]{0.73,0.13,0.13}{##1}}}
\expandafter\def\csname PY@tok@s2\endcsname{\def\PY@tc##1{\textcolor[rgb]{0.73,0.13,0.13}{##1}}}
\expandafter\def\csname PY@tok@sh\endcsname{\def\PY@tc##1{\textcolor[rgb]{0.73,0.13,0.13}{##1}}}
\expandafter\def\csname PY@tok@s1\endcsname{\def\PY@tc##1{\textcolor[rgb]{0.73,0.13,0.13}{##1}}}
\expandafter\def\csname PY@tok@mb\endcsname{\def\PY@tc##1{\textcolor[rgb]{0.40,0.40,0.40}{##1}}}
\expandafter\def\csname PY@tok@mf\endcsname{\def\PY@tc##1{\textcolor[rgb]{0.40,0.40,0.40}{##1}}}
\expandafter\def\csname PY@tok@mh\endcsname{\def\PY@tc##1{\textcolor[rgb]{0.40,0.40,0.40}{##1}}}
\expandafter\def\csname PY@tok@mi\endcsname{\def\PY@tc##1{\textcolor[rgb]{0.40,0.40,0.40}{##1}}}
\expandafter\def\csname PY@tok@il\endcsname{\def\PY@tc##1{\textcolor[rgb]{0.40,0.40,0.40}{##1}}}
\expandafter\def\csname PY@tok@mo\endcsname{\def\PY@tc##1{\textcolor[rgb]{0.40,0.40,0.40}{##1}}}
\expandafter\def\csname PY@tok@ch\endcsname{\let\PY@it=\textit\def\PY@tc##1{\textcolor[rgb]{0.25,0.50,0.50}{##1}}}
\expandafter\def\csname PY@tok@cm\endcsname{\let\PY@it=\textit\def\PY@tc##1{\textcolor[rgb]{0.25,0.50,0.50}{##1}}}
\expandafter\def\csname PY@tok@cpf\endcsname{\let\PY@it=\textit\def\PY@tc##1{\textcolor[rgb]{0.25,0.50,0.50}{##1}}}
\expandafter\def\csname PY@tok@c1\endcsname{\let\PY@it=\textit\def\PY@tc##1{\textcolor[rgb]{0.25,0.50,0.50}{##1}}}
\expandafter\def\csname PY@tok@cs\endcsname{\let\PY@it=\textit\def\PY@tc##1{\textcolor[rgb]{0.25,0.50,0.50}{##1}}}

\def\PYZbs{\char`\\}
\def\PYZus{\char`\_}
\def\PYZob{\char`\{}
\def\PYZcb{\char`\}}
\def\PYZca{\char`\^}
\def\PYZam{\char`\&}
\def\PYZlt{\char`\<}
\def\PYZgt{\char`\>}
\def\PYZsh{\char`\#}
\def\PYZpc{\char`\%}
\def\PYZdl{\char`\$}
\def\PYZhy{\char`\-}
\def\PYZsq{\char`\'}
\def\PYZdq{\char`\"}
\def\PYZti{\char`\~}
% for compatibility with earlier versions
\def\PYZat{@}
\def\PYZlb{[}
\def\PYZrb{]}
\makeatother


    % For linebreaks inside Verbatim environment from package fancyvrb. 
    \makeatletter
        \newbox\Wrappedcontinuationbox 
        \newbox\Wrappedvisiblespacebox 
        \newcommand*\Wrappedvisiblespace {\textcolor{red}{\textvisiblespace}} 
        \newcommand*\Wrappedcontinuationsymbol {\textcolor{red}{\llap{\tiny$\m@th\hookrightarrow$}}} 
        \newcommand*\Wrappedcontinuationindent {3ex } 
        \newcommand*\Wrappedafterbreak {\kern\Wrappedcontinuationindent\copy\Wrappedcontinuationbox} 
        % Take advantage of the already applied Pygments mark-up to insert 
        % potential linebreaks for TeX processing. 
        %        {, <, #, %, $, ' and ": go to next line. 
        %        _, }, ^, &, >, - and ~: stay at end of broken line. 
        % Use of \textquotesingle for straight quote. 
        \newcommand*\Wrappedbreaksatspecials {% 
            \def\PYGZus{\discretionary{\char`\_}{\Wrappedafterbreak}{\char`\_}}% 
            \def\PYGZob{\discretionary{}{\Wrappedafterbreak\char`\{}{\char`\{}}% 
            \def\PYGZcb{\discretionary{\char`\}}{\Wrappedafterbreak}{\char`\}}}% 
            \def\PYGZca{\discretionary{\char`\^}{\Wrappedafterbreak}{\char`\^}}% 
            \def\PYGZam{\discretionary{\char`\&}{\Wrappedafterbreak}{\char`\&}}% 
            \def\PYGZlt{\discretionary{}{\Wrappedafterbreak\char`\<}{\char`\<}}% 
            \def\PYGZgt{\discretionary{\char`\>}{\Wrappedafterbreak}{\char`\>}}% 
            \def\PYGZsh{\discretionary{}{\Wrappedafterbreak\char`\#}{\char`\#}}% 
            \def\PYGZpc{\discretionary{}{\Wrappedafterbreak\char`\%}{\char`\%}}% 
            \def\PYGZdl{\discretionary{}{\Wrappedafterbreak\char`\$}{\char`\$}}% 
            \def\PYGZhy{\discretionary{\char`\-}{\Wrappedafterbreak}{\char`\-}}% 
            \def\PYGZsq{\discretionary{}{\Wrappedafterbreak\textquotesingle}{\textquotesingle}}% 
            \def\PYGZdq{\discretionary{}{\Wrappedafterbreak\char`\"}{\char`\"}}% 
            \def\PYGZti{\discretionary{\char`\~}{\Wrappedafterbreak}{\char`\~}}% 
        } 
        % Some characters . , ; ? ! / are not pygmentized. 
        % This macro makes them "active" and they will insert potential linebreaks 
        \newcommand*\Wrappedbreaksatpunct {% 
            \lccode`\~`\.\lowercase{\def~}{\discretionary{\hbox{\char`\.}}{\Wrappedafterbreak}{\hbox{\char`\.}}}% 
            \lccode`\~`\,\lowercase{\def~}{\discretionary{\hbox{\char`\,}}{\Wrappedafterbreak}{\hbox{\char`\,}}}% 
            \lccode`\~`\;\lowercase{\def~}{\discretionary{\hbox{\char`\;}}{\Wrappedafterbreak}{\hbox{\char`\;}}}% 
            \lccode`\~`\:\lowercase{\def~}{\discretionary{\hbox{\char`\:}}{\Wrappedafterbreak}{\hbox{\char`\:}}}% 
            \lccode`\~`\?\lowercase{\def~}{\discretionary{\hbox{\char`\?}}{\Wrappedafterbreak}{\hbox{\char`\?}}}% 
            \lccode`\~`\!\lowercase{\def~}{\discretionary{\hbox{\char`\!}}{\Wrappedafterbreak}{\hbox{\char`\!}}}% 
            \lccode`\~`\/\lowercase{\def~}{\discretionary{\hbox{\char`\/}}{\Wrappedafterbreak}{\hbox{\char`\/}}}% 
            \catcode`\.\active
            \catcode`\,\active 
            \catcode`\;\active
            \catcode`\:\active
            \catcode`\?\active
            \catcode`\!\active
            \catcode`\/\active 
            \lccode`\~`\~ 	
        }
    \makeatother

    \let\OriginalVerbatim=\Verbatim
    \makeatletter
    \renewcommand{\Verbatim}[1][1]{%
        %\parskip\z@skip
        \sbox\Wrappedcontinuationbox {\Wrappedcontinuationsymbol}%
        \sbox\Wrappedvisiblespacebox {\FV@SetupFont\Wrappedvisiblespace}%
        \def\FancyVerbFormatLine ##1{\hsize\linewidth
            \vtop{\raggedright\hyphenpenalty\z@\exhyphenpenalty\z@
                \doublehyphendemerits\z@\finalhyphendemerits\z@
                \strut ##1\strut}%
        }%
        % If the linebreak is at a space, the latter will be displayed as visible
        % space at end of first line, and a continuation symbol starts next line.
        % Stretch/shrink are however usually zero for typewriter font.
        \def\FV@Space {%
            \nobreak\hskip\z@ plus\fontdimen3\font minus\fontdimen4\font
            \discretionary{\copy\Wrappedvisiblespacebox}{\Wrappedafterbreak}
            {\kern\fontdimen2\font}%
        }%
        
        % Allow breaks at special characters using \PYG... macros.
        \Wrappedbreaksatspecials
        % Breaks at punctuation characters . , ; ? ! and / need catcode=\active 	
        \OriginalVerbatim[#1,codes*=\Wrappedbreaksatpunct]%
    }
    \makeatother

    % Exact colors from NB
    \definecolor{incolor}{HTML}{303F9F}
    \definecolor{outcolor}{HTML}{D84315}
    \definecolor{cellborder}{HTML}{CFCFCF}
    \definecolor{cellbackground}{HTML}{F7F7F7}
    
    % prompt
    \makeatletter
    \newcommand{\boxspacing}{\kern\kvtcb@left@rule\kern\kvtcb@boxsep}
    \makeatother
    \newcommand{\prompt}[4]{
        \ttfamily\llap{{\color{#2}[#3]:\hspace{3pt}#4}}\vspace{-\baselineskip}
    }
    

    
    % Prevent overflowing lines due to hard-to-break entities
    \sloppy 
    % Setup hyperref package
    \hypersetup{
      breaklinks=true,  % so long urls are correctly broken across lines
      colorlinks=true,
      urlcolor=urlcolor,
      linkcolor=linkcolor,
      citecolor=citecolor,
      }
    % Slightly bigger margins than the latex defaults
    
    \geometry{verbose,tmargin=1in,bmargin=1in,lmargin=1in,rmargin=1in}
    
\makeatletter
\renewcommand{\@seccntformat}[1]{}
\makeatother  

\begin{document}
    \title{CS 156a - Final Exam}
    \author{Samuel Patrone, 2140749}
    \maketitle
    

The following notebook is publicly available
\href{https://github.com/spatrone/CS156A-Caltech.git}{here}.

\tableofcontents

    \hypertarget{non-linear-transform}{%
\part{Non-linear Transform}\label{non-linear-transform}}

\hypertarget{problem-1}{%
\section{Problem 1}\label{problem-1}}

\hypertarget{answer-e-none-of-the-above.}{%
\subsection{Answer: {[}e{]} None of the
above.}\label{answer-e-none-of-the-above.}}

\hypertarget{derivation}{%
\subsection{Derivation:}\label{derivation}}

Let's consider the general \(\mathcal{Q}\)th order polynomial transform
\(\Phi_\mathcal{Q}\) for the space \(\mathcal{x}=\mathbb{R}^d\). We can
find the dimensionality \(\tilde{d}\) of the feature space
\(\mathcal{Z}\) by observing that we can form \(C(d,k)\) different
monomials of order \(k\) from the \(d\) initial coordinates, where

\begin{equation}
C(d,k)={d+k-1 \choose k}\,.
\end{equation}

Since \(\Phi_\mathcal{Q}\) will have all possible monomials up to order
\(\mathcal{Q}\) as transformed coordinates, the feature space
\(\mathcal{Z}\) will have a dimensionality

\begin{equation}
\tilde{d}(Q,d)=\sum^Q_{k=1}{d+k-1 \choose k}\,.
\end{equation}

For \(d=2\) and \(Q=10\), we get

\begin{equation}
\tilde{d}(Q=10,d=2)=\sum^Q_{k=1}{k+1 \choose k}= \sum^Q_{k=1} k+1 = \frac{Q(Q+3)}{2} = \frac{10 \times 13}{2}=65\,.
\end{equation}

    \hypertarget{bias-and-variance}{%
\part{Bias and Variance}\label{bias-and-variance}}

\hypertarget{problem-2}{%
\section{Problem 2}\label{problem-2}}

\hypertarget{answer-d-mathcalh-is-the-logistic-regression-model.}{%
\subsection{\texorpdfstring{Answer: {[}d{]} \(\mathcal{H}\) is the
logistic regression
model.}{Answer: {[}d{]} \textbackslash{}mathcal\{H\} is the logistic regression model.}}\label{answer-d-mathcalh-is-the-logistic-regression-model.}}

\hypertarget{derivation}{%
\subsection{Derivation:}\label{derivation}}

By definition of average hypothesis,

\begin{equation}
\bar{g}=\frac{1}{K}\sum_{k=1}^K g_k\,,
\end{equation}

where \(g_k\) is the final hypothesis generated by the dataset
\(\mathcal{D}_k\). Since the average hypothesis is a linear combination
of hypothesis taken from the space \(\mathcal{H}\), only a linear
combination of non-linear functions can be outside of \(\mathcal{H}\).
Specifically, for the case of logistic regression we use non-linear
functions as the sigmoid or the hyperbolic tangent, whose linear
combinations are not generally sigmoids or hyperbolic tangents
respectively.

    \hypertarget{overfitting}{%
\part{Overfitting}\label{overfitting}}

\hypertarget{problem-3}{%
\section{Problem 3}\label{problem-3}}

\hypertarget{answer-d-we-can-always-determine-if-there-is-overfitting-by-comparing-the-values-of-e_out-e_in.}{%
\subsection{\texorpdfstring{Answer: {[}d{]} We can always determine
if there is overfitting by comparing the values of
\((E_{out}-E_{in})\).}{Answer: {[}d{]} We can always determine if there is overfitting by comparing the values of (E\_\{out\}-E\_\{in\}).}}\label{answer-d-we-can-always-determine-if-there-is-overfitting-by-comparing-the-values-of-e_out-e_in.}}

\hypertarget{derivation}{%
\subsection{Derivation:}\label{derivation}}

Overfitting is, by definition, the case in which, between two
hypothesis, the one with lower \(E_{in}\) is preferred, and it results
in an higher \(E_{out}\). This happens because \(E_{in}\) starts loosing
track of \(E_{out}\) since the chosen hypothesis is fitting the noise
(stochastic or deterministic). If there is overfitting, there must be
two or more hypothesis that have different values of
\((E_{out}-E_{in})\). However, comparing the values of
\((E_{out}-E_{in})\) for two hypothesis does not always determine of
there is overfitting in choosing the one with a lower \(E_{in}\). For
example, the difference between the two hypothesis in the values of
\((E_{out}-E_{in})\) may depend on underfitting (which will produce an
higher \(E_{out}\)) or stochastic properties of the test set.

    \hypertarget{problem-4}{%
\section{Problem 4}\label{problem-4}}

\hypertarget{answer-d-stochastic-noise-does-not-depend-on-the-hypothesis-set.}{%
\subsection{Answer: {[}d{]} Stochastic noise does not depend on the
hypothesis
set.}\label{answer-d-stochastic-noise-does-not-depend-on-the-hypothesis-set.}}

\hypertarget{derivation}{%
\subsection{Derivation:}\label{derivation}}

Stochastic noise is a property of the data we are given with respect to
the target function and it is independent on the hypothesis set chosen.
The amount of noise that can be captured by the fit depends on the
hypothesis set chosen: an higher complexity of the hypothesis may reduce
\(E_{in}\) although resulting in an higher \(E_{out}\), leading to the
most common case of overfitting.

    \hypertarget{regularization}{%
\part{Regularization}\label{regularization}}

\hypertarget{problem-5}{%
\section{Problem 5}\label{problem-5}}

\hypertarget{answer-a-mathbfw_regmathbfw_lin}{%
\subsection{\texorpdfstring{Answer: {[}a{]}
\(\mathbf{w}_{reg}=\mathbf{w}_{lin}\)}{Answer: {[}a{]} \textbackslash{}mathbf\{w\}\_\{reg\}=\textbackslash{}mathbf\{w\}\_\{lin\}}}\label{answer-a-mathbfw_regmathbfw_lin}}

\hypertarget{derivation}{%
\subsection{Derivation:}\label{derivation}}

If \(\mathbf{w}_{lin}\) already satisfies the constrain
\(\mathbf{w}_{lin}^T\Gamma^T\Gamma\mathbf{w}_{lin}\le C\), the Tikhonov
regularization constrain will not affect the results of linear
regression, giving \(\mathbf{w}_{reg}=\mathbf{w}_{lin}\).

This result can be easily seen by noticing that the Tikhonov regularized
weights \(\mathbf{w}_{reg}\) is a solution of the following problem:

\begin{equation}
{\rm min}_\mathbf{w} \frac{1}{N}\sum_{n=1}^N(\mathbf{w}^T\mathbf{x}_n-y_n)^2 \;\; {\rm subject \; to} \;\; \mathbf{w}^T\Gamma^T\Gamma\mathbf{w}\le C\,.
\end{equation}

We define the soft-order constrained hypothesis set \(\mathcal{H}(C)\)
as

\begin{equation}
\mathcal{H}(C)=\left\{h|h(\mathbf{x})=\mathbf{w}^T\mathbf{x}, \mathbf{w}^T\Gamma^T\Gamma\mathbf{w}\le C\right\}
\end{equation}

where \(\mathbf{w}_{reg}\in \mathcal{H}(C)\) by definition.

Since \(\mathbf{w}_{lin}^T\Gamma^T\Gamma\mathbf{w}_{lin}\le C\), we have
that \(\mathbf{w}_{lin}\in\mathcal{H}(C)\).
Therefore,\(\mathbf{w}_{reg}=\mathbf{w}_{lin}\).

    \hypertarget{problem-6}{%
\section{Problem 6}\label{problem-6}}

\hypertarget{answer-b-translated-into-augmented-error}{%
\subsection{Answer: {[}b{]} Translated into augmented
error}\label{answer-b-translated-into-augmented-error}}

\hypertarget{derivation}{%
\subsection{Derivation:}\label{derivation}}

Since soft-order constraints for polynomial models are constrained
minimization of \(E_{in}\), we can equivalently solve an unconstrained
minimization of a different function, called augmented error

\begin{equation}
E_{aug}(h,\lambda,\Omega) = E_{in}(h)+\frac{\lambda}{N}\Omega(h)\,.
\end{equation}

where \(\lambda\) is a Lagrange multiplier that controls the amount of
regularization, introducing a penalty term, and it is related to \(C\),
the parameter controlling the soft-order constrain.

For the Tikhonov regularizer above,
\(\Omega(h)=\mathbf{w}^T\Gamma^T\Gamma\mathbf{w}\).

    \hypertarget{regularized-linear-regression}{%
\part{Regularized Linear
Regression}\label{regularized-linear-regression}}

\hypertarget{problems-7-8}{%
\section{Problems 7-8}\label{problems-7-8}}

\hypertarget{answers-d-8-versus-all-b-1-versus-all}{%
\subsection{Answers: {[}d{]} 8 versus all, {[}b{]} 1 versus
all}\label{answers-d-8-versus-all-b-1-versus-all}}

\hypertarget{code}{%
\subsection{Code:}\label{code}}

    \begin{tcolorbox}[breakable, size=fbox, boxrule=1pt, pad at break*=1mm,colback=cellbackground, colframe=cellborder]
\prompt{In}{incolor}{1}{\boxspacing}
\begin{Verbatim}[commandchars=\\\{\}]
\PY{k+kn}{import} \PY{n+nn}{numpy} \PY{k}{as} \PY{n+nn}{np}
\PY{k+kn}{import} \PY{n+nn}{matplotlib}\PY{n+nn}{.}\PY{n+nn}{pyplot} \PY{k}{as} \PY{n+nn}{plt}
\PY{k+kn}{import} \PY{n+nn}{pandas} \PY{k}{as} \PY{n+nn}{pd}

\PY{c+c1}{\PYZsh{}import data}

\PY{n}{training\PYZus{}set}\PY{o}{=}\PY{n}{pd}\PY{o}{.}\PY{n}{read\PYZus{}csv}\PY{p}{(}\PY{l+s+s1}{\PYZsq{}}\PY{l+s+s1}{features.train}\PY{l+s+s1}{\PYZsq{}}\PY{p}{,}\PY{n}{header}\PY{o}{=}\PY{k+kc}{None}\PY{p}{,}\PY{n}{delim\PYZus{}whitespace}\PY{o}{=}\PY{k+kc}{True}\PY{p}{)}
\PY{n}{testing\PYZus{}set}\PY{o}{=}\PY{n}{pd}\PY{o}{.}\PY{n}{read\PYZus{}csv}\PY{p}{(}\PY{l+s+s1}{\PYZsq{}}\PY{l+s+s1}{features.test}\PY{l+s+s1}{\PYZsq{}}\PY{p}{,}\PY{n}{header}\PY{o}{=}\PY{k+kc}{None}\PY{p}{,}\PY{n}{delim\PYZus{}whitespace}\PY{o}{=}\PY{k+kc}{True}\PY{p}{)}

\PY{n}{train\PYZus{}pts}\PY{o}{=}\PY{n}{training\PYZus{}set}\PY{p}{[}\PY{p}{[}\PY{l+m+mi}{1}\PY{p}{,} \PY{l+m+mi}{2}\PY{p}{]}\PY{p}{]}\PY{o}{.}\PY{n}{to\PYZus{}numpy}\PY{p}{(}\PY{p}{)}
\PY{n}{train\PYZus{}y}\PY{o}{=}\PY{n}{training\PYZus{}set}\PY{p}{[}\PY{l+m+mi}{0}\PY{p}{]}\PY{o}{.}\PY{n}{to\PYZus{}numpy}\PY{p}{(}\PY{p}{)}

\PY{n}{test\PYZus{}pts}\PY{o}{=}\PY{n}{testing\PYZus{}set}\PY{p}{[}\PY{p}{[}\PY{l+m+mi}{1}\PY{p}{,} \PY{l+m+mi}{2}\PY{p}{]}\PY{p}{]}\PY{o}{.}\PY{n}{to\PYZus{}numpy}\PY{p}{(}\PY{p}{)}
\PY{n}{test\PYZus{}y}\PY{o}{=}\PY{n}{testing\PYZus{}set}\PY{p}{[}\PY{l+m+mi}{0}\PY{p}{]}\PY{o}{.}\PY{n}{to\PYZus{}numpy}\PY{p}{(}\PY{p}{)}
    
\PY{c+c1}{\PYZsh{} Binary classifier}

\PY{k}{def} \PY{n+nf}{label\PYZus{}n}\PY{p}{(}\PY{n}{y}\PY{p}{,}\PY{n}{n}\PY{p}{)}\PY{p}{:}
    \PY{n}{res}\PY{o}{=}\PY{o}{\PYZhy{}}\PY{n}{np}\PY{o}{.}\PY{n}{ones}\PY{p}{(}\PY{n+nb}{len}\PY{p}{(}\PY{n}{y}\PY{p}{)}\PY{p}{)}
    \PY{n}{res}\PY{p}{[}\PY{n}{y}\PY{o}{==}\PY{n}{n}\PY{p}{]}\PY{o}{=}\PY{l+m+mi}{1}
    \PY{k}{return} \PY{n}{res}

\PY{c+c1}{\PYZsh{} Non\PYZhy{}linear Transform}

\PY{k}{def} \PY{n+nf}{transform}\PY{p}{(}\PY{n}{pts}\PY{p}{,}\PY{n}{nonlin}\PY{o}{=}\PY{k+kc}{True}\PY{p}{)}\PY{p}{:}
    \PY{n}{res}\PY{o}{=}\PY{p}{[}\PY{p}{]}
    \PY{k}{if}\PY{p}{(}\PY{n}{nonlin}\PY{p}{)}\PY{p}{:}
        \PY{k}{for} \PY{n}{i} \PY{o+ow}{in} \PY{n+nb}{range}\PY{p}{(}\PY{n+nb}{len}\PY{p}{(}\PY{n}{pts}\PY{p}{)}\PY{p}{)}\PY{p}{:}
            \PY{n}{x1}\PY{o}{=}\PY{n}{pts}\PY{p}{[}\PY{n}{i}\PY{p}{]}\PY{p}{[}\PY{l+m+mi}{0}\PY{p}{]}
            \PY{n}{x2}\PY{o}{=}\PY{n}{pts}\PY{p}{[}\PY{n}{i}\PY{p}{]}\PY{p}{[}\PY{l+m+mi}{1}\PY{p}{]}
            \PY{n}{res}\PY{o}{.}\PY{n}{append}\PY{p}{(}\PY{p}{[}\PY{l+m+mi}{1}\PY{p}{,}\PY{n}{x1}\PY{p}{,}\PY{n}{x2}\PY{p}{,}\PY{n}{x1}\PY{o}{*}\PY{n}{x2}\PY{p}{,}\PY{n}{x1}\PY{o}{*}\PY{o}{*}\PY{l+m+mi}{2}\PY{p}{,}\PY{n}{x2}\PY{o}{*}\PY{o}{*}\PY{l+m+mi}{2}\PY{p}{]}\PY{p}{)}
    \PY{k}{else}\PY{p}{:}
        \PY{k}{for} \PY{n}{i} \PY{o+ow}{in} \PY{n+nb}{range}\PY{p}{(}\PY{n+nb}{len}\PY{p}{(}\PY{n}{pts}\PY{p}{)}\PY{p}{)}\PY{p}{:}
            \PY{n}{x1}\PY{o}{=}\PY{n}{pts}\PY{p}{[}\PY{n}{i}\PY{p}{]}\PY{p}{[}\PY{l+m+mi}{0}\PY{p}{]}
            \PY{n}{x2}\PY{o}{=}\PY{n}{pts}\PY{p}{[}\PY{n}{i}\PY{p}{]}\PY{p}{[}\PY{l+m+mi}{1}\PY{p}{]}
            \PY{n}{res}\PY{o}{.}\PY{n}{append}\PY{p}{(}\PY{p}{[}\PY{l+m+mi}{1}\PY{p}{,}\PY{n}{x1}\PY{p}{,}\PY{n}{x2}\PY{p}{]}\PY{p}{)}
    \PY{k}{return} \PY{n}{np}\PY{o}{.}\PY{n}{array}\PY{p}{(}\PY{n}{res}\PY{p}{)}
        
\PY{c+c1}{\PYZsh{}Linear Regression with weight decay}

\PY{k}{def} \PY{n+nf}{h}\PY{p}{(}\PY{n}{pts}\PY{p}{,}\PY{n}{w}\PY{p}{)}\PY{p}{:}
    \PY{k}{return} \PY{n}{np}\PY{o}{.}\PY{n}{sign}\PY{p}{(}\PY{n}{np}\PY{o}{.}\PY{n}{dot}\PY{p}{(}\PY{n}{w}\PY{p}{,}\PY{n}{pts}\PY{o}{.}\PY{n}{T}\PY{p}{)}\PY{p}{)}

\PY{k}{def} \PY{n+nf}{lin\PYZus{}reg\PYZus{}w\PYZus{}lam}\PY{p}{(}\PY{n}{X}\PY{p}{,}\PY{n}{y}\PY{p}{,}\PY{n}{lam}\PY{p}{)}\PY{p}{:}
    \PY{n}{pinv\PYZus{}decay}\PY{o}{=}\PY{n}{np}\PY{o}{.}\PY{n}{dot}\PY{p}{(}\PY{n}{np}\PY{o}{.}\PY{n}{linalg}\PY{o}{.}\PY{n}{inv}\PY{p}{(}\PY{n}{np}\PY{o}{.}\PY{n}{dot}\PY{p}{(}\PY{n}{X}\PY{o}{.}\PY{n}{T}\PY{p}{,}\PY{n}{X}\PY{p}{)}\PY{o}{+}\PY{n}{lam}\PY{o}{*}\PY{n}{np}\PY{o}{.}\PY{n}{identity}\PY{p}{(}\PY{n+nb}{len}\PY{p}{(}\PY{n}{X}\PY{o}{.}\PY{n}{T}\PY{p}{)}\PY{p}{)}\PY{p}{)}\PY{p}{,}\PY{n}{X}\PY{o}{.}\PY{n}{T}\PY{p}{)}
    \PY{k}{return} \PY{n}{np}\PY{o}{.}\PY{n}{dot}\PY{p}{(}\PY{n}{pinv\PYZus{}decay}\PY{p}{,}\PY{n}{y}\PY{p}{)}

\PY{k}{def} \PY{n+nf}{lin\PYZus{}reg\PYZus{}wdecay}\PY{p}{(}\PY{n}{train\PYZus{}pts}\PY{p}{,}\PY{n}{train\PYZus{}y}\PY{p}{,}\PY{n}{test\PYZus{}pts}\PY{p}{,}\PY{n}{test\PYZus{}y}\PY{p}{,}\PY{n}{lam}\PY{o}{=}\PY{l+m+mi}{1}\PY{p}{,}\PY{n}{res}\PY{o}{=}\PY{k+kc}{True}\PY{p}{)}\PY{p}{:}
    \PY{n}{N\PYZus{}train}\PY{o}{=}\PY{n+nb}{len}\PY{p}{(}\PY{n}{train\PYZus{}pts}\PY{p}{)}
    \PY{n}{N\PYZus{}test}\PY{o}{=}\PY{n+nb}{len}\PY{p}{(}\PY{n}{test\PYZus{}pts}\PY{p}{)}
    
    \PY{n}{w}\PY{o}{=}\PY{n}{lin\PYZus{}reg\PYZus{}w\PYZus{}lam}\PY{p}{(}\PY{n}{train\PYZus{}pts}\PY{p}{,}\PY{n}{train\PYZus{}y}\PY{p}{,}\PY{n}{lam}\PY{p}{)}
    
    \PY{c+c1}{\PYZsh{}Ein computation}
    \PY{n}{gin}\PY{o}{=}\PY{n}{h}\PY{p}{(}\PY{n}{train\PYZus{}pts}\PY{p}{,}\PY{n}{w}\PY{p}{)}
    \PY{n}{testgin}\PY{o}{=}\PY{p}{(}\PY{n}{gin}\PY{o}{==}\PY{n}{train\PYZus{}y}\PY{p}{)}
    \PY{n}{Ein}\PY{o}{=}\PY{n+nb}{len}\PY{p}{(}\PY{n}{np}\PY{o}{.}\PY{n}{where}\PY{p}{(}\PY{n}{testgin}\PY{o}{==}\PY{k+kc}{False}\PY{p}{)}\PY{p}{[}\PY{l+m+mi}{0}\PY{p}{]}\PY{p}{)}\PY{o}{/}\PY{n}{N\PYZus{}train}
    
    \PY{c+c1}{\PYZsh{}Eout computation}
    \PY{n}{gout}\PY{o}{=}\PY{n}{h}\PY{p}{(}\PY{n}{test\PYZus{}pts}\PY{p}{,}\PY{n}{w}\PY{p}{)}
    \PY{n}{testgout}\PY{o}{=}\PY{p}{(}\PY{n}{gout}\PY{o}{==}\PY{n}{test\PYZus{}y}\PY{p}{)}
    \PY{n}{Eout}\PY{o}{=}\PY{n+nb}{len}\PY{p}{(}\PY{n}{np}\PY{o}{.}\PY{n}{where}\PY{p}{(}\PY{n}{testgout}\PY{o}{==}\PY{k+kc}{False}\PY{p}{)}\PY{p}{[}\PY{l+m+mi}{0}\PY{p}{]}\PY{p}{)}\PY{o}{/}\PY{n}{N\PYZus{}test}
    
    \PY{c+c1}{\PYZsh{}print results}
    \PY{k}{if}\PY{p}{(}\PY{n}{res}\PY{o}{==}\PY{k+kc}{True}\PY{p}{)}\PY{p}{:}
        \PY{n+nb}{print}\PY{p}{(}\PY{n}{f}\PY{l+s+s1}{\PYZsq{}}\PY{l+s+s1}{Linear Regression results with Lambda=}\PY{l+s+si}{\PYZob{}lam:.2f\PYZcb{}}\PY{l+s+s1}{:}\PY{l+s+se}{\PYZbs{}n}\PY{l+s+s1}{Ein=}\PY{l+s+si}{\PYZob{}Ein:.2f\PYZcb{}}\PY{l+s+se}{\PYZbs{}n}\PY{l+s+s1}{Eout=}\PY{l+s+si}{\PYZob{}Eout:.2f\PYZcb{}}\PY{l+s+s1}{\PYZsq{}}\PY{p}{)}
    
    \PY{k}{return} \PY{n}{Ein}\PY{p}{,}\PY{n}{Eout}

\PY{n}{Ein}\PY{o}{=}\PY{p}{[}\PY{p}{]}
\PY{n}{Eout}\PY{o}{=}\PY{p}{[}\PY{p}{]}
\PY{n}{Ein\PYZus{}tr}\PY{o}{=}\PY{p}{[}\PY{p}{]}
\PY{n}{Eout\PYZus{}tr}\PY{o}{=}\PY{p}{[}\PY{p}{]}
\PY{n}{label}\PY{o}{=}\PY{p}{[}\PY{p}{]}

\PY{k}{for} \PY{n}{i} \PY{o+ow}{in} \PY{n+nb}{range}\PY{p}{(}\PY{l+m+mi}{10}\PY{p}{)}\PY{p}{:}
    \PY{n}{x\PYZus{}train}\PY{o}{=}\PY{n}{transform}\PY{p}{(}\PY{n}{train\PYZus{}pts}\PY{p}{,}\PY{n}{nonlin}\PY{o}{=}\PY{k+kc}{False}\PY{p}{)}
    \PY{n}{z\PYZus{}train}\PY{o}{=}\PY{n}{transform}\PY{p}{(}\PY{n}{train\PYZus{}pts}\PY{p}{)}
    \PY{n}{y\PYZus{}train}\PY{o}{=}\PY{n}{label\PYZus{}n}\PY{p}{(}\PY{n}{train\PYZus{}y}\PY{p}{,}\PY{n}{i}\PY{p}{)}
    
    \PY{n}{x\PYZus{}test}\PY{o}{=}\PY{n}{transform}\PY{p}{(}\PY{n}{test\PYZus{}pts}\PY{p}{,}\PY{n}{nonlin}\PY{o}{=}\PY{k+kc}{False}\PY{p}{)}
    \PY{n}{z\PYZus{}test}\PY{o}{=}\PY{n}{transform}\PY{p}{(}\PY{n}{test\PYZus{}pts}\PY{p}{)}
    \PY{n}{y\PYZus{}test}\PY{o}{=}\PY{n}{label\PYZus{}n}\PY{p}{(}\PY{n}{test\PYZus{}y}\PY{p}{,}\PY{n}{i}\PY{p}{)}
    
    \PY{n}{Ein\PYZus{}tmp}\PY{p}{,}\PY{n}{Eout\PYZus{}tmp}\PY{o}{=}\PY{n}{lin\PYZus{}reg\PYZus{}wdecay}\PY{p}{(}\PY{n}{x\PYZus{}train}\PY{p}{,}\PY{n}{y\PYZus{}train}\PY{p}{,} \PY{n}{x\PYZus{}test}\PY{p}{,}\PY{n}{y\PYZus{}test}\PY{p}{,}\PY{n}{res}\PY{o}{=}\PY{k+kc}{False}\PY{p}{)}
    \PY{n}{Ein\PYZus{}tr\PYZus{}tmp}\PY{p}{,}\PY{n}{Eout\PYZus{}tr\PYZus{}tmp}\PY{o}{=}\PY{n}{lin\PYZus{}reg\PYZus{}wdecay}\PY{p}{(}\PY{n}{z\PYZus{}train}\PY{p}{,}\PY{n}{y\PYZus{}train}\PY{p}{,} \PY{n}{z\PYZus{}test}\PY{p}{,}\PY{n}{y\PYZus{}test}\PY{p}{,}\PY{n}{res}\PY{o}{=}\PY{k+kc}{False}\PY{p}{)}
    
    \PY{n}{Ein}\PY{o}{.}\PY{n}{append}\PY{p}{(}\PY{n}{Ein\PYZus{}tmp}\PY{p}{)}
    \PY{n}{Ein\PYZus{}tr}\PY{o}{.}\PY{n}{append}\PY{p}{(}\PY{n}{Ein\PYZus{}tr\PYZus{}tmp}\PY{p}{)}
    \PY{n}{Eout}\PY{o}{.}\PY{n}{append}\PY{p}{(}\PY{n}{Eout\PYZus{}tmp}\PY{p}{)}
    \PY{n}{Eout\PYZus{}tr}\PY{o}{.}\PY{n}{append}\PY{p}{(}\PY{n}{Eout\PYZus{}tr\PYZus{}tmp}\PY{p}{)}
    
    \PY{n}{label}\PY{o}{.}\PY{n}{append}\PY{p}{(}\PY{l+s+s1}{\PYZsq{}}\PY{l+s+si}{\PYZpc{}i}\PY{l+s+s1}{ VS all}\PY{l+s+s1}{\PYZsq{}} \PY{o}{\PYZpc{}}\PY{k}{i})
    
\PY{n}{pd}\PY{o}{.}\PY{n}{options}\PY{o}{.}\PY{n}{display}\PY{o}{.}\PY{n}{float\PYZus{}format} \PY{o}{=} \PY{l+s+s1}{\PYZsq{}}\PY{l+s+si}{\PYZob{}:,.3f\PYZcb{}}\PY{l+s+s1}{\PYZsq{}}\PY{o}{.}\PY{n}{format}   
\PY{n}{pd}\PY{o}{.}\PY{n}{DataFrame}\PY{p}{(}\PY{n+nb}{list}\PY{p}{(}\PY{n+nb}{zip}\PY{p}{(}\PY{n}{label}\PY{p}{,}\PY{n}{Ein}\PY{p}{,}\PY{n}{Ein\PYZus{}tr}\PY{p}{,}\PY{n}{Eout}\PY{p}{,}\PY{n}{Eout\PYZus{}tr}\PY{p}{)}\PY{p}{)}\PY{p}{,} \PY{n}{columns} \PY{o}{=}\PY{p}{[}\PY{l+s+s1}{\PYZsq{}}\PY{l+s+s1}{Classifier}\PY{l+s+s1}{\PYZsq{}}\PY{p}{,} \PY{l+s+s1}{\PYZsq{}}\PY{l+s+s1}{Ein}\PY{l+s+s1}{\PYZsq{}}\PY{p}{,}\PY{l+s+s1}{\PYZsq{}}\PY{l+s+s1}{Ein\PYZus{}tr}\PY{l+s+s1}{\PYZsq{}}\PY{p}{,}\PY{l+s+s1}{\PYZsq{}}\PY{l+s+s1}{Eout}\PY{l+s+s1}{\PYZsq{}}\PY{p}{,}\PY{l+s+s1}{\PYZsq{}}\PY{l+s+s1}{Eout\PYZus{}tr}\PY{l+s+s1}{\PYZsq{}}\PY{p}{]}\PY{p}{)}
\end{Verbatim}
\end{tcolorbox}

            \begin{tcolorbox}[breakable, size=fbox, boxrule=.5pt, pad at break*=1mm, opacityfill=0]
\prompt{Out}{outcolor}{1}{\boxspacing}
\begin{Verbatim}[commandchars=\\\{\}]
  Classifier   Ein  Ein\_tr  Eout  Eout\_tr
0   0 VS all 0.109   0.102 0.115    0.107
1   1 VS all 0.015   0.012 0.022    0.022
2   2 VS all 0.100   0.100 0.099    0.099
3   3 VS all 0.090   0.090 0.083    0.083
4   4 VS all 0.089   0.089 0.100    0.100
5   5 VS all 0.076   0.076 0.080    0.079
6   6 VS all 0.091   0.091 0.085    0.085
7   7 VS all 0.088   0.088 0.073    0.073
8   8 VS all 0.074   0.074 0.083    0.083
9   9 VS all 0.088   0.088 0.088    0.088
\end{Verbatim}
\end{tcolorbox}
        
    \hypertarget{problem-9}{%
\section{Problem 9}\label{problem-9}}

\hypertarget{answer-e-the-transform-improves-the-out-of-sample-performance-of-5-versus-all-but-by-less-than-5}\label{answer-e-the-transform-improves-the-out-of-sample-performance-of-5-versus-all-but-by-less-than-5}}

\hypertarget{derivation}{%
\subsection{Derivation:}\label{derivation}}

From the table above,

\begin{itemize}
\tightlist
\item
  {[}a{]} is generally false, see for example `0 vs all'
\item
  {[}b{]} is generally false, see for example `1 vs all'
\item
  {[}c{]} is generally false, see for example `5 vs all'
\item
  {[}d{]} is generally false, see for example `0 vs all'
\item
  {[}e{]} is correct,
  \(E^{trans}_{out}({\rm 5\; vs\; all})/E_{out}({\rm 5\; vs\; all})\simeq 0.99\):
  the transform improves the out-of-sample performance by 1\%.
\end{itemize}

    \hypertarget{problem-10}{%
\section{Problem 10}\label{problem-10}}

\hypertarget{answer-a-overfitting-occurs-from-lambda1-to-lambda0.01}{%
\subsection{\texorpdfstring{Answer: {[}a{]} Overfitting occurs (from
\(\lambda=1\) to
\(\lambda=0.01\))}{Answer: {[}a{]} Overfitting occurs (from \textbackslash{}lambda=1 to \textbackslash{}lambda=0.01)}}\label{answer-a-overfitting-occurs-from-lambda1-to-lambda0.01}}

\hypertarget{derivationcode}{%
\subsection{Derivation\&Code:}\label{derivationcode}}

From the results below, it is clear that from \(\lambda=1\) to
\(\lambda=0.01\) overfitting occurs, since \(E_{in}\) is decreasing
while \(E_{out}\) is increasing.

    \begin{tcolorbox}[breakable, size=fbox, boxrule=1pt, pad at break*=1mm,colback=cellbackground, colframe=cellborder]
\prompt{In}{incolor}{2}{\boxspacing}
\begin{Verbatim}[commandchars=\\\{\}]
\PY{n}{z\PYZus{}train} \PY{o}{=} \PY{n}{transform}\PY{p}{(}\PY{n}{train\PYZus{}pts}\PY{p}{[}\PY{n}{np}\PY{o}{.}\PY{n}{logical\PYZus{}or}\PY{p}{(}\PY{n}{train\PYZus{}y}\PY{o}{==}\PY{l+m+mi}{1}\PY{p}{,} \PY{n}{train\PYZus{}y}\PY{o}{==}\PY{l+m+mi}{5}\PY{p}{)}\PY{p}{,} \PY{p}{:}\PY{p}{]}\PY{p}{)}
\PY{n}{z\PYZus{}test} \PY{o}{=} \PY{n}{transform}\PY{p}{(}\PY{n}{test\PYZus{}pts}\PY{p}{[}\PY{n}{np}\PY{o}{.}\PY{n}{logical\PYZus{}or}\PY{p}{(}\PY{n}{test\PYZus{}y}\PY{o}{==}\PY{l+m+mi}{1}\PY{p}{,} \PY{n}{test\PYZus{}y}\PY{o}{==}\PY{l+m+mi}{5}\PY{p}{)}\PY{p}{,} \PY{p}{:}\PY{p}{]}\PY{p}{)}

\PY{n}{y\PYZus{}train} \PY{o}{=} \PY{n}{label\PYZus{}n}\PY{p}{(}\PY{n}{train\PYZus{}y}\PY{p}{[}\PY{n}{np}\PY{o}{.}\PY{n}{logical\PYZus{}or}\PY{p}{(}\PY{n}{train\PYZus{}y}\PY{o}{==}\PY{l+m+mi}{1}\PY{p}{,} \PY{n}{train\PYZus{}y}\PY{o}{==}\PY{l+m+mi}{5}\PY{p}{)}\PY{p}{]}\PY{p}{,} \PY{l+m+mi}{1}\PY{p}{)}
\PY{n}{y\PYZus{}test} \PY{o}{=} \PY{n}{label\PYZus{}n}\PY{p}{(}\PY{n}{test\PYZus{}y}\PY{p}{[}\PY{n}{np}\PY{o}{.}\PY{n}{logical\PYZus{}or}\PY{p}{(}\PY{n}{test\PYZus{}y}\PY{o}{==}\PY{l+m+mi}{1}\PY{p}{,} \PY{n}{test\PYZus{}y}\PY{o}{==}\PY{l+m+mi}{5}\PY{p}{)}\PY{p}{]}\PY{p}{,} \PY{l+m+mi}{1}\PY{p}{)}


\PY{n}{lam}\PY{o}{=}\PY{p}{[}\PY{l+m+mf}{0.01}\PY{p}{,}\PY{l+m+mi}{1}\PY{p}{]}

\PY{n+nb}{print}\PY{p}{(}\PY{l+s+s2}{\PYZdq{}}\PY{l+s+s2}{RESULTS FOR 1 VS 5}\PY{l+s+se}{\PYZbs{}n}\PY{l+s+s2}{\PYZdq{}}\PY{p}{)}
\PY{k}{for} \PY{n}{i} \PY{o+ow}{in} \PY{n}{lam}\PY{p}{:}
    \PY{n}{Ein\PYZus{}tmp}\PY{p}{,}\PY{n}{Eout\PYZus{}tmp}\PY{o}{=}\PY{n}{lin\PYZus{}reg\PYZus{}wdecay}\PY{p}{(}\PY{n}{z\PYZus{}train}\PY{p}{,}\PY{n}{y\PYZus{}train}\PY{p}{,} \PY{n}{z\PYZus{}test}\PY{p}{,}\PY{n}{y\PYZus{}test}\PY{p}{,}\PY{n}{lam}\PY{o}{=}\PY{n}{i}\PY{p}{,}\PY{n}{res}\PY{o}{=}\PY{k+kc}{False}\PY{p}{)}
    \PY{n+nb}{print}\PY{p}{(}\PY{n}{f}\PY{l+s+s1}{\PYZsq{}}\PY{l+s+s1}{For lambda=}\PY{l+s+si}{\PYZob{}i:.2f\PYZcb{}}\PY{l+s+s1}{, Ein=}\PY{l+s+si}{\PYZob{}Ein\PYZus{}tmp:.3f\PYZcb{}}\PY{l+s+s1}{ and Eout=}\PY{l+s+si}{\PYZob{}Eout\PYZus{}tmp:.3f\PYZcb{}}\PY{l+s+se}{\PYZbs{}n}\PY{l+s+s1}{\PYZsq{}}\PY{p}{)}
\end{Verbatim}
\end{tcolorbox}

    \begin{Verbatim}[commandchars=\\\{\}]
RESULTS FOR 1 VS 5

For lambda=0.01, Ein=0.004 and Eout=0.028

For lambda=1.00, Ein=0.005 and Eout=0.026

    \end{Verbatim}

    \hypertarget{support-vector-machines}{%
\part{Support Vector Machines}\label{support-vector-machines}}

\hypertarget{problem-11}{%
\section{Problem 11}\label{problem-11}}

\hypertarget{answer-c-10-0.5}{%
\subsection{\texorpdfstring{Answer: {[}c{]}
\(1,0,-0.5\)}{Answer: {[}c{]} 1,0,-0.5}}\label{answer-c-10-0.5}}

\hypertarget{derivationcode}{%
\subsection{Derivation\&Code:}\label{derivationcode}}

As it can be inferred by the plot below, the only line that correctly
separates the points is given by the parameters in answer {[}c{]}. We
can simply show that this maximize the margin using geometry. Indeed,
the closest (transformed) points to the plane are
\(z_2=\phi(x_2)=(0,-1), \,z_3=\phi(x_3)=(0,3),\, z_4=\phi(x_4)=(1,2)\).
\(z_2\) and \(z_3\) are labelled with \(-1\) while \(z_4\) with \(+1\).
The distance to the plane is given by

\begin{equation}
d(z_i,w,b)=\frac{1}{||\mathbf{w}||}|\mathbf{w}\cdot(\mathbf{z}_i-\mathbf{x})|
\end{equation}

where \(\mathbf{x}=(0.5,y)\) is a generic point on the plane (with
\(y\in\mathbb{R}\)) and \(||\mathbf{w}||=1\) is the Euclidean norm of
the weights vector.

Computing explicitly the distances of the points from the plane, we
find:

\begin{equation}
d(z_2)=d(z_3)=d(z_4)=0.5\,.
\end{equation}

Since the plane is equidistant from two points (e.g. \(z_2\) and
\(z_4\)) that belongs to two different classification, it has the
fattest margin of separation.

    \begin{tcolorbox}[breakable, size=fbox, boxrule=1pt, pad at break*=1mm,colback=cellbackground, colframe=cellborder]
\prompt{In}{incolor}{3}{\boxspacing}
\begin{Verbatim}[commandchars=\\\{\}]
\PY{o}{\PYZpc{}}\PY{k}{reset} \PYZhy{}f

\PY{k+kn}{import} \PY{n+nn}{numpy} \PY{k}{as} \PY{n+nn}{np}
\PY{k+kn}{import} \PY{n+nn}{matplotlib}\PY{n+nn}{.}\PY{n+nn}{pyplot} \PY{k}{as} \PY{n+nn}{plt}

\PY{n}{x} \PY{o}{=} \PY{n}{np}\PY{o}{.}\PY{n}{array}\PY{p}{(}\PY{p}{[}\PY{p}{[}\PY{l+m+mi}{1}\PY{p}{,}\PY{l+m+mi}{0}\PY{p}{]}\PY{p}{,}\PY{p}{[}\PY{l+m+mi}{0}\PY{p}{,}\PY{l+m+mi}{1}\PY{p}{]}\PY{p}{,}\PY{p}{[}\PY{l+m+mi}{0}\PY{p}{,}\PY{o}{\PYZhy{}}\PY{l+m+mi}{1}\PY{p}{]}\PY{p}{,}\PY{p}{[}\PY{o}{\PYZhy{}}\PY{l+m+mi}{1}\PY{p}{,}\PY{l+m+mi}{0}\PY{p}{]}\PY{p}{,}\PY{p}{[}\PY{l+m+mi}{0}\PY{p}{,}\PY{l+m+mi}{2}\PY{p}{]}\PY{p}{,}\PY{p}{[}\PY{l+m+mi}{0}\PY{p}{,}\PY{o}{\PYZhy{}}\PY{l+m+mi}{2}\PY{p}{]}\PY{p}{,}\PY{p}{[}\PY{o}{\PYZhy{}}\PY{l+m+mi}{2}\PY{p}{,}\PY{l+m+mi}{0}\PY{p}{]}\PY{p}{]}\PY{p}{)}
\PY{n}{y} \PY{o}{=} \PY{n}{np}\PY{o}{.}\PY{n}{array}\PY{p}{(}\PY{p}{[}\PY{o}{\PYZhy{}}\PY{l+m+mi}{1}\PY{p}{,} \PY{o}{\PYZhy{}}\PY{l+m+mi}{1}\PY{p}{,} \PY{o}{\PYZhy{}}\PY{l+m+mi}{1}\PY{p}{,} \PY{l+m+mi}{1}\PY{p}{,} \PY{l+m+mi}{1}\PY{p}{,} \PY{l+m+mi}{1}\PY{p}{,} \PY{l+m+mi}{1}\PY{p}{]}\PY{p}{)}

\PY{k}{def} \PY{n+nf}{color\PYZus{}pts}\PY{p}{(}\PY{n}{y}\PY{p}{)}\PY{p}{:}
    \PY{c+c1}{\PYZsh{}green is +1, red is \PYZhy{}1}
    \PY{n}{col}\PY{o}{=}\PY{p}{[}\PY{p}{]}
    \PY{k}{for} \PY{n}{i} \PY{o+ow}{in} \PY{n+nb}{range}\PY{p}{(}\PY{n+nb}{len}\PY{p}{(}\PY{n}{y}\PY{p}{)}\PY{p}{)}\PY{p}{:}
        \PY{k}{if}\PY{p}{(}\PY{n}{y}\PY{p}{[}\PY{n}{i}\PY{p}{]}\PY{o}{\PYZgt{}}\PY{l+m+mi}{0}\PY{p}{)}\PY{p}{:} \PY{n}{col}\PY{o}{.}\PY{n}{append}\PY{p}{(}\PY{l+s+s1}{\PYZsq{}}\PY{l+s+s1}{green}\PY{l+s+s1}{\PYZsq{}}\PY{p}{)}
        \PY{k}{else}\PY{p}{:} \PY{n}{col}\PY{o}{.}\PY{n}{append}\PY{p}{(}\PY{l+s+s1}{\PYZsq{}}\PY{l+s+s1}{red}\PY{l+s+s1}{\PYZsq{}}\PY{p}{)}
    \PY{k}{return} \PY{n}{col}

\PY{k}{def} \PY{n+nf}{plot\PYZus{}pts}\PY{p}{(}\PY{n}{pts}\PY{p}{,}\PY{n}{y}\PY{p}{)}\PY{p}{:}
    \PY{n}{col}\PY{o}{=}\PY{n}{color\PYZus{}pts}\PY{p}{(}\PY{n}{y}\PY{p}{)}
    \PY{n}{plt}\PY{o}{.}\PY{n}{scatter}\PY{p}{(}\PY{n}{pts}\PY{p}{[}\PY{p}{:}\PY{p}{,}\PY{l+m+mi}{0}\PY{p}{]}\PY{p}{,}\PY{n}{pts}\PY{p}{[}\PY{p}{:}\PY{p}{,}\PY{l+m+mi}{1}\PY{p}{]}\PY{p}{,}\PY{n}{color}\PY{o}{=}\PY{n}{col}\PY{p}{)}

\PY{k}{def} \PY{n+nf}{transform}\PY{p}{(}\PY{n}{pts}\PY{p}{)}\PY{p}{:}
    \PY{n}{res}\PY{o}{=}\PY{p}{[}\PY{p}{]}
    \PY{k}{for} \PY{n}{i} \PY{o+ow}{in} \PY{n+nb}{range}\PY{p}{(}\PY{n+nb}{len}\PY{p}{(}\PY{n}{pts}\PY{p}{)}\PY{p}{)}\PY{p}{:}
        \PY{n}{x1}\PY{o}{=}\PY{n}{pts}\PY{p}{[}\PY{n}{i}\PY{p}{]}\PY{p}{[}\PY{l+m+mi}{0}\PY{p}{]}
        \PY{n}{x2}\PY{o}{=}\PY{n}{pts}\PY{p}{[}\PY{n}{i}\PY{p}{]}\PY{p}{[}\PY{l+m+mi}{1}\PY{p}{]}
        \PY{n}{res}\PY{o}{.}\PY{n}{append}\PY{p}{(}\PY{p}{[}\PY{n}{x2}\PY{o}{*}\PY{o}{*}\PY{l+m+mi}{2}\PY{o}{\PYZhy{}}\PY{l+m+mi}{2}\PY{o}{*}\PY{n}{x1}\PY{o}{\PYZhy{}}\PY{l+m+mi}{1}\PY{p}{,}\PY{n}{x1}\PY{o}{*}\PY{o}{*}\PY{l+m+mi}{2}\PY{o}{\PYZhy{}}\PY{l+m+mi}{2}\PY{o}{*}\PY{n}{x2}\PY{o}{+}\PY{l+m+mi}{1}\PY{p}{]}\PY{p}{)}
    \PY{k}{return} \PY{n}{np}\PY{o}{.}\PY{n}{array}\PY{p}{(}\PY{n}{res}\PY{p}{)}

\PY{k}{def} \PY{n+nf}{line}\PY{p}{(}\PY{n}{x}\PY{p}{,}\PY{n}{w}\PY{p}{,}\PY{n}{b}\PY{p}{)}\PY{p}{:}
    \PY{k}{if}\PY{p}{(}\PY{n}{w}\PY{p}{[}\PY{l+m+mi}{1}\PY{p}{]}\PY{o}{==}\PY{l+m+mi}{0}\PY{p}{)}\PY{p}{:}
        \PY{k}{return} \PY{o}{\PYZhy{}}\PY{n}{b}\PY{o}{/}\PY{n}{w}\PY{p}{[}\PY{l+m+mi}{0}\PY{p}{]}
    \PY{k}{else}\PY{p}{:}    
        \PY{k}{return} \PY{o}{\PYZhy{}}\PY{n}{w}\PY{p}{[}\PY{l+m+mi}{0}\PY{p}{]}\PY{o}{/}\PY{n}{w}\PY{p}{[}\PY{l+m+mi}{1}\PY{p}{]}\PY{o}{*}\PY{n}{x}\PY{o}{\PYZhy{}}\PY{n}{b}\PY{o}{/}\PY{n}{w}\PY{p}{[}\PY{l+m+mi}{1}\PY{p}{]}
    
\PY{n}{w1}\PY{o}{=}\PY{p}{[}\PY{o}{\PYZhy{}}\PY{l+m+mi}{1}\PY{p}{,}\PY{l+m+mi}{1}\PY{p}{]}
\PY{n}{w2}\PY{o}{=}\PY{p}{[}\PY{l+m+mi}{1}\PY{p}{,}\PY{o}{\PYZhy{}}\PY{l+m+mi}{1}\PY{p}{]}
\PY{n}{w3}\PY{o}{=}\PY{p}{[}\PY{l+m+mi}{1}\PY{p}{,}\PY{l+m+mi}{0}\PY{p}{]}
\PY{n}{w4}\PY{o}{=}\PY{p}{[}\PY{l+m+mi}{0}\PY{p}{,}\PY{l+m+mi}{1}\PY{p}{]}
\PY{n}{b}\PY{o}{=}\PY{o}{\PYZhy{}}\PY{l+m+mf}{0.5}

\PY{n}{z}\PY{o}{=}\PY{n}{transform}\PY{p}{(}\PY{n}{x}\PY{p}{)}
\PY{n}{xaxis}\PY{o}{=}\PY{n}{np}\PY{o}{.}\PY{n}{linspace}\PY{p}{(}\PY{o}{\PYZhy{}}\PY{l+m+mi}{3}\PY{p}{,}\PY{l+m+mi}{3}\PY{p}{,}\PY{l+m+mi}{100}\PY{p}{)}
\PY{n}{yaxis}\PY{o}{=}\PY{n}{np}\PY{o}{.}\PY{n}{linspace}\PY{p}{(}\PY{o}{\PYZhy{}}\PY{l+m+mi}{4}\PY{p}{,}\PY{l+m+mi}{5}\PY{p}{,}\PY{l+m+mi}{100}\PY{p}{)}

\PY{n}{w1line}\PY{o}{=}\PY{p}{[}\PY{n}{line}\PY{p}{(}\PY{n}{x}\PY{p}{,}\PY{n}{w1}\PY{p}{,}\PY{n}{b}\PY{p}{)} \PY{k}{for} \PY{n}{x} \PY{o+ow}{in} \PY{n}{xaxis}\PY{p}{]}
\PY{n}{w2line}\PY{o}{=}\PY{p}{[}\PY{n}{line}\PY{p}{(}\PY{n}{x}\PY{p}{,}\PY{n}{w2}\PY{p}{,}\PY{n}{b}\PY{p}{)} \PY{k}{for} \PY{n}{x} \PY{o+ow}{in} \PY{n}{xaxis}\PY{p}{]}
\PY{n}{w3line}\PY{o}{=}\PY{p}{[}\PY{n}{line}\PY{p}{(}\PY{n}{x}\PY{p}{,}\PY{n}{w3}\PY{p}{,}\PY{n}{b}\PY{p}{)} \PY{k}{for} \PY{n}{x} \PY{o+ow}{in} \PY{n}{xaxis}\PY{p}{]}
\PY{n}{w4line}\PY{o}{=}\PY{p}{[}\PY{n}{line}\PY{p}{(}\PY{n}{x}\PY{p}{,}\PY{n}{w4}\PY{p}{,}\PY{n}{b}\PY{p}{)} \PY{k}{for} \PY{n}{x} \PY{o+ow}{in} \PY{n}{xaxis}\PY{p}{]}

\PY{n}{plot\PYZus{}pts}\PY{p}{(}\PY{n}{z}\PY{p}{,}\PY{n}{y}\PY{p}{)}
\PY{n}{plt}\PY{o}{.}\PY{n}{plot}\PY{p}{(}\PY{n}{xaxis}\PY{p}{,} \PY{n}{w1line}\PY{p}{,}\PY{n}{label}\PY{o}{=}\PY{l+s+s2}{\PYZdq{}}\PY{l+s+s2}{[a]}\PY{l+s+s2}{\PYZdq{}}\PY{p}{)}
\PY{n}{plt}\PY{o}{.}\PY{n}{plot}\PY{p}{(}\PY{n}{xaxis}\PY{p}{,} \PY{n}{w2line}\PY{p}{,}\PY{n}{label}\PY{o}{=}\PY{l+s+s2}{\PYZdq{}}\PY{l+s+s2}{[b]}\PY{l+s+s2}{\PYZdq{}}\PY{p}{)}
\PY{n}{plt}\PY{o}{.}\PY{n}{plot}\PY{p}{(}\PY{n}{w3line}\PY{p}{,}\PY{n}{yaxis}\PY{p}{,}\PY{n}{label}\PY{o}{=}\PY{l+s+s2}{\PYZdq{}}\PY{l+s+s2}{[c]}\PY{l+s+s2}{\PYZdq{}}\PY{p}{)}
\PY{n}{plt}\PY{o}{.}\PY{n}{plot}\PY{p}{(}\PY{n}{xaxis}\PY{p}{,} \PY{n}{w4line}\PY{p}{,}\PY{n}{label}\PY{o}{=}\PY{l+s+s2}{\PYZdq{}}\PY{l+s+s2}{[d]}\PY{l+s+s2}{\PYZdq{}}\PY{p}{)}
\PY{n}{plt}\PY{o}{.}\PY{n}{legend}\PY{p}{(}\PY{p}{)}
\PY{n}{plt}\PY{o}{.}\PY{n}{show}\PY{p}{(}\PY{p}{)}
\end{Verbatim}
\end{tcolorbox}

    \begin{center}
    \adjustimage{max size={0.9\linewidth}{0.9\paperheight}}{output_13_0.png}
    \end{center}
    { \hspace*{\fill} \\}
    
    \hypertarget{problem-12}{%
\section{Problem 12}\label{problem-12}}

\hypertarget{answer-c-4-5}{%
\subsection{\texorpdfstring{Answer: {[}c{]}
\(4-5\)}{Answer: {[}c{]} 4-5}}\label{answer-c-4-5}}

\hypertarget{code}{%
\subsection{Code:}\label{code}}

    \begin{tcolorbox}[breakable, size=fbox, boxrule=1pt, pad at break*=1mm,colback=cellbackground, colframe=cellborder]
\prompt{In}{incolor}{4}{\boxspacing}
\begin{Verbatim}[commandchars=\\\{\}]
\PY{k+kn}{from} \PY{n+nn}{sklearn} \PY{k}{import} \PY{n}{svm}

\PY{n}{clf} \PY{o}{=} \PY{n}{svm}\PY{o}{.}\PY{n}{SVC}\PY{p}{(}\PY{n}{C}\PY{o}{=}\PY{n}{np}\PY{o}{.}\PY{n}{Inf}\PY{p}{,} \PY{n}{kernel}\PY{o}{=}\PY{l+s+s1}{\PYZsq{}}\PY{l+s+s1}{poly}\PY{l+s+s1}{\PYZsq{}}\PY{p}{,}\PY{n}{degree}\PY{o}{=}\PY{l+m+mi}{2}\PY{p}{,} \PY{n}{coef0}\PY{o}{=}\PY{l+m+mi}{1}\PY{p}{,}\PY{n}{gamma}\PY{o}{=}\PY{l+m+mi}{1}\PY{p}{)}
\PY{n}{clf}\PY{o}{.}\PY{n}{fit}\PY{p}{(}\PY{n}{x}\PY{p}{,} \PY{n}{y}\PY{p}{)}
\PY{n}{gin}\PY{o}{=}\PY{n}{clf}\PY{o}{.}\PY{n}{predict}\PY{p}{(}\PY{n}{x}\PY{p}{)}
\PY{n}{testgin}\PY{o}{=}\PY{p}{(}\PY{n}{gin}\PY{o}{==}\PY{n}{y}\PY{p}{)}
\PY{n}{Ein}\PY{o}{=}\PY{n+nb}{len}\PY{p}{(}\PY{n}{np}\PY{o}{.}\PY{n}{where}\PY{p}{(}\PY{n}{testgin}\PY{o}{==}\PY{k+kc}{False}\PY{p}{)}\PY{p}{[}\PY{l+m+mi}{0}\PY{p}{]}\PY{p}{)}\PY{o}{/}\PY{n+nb}{len}\PY{p}{(}\PY{n}{y}\PY{p}{)}

\PY{n+nb}{print}\PY{p}{(}\PY{n}{f}\PY{l+s+s1}{\PYZsq{}}\PY{l+s+s1}{Support vectors found: }\PY{l+s+si}{\PYZob{}clf.support\PYZus{}vectors\PYZus{}.shape[0]\PYZcb{}}\PY{l+s+s1}{\PYZsq{}}\PY{p}{)}
\PY{n+nb}{print}\PY{p}{(}\PY{n}{f}\PY{l+s+s1}{\PYZsq{}}\PY{l+s+s1}{E\PYZus{}in=}\PY{l+s+si}{\PYZob{}Ein\PYZcb{}}\PY{l+s+s1}{\PYZsq{}}\PY{p}{)}
\end{Verbatim}
\end{tcolorbox}

    \begin{Verbatim}[commandchars=\\\{\}]
Support vectors found: 5
E\_in=0.0
    \end{Verbatim}

    \hypertarget{radial-basis-functions}{%
\part{Radial Basis Functions}\label{radial-basis-functions}}

\hypertarget{preliminary-code}{%
\section{Preliminary Code}\label{preliminary-code}}

    \begin{tcolorbox}[breakable, size=fbox, boxrule=1pt, pad at break*=1mm,colback=cellbackground, colframe=cellborder]
\prompt{In}{incolor}{5}{\boxspacing}
\begin{Verbatim}[commandchars=\\\{\}]
\PY{o}{\PYZpc{}}\PY{k}{reset} \PYZhy{}f

\PY{k+kn}{import} \PY{n+nn}{numpy} \PY{k}{as} \PY{n+nn}{np}
\PY{k+kn}{import} \PY{n+nn}{matplotlib}\PY{n+nn}{.}\PY{n+nn}{pyplot} \PY{k}{as} \PY{n+nn}{plt}
\PY{k+kn}{from} \PY{n+nn}{sklearn} \PY{k}{import} \PY{n}{svm}
\PY{k+kn}{from} \PY{n+nn}{sklearn}\PY{n+nn}{.}\PY{n+nn}{cluster} \PY{k}{import} \PY{n}{KMeans}

\PY{c+c1}{\PYZsh{}Generate and Plot Data\PYZhy{}Points}

\PY{k}{def} \PY{n+nf}{color\PYZus{}pts}\PY{p}{(}\PY{n}{y}\PY{p}{)}\PY{p}{:}
    \PY{c+c1}{\PYZsh{}green is +1, red is \PYZhy{}1}
    \PY{n}{col}\PY{o}{=}\PY{p}{[}\PY{p}{]}
    \PY{k}{for} \PY{n}{i} \PY{o+ow}{in} \PY{n+nb}{range}\PY{p}{(}\PY{n+nb}{len}\PY{p}{(}\PY{n}{y}\PY{p}{)}\PY{p}{)}\PY{p}{:}
        \PY{k}{if}\PY{p}{(}\PY{n}{y}\PY{p}{[}\PY{n}{i}\PY{p}{]}\PY{o}{\PYZgt{}}\PY{l+m+mi}{0}\PY{p}{)}\PY{p}{:} \PY{n}{col}\PY{o}{.}\PY{n}{append}\PY{p}{(}\PY{l+s+s1}{\PYZsq{}}\PY{l+s+s1}{green}\PY{l+s+s1}{\PYZsq{}}\PY{p}{)}
        \PY{k}{else}\PY{p}{:} \PY{n}{col}\PY{o}{.}\PY{n}{append}\PY{p}{(}\PY{l+s+s1}{\PYZsq{}}\PY{l+s+s1}{red}\PY{l+s+s1}{\PYZsq{}}\PY{p}{)}
    \PY{k}{return} \PY{n}{col}

\PY{k}{def} \PY{n+nf}{plot\PYZus{}pts}\PY{p}{(}\PY{n}{pts}\PY{p}{,}\PY{n}{y}\PY{p}{,}\PY{n}{marker}\PY{o}{=}\PY{l+s+s1}{\PYZsq{}}\PY{l+s+s1}{o}\PY{l+s+s1}{\PYZsq{}}\PY{p}{,}\PY{n}{size}\PY{o}{=}\PY{l+m+mi}{10}\PY{p}{)}\PY{p}{:}
    \PY{n}{col}\PY{o}{=}\PY{n}{color\PYZus{}pts}\PY{p}{(}\PY{n}{y}\PY{p}{)}
    \PY{n}{plt}\PY{o}{.}\PY{n}{scatter}\PY{p}{(}\PY{n}{pts}\PY{p}{[}\PY{p}{:}\PY{p}{,}\PY{l+m+mi}{0}\PY{p}{]}\PY{p}{,}\PY{n}{pts}\PY{p}{[}\PY{p}{:}\PY{p}{,}\PY{l+m+mi}{1}\PY{p}{]}\PY{p}{,}\PY{n}{color}\PY{o}{=}\PY{n}{col}\PY{p}{,}\PY{n}{marker}\PY{o}{=}\PY{n}{marker}\PY{p}{,}\PY{n}{s}\PY{o}{=}\PY{n}{size}\PY{p}{)}

\PY{k}{def} \PY{n+nf}{gen\PYZus{}uniform\PYZus{}points}\PY{p}{(}\PY{n}{N}\PY{p}{,}\PY{n}{d}\PY{o}{=}\PY{l+m+mi}{2}\PY{p}{,}\PY{n}{vmin}\PY{o}{=}\PY{p}{[}\PY{o}{\PYZhy{}}\PY{l+m+mi}{1}\PY{p}{,}\PY{o}{\PYZhy{}}\PY{l+m+mi}{1}\PY{p}{]}\PY{p}{,}\PY{n}{vmax}\PY{o}{=}\PY{p}{[}\PY{l+m+mi}{1}\PY{p}{,}\PY{l+m+mi}{1}\PY{p}{]}\PY{p}{)}\PY{p}{:}
    \PY{k}{if}\PY{p}{(}\PY{n}{d}\PY{o}{!=}\PY{n+nb}{len}\PY{p}{(}\PY{n}{vmin}\PY{p}{)}\PY{o}{|}\PY{n}{d}\PY{o}{!=}\PY{n+nb}{len}\PY{p}{(}\PY{n}{vmax}\PY{p}{)}\PY{p}{)}\PY{p}{:} 
        \PY{k}{raise} \PY{n+ne}{Exception}\PY{p}{(}\PY{l+s+s1}{\PYZsq{}}\PY{l+s+s1}{WARNING: Boundary values do not match the dimensionality of the problem!}\PY{l+s+s1}{\PYZsq{}}\PY{p}{)}
    \PY{n}{pts}\PY{o}{=}\PY{n}{np}\PY{o}{.}\PY{n}{random}\PY{o}{.}\PY{n}{uniform}\PY{p}{(}\PY{n}{low}\PY{o}{=}\PY{n}{vmin}\PY{p}{,}\PY{n}{high}\PY{o}{=}\PY{n}{vmax}\PY{p}{,}\PY{n}{size}\PY{o}{=}\PY{p}{(}\PY{n}{N}\PY{p}{,}\PY{n}{d}\PY{p}{)}\PY{p}{)}
    \PY{k}{return} \PY{n}{pts}

\PY{k}{def} \PY{n+nf}{f}\PY{p}{(}\PY{n}{x}\PY{p}{)}\PY{p}{:}
    \PY{k}{return} \PY{n}{x}\PY{p}{[}\PY{l+m+mi}{1}\PY{p}{]}\PY{o}{\PYZhy{}}\PY{n}{x}\PY{p}{[}\PY{l+m+mi}{0}\PY{p}{]}\PY{o}{+}\PY{l+m+mf}{0.25}\PY{o}{*}\PY{n}{np}\PY{o}{.}\PY{n}{sin}\PY{p}{(}\PY{n}{np}\PY{o}{.}\PY{n}{pi}\PY{o}{*}\PY{n}{x}\PY{p}{[}\PY{l+m+mi}{1}\PY{p}{]}\PY{p}{)}

\PY{k}{def} \PY{n+nf}{label\PYZus{}f}\PY{p}{(}\PY{n}{pts}\PY{p}{,}\PY{n}{f}\PY{p}{)}\PY{p}{:}
    \PY{n}{y}\PY{o}{=}\PY{p}{[}\PY{p}{]}
    \PY{k}{for} \PY{n}{i} \PY{o+ow}{in} \PY{n+nb}{range}\PY{p}{(}\PY{n+nb}{len}\PY{p}{(}\PY{n}{pts}\PY{p}{)}\PY{p}{)}\PY{p}{:}
        \PY{n}{y}\PY{o}{.}\PY{n}{append}\PY{p}{(}\PY{n}{np}\PY{o}{.}\PY{n}{sign}\PY{p}{(}\PY{n}{f}\PY{p}{(}\PY{n}{pts}\PY{p}{[}\PY{n}{i}\PY{p}{]}\PY{p}{)}\PY{p}{)}\PY{p}{)}
    \PY{k}{return} \PY{n}{np}\PY{o}{.}\PY{n}{array}\PY{p}{(}\PY{n}{y}\PY{p}{)}

\PY{c+c1}{\PYZsh{}HardSVM algo}

\PY{k}{def} \PY{n+nf}{HardSVM\PYZus{}rbf}\PY{p}{(}\PY{n}{x}\PY{p}{,}\PY{n}{y}\PY{p}{,}\PY{n}{gamma}\PY{p}{,}\PY{n}{Nval}\PY{o}{=}\PY{l+m+mi}{1000}\PY{p}{)}\PY{p}{:}
    \PY{n}{clf} \PY{o}{=} \PY{n}{svm}\PY{o}{.}\PY{n}{SVC}\PY{p}{(}\PY{n}{C}\PY{o}{=}\PY{n}{np}\PY{o}{.}\PY{n}{Inf}\PY{p}{,}\PY{n}{kernel}\PY{o}{=}\PY{l+s+s1}{\PYZsq{}}\PY{l+s+s1}{rbf}\PY{l+s+s1}{\PYZsq{}}\PY{p}{,}\PY{n}{gamma}\PY{o}{=}\PY{n}{gamma}\PY{p}{)}
    \PY{n}{clf}\PY{o}{.}\PY{n}{fit}\PY{p}{(}\PY{n}{x}\PY{p}{,} \PY{n}{y}\PY{p}{)}
    \PY{n}{svs}\PY{o}{=}\PY{n}{clf}\PY{o}{.}\PY{n}{support\PYZus{}vectors\PYZus{}}\PY{o}{.}\PY{n}{shape}\PY{p}{[}\PY{l+m+mi}{0}\PY{p}{]}
    
    \PY{c+c1}{\PYZsh{}Ein computation}
    \PY{n}{gin}\PY{o}{=}\PY{n}{clf}\PY{o}{.}\PY{n}{predict}\PY{p}{(}\PY{n}{x}\PY{p}{)}
    \PY{n}{testgin}\PY{o}{=}\PY{p}{(}\PY{n}{gin}\PY{o}{==}\PY{n}{y}\PY{p}{)}
    \PY{n}{Ein}\PY{o}{=}\PY{n+nb}{len}\PY{p}{(}\PY{n}{np}\PY{o}{.}\PY{n}{where}\PY{p}{(}\PY{n}{testgin}\PY{o}{==}\PY{k+kc}{False}\PY{p}{)}\PY{p}{[}\PY{l+m+mi}{0}\PY{p}{]}\PY{p}{)}\PY{o}{/}\PY{n+nb}{len}\PY{p}{(}\PY{n}{y}\PY{p}{)}
    
    \PY{c+c1}{\PYZsh{}Eout computation}
    \PY{n}{val\PYZus{}pts}\PY{o}{=}\PY{n}{gen\PYZus{}uniform\PYZus{}points}\PY{p}{(}\PY{n}{Nval}\PY{p}{)}
    \PY{n}{y\PYZus{}val}\PY{o}{=}\PY{n}{label\PYZus{}f}\PY{p}{(}\PY{n}{val\PYZus{}pts}\PY{p}{,}\PY{n}{f}\PY{p}{)}
    \PY{n}{gout}\PY{o}{=}\PY{n}{clf}\PY{o}{.}\PY{n}{predict}\PY{p}{(}\PY{n}{val\PYZus{}pts}\PY{p}{)}
    \PY{n}{testgout}\PY{o}{=}\PY{p}{(}\PY{n}{gout}\PY{o}{==}\PY{n}{y\PYZus{}val}\PY{p}{)}
    \PY{n}{Eout}\PY{o}{=}\PY{n+nb}{len}\PY{p}{(}\PY{n}{np}\PY{o}{.}\PY{n}{where}\PY{p}{(}\PY{n}{testgout}\PY{o}{==}\PY{k+kc}{False}\PY{p}{)}\PY{p}{[}\PY{l+m+mi}{0}\PY{p}{]}\PY{p}{)}\PY{o}{/}\PY{n+nb}{len}\PY{p}{(}\PY{n}{y\PYZus{}val}\PY{p}{)}
    
    \PY{k}{return} \PY{n}{Ein}\PY{p}{,}\PY{n}{Eout}

\PY{c+c1}{\PYZsh{}Regular RBF}

\PY{k}{def} \PY{n+nf}{phi}\PY{p}{(}\PY{n}{x}\PY{p}{,}\PY{n}{mu}\PY{p}{,}\PY{n}{gamma}\PY{p}{)}\PY{p}{:}
    \PY{k}{return} \PY{n}{np}\PY{o}{.}\PY{n}{exp}\PY{p}{(}\PY{o}{\PYZhy{}}\PY{n}{gamma}\PY{o}{*}\PY{n}{np}\PY{o}{.}\PY{n}{sum}\PY{p}{(}\PY{p}{(}\PY{n}{x}\PY{o}{\PYZhy{}}\PY{n}{mu}\PY{p}{)}\PY{o}{*}\PY{o}{*}\PY{l+m+mi}{2}\PY{p}{)}\PY{p}{)}

\PY{k}{def} \PY{n+nf}{Z\PYZus{}rbf}\PY{p}{(}\PY{n}{x}\PY{p}{,}\PY{n}{center}\PY{p}{,}\PY{n}{gamma}\PY{p}{)}\PY{p}{:}
    \PY{n}{exp}\PY{o}{=}\PY{n}{np}\PY{o}{.}\PY{n}{array}\PY{p}{(}\PY{p}{[}\PY{p}{[}\PY{n}{phi}\PY{p}{(}\PY{n}{i}\PY{p}{,}\PY{n}{j}\PY{p}{,}\PY{n}{gamma}\PY{p}{)} \PY{k}{for} \PY{n}{j} \PY{o+ow}{in} \PY{n}{center}\PY{p}{]} \PY{k}{for} \PY{n}{i} \PY{o+ow}{in} \PY{n}{x}\PY{p}{]}\PY{p}{)}
    \PY{k}{return} \PY{n}{np}\PY{o}{.}\PY{n}{concatenate}\PY{p}{(}\PY{p}{(}\PY{n}{np}\PY{o}{.}\PY{n}{ones}\PY{p}{(}\PY{n+nb}{len}\PY{p}{(}\PY{n}{x}\PY{p}{)}\PY{p}{)}\PY{p}{[}\PY{p}{:}\PY{p}{,} \PY{n}{np}\PY{o}{.}\PY{n}{newaxis}\PY{p}{]}\PY{p}{,} \PY{n}{exp}\PY{p}{)}\PY{p}{,} \PY{n}{axis}\PY{o}{=}\PY{l+m+mi}{1}\PY{p}{)}

\PY{k}{def} \PY{n+nf}{RBF}\PY{p}{(}\PY{n}{x}\PY{p}{,}\PY{n}{y}\PY{p}{,}\PY{n}{gamma}\PY{p}{,}\PY{n}{k}\PY{p}{,}\PY{n}{plot\PYZus{}cent}\PY{o}{=}\PY{k+kc}{False}\PY{p}{,}\PY{n}{Nval}\PY{o}{=}\PY{l+m+mi}{1000}\PY{p}{)}\PY{p}{:}
    \PY{n}{center}\PY{o}{=}\PY{n}{KMeans}\PY{p}{(}\PY{n}{n\PYZus{}clusters}\PY{o}{=}\PY{n}{k}\PY{p}{)}\PY{o}{.}\PY{n}{fit}\PY{p}{(}\PY{n}{x}\PY{p}{)}\PY{o}{.}\PY{n}{cluster\PYZus{}centers\PYZus{}}
    \PY{n}{y\PYZus{}center}\PY{o}{=}\PY{n}{label\PYZus{}f}\PY{p}{(}\PY{n}{center}\PY{p}{,}\PY{n}{f}\PY{p}{)}
    \PY{k}{if}\PY{p}{(}\PY{n}{plot\PYZus{}cent}\PY{o}{==}\PY{k+kc}{True}\PY{p}{)}\PY{p}{:}
        \PY{n}{plot\PYZus{}pts}\PY{p}{(}\PY{n}{center}\PY{p}{,}\PY{n}{y\PYZus{}center}\PY{p}{,}\PY{n}{marker}\PY{o}{=}\PY{l+s+s1}{\PYZsq{}}\PY{l+s+s1}{X}\PY{l+s+s1}{\PYZsq{}}\PY{p}{,}\PY{n}{size}\PY{o}{=}\PY{l+m+mi}{100}\PY{p}{)}
    
    \PY{n}{Z}\PY{o}{=}\PY{n}{Z\PYZus{}rbf}\PY{p}{(}\PY{n}{x}\PY{p}{,}\PY{n}{center}\PY{p}{,}\PY{n}{gamma}\PY{p}{)}
    
    \PY{n}{w}\PY{o}{=}\PY{n}{np}\PY{o}{.}\PY{n}{dot}\PY{p}{(}\PY{n}{np}\PY{o}{.}\PY{n}{linalg}\PY{o}{.}\PY{n}{pinv}\PY{p}{(}\PY{n}{Z}\PY{p}{)}\PY{p}{,}\PY{n}{y}\PY{p}{)}
    
    \PY{c+c1}{\PYZsh{}Ein}
    \PY{n}{gin}\PY{o}{=}\PY{n}{np}\PY{o}{.}\PY{n}{sign}\PY{p}{(}\PY{n}{np}\PY{o}{.}\PY{n}{dot}\PY{p}{(}\PY{n}{Z}\PY{p}{,}\PY{n}{w}\PY{p}{)}\PY{p}{)}
    \PY{n}{testgin}\PY{o}{=}\PY{p}{(}\PY{n}{gin}\PY{o}{==}\PY{n}{y}\PY{p}{)}
    \PY{n}{Ein}\PY{o}{=}\PY{n+nb}{len}\PY{p}{(}\PY{n}{np}\PY{o}{.}\PY{n}{where}\PY{p}{(}\PY{n}{testgin}\PY{o}{==}\PY{k+kc}{False}\PY{p}{)}\PY{p}{[}\PY{l+m+mi}{0}\PY{p}{]}\PY{p}{)}\PY{o}{/}\PY{n+nb}{len}\PY{p}{(}\PY{n}{y}\PY{p}{)}
    
    \PY{c+c1}{\PYZsh{}Eout}
    \PY{n}{val\PYZus{}pts}\PY{o}{=}\PY{n}{gen\PYZus{}uniform\PYZus{}points}\PY{p}{(}\PY{n}{Nval}\PY{p}{)}
    \PY{n}{Z\PYZus{}val}\PY{o}{=}\PY{n}{Z\PYZus{}rbf}\PY{p}{(}\PY{n}{val\PYZus{}pts}\PY{p}{,}\PY{n}{center}\PY{p}{,}\PY{n}{gamma}\PY{p}{)}
    \PY{n}{y\PYZus{}val}\PY{o}{=}\PY{n}{label\PYZus{}f}\PY{p}{(}\PY{n}{val\PYZus{}pts}\PY{p}{,}\PY{n}{f}\PY{p}{)}
    \PY{n}{gout}\PY{o}{=}\PY{n}{np}\PY{o}{.}\PY{n}{sign}\PY{p}{(}\PY{n}{np}\PY{o}{.}\PY{n}{dot}\PY{p}{(}\PY{n}{Z\PYZus{}val}\PY{p}{,}\PY{n}{w}\PY{p}{)}\PY{p}{)}
    \PY{n}{testgout}\PY{o}{=}\PY{p}{(}\PY{n}{gout}\PY{o}{==}\PY{n}{y\PYZus{}val}\PY{p}{)}
    \PY{n}{Eout}\PY{o}{=}\PY{n+nb}{len}\PY{p}{(}\PY{n}{np}\PY{o}{.}\PY{n}{where}\PY{p}{(}\PY{n}{testgout}\PY{o}{==}\PY{k+kc}{False}\PY{p}{)}\PY{p}{[}\PY{l+m+mi}{0}\PY{p}{]}\PY{p}{)}\PY{o}{/}\PY{n+nb}{len}\PY{p}{(}\PY{n}{y\PYZus{}val}\PY{p}{)}
    
    \PY{k}{return} \PY{n}{Ein}\PY{p}{,}\PY{n}{Eout}

\PY{c+c1}{\PYZsh{}Plotted Example with regular RBF: the crosses are the centers of the clusters found via Lloyd algorithm}

\PY{n}{pts}\PY{o}{=}\PY{n}{gen\PYZus{}uniform\PYZus{}points}\PY{p}{(}\PY{l+m+mi}{100}\PY{p}{)}
\PY{n}{y}\PY{o}{=}\PY{n}{label\PYZus{}f}\PY{p}{(}\PY{n}{pts}\PY{p}{,}\PY{n}{f}\PY{p}{)}
\PY{n}{plot\PYZus{}pts}\PY{p}{(}\PY{n}{pts}\PY{p}{,}\PY{n}{y}\PY{p}{)}
\PY{n}{res}\PY{o}{=}\PY{n}{RBF}\PY{p}{(}\PY{n}{pts}\PY{p}{,}\PY{n}{y}\PY{p}{,}\PY{l+m+mf}{1.5}\PY{p}{,}\PY{l+m+mi}{9}\PY{p}{,}\PY{n}{plot\PYZus{}cent}\PY{o}{=}\PY{k+kc}{True}\PY{p}{,}\PY{n}{Nval}\PY{o}{=}\PY{l+m+mi}{1000}\PY{p}{)}
\end{Verbatim}
\end{tcolorbox}

    \begin{center}
    \adjustimage{max size={0.9\linewidth}{0.9\paperheight}}{output_17_0.png}
    \end{center}
    { \hspace*{\fill} \\}
    
    \hypertarget{problems-13-14}{%
\section{Problems 13-14}\label{problems-13-14}}

\hypertarget{answers-a-le-5-of-the-time-e-ge-75-of-the-time}{%
\subsection{\texorpdfstring{Answers: {[}a{]} \(\le 5\%\) of the time,
{[}e{]} \(\ge 75\%\) of the
time}{Answers: {[}a{]} \textbackslash{}le 5\textbackslash{}\% of the time, {[}e{]} \textbackslash{}ge 75\textbackslash{}\% of the time}}\label{answers-a-le-5-of-the-time-e-ge-75-of-the-time}}

\hypertarget{code}{%
\subsection{Code:}\label{code}}

    \begin{tcolorbox}[breakable, size=fbox, boxrule=1pt, pad at break*=1mm,colback=cellbackground, colframe=cellborder]
\prompt{In}{incolor}{6}{\boxspacing}
\begin{Verbatim}[commandchars=\\\{\}]
\PY{n}{Nruns}\PY{o}{=}\PY{l+m+mi}{100}
\PY{n}{Npts}\PY{o}{=}\PY{l+m+mi}{100}
\PY{n}{gamma}\PY{o}{=}\PY{l+m+mf}{1.5}
\PY{n}{k}\PY{o}{=}\PY{l+m+mi}{9}
\PY{n}{counter\PYZus{}p13}\PY{o}{=}\PY{l+m+mi}{0}
\PY{n}{counter\PYZus{}p14}\PY{o}{=}\PY{l+m+mi}{0} 

\PY{k}{for} \PY{n}{i} \PY{o+ow}{in} \PY{n+nb}{range}\PY{p}{(}\PY{n}{Nruns}\PY{p}{)}\PY{p}{:}
    \PY{n}{pts}\PY{o}{=}\PY{n}{gen\PYZus{}uniform\PYZus{}points}\PY{p}{(}\PY{n}{Npts}\PY{p}{)}
    \PY{n}{y}\PY{o}{=}\PY{n}{label\PYZus{}f}\PY{p}{(}\PY{n}{pts}\PY{p}{,}\PY{n}{f}\PY{p}{)}
    \PY{n}{EinSVM}\PY{p}{,}\PY{n}{EoutSVM}\PY{o}{=}\PY{n}{HardSVM\PYZus{}rbf}\PY{p}{(}\PY{n}{pts}\PY{p}{,}\PY{n}{y}\PY{p}{,}\PY{n}{gamma}\PY{p}{)}
    \PY{n}{EinRBF}\PY{p}{,}\PY{n}{EoutRBF}\PY{o}{=}\PY{n}{RBF}\PY{p}{(}\PY{n}{pts}\PY{p}{,}\PY{n}{y}\PY{p}{,}\PY{n}{gamma}\PY{p}{,}\PY{n}{k}\PY{p}{)}
    \PY{k}{if}\PY{p}{(}\PY{n}{EinSVM}\PY{o}{!=}\PY{l+m+mi}{0}\PY{p}{)}\PY{p}{:} \PY{n}{counter\PYZus{}p13}\PY{o}{+}\PY{o}{=}\PY{l+m+mi}{1}
    \PY{k}{if}\PY{p}{(}\PY{n}{EoutSVM}\PY{o}{\PYZlt{}}\PY{n}{EoutRBF}\PY{p}{)}\PY{p}{:} \PY{n}{counter\PYZus{}p14}\PY{o}{+}\PY{o}{=}\PY{l+m+mi}{1}
        
\PY{n+nb}{print}\PY{p}{(}\PY{n}{f}\PY{l+s+s1}{\PYZsq{}}\PY{l+s+s1}{Problem 13:}\PY{l+s+se}{\PYZbs{}n}\PY{l+s+s1}{Out of Nruns=}\PY{l+s+si}{\PYZob{}Nruns\PYZcb{}}\PY{l+s+s1}{, the datasets are not separable }\PY{l+s+se}{\PYZbs{}n}\PY{l+s+s1}{(using hard\PYZhy{}margin SVM) }\PY{l+s+s1}{\PYZob{}}\PY{l+s+s1}{counter\PYZus{}p13/Nruns*100:.0f\PYZcb{}}\PY{l+s+si}{\PYZpc{} o}\PY{l+s+s1}{f the time.}\PY{l+s+se}{\PYZbs{}n}\PY{l+s+s1}{\PYZsq{}}\PY{p}{)}
\PY{n+nb}{print}\PY{p}{(}\PY{n}{f}\PY{l+s+s1}{\PYZsq{}}\PY{l+s+s1}{Problem 14:}\PY{l+s+se}{\PYZbs{}n}\PY{l+s+s1}{Out of Nruns=}\PY{l+s+si}{\PYZob{}Nruns\PYZcb{}}\PY{l+s+s1}{, kernel beats regular RBF (gamma=}\PY{l+s+si}{\PYZob{}gamma\PYZcb{}}\PY{l+s+s1}{ and k=}\PY{l+s+si}{\PYZob{}k\PYZcb{}}\PY{l+s+s1}{) }\PY{l+s+se}{\PYZbs{}n}\PY{l+s+s1}{\PYZob{}}\PY{l+s+s1}{counter\PYZus{}p14/Nruns*100:.0f\PYZcb{}}\PY{l+s+si}{\PYZpc{} o}\PY{l+s+s1}{f the time.}\PY{l+s+s1}{\PYZsq{}}\PY{p}{)}
\end{Verbatim}
\end{tcolorbox}

    \begin{Verbatim}[commandchars=\\\{\}]
Problem 13:
Out of Nruns=100, the datasets are not separable
(using hard-margin SVM) 0\% of the time.

Problem 14:
Out of Nruns=100, kernel beats regular RBF (gamma=1.5 and k=9)
81\% of the time.
    \end{Verbatim}

    \hypertarget{problem-15}{%
\section{Problem 15}\label{problem-15}}

\hypertarget{answer-d-60-but-le-90-of-the-time}{%
\subsection{\texorpdfstring{Answer: {[}d{]} \(> 60\%\) but
\(\le 90\%\) of the
time}{Answer: {[}d{]} \textgreater{} 60\textbackslash{}\% but \textbackslash{}le 90\textbackslash{}\% of the time}}\label{answer-d-60-but-le-90-of-the-time}}

\hypertarget{code}{%
\subsection{Code:}\label{code}}

    \begin{tcolorbox}[breakable, size=fbox, boxrule=1pt, pad at break*=1mm,colback=cellbackground, colframe=cellborder]
\prompt{In}{incolor}{7}{\boxspacing}
\begin{Verbatim}[commandchars=\\\{\}]
\PY{n}{Nruns}\PY{o}{=}\PY{l+m+mi}{100}
\PY{n}{Npts}\PY{o}{=}\PY{l+m+mi}{100}
\PY{n}{gamma}\PY{o}{=}\PY{l+m+mf}{1.5}
\PY{n}{k}\PY{o}{=}\PY{l+m+mi}{12}
\PY{n}{counter\PYZus{}p15}\PY{o}{=}\PY{l+m+mi}{0}

\PY{k}{for} \PY{n}{i} \PY{o+ow}{in} \PY{n+nb}{range}\PY{p}{(}\PY{n}{Nruns}\PY{p}{)}\PY{p}{:}
    \PY{n}{pts}\PY{o}{=}\PY{n}{gen\PYZus{}uniform\PYZus{}points}\PY{p}{(}\PY{n}{Npts}\PY{p}{)}
    \PY{n}{y}\PY{o}{=}\PY{n}{label\PYZus{}f}\PY{p}{(}\PY{n}{pts}\PY{p}{,}\PY{n}{f}\PY{p}{)}
    \PY{n}{EinSVM}\PY{p}{,}\PY{n}{EoutSVM}\PY{o}{=}\PY{n}{HardSVM\PYZus{}rbf}\PY{p}{(}\PY{n}{pts}\PY{p}{,}\PY{n}{y}\PY{p}{,}\PY{n}{gamma}\PY{p}{)}
    \PY{n}{EinRBF}\PY{p}{,}\PY{n}{EoutRBF}\PY{o}{=}\PY{n}{RBF}\PY{p}{(}\PY{n}{pts}\PY{p}{,}\PY{n}{y}\PY{p}{,}\PY{n}{gamma}\PY{p}{,}\PY{n}{k}\PY{p}{)}
    \PY{k}{if}\PY{p}{(}\PY{n}{EoutSVM}\PY{o}{\PYZlt{}}\PY{n}{EoutRBF}\PY{p}{)}\PY{p}{:} \PY{n}{counter\PYZus{}p15}\PY{o}{+}\PY{o}{=}\PY{l+m+mi}{1}
        
\PY{n+nb}{print}\PY{p}{(}\PY{n}{f}\PY{l+s+s1}{\PYZsq{}}\PY{l+s+s1}{Problem 15:}\PY{l+s+se}{\PYZbs{}n}\PY{l+s+s1}{Out of Nruns=}\PY{l+s+si}{\PYZob{}Nruns\PYZcb{}}\PY{l+s+s1}{, kernel beats regular RBF (gamma=}\PY{l+s+si}{\PYZob{}gamma\PYZcb{}}\PY{l+s+s1}{ and k=}\PY{l+s+si}{\PYZob{}k\PYZcb{}}\PY{l+s+s1}{) }\PY{l+s+se}{\PYZbs{}n}\PY{l+s+s1}{\PYZob{}}\PY{l+s+s1}{counter\PYZus{}p15/Nruns*100:.0f\PYZcb{}}\PY{l+s+si}{\PYZpc{} o}\PY{l+s+s1}{f the time.}\PY{l+s+s1}{\PYZsq{}}\PY{p}{)}
\end{Verbatim}
\end{tcolorbox}

    \begin{Verbatim}[commandchars=\\\{\}]
Problem 15:
Out of Nruns=100, kernel beats regular RBF (gamma=1.5 and k=12)
70\% of the time.
    \end{Verbatim}

    \hypertarget{problem-16}{%
\section{Problem 16}\label{problem-16}}

\hypertarget{answer-d-both-e_in-and-e_out-go-down}{%
\subsection{\texorpdfstring{Answer: {[}d{]} Both \(E_{in}\) and
\(E_{out}\) go
down}{Answer: {[}d{]} Both E\_\{in\} and E\_\{out\} go down}}\label{answer-d-both-e_in-and-e_out-go-down}}

\hypertarget{code}{%
\subsection{Code:}\label{code}}

    \begin{tcolorbox}[breakable, size=fbox, boxrule=1pt, pad at break*=1mm,colback=cellbackground, colframe=cellborder]
\prompt{In}{incolor}{8}{\boxspacing}
\begin{Verbatim}[commandchars=\\\{\}]
\PY{c+c1}{\PYZsh{}Problem 16}
\PY{n}{Nruns}\PY{o}{=}\PY{l+m+mi}{100}
\PY{n}{Npts}\PY{o}{=}\PY{l+m+mi}{100}
\PY{n}{gamma}\PY{o}{=}\PY{l+m+mf}{1.5}
\PY{n}{counter\PYZus{}p16}\PY{o}{=}\PY{n}{np}\PY{o}{.}\PY{n}{zeros}\PY{p}{(}\PY{l+m+mi}{5}\PY{p}{)}

\PY{k}{for} \PY{n}{i} \PY{o+ow}{in} \PY{n+nb}{range}\PY{p}{(}\PY{n}{Nruns}\PY{p}{)}\PY{p}{:}
    \PY{n}{pts}\PY{o}{=}\PY{n}{gen\PYZus{}uniform\PYZus{}points}\PY{p}{(}\PY{n}{Npts}\PY{p}{)}
    \PY{n}{y}\PY{o}{=}\PY{n}{label\PYZus{}f}\PY{p}{(}\PY{n}{pts}\PY{p}{,}\PY{n}{f}\PY{p}{)}
    \PY{n}{Ein\PYZus{}k9}\PY{p}{,}\PY{n}{Eout\PYZus{}k9}\PY{o}{=}\PY{n}{RBF}\PY{p}{(}\PY{n}{pts}\PY{p}{,}\PY{n}{y}\PY{p}{,}\PY{n}{gamma}\PY{p}{,}\PY{l+m+mi}{9}\PY{p}{)}
    \PY{n}{Ein\PYZus{}k12}\PY{p}{,}\PY{n}{Eout\PYZus{}k12}\PY{o}{=}\PY{n}{RBF}\PY{p}{(}\PY{n}{pts}\PY{p}{,}\PY{n}{y}\PY{p}{,}\PY{n}{gamma}\PY{p}{,}\PY{l+m+mi}{12}\PY{p}{)}
    \PY{k}{if}\PY{p}{(}\PY{p}{(}\PY{n}{Ein\PYZus{}k9}\PY{o}{\PYZgt{}}\PY{n}{Ein\PYZus{}k12}\PY{p}{)}\PY{o}{\PYZam{}}\PY{p}{(}\PY{n}{Eout\PYZus{}k9}\PY{o}{\PYZlt{}}\PY{n}{Eout\PYZus{}k12}\PY{p}{)}\PY{p}{)}\PY{p}{:} \PY{n}{counter\PYZus{}p16}\PY{p}{[}\PY{l+m+mi}{0}\PY{p}{]}\PY{o}{+}\PY{o}{=}\PY{l+m+mi}{1}
    \PY{k}{if}\PY{p}{(}\PY{p}{(}\PY{n}{Ein\PYZus{}k9}\PY{o}{\PYZlt{}}\PY{n}{Ein\PYZus{}k12}\PY{p}{)}\PY{o}{\PYZam{}}\PY{p}{(}\PY{n}{Eout\PYZus{}k9}\PY{o}{\PYZgt{}}\PY{n}{Eout\PYZus{}k12}\PY{p}{)}\PY{p}{)}\PY{p}{:} \PY{n}{counter\PYZus{}p16}\PY{p}{[}\PY{l+m+mi}{1}\PY{p}{]}\PY{o}{+}\PY{o}{=}\PY{l+m+mi}{1}
    \PY{k}{if}\PY{p}{(}\PY{p}{(}\PY{n}{Ein\PYZus{}k9}\PY{o}{\PYZlt{}}\PY{n}{Ein\PYZus{}k12}\PY{p}{)}\PY{o}{\PYZam{}}\PY{p}{(}\PY{n}{Eout\PYZus{}k9}\PY{o}{\PYZlt{}}\PY{n}{Eout\PYZus{}k12}\PY{p}{)}\PY{p}{)}\PY{p}{:} \PY{n}{counter\PYZus{}p16}\PY{p}{[}\PY{l+m+mi}{2}\PY{p}{]}\PY{o}{+}\PY{o}{=}\PY{l+m+mi}{1}
    \PY{k}{if}\PY{p}{(}\PY{p}{(}\PY{n}{Ein\PYZus{}k9}\PY{o}{\PYZgt{}}\PY{n}{Ein\PYZus{}k12}\PY{p}{)}\PY{o}{\PYZam{}}\PY{p}{(}\PY{n}{Eout\PYZus{}k9}\PY{o}{\PYZgt{}}\PY{n}{Eout\PYZus{}k12}\PY{p}{)}\PY{p}{)}\PY{p}{:} \PY{n}{counter\PYZus{}p16}\PY{p}{[}\PY{l+m+mi}{3}\PY{p}{]}\PY{o}{+}\PY{o}{=}\PY{l+m+mi}{1}
    \PY{k}{if}\PY{p}{(}\PY{p}{(}\PY{n}{Ein\PYZus{}k9}\PY{o}{==}\PY{n}{Ein\PYZus{}k12}\PY{p}{)}\PY{o}{\PYZam{}}\PY{p}{(}\PY{n}{Eout\PYZus{}k9}\PY{o}{==}\PY{n}{Eout\PYZus{}k12}\PY{p}{)}\PY{p}{)}\PY{p}{:} \PY{n}{counter\PYZus{}p16}\PY{p}{[}\PY{l+m+mi}{4}\PY{p}{]}\PY{o}{+}\PY{o}{=}\PY{l+m+mi}{1}
    
\PY{n+nb}{print}\PY{p}{(}\PY{n}{f}\PY{l+s+s1}{\PYZsq{}}\PY{l+s+s1}{Problem 16:}\PY{l+s+se}{\PYZbs{}n}\PY{l+s+s1}{Out of Nruns=}\PY{l+s+si}{\PYZob{}Nruns\PYZcb{}}\PY{l+s+s1}{,}\PY{l+s+se}{\PYZbs{}n}\PY{l+s+s1}{[a] Ein goes down, but Eout goes up }\PY{l+s+si}{\PYZob{}counter\PYZus{}p16[0]:.0f\PYZcb{}}\PY{l+s+s1}{ times}\PY{l+s+se}{\PYZbs{}n}\PY{l+s+s1}{[b] Ein goes up, but Eout goes down }\PY{l+s+si}{\PYZob{}counter\PYZus{}p16[1]:.0f\PYZcb{}}\PY{l+s+s1}{ times}\PY{l+s+se}{\PYZbs{}n}\PY{l+s+s1}{[c] Both Ein and Eout go up }\PY{l+s+si}{\PYZob{}counter\PYZus{}p16[2]:.0f\PYZcb{}}\PY{l+s+s1}{ times}\PY{l+s+se}{\PYZbs{}n}\PY{l+s+s1}{[d] Both Ein and Eout go down }\PY{l+s+si}{\PYZob{}counter\PYZus{}p16[3]:.0f\PYZcb{}}\PY{l+s+s1}{ times}\PY{l+s+se}{\PYZbs{}n}\PY{l+s+s1}{[e] Ein and Eout remain the same }\PY{l+s+si}{\PYZob{}counter\PYZus{}p16[4]:.0f\PYZcb{}}\PY{l+s+s1}{ times.}\PY{l+s+s1}{\PYZsq{}}\PY{p}{)}
\end{Verbatim}
\end{tcolorbox}

    \begin{Verbatim}[commandchars=\\\{\}]
Problem 16:
Out of Nruns=100,
[a] Ein goes down, but Eout goes up 16 times
[b] Ein goes up, but Eout goes down 3 times
[c] Both Ein and Eout go up 7 times
[d] Both Ein and Eout go down 34 times
[e] Ein and Eout remain the same 0 times.
    \end{Verbatim}

    \hypertarget{problem-17}{%
\section{Problem 17}\label{problem-17}}

\hypertarget{answer-c-both-e_in-and-e_out-go-up}{%
\subsection{\texorpdfstring{Answer: {[}c{]} Both \(E_{in}\) and
\(E_{out}\) go
up}{Answer: {[}c{]} Both E\_\{in\} and E\_\{out\} go up}}\label{answer-c-both-e_in-and-e_out-go-up}}

\hypertarget{code}{%
\subsection{Code:}\label{code}}

    \begin{tcolorbox}[breakable, size=fbox, boxrule=1pt, pad at break*=1mm,colback=cellbackground, colframe=cellborder]
\prompt{In}{incolor}{9}{\boxspacing}
\begin{Verbatim}[commandchars=\\\{\}]
\PY{n}{Nruns}\PY{o}{=}\PY{l+m+mi}{100}
\PY{n}{Npts}\PY{o}{=}\PY{l+m+mi}{100}
\PY{n}{k}\PY{o}{=}\PY{l+m+mi}{9}
\PY{n}{counter\PYZus{}p17}\PY{o}{=}\PY{n}{np}\PY{o}{.}\PY{n}{zeros}\PY{p}{(}\PY{l+m+mi}{5}\PY{p}{)}

\PY{k}{for} \PY{n}{i} \PY{o+ow}{in} \PY{n+nb}{range}\PY{p}{(}\PY{n}{Nruns}\PY{p}{)}\PY{p}{:}
    \PY{n}{pts}\PY{o}{=}\PY{n}{gen\PYZus{}uniform\PYZus{}points}\PY{p}{(}\PY{n}{Npts}\PY{p}{)}
    \PY{n}{y}\PY{o}{=}\PY{n}{label\PYZus{}f}\PY{p}{(}\PY{n}{pts}\PY{p}{,}\PY{n}{f}\PY{p}{)}
    \PY{n}{Ein\PYZus{}g1}\PY{p}{,}\PY{n}{Eout\PYZus{}g1}\PY{o}{=}\PY{n}{RBF}\PY{p}{(}\PY{n}{pts}\PY{p}{,}\PY{n}{y}\PY{p}{,}\PY{l+m+mf}{1.5}\PY{p}{,}\PY{n}{k}\PY{p}{)}
    \PY{n}{Ein\PYZus{}g2}\PY{p}{,}\PY{n}{Eout\PYZus{}g2}\PY{o}{=}\PY{n}{RBF}\PY{p}{(}\PY{n}{pts}\PY{p}{,}\PY{n}{y}\PY{p}{,}\PY{l+m+mi}{2}\PY{p}{,}\PY{n}{k}\PY{p}{)}
    \PY{k}{if}\PY{p}{(}\PY{p}{(}\PY{n}{Ein\PYZus{}g1}\PY{o}{\PYZgt{}}\PY{n}{Ein\PYZus{}g2}\PY{p}{)}\PY{o}{\PYZam{}}\PY{p}{(}\PY{n}{Eout\PYZus{}g1}\PY{o}{\PYZlt{}}\PY{n}{Eout\PYZus{}g2}\PY{p}{)}\PY{p}{)}\PY{p}{:} \PY{n}{counter\PYZus{}p17}\PY{p}{[}\PY{l+m+mi}{0}\PY{p}{]}\PY{o}{+}\PY{o}{=}\PY{l+m+mi}{1}
    \PY{k}{if}\PY{p}{(}\PY{p}{(}\PY{n}{Ein\PYZus{}g1}\PY{o}{\PYZlt{}}\PY{n}{Ein\PYZus{}g2}\PY{p}{)}\PY{o}{\PYZam{}}\PY{p}{(}\PY{n}{Eout\PYZus{}g1}\PY{o}{\PYZgt{}}\PY{n}{Eout\PYZus{}g2}\PY{p}{)}\PY{p}{)}\PY{p}{:} \PY{n}{counter\PYZus{}p17}\PY{p}{[}\PY{l+m+mi}{1}\PY{p}{]}\PY{o}{+}\PY{o}{=}\PY{l+m+mi}{1}
    \PY{k}{if}\PY{p}{(}\PY{p}{(}\PY{n}{Ein\PYZus{}g1}\PY{o}{\PYZlt{}}\PY{n}{Ein\PYZus{}g2}\PY{p}{)}\PY{o}{\PYZam{}}\PY{p}{(}\PY{n}{Eout\PYZus{}g1}\PY{o}{\PYZlt{}}\PY{n}{Eout\PYZus{}g2}\PY{p}{)}\PY{p}{)}\PY{p}{:} \PY{n}{counter\PYZus{}p17}\PY{p}{[}\PY{l+m+mi}{2}\PY{p}{]}\PY{o}{+}\PY{o}{=}\PY{l+m+mi}{1}
    \PY{k}{if}\PY{p}{(}\PY{p}{(}\PY{n}{Ein\PYZus{}g1}\PY{o}{\PYZgt{}}\PY{n}{Ein\PYZus{}g2}\PY{p}{)}\PY{o}{\PYZam{}}\PY{p}{(}\PY{n}{Eout\PYZus{}g1}\PY{o}{\PYZgt{}}\PY{n}{Eout\PYZus{}g2}\PY{p}{)}\PY{p}{)}\PY{p}{:} \PY{n}{counter\PYZus{}p17}\PY{p}{[}\PY{l+m+mi}{3}\PY{p}{]}\PY{o}{+}\PY{o}{=}\PY{l+m+mi}{1}
    \PY{k}{if}\PY{p}{(}\PY{p}{(}\PY{n}{Ein\PYZus{}g1}\PY{o}{==}\PY{n}{Ein\PYZus{}g2}\PY{p}{)}\PY{o}{\PYZam{}}\PY{p}{(}\PY{n}{Eout\PYZus{}g1}\PY{o}{==}\PY{n}{Eout\PYZus{}g2}\PY{p}{)}\PY{p}{)}\PY{p}{:} \PY{n}{counter\PYZus{}p17}\PY{p}{[}\PY{l+m+mi}{4}\PY{p}{]}\PY{o}{+}\PY{o}{=}\PY{l+m+mi}{1}
              
\PY{n+nb}{print}\PY{p}{(}\PY{n}{f}\PY{l+s+s1}{\PYZsq{}}\PY{l+s+s1}{Problem 17:}\PY{l+s+se}{\PYZbs{}n}\PY{l+s+s1}{Out of Nruns=}\PY{l+s+si}{\PYZob{}Nruns\PYZcb{}}\PY{l+s+s1}{,}\PY{l+s+se}{\PYZbs{}n}\PY{l+s+s1}{[a] Ein goes down, but Eout goes up }\PY{l+s+si}{\PYZob{}counter\PYZus{}p17[0]:.0f\PYZcb{}}\PY{l+s+s1}{ times}\PY{l+s+se}{\PYZbs{}n}\PY{l+s+s1}{[b] Ein goes up, but Eout goes down }\PY{l+s+si}{\PYZob{}counter\PYZus{}p17[1]:.0f\PYZcb{}}\PY{l+s+s1}{ times}\PY{l+s+se}{\PYZbs{}n}\PY{l+s+s1}{[c] Both Ein and Eout go up }\PY{l+s+si}{\PYZob{}counter\PYZus{}p17[2]:.0f\PYZcb{}}\PY{l+s+s1}{ times}\PY{l+s+se}{\PYZbs{}n}\PY{l+s+s1}{[d] Both Ein and Eout go down }\PY{l+s+si}{\PYZob{}counter\PYZus{}p17[3]:.0f\PYZcb{}}\PY{l+s+s1}{ times}\PY{l+s+se}{\PYZbs{}n}\PY{l+s+s1}{[e] Ein and Eout remain the same }\PY{l+s+si}{\PYZob{}counter\PYZus{}p17[4]:.0f\PYZcb{}}\PY{l+s+s1}{ times.}\PY{l+s+s1}{\PYZsq{}}\PY{p}{)}
\end{Verbatim}
\end{tcolorbox}

    \begin{Verbatim}[commandchars=\\\{\}]
Problem 17:
Out of Nruns=100,
[a] Ein goes down, but Eout goes up 5 times
[b] Ein goes up, but Eout goes down 14 times
[c] Both Ein and Eout go up 21 times
[d] Both Ein and Eout go down 9 times
[e] Ein and Eout remain the same 1 times.
    \end{Verbatim}

    \hypertarget{problem-18}{%
\section{Problem 18}\label{problem-18}}

\hypertarget{answer-a-le-10-of-the-time}{%
\subsection{\texorpdfstring{Answer: {[}a{]} \(\le 10 \%\) of the
time}{Answer: {[}a{]} \textbackslash{}le 10 \textbackslash{}\% of the time}}\label{answer-a-le-10-of-the-time}}

\hypertarget{code}{%
\subsection{Code:}\label{code}}

    \begin{tcolorbox}[breakable, size=fbox, boxrule=1pt, pad at break*=1mm,colback=cellbackground, colframe=cellborder]
\prompt{In}{incolor}{10}{\boxspacing}
\begin{Verbatim}[commandchars=\\\{\}]
\PY{n}{Nruns}\PY{o}{=}\PY{l+m+mi}{100}
\PY{n}{Npts}\PY{o}{=}\PY{l+m+mi}{100}
\PY{n}{k}\PY{o}{=}\PY{l+m+mi}{9}
\PY{n}{gamma}\PY{o}{=}\PY{l+m+mf}{1.5}
\PY{n}{counter\PYZus{}p18}\PY{o}{=}\PY{l+m+mi}{0}

\PY{k}{for} \PY{n}{i} \PY{o+ow}{in} \PY{n+nb}{range}\PY{p}{(}\PY{n}{Nruns}\PY{p}{)}\PY{p}{:}
    \PY{n}{pts}\PY{o}{=}\PY{n}{gen\PYZus{}uniform\PYZus{}points}\PY{p}{(}\PY{n}{Npts}\PY{p}{)}
    \PY{n}{y}\PY{o}{=}\PY{n}{label\PYZus{}f}\PY{p}{(}\PY{n}{pts}\PY{p}{,}\PY{n}{f}\PY{p}{)}
    \PY{n}{Ein}\PY{p}{,}\PY{n}{Eout}\PY{o}{=}\PY{n}{RBF}\PY{p}{(}\PY{n}{pts}\PY{p}{,}\PY{n}{y}\PY{p}{,}\PY{n}{gamma}\PY{p}{,}\PY{n}{k}\PY{p}{)}
    \PY{k}{if}\PY{p}{(}\PY{n}{Ein}\PY{o}{==}\PY{l+m+mi}{0}\PY{p}{)}\PY{p}{:} \PY{n}{counter\PYZus{}p18}\PY{o}{+}\PY{o}{=}\PY{l+m+mi}{1}
              
\PY{n+nb}{print}\PY{p}{(}\PY{n}{f}\PY{l+s+s1}{\PYZsq{}}\PY{l+s+s1}{Problem 18:}\PY{l+s+se}{\PYZbs{}n}\PY{l+s+s1}{Out of Nruns=}\PY{l+s+si}{\PYZob{}Nruns\PYZcb{}}\PY{l+s+s1}{,}\PY{l+s+se}{\PYZbs{}n}\PY{l+s+s1}{regular RBF achieves Ein=0 (with gamma=}\PY{l+s+si}{\PYZob{}gamma\PYZcb{}}\PY{l+s+s1}{ and k=}\PY{l+s+si}{\PYZob{}k\PYZcb{}}\PY{l+s+s1}{) }\PY{l+s+s1}{\PYZob{}}\PY{l+s+s1}{counter\PYZus{}p18/Nruns*100:.0f\PYZcb{}}\PY{l+s+si}{\PYZpc{} o}\PY{l+s+s1}{f the times}\PY{l+s+s1}{\PYZsq{}}\PY{p}{)}
\end{Verbatim}
\end{tcolorbox}

    \begin{Verbatim}[commandchars=\\\{\}]
Problem 18:
Out of Nruns=100,
regular RBF achieves Ein=0 (with gamma=1.5 and k=9) 7\% of the times
    \end{Verbatim}

    \hypertarget{bayesian-priors}{%
\part{Bayesian Priors}\label{bayesian-priors}}

\hypertarget{problem-19}{%
\section{Problem 19}\label{problem-19}}

\hypertarget{answer-b-the-posterior-increases-linearly-over-01.}{%
\subsection{\texorpdfstring{Answer: {[}b{]} The posterior increases
linearly over
\([0,1]\).}{Answer: {[}b{]} The posterior increases linearly over {[}0,1{]}.}}\label{answer-b-the-posterior-increases-linearly-over-01.}}

\hypertarget{derivation}{%
\subsection{Derivation:}\label{derivation}}

According to the Bayesian approach, the posterior, i.e.~the probability
distribution \(\mathcal{P}(h=f|\mathcal{D})\) that the hypothesis \(h\)
is equal to the target function \(f\) given the data \(\mathcal{D}\),
can be related to the prior \(\mathcal{P}(h=f)\) and the likelihood
\(\mathcal{P}(\mathcal{D}|h=f)\) up to a normalization constant as the
following

\begin{equation}
\mathcal{P}(h=f|\mathcal{D})\propto \mathcal{P}(h=f) \mathcal{P}(\mathcal{D}|h=f)\,.
\end{equation}

In our specific example, we are choosing an (uninformative) uniform
prior over the interval \([0,1]\), so that the probability density can
be written as:

\begin{equation}
\mathcal{p}(h=f)|_{ {\rm\, over\,} [a=0,b=1]}= \frac{1}{b-a}=\frac{1}{1-0}=1\,,
\end{equation}

where \(d\mathcal{P}(x)=p(x)dx\). Since the event has a binary outcome
(either a person had an heart attack or not), we can choose the
likelihood to be the binomial distribution. Setting the (unknown)
probability of getting an heart attack equal to \(h\), the probability
that in a sample of \(N\) people, \(k\) of them have had an heart attack
is equal to:

\begin{equation}
\mathcal{P}(N,k;h)= {N\choose k}h^k(1-h)^{N-k}\,.
\end{equation}

In our case \(N=k=1\), leading to \(\mathcal{P}(1,1;h)=h\), i.e.~the
likelihood is equal to the probability of getting an heart attack.
Finally, we have that

\begin{equation}
d\mathcal{P}(h=f|\mathcal{D};x)\propto h \,dx \,\rightarrow \mathcal{P}(h=f|\mathcal{D};x)=hx \,,
\end{equation}

where \(x\in[0,1]\), i.e.~the posterior probability that \(h=f\)
increases linearly over \([0,1]\).

    \hypertarget{aggregation}{%
\part{Aggregation}\label{aggregation}}

\hypertarget{problem-20}{%
\section{Problem 20}\label{problem-20}}

\hypertarget{answer-c-e_outg-cannot-be-worse-than-the-average-of-e_outg_1-and-e_outg_2.}{%
\subsection{\texorpdfstring{Answer: {[}c{]} \(E_{out}(g)\) cannot be
worse than the average of \(E_{out}(g_1)\) and
\(E_{out}(g_2)\).}{Answer: {[}c{]} E\_\{out\}(g) cannot be worse than the average of E\_\{out\}(g\_1) and E\_\{out\}(g\_2).}}\label{answer-c-e_outg-cannot-be-worse-than-the-average-of-e_outg_1-and-e_outg_2.}}

\hypertarget{derivation}{%
\subsection{Derivation:}\label{derivation}}

Using the mean-square error definition, we have that the error
\(E_{out}(g)\) on the aggregate hypothesis \(g=\frac{1}{2} (g_1+g_2)\)
is
\begin{equation}
E_{out}(g)=\mathbb{E}\left[\left(f-g\right)^2\right]=\mathbb{E}\left[\left(f-\frac{1}{2}(g_1+g_2)\right)^2\right]\,,
\end{equation}

where we dropped the explicit dependence on the data points \(x\in\chi\)
on which the expectation value \(\mathbb{E}\) is computed on.

After some algebraic manipulation, using the linearity of \(\mathbb{E}\)
we can write:
\begin{equation}
\begin{split}
E_{out}(g)&=\frac{1}{2}\left\{\mathbb{E}\left[\left(f-g_1\right)^2\right]+\mathbb{E}\left[\left(f-g_2\right)^2\right]\right\}-\frac{1}{4}\mathbb{E}\left[\left(g_1-g_2\right)^2\right]\\ &=\frac{E_{out}(g_1)+E_{out}(g_2)}{2}-\sigma^2=\bar{E}_{out}-\sigma^2\,,
\end{split}
\end{equation}
where
\(\sigma^2\equiv\frac{1}{4}\mathbb{E}\left[\left(g_1-g_2\right)^2\right]\)
is a positive defined quantity. This last relation shows that
\(E_{out}(g)\) is equal or smaller than the average \(\bar{E}_{out}\) of
\(E_{out}(g_1)\) and \(E_{out}(g_2)\).

    \begin{tcolorbox}[breakable, size=fbox, boxrule=1pt, pad at break*=1mm,colback=cellbackground, colframe=cellborder]
\prompt{In}{incolor}{ }{\boxspacing}
\begin{Verbatim}[commandchars=\\\{\}]

\end{Verbatim}
\end{tcolorbox}


    % Add a bibliography block to the postdoc
    
    
    
\end{document}
